%===========================================================
% summary.tex - 总结与作业
%===========================================================

\section*{总结与作业}

\begin{frame}{差异化学习路径}
    \textbf{三种角色,三种任务:}

    \vspace{0.3cm}

    \begin{table}
        \centering
        \small
        \begin{tabular}{p{2.5cm}p{4cm}p{4cm}}
            \toprule
            \textbf{角色} \u0026 \textbf{基础任务(60分)} \u0026 \textbf{进阶/挑战任务(40分)} \\
            \midrule
            \textbf{观察者} \u0026
            跟着运行PaddleOCR手写识别代码,理解识别流程 \u0026
            分析不同预处理方法(去噪强度、CLAHE参数)对识别效果的影响差异 \\
            \midrule
            \textbf{使用者} \u0026
            完成手写文字识别功能,能识别简单手写答案 \u0026
            对比PaddleOCR和TrOCR的识别效果,分析准确率和速度差异 \\
            \midrule
            \textbf{创造者} \u0026
            完成所有基础功能 \u0026
            设计新的手写识别场景(如签名识别、公式识别),并实现端到端识别系统 \\
            \bottomrule
        \end{tabular}
    \end{table}

    \vspace{0.3cm}

    \textbf{选择建议:}
    \begin{itemize}
        \item 无编程基础或时间紧张:选择\textbf{"观察者"}
        \item 有基础但时间有限:选择\textbf{"使用者"}
        \item 有基础且兴趣浓厚:选择\textbf{"创造者"}
    \end{itemize}

    \vspace{0.3cm}

    \begin{alertblock}{提示}
        不同角色可以使用不同的OCR工具:
        \begin{itemize}
            \item 观察者:使用PaddleOCR基础版
            \item 使用者:使用PaddleOCR + 参数优化
            \item 创造者:尝试TrOCR或对比多种工具
        \end{itemize}
    \end{alertblock}
\end{frame}

\begin{frame}{本周要点总结}
    \begin{columns}
        \column{0.5\textwidth}
        \textbf{手写识别概述}
        \begin{itemize}
            \item 字形差异大、连笔问题
            \item 准确率95\%+ (TrOCR)
            \item 工具选择:PaddleOCR vs TrOCR
        \end{itemize}

        \vspace{0.3cm}

        \textbf{PaddleOCR}
        \begin{itemize}
            \item 中文支持好
            \item 速度快
            \item 配置灵活
        \end{itemize}

        \column{0.5\textwidth}
        \textbf{TrOCR}
        \begin{itemize}
            \item 准确率最高
            \item Transformer架构
            \item 适合高精度需求
        \end{itemize}

        \vspace{0.3cm}

        \textbf{预处理优化}
        \begin{itemize}
            \item 灰度化 + 去噪
            \item 对比度增强
            \item 二值化
        \end{itemize}
    \end{columns}
\end{frame}

%----------------------------------------------------------
% 新增:手写识别完整流程图
%----------------------------------------------------------
\begin{frame}{手写识别完整流程图}
    \centering
    \begin{tikzpicture}[
        node distance=0.8cm and 1.2cm,
        box/.style={
            draw=themecolor,
            fill=white,
            minimum width=2.2cm,
            minimum height=0.7cm,
            font=\scriptsize,
            align=center,
            rounded corners=2pt
        },
        stage/.style={
            draw=themecolor!70,
            fill=themecolor!10,
            minimum width=2.2cm,
            minimum height=0.7cm,
            font=\scriptsize\bfseries,
            align=center,
            rounded corners=2pt
        },
        arrow/.style={
            ->,
            >=stealth,
            thick,
            themecolor
        },
        tech/.style={
            font=\tiny\color{gray},
            align=center
        }
    ]
        % 第一行:输入
        \node[box, fill=themecolor!20] (input) {手写图像};

        % 第二行:预处理
        \node[stage, below=0.5cm of input] (pre) {预处理};

        % 预处理技术节点
        \node[box, below left=0.3cm and 0.1cm of pre] (gray) {灰度化};
        \node[box, below=0.3cm of pre] (denoise) {去噪};
        \node[box, below right=0.3cm and 0.1cm of pre] (enhance) {对比度增强};

        % 第三行:文本行检测
        \node[stage, below=1.5cm of pre] (detect) {文本行检测};

        % 第四行:字符分割
        \node[stage, below=0.5cm of detect] (split) {字符分割};

        % 第五行:特征提取与识别
        \node[stage, below=0.5cm of split] (recognize) {特征提取与识别};

        % 第六行:后处理
        \node[stage, below=0.5cm of recognize] (post) {后处理};

        % 第七行:输出
        \node[box, fill=themecolor!20, below=0.5cm of post] (output) {识别文本};

        % 连接箭头
        \draw[arrow] (input) -- (pre);
        \draw[arrow] (gray) -- (detect);
        \draw[arrow] (denoise) -- (detect);
        \draw[arrow] (enhance) -- (detect);
        \draw[arrow] (detect) -- (split);
        \draw[arrow] (split) -- (recognize);
        \draw[arrow] (recognize) -- (post);
        \draw[arrow] (post) -- (output);

    \end{tikzpicture}
\end{frame}

\begin{frame}{手写识别流程详解}
    \begin{columns}
        \column{0.5\textwidth}
        \begin{block}{1. 预处理阶段}
            \footnotesize
            \begin{itemize}
                \item \textbf{灰度化}:降低计算复杂度
                \item \textbf{去噪}:高斯滤波、中值滤波
                \item \textbf{对比度增强}:直方图均衡化、CLAHE
                \item \textbf{二值化}:自适应阈值分割
            \end{itemize}
        \end{block}

        \vspace{0.2cm}

        \begin{block}{2. 文本行检测}
            \footnotesize
            \begin{itemize}
                \item \textbf{投影法}:水平投影分割行
                \item \textbf{连通域分析}:DBSCAN、MSER
                \item \textbf{深度学习方法}:EAST、CRAFT、PSENet
            \end{itemize}
        \end{block}

        \column{0.5\textwidth}
        \begin{block}{3. 字符分割与识别}
            \footnotesize
            \begin{itemize}
                \item \textbf{字符分割}:垂直投影、Viterbi算法
                \item \textbf{特征提取}:
                \begin{itemize}
                    \footnotesize
                    \item 传统:HOG、SIFT、Gabor特征
                    \item 深度学习:CNN特征、Transformer特征
                \end{itemize}
                \item \textbf{分类识别}:SVM、CNN、Transformer
            \end{itemize}
        \end{block}

        \vspace{0.2cm}

        \begin{block}{4. 后处理优化}
            \footnotesize
            \begin{itemize}
                \item \textbf{语言模型校正}:N-gram、RNNLM
                \item \textbf{词典匹配}:模糊匹配、编辑距离
                \item \textbf{上下文校正}:BERT-based纠错
            \end{itemize}
        \end{block}

    \end{columns}
\end{frame}

%----------------------------------------------------------
% 新增:传统方法 vs 深度学习方法对比表
%----------------------------------------------------------
\begin{frame}{传统方法 vs 深度学习方法对比}
    \begin{table}
        \centering
        \footnotesize
        \begin{tabular}{>{\raggedright\arraybackslash}p{2.2cm}|>{\raggedright\arraybackslash}p{4.5cm}|>{\raggedright\arraybackslash}p{4.5cm}}
            \toprule
            \textbf{对比维度} & \textbf{传统方法} & \textbf{深度学习方法} \\
            \midrule
            \textbf{特征提取} &
            人工设计特征:HOG、SIFT、Gabor特征、投影特征等。需要领域专家知识,特征设计耗时。 &
            自动学习特征:CNN自动提取层次化特征,Transformer捕捉长距离依赖。无需人工设计。 \\
            \midrule
            \textbf{模型复杂度} &
            模型简单:SVM、KNN、HMM等,参数量小,可解释性强。 &
            模型复杂:深度神经网络,参数量大(百万至十亿级),计算量大。 \\
            \midrule
            \textbf{数据需求} &
            数据需求小:在少量样本上也能取得不错效果,适合数据稀缺场景。 &
            数据需求大:需要大量标注数据训练,但在大数据下性能显著优于传统方法。 \\
            \bottomrule
        \end{tabular}
        \caption{传统方法与深度学习方法对比(一)}
    \end{table}
\end{frame}

\begin{frame}{传统方法 vs 深度学习方法对比(续)}
    \begin{table}
        \centering
        \footnotesize
        \begin{tabular}{>{\raggedright\arraybackslash}p{2.2cm}|>{\raggedright\arraybackslash}p{4.5cm}|>{\raggedright\arraybackslash}p{4.5cm}}
            \toprule
            \textbf{对比维度} & \textbf{传统方法} & \textbf{深度学习方法} \\
            \midrule
            \textbf{准确率} &
            准确率有限:手写汉字识别率通常在70-85\%,对复杂变形、连笔、噪声敏感。 &
            准确率高:TrOCR等模型可达95\%+,对复杂场景鲁棒性强,泛化能力好。 \\
            \midrule
            \textbf{推理速度} &
            速度快:轻量级模型,CPU上可实时处理,适合边缘设备部署。 &
            速度较慢:需要GPU加速,大模型推理延迟较高,但可通过量化、剪枝优化。 \\
            \midrule
            \textbf{可解释性} &
            可解释性强:特征工程清晰,决策过程透明,易于调试和分析错误。 &
            可解释性弱:黑盒模型,难以理解内部决策机制,需要特殊工具(如注意力可视化)。 \\
            \midrule
            \textbf{适用场景} &
            资源受限、数据稀缺、需要快速部署、对可解释性要求高的场景。 &
            追求高准确率、有充足数据和计算资源、对实时性要求不极端的场景。 \\
            \bottomrule
        \end{tabular}
        \caption{传统方法与深度学习方法对比(二)}
    \end{table}
\end{frame}

%----------------------------------------------------------
% 新增:各工具/框架对比表
%----------------------------------------------------------
\begin{frame}{主流OCR工具/框架对比}
    \begin{table}
        \centering
        \scriptsize
        \begin{tabular}{>{\raggedright\arraybackslash}p{1.8cm}|>{\centering\arraybackslash}p{1.8cm}|>{\centering\arraybackslash}p{1.8cm}|>{\centering\arraybackslash}p{1.8cm}|>{\centering\arraybackslash}p{1.8cm}}
            \toprule
            \textbf{对比维度} & \textbf{PaddleOCR} & \textbf{TrOCR} & \textbf{Tesseract} & \textbf{EasyOCR} \\
            \midrule
            \textbf{开发团队} & 百度 & Microsoft & Google & 社区开源 \\
            \midrule
            \textbf{支持语言} & 80+种,中文优秀 & 多语言,英文最优 & 100+种 & 80+种 \\
            \midrule
            \textbf{手写识别} & 良好(需微调) & 优秀(原生支持) & 一般 & 一般 \\
            \midrule
            \textbf{印刷体识别} & 优秀 & 优秀 & 良好 & 良好 \\
            \midrule
            \textbf{准确率} & 85-90\% & 95\%+ & 70-80\% & 75-85\% \\
            \midrule
            \textbf{推理速度} & 快(GPU/CPU) & 中等(需GPU) & 快(CPU) & 中等 \\
            \midrule
            \textbf{模型大小} & 轻量-中等 & 大(Transformer) & 轻量 & 中等 \\
            \midrule
            \textbf{部署难度} & 简单 & 中等 & 简单 & 简单 \\
            \midrule
            \textbf{文档完善度} & 优秀 & 良好 & 良好 & 一般 \\
            \midrule
            \textbf{社区活跃度} & 高(Star 40k+) & 中等 & 高 & 中等 \\
            \midrule
            \textbf{适用场景} & 中文场景、移动端 & 高精度需求 & 多语言、轻量 & 快速原型 \\
            \bottomrule
        \end{tabular}
        \caption{主流OCR工具/框架全面对比}
    \end{table}
\end{frame}

%----------------------------------------------------------
% 新增:工具选择建议
%----------------------------------------------------------
\begin{frame}{OCR工具选择决策树}
    \begin{columns}
        \column{0.6\textwidth}
        \begin{tikzpicture}[
            node distance=0.5cm,
            decision/.style={
                draw=themecolor,
                fill=themecolor!10,
                minimum width=3cm,
                minimum height=0.6cm,
                font=\scriptsize,
                align=center,
                rounded corners=2pt
            },
            result/.style={
                draw=themecolor!70,
                fill=white,
                minimum width=2.5cm,
                minimum height=0.6cm,
                font=\scriptsize\bfseries,
                align=center,
                rounded corners=2pt
            },
            arrow/.style={
                ->,
                >=stealth,
                thick,
                themecolor
            }
        ]
            % 根节点
            \node[decision] (root) {主要识别语言?};

            % 第二层
            \node[decision, below left=0.6cm and 0.3cm of root] (chinese) {中文为主};
            \node[decision, below right=0.6cm and 0.3cm of root] (english) {英文为主};

            % 第三层
            \node[decision, below=0.5cm of chinese] (gpu) {有GPU资源?};
            \node[decision, below=0.5cm of english] (accuracy) {精度要求?};

            % 第四层 - 结果
            \node[result, below left=0.5cm and 0.1cm of gpu] (paddle) {PaddleOCR};
            \node[result, below right=0.5cm and 0.1cm of gpu] (tesseract) {Tesseract};
            \node[result, below left=0.5cm and 0.1cm of accuracy] (trocr) {TrOCR};
            \node[result, below right=0.5cm and 0.1cm of accuracy] (easyocr) {EasyOCR};

            % 连接
            \draw[arrow] (root) -- (chinese);
            \draw[arrow] (root) -- (english);
            \draw[arrow] (chinese) -- (gpu);
            \draw[arrow] (english) -- (accuracy);
            \draw[arrow] (gpu) -- (paddle);
            \draw[arrow] (gpu) -- (tesseract);
            \draw[arrow] (accuracy) -- (trocr);
            \draw[arrow] (accuracy) -- (easyocr);

        \end{tikzpicture}

        \column{0.4\textwidth}
        \footnotesize
        \begin{block}{选择建议}
            \textbf{中文场景:}
            \begin{itemize}
                \item 有GPU:PaddleOCR
                \item 无GPU:Tesseract
            \end{itemize}

            \textbf{英文场景:}
            \begin{itemize}
                \item 高精度:TrOCR
                \item 快速部署:EasyOCR
            \end{itemize}
        \end{block}

        \begin{alertblock}{提示}
            实际选择时还需考虑:\\
            - 硬件资源限制\\
            - 实时性要求\\
            - 维护成本
        \end{alertblock}
    \end{columns}
\end{frame}

%----------------------------------------------------------
% 新增:常见问题解决方案
%----------------------------------------------------------
\begin{frame}{常见问题与解决方案(一)}
    \begin{block}{问题1:识别准确率低}
        \footnotesize
        \textbf{原因分析:}
        \begin{itemize}
            \item 图像质量差:噪声、模糊、光照不均
            \item 预处理不足:未进行有效的去噪和增强
            \item 模型不匹配:选择的模型不适合特定场景
            \item 字体/书写风格差异:训练数据覆盖不足
        \end{itemize}

        \textbf{解决方案:}
        \begin{enumerate}
            \item 加强预处理:使用自适应阈值、CLAHE增强对比度
            \item 尝试多种去噪算法:高斯滤波、中值滤波、双边滤波
            \item 更换或组合模型:如PaddleOCR + TrOCR结果融合
            \item 针对性微调:使用领域数据对模型进行微调
        \end{enumerate}
    \end{block}
\end{frame}

\begin{frame}{常见问题与解决方案(二)}
    \begin{columns}
        \column{0.5\textwidth}
        \begin{block}{问题2:处理速度慢}
            \footnotesize
            \textbf{优化方案:}
            \begin{itemize}
                \item \textbf{硬件加速:}使用GPU替代CPU,批量处理
                \item \textbf{模型轻量化:}使用轻量级模型(如PaddleOCR mobile版)
                \item \textbf{推理优化:}ONNX导出、TensorRT加速、量化压缩
                \item \textbf{多线程并行:}利用多核CPU并行处理多图像
                \item \textbf{预处理优化:}降低输入图像分辨率(保持宽高比)
            \end{itemize}
        \end{block}

        \column{0.5\textwidth}
        \begin{block}{问题3:内存不足}
            \footnotesize
            \textbf{解决方案:}
            \begin{itemize}
                \item \textbf{分批处理:}避免一次性加载所有图像
                \item \textbf{降低batch size:}减少同时处理的样本数
                \item \textbf{模型精简:}使用更小的模型变体
                \item \textbf{清空缓存:}及时处理释放不再使用的变量
                \item \textbf{使用16位浮点:}FP16混合精度训练/推理
            \end{itemize}
        \end{block}

        \vspace{0.2cm}

        \begin{block}{问题4:特殊字符识别困难}
            \footnotesize
            \begin{itemize}
                \item 扩充训练数据覆盖特殊字符
                \item 使用专用字体数据集微调
                \item 添加字符级后处理规则
            \end{itemize}
        \end{block}
    \end{columns}
\end{frame}

\begin{frame}{常见问题解答}
    \begin{block}{Q1:手写识别准确率能达到多少?}
        A:TrOCR可以达到95\%+,PaddleOCR约85-90\%,取决于预处理质量。
    \end{block}

    \vspace{0.3cm}

    \begin{block}{Q2:如何提高识别准确率?}
        A:
        \begin{enumerate}
            \item 优化图像预处理(去噪、增强对比度)
            \item 使用更高精度的模型(TrOCR large)
            \item 针对特定场景进行微调训练
        \end{enumerate}
    \end{block}

    \vspace{0.3cm}

    \begin{block}{Q3:连笔字如何处理?}
        A:TrOCR对连笔字识别效果更好,因为Transformer能捕捉上下文关系。
    \end{block}
\end{frame}

\begin{frame}{作业提交要求}
    \begin{block}{作业名称}
        week08\_手写简答题识别.ipynb
    \end{block}

    \vspace{0.3cm}

    \textbf{提交内容:}
    \begin{enumerate}
        \item \textbf{代码部分}(60\%)
        \begin{itemize}
            \item 模型配置与加载
            \item 预处理函数实现
            \item 手写文字识别
            \item 结果可视化
        \end{itemize}

        \item \textbf{分析报告}(40\%)
        \begin{itemize}
            \item 识别准确率分析
            \item 不同模型的对比
            \item 遇到的问题及解决方案
        \end{itemize}
    \end{enumerate}
\end{frame}

\begin{frame}{评分标准细则}
    \begin{table}
        \centering
        \small
        \begin{tabular}{lp{6cm}c}
            \toprule
            \textbf{评分项} & \textbf{具体要求} & \textbf{分值} \\
            \midrule
            模型配置 & 正确加载PaddleOCR或TrOCR模型 & 25分 \\
            预处理优化 & 实现灰度化、去噪、增强、二值化 & 25分 \\
            识别效果 & 能正确识别手写文字,准确率高 & 30分 \\
            分析报告 & 对比分析、问题总结、改进建议 & 20分 \\
            \midrule
            \textbf{总分} & & \textbf{100分} \\
            \bottomrule
        \end{tabular}
    \end{table}

    \vspace{0.3cm}

    \begin{alertblock}{加分项}
        \begin{itemize}
            \item 同时实现PaddleOCR和TrOCR并对比(+5分)
            \item 实现批处理多张手写图片(+5分)
            \item 可视化识别结果(标注识别区域)(+5分)
        \end{itemize}
    \end{alertblock}
\end{frame}
