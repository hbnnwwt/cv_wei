\documentclass[aspectratio=169, 12pt]{beamer}
\usepackage[UTF8]{ctex}
\usepackage{graphicx}
\usepackage{booktabs}
\usepackage{listings}
\usepackage{xcolor}
\usepackage{tikz}
\usepackage{hyperref}

\usetheme{Madrid}
\usecolortheme{whale}
\usefonttheme{professionalfonts}

% 页脚logo(缩小显示)
\logo{\includegraphics[height=0.8cm]{../xiaohui.png}}

\lstset{
    language=Python,
    basicstyle=\ttfamily\small,
    keywordstyle=\color{blue},
    commentstyle=\color{green!60!black},
    stringstyle=\color{orange},
    breaklines=true,
    frame=single,
    showstringspaces=false,
    backgroundcolor=\color{gray!10}
}

\title[手写简答题识别]{第8周:手写简答题识别}
\subtitle{能看懂学生写的答案吗?}
\author{北京石油化工学院\\人工智能研究院\\王文通}
\institute{通选课}
\date{2025-2026 学年}
\titlegraphic{
    \includegraphics[height=1.2cm]{../xiaohui.png}\hspace{2cm}
    \includegraphics[height=1.2cm]{../name.png}
}

\begin{document}

\begin{frame}
    \titlepage
\end{frame}

\begin{frame}{课程概览}
    \tableofcontents
\end{frame}

\section{手写识别概述}

\begin{frame}{手写识别的挑战}
    \begin{columns}
        \column{0.5\textwidth}
        \textbf{印刷文字:}
        \begin{itemize}
            \item 字体统一
            \item 字形稳定
            \item 间距规范
        \end{itemize}

        \column{0.5\textwidth}
        \textbf{手写文字:}
        \begin{itemize}
            \item 每人不同
            \item 形状变化大
            \item 间距随意
        \end{itemize}
    \end{columns}

    \vspace{0.5cm}

    \textbf{主要难点:}
    \begin{enumerate}
        \item 字形差异大
        \item 连笔问题
        \item 倾斜旋转
        \item 图像质量
    \end{enumerate}
\end{frame}

\begin{frame}{手写识别技术发展}
    \begin{table}
        \centering
        \begin{tabular}{lp{6cm}l}
            \toprule
            \textbf{阶段} & \textbf{技术} & \textbf{准确率} \\
            \midrule
            传统方法 & 特征工程 + SVM/HMM & 60-70\% \\
            CNN时代 & LeNet, AlexNet & 80-85\% \\
            RNN+Attention & RNN+Attention & 90-92\% \\
            Transformer & \textbf{TrOCR, Donut} & \textbf{95\%+} \\
            \bottomrule
        \end{tabular}
    \end{table}

    \textbf{当前最佳:} TrOCR(Microsoft)
\end{frame}

\begin{frame}{主流手写识别工具}
    \begin{table}
        \centering
        \begin{tabular}{lp{4cm}lp{3cm}}
            \toprule
            \textbf{工具} & \textbf{优点} & \textbf{缺点} & \textbf{适用} \\
            \midrule
            TrOCR & 准确率最高 & 速度慢 & 高精度 \\
            PaddleOCR & 速度快 & 手写略弱 & 实时 \\
            Tesseract & 免费 & 中文差 & 英文 \\
            \bottomrule
        \end{tabular}
    \end{table}

    \textbf{推荐:}
    \begin{itemize}
        \item 高精度需求 $\to$ TrOCR
        \item 实时应用 $\to$ PaddleOCR
    \end{itemize}
\end{frame}

\section{PaddleOCR手写识别}

\begin{frame}[fragile]{PaddleOCR手写配置}
    \begin{lstlisting}
from paddleocr import PaddleOCR

# 初始化手写识别OCR
ocr_handwrite = PaddleOCR(
    use_angle_cls=True,    # 启用方向分类
    lang='ch',             # 中文
    show_log=False
)

# 识别
result = ocr_handwrite.ocr('handwriting.jpg', cls=True)

# 打印结果
for line in result:
    text = line[1][0]
    confidence = line[1][1]
    print(f"{text} ({confidence:.4f})")
    \end{lstlisting}

    \textbf{注意:} PaddleOCR默认已支持手写识别
\end{frame}

\section{TrOCR高精度识别}

\begin{frame}{TrOCR简介}
    \textbf{TrOCR:} Transformer-based OCR

    \begin{itemize}
        \item 开发者:Microsoft
        \item 架构:ViT (图像编码器) + GPT-2 (文本解码器)
        \item 优势:端到端训练,准确率最高
    \end{itemize}

    \vspace{0.3cm}

    \textbf{模型:}
    \begin{itemize}
        \item trocr-base-handwritten
        \item trocr-large-handwritten
    \end{itemize}
\end{frame}

\begin{frame}[fragile]{TrOCR使用}
    \begin{lstlisting}
# 安装
pip install transformers torch pillow

# 使用
from transformers import TrOCRProcessor, VisionEncoderDecoderModel
from PIL import Image

# 加载模型
processor = TrOCRProcessor.from_pretrained(
    'microsoft/trocr-base-handwritten'
)
model = VisionEncoderDecoderModel.from_pretrained(
    'microsoft/trocr-base-handwritten'
)

# 识别
image = Image.open('handwriting.jpg')
pixel_values = processor(images=image, return_tensors="pt").pixel_values
generated_ids = model.generate(pixel_values)
text = processor.batch_decode(generated_ids, skip_special_tokens=True)[0]
    \end{lstlisting}
\end{frame}

\section{图像预处理优化}

\begin{frame}[fragile]{手写图像预处理}
    \begin{lstlisting}
def preprocess_handwriting(image):
    """手写图像预处理"""
    # 转灰度
    gray = cv2.cvtColor(image, cv2.COLOR_BGR2GRAY)

    # 去噪
    denoised = cv2.fastNlMeansDenoising(gray)

    # 对比度增强
    clahe = cv2.createCLAHE(clipLimit=2.0, tileGridSize=(8, 8))
    enhanced = clahe.apply(denoised)

    # 二值化
    _, binary = cv2.threshold(
        enhanced, 0, 255,
        cv2.THRESH_BINARY + cv2.THRESH_OTSU
    )

    return binary
    \end{lstlisting}

    \textbf{关键:} 手写图像需要更强的预处理!
\end{frame}

\section{思考题}

\begin{frame}{课堂思考题}
    \begin{block}{问题1:手写识别挑战}
        \begin{itemize}
            \item 为什么手写识别比印刷识别难?
            \item 连笔字如何处理?
        \end{itemize}
    \end{block}

    \vspace{0.3cm}

    \begin{block}{问题2:模型选择}
        \begin{itemize}
            \item 什么时候用PaddleOCR?什么时候用TrOCR?
            \item 如何平衡准确率和速度?
        \end{itemize}
    \end{block}
\end{frame}

\section{课后作业}

\begin{frame}{课后作业}
    \begin{block}{题目}
        实现简答题手写识别模块
    \end{block}

    \textbf{要求:}
    \begin{enumerate}
        \item 使用PaddleOCR或TrOCR识别手写文字
        \item 实现图像预处理优化
        \item 识别手写答案内容
        \item 分析识别准确率
    \end{enumerate}

    \vspace{0.2cm}

    \textbf{评分标准:}
    \begin{itemize}
        \item 模型配置:25分
        \item 预处理优化:25分
        \item 识别效果:30分
        \item 分析报告:20分
    \end{itemize}
\end{frame}

\begin{frame}{下节预告}
    \begin{center}
        \Large \textbf{第9周:系统架构与分组开发}

        \vspace{0.5cm}

        \normalsize
        故事问题:\textcolor{blue}{把所有模块组合起来}

        \vspace{0.3cm}

        你将学会:
        \begin{itemize}
            \item 系统架构设计
            \item 模块集成方法
            \item 团队协作开发
        \end{itemize}

        \vspace{0.3cm}

        \textbf{课程设计开始!}
    \end{center}
\end{frame}

\begin{frame}
    \begin{center}
        \Huge \textbf{谢谢!}
    \end{center}
\end{frame}

\end{document}
