%===========================================================
% 05_exercises.tex - 思考题与作业
%===========================================================

\section{思考题与作业}

\begin{frame}{课堂思考题}
    \begin{block}{问题1:手写识别挑战}
        \begin{itemize}
            \item 为什么手写识别比印刷识别难?
            \item 连笔字如何处理?
        \end{itemize}
    \end{block}

    \vspace{0.3cm}

    \begin{block}{问题2:模型选择}
        \begin{itemize}
            \item 什么时候用PaddleOCR?什么时候用TrOCR?
            \item 如何平衡准确率和速度?
        \end{itemize}
    \end{block}
\end{frame}

\begin{frame}{课堂互动Quiz - 手写识别技术选择}
    \begin{block}{Quiz 1:单选题 - OCR工具选择}
        \textbf{场景:}需要识别一份手写简答题试卷,要求准确率90\%以上,服务器有GPU。

        \vspace{0.2cm}

        \textbf{最佳选择是?}

        \begin{enumerate}[A.]
            \item 传统模板匹配方法
            \item PaddleOCR + 预处理优化
            \item 仅使用TrOCR,不做预处理
            \item 手工录入所有答案
        \end{enumerate}

        \vspace{0.2cm}

        \textbf{正确答案:B}

        \textit{解析:PaddleOCR对中文支持好,配合预处理可以达到90\%+准确率。}
    \end{block}
\end{frame}

\begin{frame}{课堂互动Quiz - 对比分析}
    \begin{columns}
        \begin{column}{0.5\textwidth}
            \begin{block}{Quiz 2:多选题 - PaddleOCR vs TrOCR}
                关于PaddleOCR和TrOCR的比较,以下说法正确的是:

                \vspace{0.2cm}

                \begin{enumerate}[(1)]
                    \item PaddleOCR对中文手写体支持更好
                    \item TrOCR更适合英文手写识别
                    \item PaddleOCR部署更简单
                    \item TrOCR不需要文本行检测
                    \item PaddleOCR支持更多预处理选项
                \end{enumerate}

                \vspace{0.2cm}

                \textbf{正确答案:(1)(2)(4)(5)}
            \end{block}
        \end{column}
        \begin{column}{0.5\textwidth}
            \begin{block}{Quiz 3:排序题 - 预处理步骤}
                将以下预处理步骤按正确顺序排列:

                \vspace{0.2cm}

                \begin{itemize}
                    \item[A] 去噪
                    \item[B] 二值化
                    \item[C] 灰度化
                    \item[D] 对比度增强
                \end{itemize}

                \vspace{0.3cm}

                \textbf{正确顺序:C → A → D → B}

                \vspace{0.3cm}

                \textit{解释:}
                \begin{enumerate}
                    \item 先转灰度
                    \item 去噪避免增强噪声
                    \item 增强对比度
                    \item 最后二值化
                \end{enumerate}
            \end{block}
        \end{column}
    \end{columns}
\end{frame}

\begin{frame}{课堂互动Quiz - 实时投票}
    \begin{block}{投票1:你认为手写识别最大的挑战是?}
        \begin{columns}
            \begin{column}{0.6\textwidth}
                \begin{enumerate}[A.]
                    \item 字体风格差异大(草书vs楷书)
                    \item 书写质量参差不齐
                    \item 中文同音字太多
                    \item 计算资源限制
                \end{enumerate}

                \vspace{0.3cm}

                \textit{请举手或使用在线投票工具选择你的答案}
            \end{column}
            \begin{column}{0.4\textwidth}
                \begin{tikzpicture}
                    \pie[rotate=90, text=legend, radius=1.5]{
                        40/字体风格差异,
                        30/书写质量,
                        20/同音字问题,
                        10/计算资源
                    }
                \end{tikzpicture}

                \vspace{0.2cm}

                \footnotesize\textit{示例投票结果}
            \end{column}
        \end{columns}
    \end{block}
\end{frame}

\begin{frame}{参数调优实验挑战 - 对比实验设计}
    \begin{block}{挑战1:不同预处理参数对比}
        \textbf{实验目标:}找到最优的预处理参数组合

        \vspace{0.2cm}

        \textbf{实验设计:}

        \begin{table}
            \centering
            \small
            \begin{tabular}{c|ccc|c}
                \toprule
                \textbf{实验组} & \textbf{去噪强度} & \textbf{CLAHE clip} & \textbf{二值化方式} & \textbf{OCR准确率} \\
                \midrule
                A & 无 & - & 简单阈值 & ? \\
                B & h=10 & 2.0 & OTSU & ? \\
                C & h=15 & 3.0 & OTSU & ? \\
                D & h=20 & 4.0 & 自适应 & ? \\
                \bottomrule
            \end{tabular}
        \end{table}

        \textbf{任务:}在提供的测试图片上运行各组参数,记录OCR准确率,找出最优参数组合。
    \end{block}
\end{frame}

\begin{frame}{参数调优实验挑战 - 模型参数调优}
    \begin{columns}
        \begin{column}{0.5\textwidth}
            \begin{block}{挑战2:PaddleOCR参数调优}
                \textbf{可调参数:}

                \begin{itemize}
                    \item \texttt{det\_limit\_side\_len}: 检测尺寸限制
                    \item \texttt{det\_db\_thresh}: DB阈值
                    \item \texttt{det\_db\_box\_thresh}: 文本框阈值
                    \item \texttt{rec\_batch\_num}: 识别批大小
                    \item \texttt{drop\_score}: 置信度过滤阈值
                \end{itemize}

                \vspace{0.2cm}

                \textbf{实验任务:}
                \begin{enumerate}
                    \item 固定其他参数
                    \item 每次只调整一个参数
                    \item 记录识别准确率变化
                    \item 绘制参数-准确率曲线
                \end{enumerate}
            \end{block}
        \end{column}
        \begin{column}{0.5\textwidth}
            \begin{block}{挑战3:TrOCR beam size调优}
                \textbf{什么是beam search?}

                \textit{束搜索是一种解码策略,在每个时间步保留得分最高的k个候选序列,平衡搜索质量和效率。}

                \vspace{0.2cm}

                \textbf{beam size对结果的影响:}

                \begin{table}
                    \centering
                    \small
                    \begin{tabular}{c|cc}
                        \toprule
                        \textbf{beam size} & \textbf{准确率} & \textbf{速度} \\
                        \midrule
                        1 (贪婪) & 较低 & 最快 \\
                        5 & 中等 & 较快 \\
                        10 & 较高 & 中等 \\
                        20 & 最高 & 较慢 \\
                        \bottomrule
                    \end{tabular}
                \end{table}

                \vspace{0.2cm}

                \textbf{实验任务:}
                \begin{enumerate}
                    \item 在相同测试集上
                    \item 测试beam size = [1, 5, 10, 20]
                    \item 记录准确率和推理时间
                    \item 绘制trade-off曲线
                    \item 找出最佳平衡点
                \end{enumerate}
            \end{block}
        \end{column}
    \end{columns}
\end{frame}

\begin{frame}{案例分析:识别失败诊断}
    \begin{block}{案例1:粘连字符识别失败}
        \begin{columns}
            \begin{column}{0.4\textwidth}
                \fbox{\parbox{0.9\textwidth}{
                    \centering
                    \vspace{1cm}
                    [手写"国国"粘连]\\[0.5cm]
                    识别结果: "國"\\[0.3cm]
                    正确结果: "国国"\\[0.5cm]
                    \vspace{1cm}
                }}
            \end{column}
            \begin{column}{0.6\textwidth}
                \textbf{失败原因分析:}
                \begin{enumerate}
                    \item \textbf{过度粘连:}两个字笔画交叉过多
                    \item \textbf{预处理不足:}二值化阈值不合适
                    \item \textbf{分割失败:}字符分割算法无法区分
                    \item \textbf{模型局限:}训练集中粘连样本少
                \end{enumerate}

                \vspace{0.3cm}

                \textbf{改进方案:}
                \begin{itemize}
                    \item 优化预处理参数,增强对比度
                    \item 使用笔画宽度变换(SWT)辅助分割
                    \item 尝试过度分割+合并策略
                    \item 对粘连样本进行数据增强训练
                \end{itemize}
            \end{column}
        \end{columns}
    \end{block}
\end{frame}

\begin{frame}{案例分析:识别失败诊断(续)}
    \begin{columns}
        \begin{column}{0.5\textwidth}
            \begin{block}{案例2:潦草字迹识别失败}
                \fbox{\parbox{0.85\textwidth}{
                    \centering
                    \vspace{0.5cm}
                    [潦草手写"解决问题"]\\[0.3cm]
                    识别: "觖央问趣"\\[0.3cm]
                    正确: "解决问题"\\[0.5cm]
                }}

                \textbf{原因分析:}
                \begin{itemize}
                    \item 连笔过多,笔画断裂
                    \item 字形变形严重
                    \item 笔画顺序不规律
                \end{itemize}

                \textbf{改进方案:}
                \begin{itemize}
                    \item 使用更强的去噪和对比度增强
                    \item 尝试骨架提取辅助识别
                    \item 结合上下文语义校正
                \end{itemize}
            \end{block}
        \end{column}
        \begin{column}{0.5\textwidth}
            \begin{block}{案例3:背景干扰识别失败}
                \fbox{\parbox{0.85\textwidth}{
                    \centering
                    \vspace{0.5cm}
                    [带横线的答题纸]\\[0.3cm]
                    识别: "一答案一"\\[0.3cm]
                    正确: "答案"\\[0.5cm]
                }}

                \textbf{原因分析:}
                \begin{itemize}
                    \item 横线与文字粘连
                    \item 二值化后横线未去除
                    \item 预处理未针对横线优化
                \end{itemize}

                \textbf{改进方案:}
                \begin{itemize}
                    \item 使用形态学操作去除横线
                    \item 霍夫变换检测并擦除直线
                    \item 自适应阈值处理横线区域
                    \item 使用深度学习去噪模型
                \end{itemize}
            \end{block}
        \end{column}
    \end{columns}
\end{frame}

\begin{frame}{学生讨论与分享}
    \begin{block}{讨论题目}
        \textbf{1. 识别失败案例分享}
        \begin{itemize}
            \item 你在使用OCR时遇到过什么问题?
            \item 是如何解决的?
            \item 有什么经验可以分享?
        \end{itemize}

        \textbf{2. 改进方案设计}
        \begin{itemize}
            \item 针对刚才的案例,你有什么改进想法?
            \item 如何在准确率和速度之间取得平衡?
        \end{itemize}
    \end{block}

    \vspace{0.3cm}

    \begin{alertblock}{分享规则}
        \begin{enumerate}
            \item 每组派代表分享一个案例(2分钟)
            \item 其他组可以补充或提问
            \item 最后老师总结点评
        \end{enumerate}
    \end{alertblock}
\end{frame}

\begin{frame}{最佳实践投票与讨论}
    \begin{block}{投票1:不同场景下的工具选择}
        \textbf{场景1:}处理1000份手写简答题试卷,要求准确率高,服务器有GPU。

        \begin{columns}
            \begin{column}{0.6\textwidth}
                \begin{itemize}
                    \item[A] PaddleOCR + 完整预处理
                    \item[B] TrOCR + 简单预处理
                    \item[C] 商业API(百度/腾讯OCR)
                    \item[D] 自己训练专用模型
                \end{itemize}
            \end{column}
            \begin{column}{0.4\textwidth}
                \textbf{推荐:A或D}

                \vspace{0.3cm}

                \textit{大量数据时,专用模型效果最佳;快速部署选PaddleOCR。}
            \end{column}
        \end{columns}
    \end{block}

    \vspace{0.2cm}

    \begin{block}{投票2:实际部署经验}
        \textbf{问题:}在生产环境中,你最常遇到的OCR问题是?

        \begin{columns}
            \begin{column}{0.6\textwidth}
                \begin{itemize}
                    \item[A] 识别准确率不够高
                    \item[B] 推理速度太慢
                    \item[C] 预处理参数难调
                    \item[D] 模型部署复杂
                    \item[E] 数据安全和隐私
                \end{itemize}
            \end{column}
            \begin{column}{0.4\textwidth}
                \textbf{经验分享:}

                \vspace{0.3cm}

                \textit{欢迎有实际部署经验的同学分享你的解决方案和踩坑经历!}
            \end{column}
        \end{columns}
    \end{block}
\end{frame}

\begin{frame}{常见问题讨论}
    \begin{columns}
        \begin{column}{0.5\textwidth}
            \begin{block}{Q1: 如何提高潦草字迹的识别率?}
                \textbf{解决方案:}
                \begin{itemize}
                    \item 增强预处理(CLAHE去噪)
                    \item 使用骨架提取辅助
                    \item 结合上下文语义校正
                    \item 对潦草样本数据增强
                \end{itemize}
            \end{block}

            \vspace{0.2cm}

            \begin{block}{Q2: 识别速度太慢怎么办?}
                \textbf{优化方案:}
                \begin{itemize}
                    \item 降低输入图像分辨率
                    \item 使用更小的模型(Mobile版)
                    \item 批处理多张图片
                    \item GPU加速或使用TensorRT
                    \item 缓存常用结果
                \end{itemize}
            \end{block}
        \end{column}
        \begin{column}{0.5\textwidth}
            \begin{block}{Q3: 如何去除答题纸的横线干扰?}
                \textbf{解决方案:}
                \begin{itemize}
                    \item 霍夫变换检测直线并擦除
                    \item 形态学开运算去除横线
                    \item 使用in painting修复
                    \item 训练专门的去线模型
                \end{itemize}
            \end{block}

            \vspace{0.2cm}

            \begin{block}{Q4: 粘连字符如何分割?}
                \textbf{解决方案:}
                \begin{itemize}
                    \item 垂直投影+动态阈值
                    \item 最小生成树(MST)分割
                    \item 使用笔画宽度变换
                    \item 基于深度学习的分割
                    \item 过度分割+合并策略
                \end{itemize}
            \end{block}
        \end{column}
    \end{columns}
\end{frame}

\begin{frame}{课后作业}
    \begin{block}{题目}
        实现简答题手写识别模块
    \end{block}

    \textbf{要求:}
    \begin{enumerate}
        \item 使用PaddleOCR或TrOCR识别手写文字
        \item 实现图像预处理优化(至少包含灰度化、去噪、对比度增强、二值化)
        \item 实现文本行分割(水平投影法或连通域分析)
        \item 识别手写答案内容并分析识别准确率
        \item 撰写实验报告,包含参数调优过程和结果分析
    \end{enumerate}

    \vspace{0.2cm}

    \textbf{评分标准:}
    \begin{itemize}
        \item 模型配置与调用:20分
        \item 预处理实现:25分
        \item 文本行分割:20分
        \item 识别效果与分析:25分
        \item 实验报告质量:10分
    \end{itemize}
\end{frame}

\begin{frame}{下节预告}
    \begin{center}
        \Large \textbf{第9周:系统架构与分组开发}

        \vspace{0.5cm}

        \normalsize
        故事问题:\textcolor{blue}{把所有模块组合起来}

        \vspace{0.3cm}

        你将学会:
        \begin{itemize}
            \item 系统架构设计
            \item 模块集成方法
            \item 团队协作开发
        \end{itemize}

        \vspace{0.3cm}

        \textbf{课程设计开始!}
    \end{center}
\end{frame}

\begin{frame}
    \begin{center}
        \Huge \textbf{谢谢!}
    \end{center}
\end{frame}
