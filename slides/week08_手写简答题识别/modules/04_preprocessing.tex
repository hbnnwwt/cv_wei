%===========================================================
% 04_preprocessing.tex - 图像预处理优化
%===========================================================

\section{图像预处理优化}

\begin{frame}[fragile]{手写图像预处理}
    \begin{lstlisting}
def preprocess_handwriting(image):
    """手写图像预处理完整流程"""
    # TODO: 使用AI助手补全手写图像预处理代码
    # 提示词:"OpenCV手写图像预处理:灰度化→去噪→对比度增强→二值化"

    # ========== 第一步:转灰度 ==========
    # TODO: 使用cv2.cvtColor将图像转为灰度图
    # 提示词:"cv2.cvtColor BGR2GRAY转换"
    gray = cv2.cvtColor(image, cv2.COLOR_BGR2GRAY)

    # ========== 第二步:去噪 ==========
    # TODO: 使用fastNlMeansDenoising进行非局部均值去噪
    # 提示词:"cv2.fastNlMeansDenoising参数说明和推荐值"
    denoised = cv2.fastNlMeansDenoising(gray, None, 10, 7, 21)

    # ========== 第三步:对比度增强 ==========
    # TODO: 创建CLAHE对象并应用自适应直方图均衡化
    # 提示词:"cv2.createCLAHE clipLimit和tileGridSize参数调优"
    clahe = cv2.createCLAHE(clipLimit=2.0, tileGridSize=(8, 8))
    enhanced = clahe.apply(denoised)

    # ========== 第四步:二值化 ==========
    # TODO: 使用Otsu自适应阈值进行二值化
    # 提示词:"cv2.threshold OTSU二值化参数设置"
    _, binary = cv2.threshold(
        enhanced, 0, 255,
        cv2.THRESH_BINARY + cv2.THRESH_OTSU
    )

    return binary
    \end{lstlisting}

    \textbf{关键提示:} 手写图像需要更强的预处理,特别是去噪和对比度增强!
\end{frame}

\begin{frame}{文本行检测与分割}{水平投影法}
    \begin{columns}
        \begin{column}{0.5\textwidth}
            \textbf{水平投影法原理}
            \begin{itemize}
                \item 对二值图像按行求和
                \item 文字区域投影值高
                \item 空白行投影值接近0
                \item 通过阈值分割出行位置
            \end{itemize}

            \vspace{0.3cm}
            \textbf{适用场景:}
            \begin{itemize}
                \item 工整手写或印刷体
                \item 水平排列文本
                \item 行间距较明显
            \end{itemize}
        \end{column}
        \begin{column}{0.5\textwidth}
            \begin{lstlisting}[basicstyle=\tiny\ttfamily]
def horizontal_projection(binary_img):
    """水平投影法行分割"""
    # TODO: 使用AI助手补全水平投影代码
    # 提示词:"NumPy水平投影计算:对每行求和"

    # 对每一行求和
    h_proj = np.sum(binary_img, axis=1)

    # 可视化投影
    plt.figure(figsize=(10, 5))
    plt.subplot(121)
    plt.imshow(binary_img, cmap='gray')
    plt.title('Binary Image')

    plt.subplot(122)
    plt.plot(h_proj, range(len(h_proj)))
    plt.title('Horizontal Projection')
    plt.gca().invert_yaxis()

    return h_proj

def find_text_lines(h_proj, threshold=100):
    """从投影中找到文本行位置"""
    # TODO: 使用AI助手补全文本行检测逻辑
    # 提示词:"基于投影阈值分割文本行,处理行间隙"

    text_lines = []
    in_line = False
    start = 0

    for i, proj in enumerate(h_proj):
        if proj > threshold and not in_line:
            # 行开始
            start = i
            in_line = True
        elif proj <= threshold and in_line:
            # 行结束
            end = i
            text_lines.append((start, end))
            in_line = False

    # 处理最后一个字符
    if not in_line:
        text_lines.append((start, len(h_proj)))

    return text_lines
            \end{lstlisting}
        \end{column}
    \end{columns}
\end{frame}

\begin{frame}{文本行检测与分割}{连通域分析与深度学习方法}
    \begin{columns}
        \begin{column}{0.5\textwidth}
            \textbf{连通域分析法}

            \textbf{步骤:}
            \begin{enumerate}
                \item 对二值图像进行连通域标记
                \item 计算每个连通域的外接矩形
                \item 根据垂直位置进行聚类
                \item 合并同一行的连通域
            \end{enumerate}

            \vspace{0.3cm}
            \textbf{优点:}
            \begin{itemize}
                \item 能处理倾斜文本
                \item 对行间距不均匀适应性好
                \item 可识别弯曲行
            \end{itemize}
        \end{column}
        \begin{column}{0.5\textwidth}
            \textbf{基于深度学习的方法}

            \textbf{DBNet (Differentiable Binarization)}
            \begin{itemize}
                \item 端到端文本检测
                \item 可检测任意形状文本
                \item 速度快、精度高
            \end{itemize}

            \vspace{0.3cm}
            \textbf{EAST}
            \begin{itemize}
                \item 全卷积网络
                \item 直接预测文本框
                \item 支持多方向文本
            \end{itemize}

            \vspace{0.3cm}
            \textbf{适用场景:}
            \begin{itemize}
                \item 复杂背景文档
                \item 自然场景文本
                \item 弯曲/倾斜文本行
            \end{itemize}
        \end{column}
    \end{columns}
\end{frame}

\begin{frame}[fragile]{单字/单词分割}{垂直投影与改进方法}
    \begin{columns}
        \begin{column}{0.5\textwidth}
            \textbf{垂直投影分割原理}
            \begin{itemize}
                \item 对文本行图像按列求和
                \item 字符区域投影值高
                \item 字符间隙投影值接近0
                \item 通过阈值确定分割点
            \end{itemize}

            \vspace{0.3cm}
            \textbf{存在的问题:}
            \begin{itemize}
                \item 粘连字符难以分割
                \item 汉字偏旁可能被分开
                \item 间距不均影响效果
                \item 对手写体适应性差
            \end{itemize}
        \end{column}
        \begin{column}{0.5\textwidth}
            \begin{lstlisting}[basicstyle=\tiny\ttfamily]
def vertical_projection_split(line_img,
                                threshold=50):
    """
    垂直投影法分割字符
    """
    # TODO: 使用AI助手补全垂直投影分割代码
    # 提示词:"NumPy垂直投影:对每列求和找字符间隙"

    # 对每一列求和
    v_proj = np.sum(line_img, axis=0)

    # 找到投影小于阈值的列(间隙)
    gaps = v_proj < threshold

    # 找出间隙的起止位置
    char_regions = []
    in_gap = False
    start = 0

    for i, is_gap in enumerate(gaps):
        if is_gap and not in_gap:
            # 字符区域结束
            end = i
            char_regions.append((start, end))
            in_gap = True
        elif not is_gap and in_gap:
            # 字符区域开始
            start = i
            in_gap = False

    # 处理最后一个字符
    if not in_gap:
        char_regions.append((start, len(v_proj)))

    return char_regions, v_proj
            \end{lstlisting}
        \end{column}
    \end{columns}
\end{frame}

\begin{frame}{单字/单词分割}{高级分割方法}
    \begin{columns}
        \begin{column}{0.48\textwidth}
            \textbf{最小生成树(MST)分割}

            \textbf{算法步骤:}
            \begin{enumerate}
                \item 提取所有连通域
                \item 构建连通域之间的距离图
                \item 使用Kruskal算法构建MST
                \item 切割权重最大的边
                \item 得到分割结果
            \end{enumerate}

            \vspace{0.3cm}
            \textbf{优点:}
            \begin{itemize}
                \item 考虑全局信息
                \item 对粘连字符效果好
                \item 分割边界合理
            \end{itemize}
        \end{column}
        \begin{column}{0.48\textwidth}
            \textbf{基于聚类的分割}

            \textbf{K-means聚类:}
            \begin{itemize}
                \item 提取每个连通域特征
                \item 使用笔画宽度、宽高比等
                \item 聚类确定字符分组
            \end{itemize}

            \vspace{0.3cm}
            \textbf{过度分割与合并策略:}

            \textbf{过度分割:}
            \begin{itemize}
                \item 先进行细粒度分割
                \item 保证每个字符都被切开
                \item 可能产生过多片段
            \end{itemize}

            \textbf{合并策略:}
            \begin{itemize}
                \item 根据几何特征合并
                \item 笔画宽度一致性
                \item 相邻片段的空间关系
            \end{itemize}
        \end{column}
    \end{columns}
\end{frame}

\begin{frame}{文字规范化}{倾斜校正与尺寸归一化}
    \begin{columns}
        \begin{column}{0.5\textwidth}
            \textbf{倾斜校正算法}

            \textbf{1. 基于投影的校正:}
            \begin{itemize}
                \item 尝试不同角度旋转
                \item 计算水平投影的方差
                \item 选择方差最大的角度
            \end{itemize}

            \textbf{2. 基于Hough变换:}
            \begin{itemize}
                \item 检测文本行基线
                \item 计算倾斜角度
                \item 进行旋转校正
            \end{itemize}

            \textbf{3. 基于主成分分析(PCA):}
            \begin{itemize}
                \item 计算文字点云的主轴
                \item 确定倾斜角度
                \item 旋转校正
            \end{itemize}
        \end{column}
        \begin{column}{0.5\textwidth}
            \textbf{尺寸归一化}

            \textbf{为什么要归一化:}
            \begin{itemize}
                \item 统一输入尺寸
                \item 消除大小差异
                \item 提高模型泛化能力
            \end{itemize}

            \textbf{常用方法:}

            \textbf{1. 线性缩放:}
            \begin{itemize}
                \item 直接缩放到固定尺寸
                \item 保持宽高比或拉伸
            \end{itemize}

            \textbf{2. 保持比例填充:}
            \begin{itemize}
                \item 等比例缩放
                \item 不足部分填充背景色
                \item 保持字符原有比例
            \end{itemize}

            \textbf{3. 重心对齐:}
            \begin{itemize}
                \item 计算字符重心
                \item 归一化到固定位置
                \item 提高识别鲁棒性
            \end{itemize}
        \end{column}
    \end{columns}
\end{frame}

\begin{frame}[fragile]{文字规范化}{笔画宽度归一化与骨架提取}
    \begin{columns}
        \begin{column}{0.5\textwidth}
            \textbf{笔画宽度归一化}

            \textbf{目的:}
            \begin{itemize}
                \item 消除不同笔迹粗细差异
                \item 统一笔画宽度便于识别
                \item 提高模型泛化能力
            \end{itemize}

            \vspace{0.3cm}
            \textbf{实现方法:}

            \textbf{1. 笔画宽度变换(SWT):}
            \begin{itemize}
                \item 计算每个像素笔画宽度
                \item 根据宽度进行归一化
                \item 保持笔画连通性
            \end{itemize}

            \textbf{2. 形态学操作:}
            \begin{itemize}
                \item 膨胀/腐蚀调整笔画粗细
                \item 开运算去除细小噪声
                \item 闭运算连接断裂笔画
            \end{itemize}
        \end{column}
        \begin{column}{0.5\textwidth}
            \textbf{骨架提取}

            \textbf{什么是骨架:}
            \begin{itemize}
                \item 保持字符拓扑结构的细线
                \item 笔画中心线
                \item 宽度为1像素的线条
            \end{itemize}

            \vspace{0.3cm}
            \textbf{常用算法:}

            \textbf{1. 中轴变换(Medial Axis Transform)}

            \textbf{2. 细化算法(Thinning):}
            \begin{itemize}
                \item Zhang-Suen算法
                \item Guo-Hall算法
            \end{itemize}

            \vspace{0.3cm}
            \begin{lstlisting}[basicstyle=\tiny\ttfamily]
import cv2
import numpy as np

def skeletonize(image):
    """
    使用OpenCV提取骨架
    """
    # TODO: 使用AI助手补全骨架提取代码
    # 提示词:"OpenCV骨架提取:使用ximgproc.thinning"

    # 确保二值图像
    # TODO: 使用阈值处理确保图像为二值
    _, binary = cv2.threshold(image, 0, 255,
                              cv2.THRESH_BINARY)

    # 使用OpenCV的细化算法
    # TODO: 调用cv2.ximgproc.thinning提取骨架
    skeleton = cv2.ximgproc.thinning(binary)

    return skeleton
            \end{lstlisting}
        \end{column}
    \end{columns}
\end{frame}

\begin{frame}[fragile]{完整预处理流程可视化}
    \begin{lstlisting}
def visualize_preprocessing(image_path):
    """
    预处理全流程可视化(TODO脚手架版本)
    """
    # TODO: 使用AI助手补全可视化代码
    # 提示词:"matplotlib子图展示图像预处理各阶段结果"

    import matplotlib.pyplot as plt

    # TODO: 读取原图
    original = cv2.imread(_____)

    # 创建子图布局
    fig, axes = plt.subplots(2, 3, figsize=(15, 10))

    # ========== 1. 原图 ==========
    # TODO: 显示原图
    # 提示词:"OpenCV BGR转RGB用于matplotlib显示"
    axes[0, 0].imshow(cv2.cvtColor(original, cv2._______))
    axes[0, 0].set_title('1. Original Image')
    axes[0, 0].axis('off')

    # ========== 2. 灰度图 ==========
    # TODO: 转灰度图
    gray = cv2.cvtColor(original, cv2._______)
    axes[0, 1].imshow(gray, cmap='gray')
    axes[0, 1].set_title('2. Grayscale')
    axes[0, 1].axis('off')

    # ========== 3. 去噪 ==========
    # TODO: 使用fastNlMeansDenoising去噪
    # 提示词:"fastNlMeansDenoising参数:h=10, templateWindow=7, searchWindow=21"
    denoised = cv2.fastNlMeansDenoising(gray, None, __, __, __)
    axes[0, 2].imshow(denoised, cmap='gray')
    axes[0, 2].set_title('3. Denoised')
    axes[0, 2].axis('off')

    # ========== 4. 对比度增强 ==========
    # TODO: 创建CLAHE并应用
    # 提示词:"CLAHE clipLimit=2.0, tileGridSize=(8,8)"
    clahe = cv2.createCLAHE(clipLimit=____, tileGridSize=(__, __))
    enhanced = clahe.apply(_____)
    axes[1, 0].imshow(enhanced, cmap='gray')
    axes[1, 0].set_title('4. Contrast Enhanced')
    axes[1, 0].axis('off')

    # ========== 5. 二值化 ==========
    # TODO: 使用Otsu阈值进行二值化
    _, binary = cv2.threshold(enhanced, __, __,
                               cv2.THRESH_BINARY + cv2._____)
    axes[1, 1].imshow(binary, cmap='gray')
    axes[1, 1].set_title('5. Binary (Final)')
    axes[1, 1].axis('off')

    # ========== 6. 水平投影 ==========
    # TODO: 计算水平投影用于行分割
    h_proj = np.sum(binary, axis=__)
    axes[1, 2].plot(h_proj, range(len(h_proj)))
    axes[1, 2].set_title('6. Horizontal Projection')
    axes[1, 2].invert_yaxis()

    plt.tight_layout()
    plt.savefig('preprocessing_pipeline.png', dpi=150)
    plt.show()

    return binary
    \end{lstlisting}
\end{frame}

\begin{frame}{预处理效果对比}
    \begin{table}
        \centering
        \begin{tabular}{lccc}
            \toprule
            \textbf{预处理步骤} & \textbf{OCR准确率} & \textbf{速度} & \textbf{推荐} \\
            \midrule
            无预处理 & 65\% & 快 & \textcolor{red}{✗} \\
            仅灰度化 & 72\% & 快 & \textcolor{red}{✗} \\
            灰度+去噪 & 80\% & 中等 & \textcolor{orange}{△} \\
            完整预处理 & \textbf{90\%+} & 慢 & \textcolor{green}{✓} \\
            \bottomrule
        \end{tabular}
    \end{table}

    \vspace{0.3cm}

    \begin{columns}
        \begin{column}{0.5\textwidth}
            \textbf{建议:}
            \begin{itemize}
                \item 生产环境:使用完整预处理
                \item 快速测试:至少灰度+去噪
                \item 实时应用:平衡速度与精度
            \end{itemize}
        \end{column}
        \begin{column}{0.5\textwidth}
            \textbf{关键参数:}
            \begin{itemize}
                \item CLAHE clipLimit: 2.0-4.0
                \item 去噪强度: h=10-20
                \item 二值化阈值: OTSU自适应
            \end{itemize}
        \end{column}
    \end{columns}
\end{frame}
