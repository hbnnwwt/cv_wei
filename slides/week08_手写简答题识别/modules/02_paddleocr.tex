%===========================================================
% 02_paddleocr.tex - PaddleOCR手写识别
%===========================================================

\section{PaddleOCR手写识别}

\begin{frame}{PaddleOCR架构概述}
    \begin{columns}
        \begin{column}{0.55\textwidth}
            \textbf{PP-OCRv4三大核心模块}
            \vspace{0.3cm}

            \begin{enumerate}
                \item \textbf{文本检测} (Text Detection)
                \begin{itemize}
                    \item DBNet算法
                    \item 可微分二值化
                \end{itemize}

                \item \textbf{方向分类} (Orientation Classification)
                \begin{itemize}
                    \item 0/90/180/270度分类
                    \item 轻量级CNN
                \end{itemize}

                \item \textbf{文本识别} (Text Recognition)
                \begin{itemize}
                    \item SVTR\_LCNet架构
                    \item CTC解码
                \end{itemize}
            \end{enumerate}
        \end{column}
        \begin{column}{0.43\textwidth}
            \begin{tikzpicture}[
                node distance=0.6cm,
                box/.style={draw, rounded corners, fill=blue!10, minimum width=3.5cm, minimum height=0.7cm, align=center, font=\small}
            ]
                \node[box] (input) {输入图像};
                \node[box, below=of input, fill=red!10] (det) {文本检测\\DBNet};
                \node[box, below=of det, fill=yellow!10] (cls) {方向分类\\CNN};
                \node[box, below=of cls, fill=green!10] (rec) {文本识别\\SVTR};
                \node[box, below=of rec] (output) {识别结果};

                \draw[->, thick] (input) -- (det);
                \draw[->, thick] (det) -- (cls);
                \draw[->, thick] (cls) -- (rec);
                \draw[->, thick] (rec) -- (output);
            \end{tikzpicture}
        \end{column}
    \end{columns}
\end{frame}

\begin{frame}{检测模型:DBNet原理}
    \begin{columns}
        \begin{column}{0.5\textwidth}
            \textbf{DBNet (Differentiable Binarization)}

            \vspace{0.3cm}
            \textbf{核心创新:可微分二值化}
            \begin{itemize}
                \item 传统方法:二值化不可导,无法端到端训练
                \item DBNet:将二值化近似为可微分函数
                \item 梯度可直接传播到阈值预测
            \end{itemize}

            \vspace{0.3cm}
            \textbf{二值化公式:}
            \[
            B_{i,j} = \frac{1}{1 + e^{-k(P_{i,j} - T_{i,j})}}
            \]
            其中:$P$为概率图,$T$为阈值图,$k$为放大因子
        \end{column}
        \begin{column}{0.48\textwidth}
            \begin{tikzpicture}[
                node distance=0.4cm,
                smallbox/.style={draw, rounded corners, fill=blue!10, minimum width=4cm, minimum height=0.6cm, align=center, font=\small}
            ]
                \node[smallbox] (input) {输入图像};
                \node[smallbox, below=of input] (backbone) {Backbone (ResNet)};
                \node[smallbox, below=of backbone] (neck) {FPN (特征融合)};
                \node[smallbox, below=of neck, fill=green!10] (prob) {概率图 $P$};
                \node[smallbox, below=of prob, fill=yellow!10] (thresh) {阈值图 $T$};
                \node[smallbox, below=of thresh, fill=red!10] (binary) {近似二值图 $\hat{B}$};

                \draw[->, thick] (input) -- (backbone);
                \draw[->, thick] (backbone) -- (neck);
                \draw[->, thick] (neck) -- (prob);
                \draw[->, thick] (neck) -- (thresh);
                \draw[->, thick] (prob) -- (binary);
                \draw[->, thick] (thresh) -- (binary);
            \end{tikzpicture}
        \end{column}
    \end{columns}
\end{frame}

\begin{frame}{识别模型:SVTR\_LCNet原理}
    \begin{columns}
        \begin{column}{0.5\textwidth}
            \textbf{SVTR\_LCNet架构}

            \vspace{0.3cm}
            \textbf{设计目标:}
            \begin{itemize}
                \item 中文场景文字识别
                \item 支持横排和竖排文本
                \item 平衡精度与速度
            \end{itemize}

            \vspace{0.3cm}
            \textbf{网络结构:}
            \begin{enumerate}
                \item \textbf{LCNet骨干网络}
                \begin{itemize}
                    \item 轻量级CNN架构
                    \item 深度可分离卷积
                \end{itemize}
                \item \textbf{SVTR编码器}
                \begin{itemize}
                    \item 基于Sub-Block的Transformer
                    \item 全局特征建模
                \end{itemize}
                \item \textbf{CTC解码器}
                \begin{itemize}
                    \item 无需字符级对齐
                    \item 处理不定长序列
                \end{itemize}
            \end{enumerate}
        \end{column}
        \begin{column}{0.48\textwidth}
            \begin{tikzpicture}[
                node distance=0.35cm,
                box/.style={draw, rounded corners, minimum width=4cm, minimum height=0.55cm, align=center, font=\small}
            ]
                \node[box, fill=blue!10] (input) {输入图像 (H×W×C)};
                \node[box, below=of input, fill=cyan!10] (lcnet) {LCNet骨干网络};
                \node[box, below=of lcnet, fill=green!10] (feature) {特征图 (H'×W'×C')};
                \node[box, below=of feature, fill=yellow!10] (svtr) {SVTR编码器};
                \node[box, below=of svtr, fill=orange!10] (seq) {序列特征 (T×D)};
                \node[box, below=of seq, fill=red!10] (ctc) {CTC解码器};
                \node[box, below=of ctc, fill=purple!10] (output) {识别结果};

                \draw[->, thick] (input) -- (lcnet);
                \draw[->, thick] (lcnet) -- (feature);
                \draw[->, thick] (feature) -- (svtr);
                \draw[->, thick] (svtr) -- (seq);
                \draw[->, thick] (seq) -- (ctc);
                \draw[->, thick] (ctc) -- (output);
            \end{tikzpicture}

            \vspace{0.5cm}
            \textbf{CTC解码公式:}
            \[
            p(l|x) = \sum_{\pi \in B^{-1}(l)} p(\pi|x)
            \]
        \end{column}
    \end{columns}
\end{frame}

\begin{frame}[fragile]{PaddleOCR手写配置}
    \begin{lstlisting}
from paddleocr import PaddleOCR

# TODO: 使用AI助手补全PaddleOCR初始化
# 提示词:"PaddleOCR手写识别配置参数说明"
ocr_handwrite = PaddleOCR(
    use_angle_cls=______,  # TODO: 是否启用方向分类(True/False)
    lang=______,           # TODO: 语言设置('ch'/'en'/...)
    show_log=______        # TODO: 是否显示日志(True/False)
)

# TODO: 调用OCR识别方法
# 提示词:"PaddleOCR ocr()方法参数说明"
result = ocr_handwrite.ocr(
    ______,  # TODO: 输入图像路径或数组
    cls=______  # TODO: 是否启用方向分类(True/False)
)

# TODO: 解析并打印识别结果
# 提示词:"PaddleOCR识别结果结构解析"
for line in result:
    text = ______  # TODO: 提取文字内容
    confidence = ______  # TODO: 提取置信度
    print(f"{text} ({confidence:.4f})")
    \end{lstlisting}

    \textbf{注意:} PaddleOCR默认已支持手写识别
\end{frame}

\begin{frame}[fragile]{手写识别参数调优}
    \begin{lstlisting}
# TODO: 使用AI助手补全手写识别优化配置
# 提示词:"PaddleOCR手写识别参数调优指南"
ocr_optimized = PaddleOCR(
    use_angle_cls=______,  # TODO: 是否启用方向分类
    lang=______,           # TODO: 语言设置
    # TODO: 模型路径配置
    det_model_dir=______,  # TODO: 检测模型路径
    rec_model_dir=______,  # TODO: 识别模型路径
    cls_model_dir=______,  # TODO: 方向分类模型路径
    # TODO: 性能参数调优
    rec_batch_num=______,   # TODO: 批量识别数(整数)
    max_text_length=______, # TODO: 最大文本长度(整数)
    drop_score=______,      # TODO: 置信度阈值(0-1浮点数)
    show_log=______         # TODO: 是否显示日志
)
    \end{lstlisting}
\end{frame}

\begin{frame}[fragile]{多行手写文字识别}
    \begin{lstlisting}
import cv2
from paddleocr import PaddleOCR

# TODO: 使用AI助手补全图像读取代码
# 提示词:"OpenCV读取图像的常用方法"
image = cv2.______  # TODO: 读取图像文件(如'answer_sheet.jpg')

# TODO: 初始化OCR
# 提示词:"PaddleOCR初始化参数说明"
ocr = PaddleOCR(
    use_angle_cls=______,  # TODO: 是否启用方向分类
    lang=______           # TODO: 语言设置
)

# TODO: 执行OCR识别
result = ocr.ocr(
    ______,  # TODO: 输入图像
    cls=______  # TODO: 是否启用方向分类
)

# TODO: 提取文字块并按位置排序
# 提示词:"如何根据边界框坐标对OCR结果排序"
text_blocks = []
for line in result[0]:
    box = ______  # TODO: 提取边界框坐标
    text = ______  # TODO: 提取文字内容
    conf = ______  # TODO: 提取置信度
    # TODO: 计算中心点y坐标
    center_y = ______  # TODO: 计算边界框中心y坐标
    text_blocks.append((center_y, text, conf))

# TODO: 按y坐标排序(从上到下)
text_blocks.sort(key=______)
    \end{lstlisting}
\end{frame}

\begin{frame}[fragile]{增强版手写识别(含错误处理)}
    \begin{lstlisting}
import os
import cv2
import numpy as np
from paddleocr import PaddleOCR
from typing import List, Dict, Optional, Union

class HandwritingOCR:
    """增强版手写OCR识别器"""

    def __init__(self,
                 use_gpu: bool = False,
                 det_model_dir: Optional[str] = None,
                 rec_model_dir: Optional[str] = None,
                 drop_score: float = 0.3):
        """
        初始化OCR识别器

        Args:
            use_gpu: 是否使用GPU
            det_model_dir: 检测模型路径
            rec_model_dir: 识别模型路径
            drop_score: 置信度过滤阈值
        """
        try:
            self.ocr = PaddleOCR(
                use_angle_cls=True,
                lang='ch',
                use_gpu=use_gpu,
                det_model_dir=det_model_dir,
                rec_model_dir=rec_model_dir,
                drop_score=drop_score,
                show_log=False
            )
            self.drop_score = drop_score
            print("[INFO] OCR模型初始化成功")
        except Exception as e:
            raise RuntimeError(f"OCR初始化失败: {str(e)}")
    \end{lstlisting}
\end{frame}

\begin{frame}[fragile]{批量处理与结果可视化}
    \begin{lstlisting}
    def batch_recognize(self,
                       image_paths: List[str],
                       save_results: bool = True,
                       output_dir: str = "./results") -> List[Dict]:
        """
        批量识别手写文字

        Args:
            image_paths: 图像路径列表
            save_results: 是否保存可视化结果
            output_dir: 输出目录

        Returns:
            识别结果列表,每项包含:
            - image_path: 图像路径
            - texts: 识别文字列表
            - confidences: 置信度列表
            - avg_confidence: 平均置信度
        """
        if save_results and not os.path.exists(output_dir):
            os.makedirs(output_dir)

        results = []
        for img_path in image_paths:
            if not os.path.exists(img_path):
                print(f"[WARN] 文件不存在: {img_path}")
                continue

            try:
                result = self.recognize_single(img_path,
                                               save_vis=save_results,
                                               output_dir=output_dir)
                results.append(result)
            except Exception as e:
                print(f"[ERROR] 处理 {img_path} 失败: {str(e)}")

        return results
    \end{lstlisting}
\end{frame}

\begin{frame}[fragile]{单张图像识别与可视化}
    \begin{lstlisting}
    def recognize_single(self,
                        image_path: str,
                        save_vis: bool = True,
                        output_dir: str = "./results") -> Dict:
        """
        单张图像手写文字识别

        Args:
            image_path: 图像路径
            save_vis: 是否保存可视化结果
            output_dir: 输出目录

        Returns:
            识别结果字典
        """
        # 读取图像
        image = cv2.imread(image_path)
        if image is None:
            raise ValueError(f"无法读取图像: {image_path}")

        # 执行OCR识别
        ocr_result = self.ocr.ocr(image_path, cls=True)

        # 解析结果
        texts = []
        confidences = []
        boxes = []

        if ocr_result and ocr_result[0]:
            for line in ocr_result[0]:
                box = line[0]  # 边界框坐标
                text = line[1][0]  # 识别文本
                conf = line[1][1]  # 置信度

                # 过滤低置信度结果
                if conf >= self.drop_score:
                    boxes.append(box)
                    texts.append(text)
                    confidences.append(conf)

        # 按位置排序(从上到下)
        if boxes:
            sorted_data = sorted(zip(boxes, texts, confidences),
                                key=lambda x: (x[0][0][1] + x[0][2][1]) / 2)
            boxes, texts, confidences = zip(*sorted_data) if sorted_data else ([], [], [])

        # 可视化结果
        vis_image = None
        if save_vis:
            vis_image = self._visualize_result(image, boxes, texts, confidences)
            output_path = os.path.join(output_dir, f"vis_{os.path.basename(image_path)}")
            cv2.imwrite(output_path, vis_image)
            print(f"[INFO] 可视化结果已保存: {output_path}")

        # 计算平均置信度
        avg_conf = sum(confidences) / len(confidences) if confidences else 0.0

        return {
            'image_path': image_path,
            'texts': list(texts),
            'confidences': list(confidences),
            'avg_confidence': avg_conf,
            'boxes': list(boxes),
            'visualization': vis_image
        }
    \end{lstlisting}
\end{frame}

\begin{frame}[fragile]{结果可视化方法}
    \begin{lstlisting}
    def _visualize_result(self,
                         image: np.ndarray,
                         boxes: List[List],
                         texts: List[str],
                         confidences: List[float],
                         box_color: Tuple[int, int, int] = (0, 255, 0),
                         text_color: Tuple[int, int, int] = (255, 0, 0)) -> np.ndarray:
        """
        可视化识别结果

        Args:
            image: 原始图像
            boxes: 边界框列表
            texts: 识别文本列表
            confidences: 置信度列表
            box_color: 框颜色 (BGR)
            text_color: 文字颜色 (BGR)

        Returns:
            可视化后的图像
        """
        vis_image = image.copy()

        for i, (box, text, conf) in enumerate(zip(boxes, texts, confidences)):
            # 绘制边界框
            box_pts = np.array(box, dtype=np.int32).reshape((-1, 1, 2))
            cv2.polylines(vis_image, [box_pts], True, box_color, 2)

            # 准备显示文本
            display_text = f"{text} ({conf:.2f})"

            # 计算文本位置(框的左上角)
            text_pos = (int(box[0][0]), int(box[0][1]) - 5)

            # 获取文本大小以绘制背景
            (text_w, text_h), _ = cv2.getTextSize(
                display_text, cv2.FONT_HERSHEY_SIMPLEX, 0.5, 2
            )

            # 绘制文本背景
            cv2.rectangle(
                vis_image,
                (text_pos[0], text_pos[1] - text_h - 4),
                (text_pos[0] + text_w, text_pos[1] + 4),
                (0, 0, 0),
                -1
            )

            # 绘制文本
            cv2.putText(
                vis_image, display_text, text_pos,
                cv2.FONT_HERSHEY_SIMPLEX, 0.5, text_color, 2
            )

        return vis_image


# 使用示例
if __name__ == "__main__":
    # 初始化OCR
    ocr = HandwritingOCR(use_gpu=False, drop_score=0.3)

    # 单张识别
    result = ocr.recognize_single("handwriting.jpg", save_vis=True)
    print(f"识别结果: {result['texts']}")
    print(f"平均置信度: {result['avg_confidence']:.4f}")
    \end{lstlisting}
\end{frame}

% 注意:重复的PaddleOCR配置页面已删除,相关内容见上方"PaddleOCR手写配置"和"手写识别参数调优"页面
