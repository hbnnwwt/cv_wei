%===========================================================
% 01_overview.tex - 手写识别概述
%===========================================================

\section{手写识别概述}

\begin{frame}{手写识别的挑战}
    \begin{columns}
        \column{0.5\textwidth}
        \textbf{印刷文字:}
        \begin{itemize}
            \item 字体统一
            \item 字形稳定
            \item 间距规范
        \end{itemize}

        \column{0.5\textwidth}
        \textbf{手写文字:}
        \begin{itemize}
            \item 每人不同
            \item 形状变化大
            \item 间距随意
        \end{itemize}
    \end{columns}

    \vspace{0.5cm}

    \textbf{主要难点:}
    \begin{enumerate}
        \item 字形差异大
        \item 连笔问题
        \item 倾斜旋转
        \item 图像质量
    \end{enumerate}
\end{frame}

\begin{frame}{脱机识别 vs 联机识别}
    \begin{table}
        \centering
        \small
        \begin{tabular}{p{3cm}p{5cm}p{5cm}}
            \toprule
            \textbf{特性} & \textbf{脱机识别 (Offline HWR)} & \textbf{联机识别 (Online HWR)} \\
            \midrule
            输入方式 & 静态图像(扫描/拍照) & 动态笔画序列(触摸屏/数位板) \\
            可用信息 & 仅图像像素 & 笔画轨迹、速度、压力、时间 \\
            难度 & 更难(信息损失) & 相对容易 \\
            应用场景 & 文档数字化、阅卷系统 & 手写输入、签名验证 \\
            典型方法 & CNN+RNN+CTC & LSTM、Transformer \\
            \bottomrule
        \end{tabular}
    \end{table}

    \begin{block}{课程重点}
        本课程重点讲解\textbf{脱机手写识别},用于智能阅卷系统中的简答题识别。
    \end{block}
\end{frame}

\begin{frame}{手写识别在智能阅卷中的重要性}
    \begin{columns}
        \column{0.5\textwidth}
        \textbf{传统阅卷痛点:}
        \begin{itemize}
            \item 人工阅卷效率低
            \item 主观评分不一致
            \item 教师工作负担重
            \item 成绩统计耗时
        \end{itemize}

        \vspace{0.5cm}

        \textbf{AI阅卷优势:}
        \begin{itemize}
            \item 秒级识别手写答案
            \item 客观题自动评分
            \item 简答题辅助批改
            \item 自动成绩统计
        \end{itemize}

        \column{0.5\textwidth}
        \begin{block}{手写识别的关键作用}
            \textbf{1. 答案区域定位}
            \begin{itemize}
                \item 检测简答题答题框
                \item 分割手写文字区域
            \end{itemize}

            \vspace{0.3cm}

            \textbf{2. 手写文字识别}
            \begin{itemize}
                \item 识别学生手写答案
                \item 转换为可编辑文本
            \end{itemize}

            \vspace{0.3cm}

            \textbf{3. 语义理解辅助评分}
            \begin{itemize}
                \item 提取关键词
                \item 匹配标准答案
            \end{itemize}
        \end{block}
    \end{columns}
\end{frame}

\begin{frame}{手写文字与印刷文字的区别}
    \begin{columns}
        \column{0.5\textwidth}
        \textbf{书写风格差异:}
        \begin{itemize}
            \item \textbf{印刷文字:}统一字体,字形规范
            \item \textbf{手写文字:}因人而异,风格多样
            \item 连笔、省略笔画现象普遍
        \end{itemize}

        \vspace{0.5cm}

        \textbf{字形变化与连笔问题:}
        \begin{itemize}
            \item 同一字的不同写法
            \item 笔画顺序变化
            \item 连笔导致字形模糊
            \item 草书难以辨认
        \end{itemize}

        \column{0.5\textwidth}
        \textbf{笔画粗细与倾斜:}
        \begin{itemize}
            \item \textbf{印刷:}笔画宽度一致
            \item \textbf{手写:}笔尖压力变化导致粗细不均
            \item 整体或局部倾斜
            \item 基线不水平
        \end{itemize}

        \vspace{0.5cm}

        \textbf{字间距与行间距变化:}
        \begin{itemize}
            \item 印刷文字间距固定
            \item 手写间距随意
            \item 字重叠或分离
            \item 行间距不均匀
        \end{itemize}
    \end{columns}
\end{frame}

\begin{frame}{手写简答题识别的难点}
    \begin{columns}
        \column{0.5\textwidth}
        \textbf{1. 自由书写,无固定格式}
        \begin{itemize}
            \item 答案长度不固定
            \item 书写位置不固定
            \item 答案组织方式多样
            \item 编号、序号使用不一致
        \end{itemize}

        \vspace{0.5cm}

        \textbf{2. 混合中英文与公式}
        \begin{itemize}
            \item 中英混杂
            \item 数学公式、符号
            \item 特殊字符
            \item 不同语言的识别要求
        \end{itemize}

        \column{0.5\textwidth}
        \textbf{3. 涂改与补充痕迹}
        \begin{itemize}
            \item 划掉的文字
            \item 修改痕迹
            \item 补充内容
            \item 箭头、连线等标记
        \end{itemize}

        \vspace{0.5cm}

        \textbf{4. 书写质量差异大}
        \begin{itemize}
            \item 字迹清晰度不同
            \item 笔画完整性差异
            \item 字体大小不统一
            \item 书写压力不同
        \end{itemize}

        \vspace{0.3cm}

        \textbf{5. 背景干扰(格线、印刷文字)}
        \begin{itemize}
            \item 答题框线
            \item 试卷印刷文字
            \item 水印、底纹
        \end{itemize}
    \end{columns}
\end{frame}

\begin{frame}{手写识别技术发展}
    \begin{table}
        \centering
        \small
        \begin{tabular}{p{2.5cm}p{5cm}p{3cm}p{2cm}}
            \toprule
            \textbf{阶段} & \textbf{技术} & \textbf{代表模型/方法} & \textbf{准确率} \\
            \midrule
            传统方法 (1960s-2000s) & 特征工程 + 统计分类器 & HMM、SVM、KNN & 60-70\% \\
            CNN时代 (2012-2015) & 卷积神经网络 & LeNet, AlexNet, VGG & 80-85\% \\
            RNN+Attention (2015-2018) & 循环神经网络+注意力 & LSTM+Attention & 90-92\% \\
            Transformer (2018-至今) & 自注意力机制 & \textbf{TrOCR, Donut, PARSeq} & \textbf{95\%+} \\
            \bottomrule
        \end{tabular}
    \end{table}

    \vspace{0.5cm}

    \begin{columns}
        \column{0.5\textwidth}
        \textbf{关键里程碑:}
        \begin{itemize}
            \item 1998: LeNet-5 手写数字识别
            \item 2012: AlexNet 开启深度学习时代
            \item 2015: CTC + LSTM 序列识别
            \item 2017: Transformer 注意力机制
            \item 2021: TrOCR 端到端OCR
        \end{itemize}

        \column{0.5\textwidth}
        \begin{block}{当前最佳:TrOCR(Microsoft)}
            \begin{itemize}
                \item 架构:ViT (图像编码器) + GPT-2 (文本解码器)
                \item 端到端训练,无需字符级标注
                \item 支持手写和印刷体混合识别
                \item 在IAM手写数据集上达到SOTA
            \end{itemize}
        \end{block}
    \end{columns}
\end{frame}

\begin{frame}{主流手写识别工具对比}
    \begin{table}
        \centering
        \small
        \begin{tabular}{p{2cm}p{3.5cm}p{3cm}p{3cm}p{2cm}}
            \toprule
            \textbf{工具} & \textbf{核心技术} & \textbf{优点} & \textbf{缺点} & \textbf{适用场景} \\
            \midrule
            TrOCR & ViT + GPT-2 Transformer & 准确率最高(95\%+),支持手写印刷混合 & 速度慢,GPU需求高 & 高精度OCR \\
            \midrule
            PaddleOCR & DB + CRNN + CTC & 速度快,中文支持好,轻量级 & 手写准确率略低(85-90\%) & 实时OCR、移动端 \\
            \midrule
            Tesseract & LSTM + 传统CV & 免费开源,多语言支持 & 中文手写支持差 & 印刷体英文 \\
            \midrule
            EasyOCR & CRAFT + Seq2Seq & 支持80+语言,易用 & 中文手写效果一般 & 多语言通用 \\
            \bottomrule
        \end{tabular}
    \end{table}

    \vspace{0.5cm}

    \begin{columns}
        \column{0.5\textwidth}
        \textbf{选择建议:}
        \begin{itemize}
            \item \textbf{高精度需求} $\to$ TrOCR
            \item \textbf{实时应用} $\to$ PaddleOCR
            \item \textbf{移动端部署} $\to$ PaddleOCR轻量版
            \item \textbf{多语言支持} $\to$ EasyOCR
        \end{itemize}

        \column{0.5\textwidth}
        \begin{block}{本课程推荐}
            \begin{itemize}
                \item \textbf{学习阶段:} PaddleOCR(易上手)
                \item \textbf{生产环境:} TrOCR(高精度)
                \item \textbf{对比实验:} 两者都实现
            \end{itemize}
        \end{block}
    \end{columns}
\end{frame}
