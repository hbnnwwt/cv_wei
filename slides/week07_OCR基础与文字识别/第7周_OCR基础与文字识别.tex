\documentclass[aspectratio=169, 12pt]{beamer}
\usepackage[UTF8]{ctex}
\usepackage{graphicx}
\usepackage{booktabs}
\usepackage{listings}
\usepackage{xcolor}
\usepackage{tikz}
\usepackage{hyperref}

\usetheme{Madrid}
\usecolortheme{whale}
\usefonttheme{professionalfonts}

\lstset{
    language=Python,
    basicstyle=\ttfamily\small,
    keywordstyle=\color{blue},
    commentstyle=\color{green!60!black},
    stringstyle=\color{orange},
    breaklines=true,
    frame=single,
    showstringspaces=false,
    backgroundcolor=\color{gray!10}
}

\title[OCR基础与文字识别]{第7周:OCR基础与文字识别}
\subtitle{怎么让机器"阅读"文字?}
\author{北京石油化工学院\\人工智能研究院\\王文通}
\institute{通选课}
\date{2025-2026 学年}
\titlegraphic{
    \includegraphics[height=1.2cm]{../xiaohui.png}\hspace{2cm}
    \includegraphics[height=1.2cm]{../name.png}
}

\begin{document}

\begin{frame}
    \titlepage
\end{frame}

\begin{frame}{课程概览}
    \tableofcontents
\end{frame}

\section{OCR技术概述}

\begin{frame}{什么是OCR?}
    \begin{block}{OCR:Optical Character Recognition}
        光学字符识别 —— 将图像中的文字转换为计算机可编辑的文本
    \end{block}

    \vspace{0.3cm}

    \textbf{应用场景:}
    \begin{itemize}
        \item 文档数字化
        \item 车牌识别
        \item 证件识别
        \item \textbf{试卷文字识别}
    \end{itemize}
\end{frame}

\begin{frame}{OCR技术发展}
    \begin{table}
        \centering
        \begin{tabular}{lp{6cm}l}
            \toprule
            \textbf{阶段} & \textbf{技术} & \textbf{准确率} \\
            \midrule
            传统方法 & 模板匹配、特征提取 & 60-70\% \\
            机器学习 & SVM、HMM & 80-85\% \\
            深度学习 & CNN、RNN & 90-92\% \\
            Transformer & TrOCR、Donut & 95\%+ \\
            \bottomrule
        \end{tabular}
    \end{table}

    \textbf{当前:} 端到端大模型时代
\end{frame}

\begin{frame}{主流OCR工具对比}
    \begin{table}
        \centering
        \small
        \begin{tabular}{lp{4cm}lp{3cm}}
            \toprule
            \textbf{工具} & \textbf{优点} & \textbf{缺点} & \textbf{适用} \\
            \midrule
            Tesseract & 开源免费 & 中文准确率一般 & 英文文档 \\
            PaddleOCR & 中文准确率高 & 需要下载模型 & \textbf{中文场景} \\
            EasyOCR & 多语言支持 & 速度较慢 & 多语言混合 \\
            TrOCR & 准确率最高 & 资源占用大 & 高精度场景 \\
            \bottomrule
        \end{tabular}
    \end{table}

    \textbf{本课程推荐:} PaddleOCR(中文友好、易用)
\end{frame}

\section{PaddleOCR快速上手}

\begin{frame}[fragile]{PaddleOCR安装}
    \begin{lstlisting}
# 安装PaddlePaddle
pip install paddlepaddle

# 安装PaddleOCR
pip install paddleocr

# 安装其他依赖
pip install opencv-python pillow
    \end{lstlisting}

    \begin{alertblock}{注意}
        \begin{itemize}
            \item 首次使用会自动下载模型
            \item 需要联网
            \item 首次运行较慢
        \end{itemize}
    \end{alertblock}
\end{frame}

\begin{frame}[fragile]{基础使用}
    \begin{lstlisting}
from paddleocr import PaddleOCR

# 初始化OCR
ocr = PaddleOCR(use_angle_cls=True, lang='ch')

# 识别图像
result = ocr.ocr('exam_paper.jpg', cls=True)

# 打印结果
for line in result:
    text = line[1][0]      # 文字内容
    confidence = line[1][1]  # 置信度
    print(f"{text} ({confidence:.4f})")
    \end{lstlisting}
\end{frame}

\begin{frame}{结果格式}
    \textbf{返回格式:}
    \begin{verbatim}
[
  [
    [[x1,y1], [x2,y2], [x3,y3], [x4,y4]],  # 文字框坐标
    ('文字内容', 0.98)                        # 文字+置信度
  ],
  ...
]
    \end{verbatim}

    \vspace{0.2cm}

    \textbf{字段说明:}
    \begin{itemize}
        \item 坐标:文字框的四个角点
        \item 文字内容:识别出的文本
        \item 置信度:0-1之间,越接近1越准确
    \end{itemize}
\end{frame}

\section{试卷文字识别实战}

\begin{frame}[fragile]{识别试卷标题}
    \begin{lstlisting}[basicstyle=\ttfamily\scriptsize]
def extract_title(ocr_result):
    """提取试卷标题"""
    boxes_with_height = []

    for line in ocr_result:
        text = line[1][0]
        boxes = line[0]

        # 计算字号(高度)
        h = np.linalg.norm(np.array(boxes[0]) - np.array(boxes[3]))
        boxes_with_height.append((text, h, boxes))

    # 按高度排序,取最大的作为标题
    boxes_with_height.sort(key=lambda x: x[1], reverse=True)
    title = ' '.join([t[0] for t in boxes_with_height[:3]])

    return title
    \end{lstlisting}
\end{frame}

\begin{frame}[fragile]{识别题号}
    \begin{lstlisting}[basicstyle=\ttfamily\scriptsize]
import re

def extract_question_numbers(ocr_result):
    """提取题号"""
    questions = []

    for line in ocr_result:
        text = line[1][0]

        # 匹配题号模式
        patterns = [
            r'^(\d+)[.、.]',      # 1. 1、1.
            r'^[((](\d+)[))]',   # (1)
            r'^第?(\d+)题'        # 第1题
        ]

        for pattern in patterns:
            match = re.match(pattern, text.strip())
            if match:
                num = int(match.group(1))
                questions.append(num)
                break

    return questions
    \end{lstlisting}
\end{frame}

\section{OCR优化}

\begin{frame}{图像预处理优化}
    \textbf{专门的OCR预处理:}
    \begin{lstlisting}
def preprocess_for_ocr(image):
    """OCR优化预处理"""
    # 转灰度
    gray = cv2.cvtColor(image, cv2.COLOR_BGR2GRAY)

    # 去噪
    denoised = cv2.fastNlMeansDenoising(gray)

    # 二值化
    _, binary = cv2.threshold(
        denoised, 0, 255,
        cv2.THRESH_BINARY + cv2.THRESH_OTSU
    )

    return binary
    \end{lstlisting}

    \textbf{关键:} 图像质量直接影响OCR效果!
\end{frame}

\section{思考题}

\begin{frame}{课堂思考题}
    \begin{block}{问题1:OCR技术}
        \begin{itemize}
            \item 为什么深度学习让OCR准确率大幅提升?
            \item Tesseract和PaddleOCR各有什么优势?
        \end{itemize}
    \end{block}

    \vspace{0.3cm}

    \begin{block}{问题2:文字识别}
        \begin{itemize}
            \item OCR识别结果中"confidence"是什么意思?
            \item 如何处理OCR识别错误的文字?
        \end{itemize}
    \end{block}
\end{frame}

\section{课后作业}

\begin{frame}{课后作业}
    \begin{block}{题目}
        用OCR识别试卷中的印刷文字
    \end{block}

    \textbf{要求:}
    \begin{enumerate}
        \item 安装配置PaddleOCR
        \item 识别试卷标题和题号
        \item 提取所有文字内容
        \item 可视化标注识别结果
    \end{enumerate}

    \vspace{0.2cm}

    \textbf{评分标准:}
    \begin{itemize}
        \item 环境配置:20分
        \item 识别效果:40分
        \item 信息提取:25分
        \item 可视化:15分
    \end{itemize}
\end{frame}

\begin{frame}{下节预告}
    \begin{center}
        \Large \textbf{第8周:手写简答题识别}

        \vspace{0.5cm}

        \normalsize
        故事问题:\textcolor{blue}{能看懂学生写的答案吗?}

        \vspace{0.3cm}

        你将学会:
        \begin{itemize}
            \item 手写识别的挑战
            \item TrOCR/PaddleOCR手写模型
            \item 手写文字识别
        \end{itemize}
    \end{center}
\end{frame}

\begin{frame}
    \begin{center}
        \Huge \textbf{谢谢!}
    \end{center}
\end{frame}

\end{document}
