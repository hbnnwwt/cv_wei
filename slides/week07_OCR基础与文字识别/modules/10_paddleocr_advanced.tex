%===========================================================
% modules/10_paddleocr_advanced.tex - PaddleOCR高级实战
%===========================================================

\section{PaddleOCR高级实战}

\begin{frame}{PaddleOCR架构解析}
    \begin{columns}
        \column{0.5\textwidth}
        \textbf{PP-OCR三大模块:}
        \begin{enumerate}
            \item \textbf{文本检测(DBNet)}
            \begin{itemize}
                \item 可微二值化网络
                \item 轻量级骨干网络
                \item 实时文字定位
            \end{itemize}
            \item \textbf{方向分类(MobileNet)}
            \begin{itemize}
                \item 0/90/180/270度分类
                \item 轻量级分类器
                \item 提高识别准确率
            \end{itemize}
            \item \textbf{文本识别(SVTR)}
            \begin{itemize}
                \item 视觉Transformer
                \item 视觉-语言预训练
                \item 中英文高精度识别
            \end{itemize}
        \end{enumerate}

        \column{0.5\textwidth}
        \textbf{数据处理流程:}
        \begin{center}
            \begin{tikzpicture}[
                node distance=0.4cm,
                box/.style={rectangle, draw=blue!60, fill=blue!10, rounded corners, minimum width=3cm, minimum height=0.5cm, font=\scriptsize},
                arrow/.style={->, thick}
            ]
                \node[box] (img) {输入图像};
                \node[box, below=of img] (det) {文本检测 (DBNet)};
                \node[box, below=of det] (crop) {文字区域裁剪};
                \node[box, below=of crop] (cls) {方向分类 (MobileNet)};
                \node[box, below=of cls] (rec) {文本识别 (SVTR)};
                \node[box, below=of rec] (out) {输出结果};

                \draw[arrow] (img) -- (det);
                \draw[arrow] (det) -- (crop);
                \draw[arrow] (crop) -- (cls);
                \draw[arrow] (cls) -- (rec);
                \draw[arrow] (rec) -- (out);
            \end{tikzpicture}
        \end{center}
    \end{columns}
\end{frame}

\begin{frame}[fragile]{PaddleOCR模型配置与优化(70\%脚手架)}
    \begin{lstlisting}[basicstyle=\ttfamily\scriptsize]
from paddleocr import PaddleOCR

# TODO: 使用AI助手配置高性能OCR
# 提示词:"配置PaddleOCR高性能模式,需要GPU支持、
# 多线程处理、大模型版本,用于生产环境"

ocr = PaddleOCR(
    # ===== 基础配置 =====
    # TODO: 语言设置(支持多语言组合)
    lang=____,  # 如 'ch', 'en', 'ch_en', 'japan', 'korean'

    # TODO: 模型版本(提示词:"PaddleOCR模型版本选择")
    det_model_dir=____,  # 检测模型路径
    rec_model_dir=____,  # 识别模型路径
    cls_model_dir=____,  # 方向分类模型路径

    # ===== 性能优化配置 =====
    # TODO: GPU配置(提示词:"PaddleOCR GPU加速配置参数")
    use_gpu=____,
    gpu_mem=____,  # GPU显存限制

    # TODO: 多线程配置
    use_tensorrt=____,  # TensorRT加速
    use_fp16=____,      # 半精度推理
    enable_mkldnn=____, # MKL-DNN加速
    cpu_threads=____,   # CPU线程数

    # ===== 检测模型参数 =====
    # TODO: 检测参数调优(提示词:"DBNet检测参数调优")
    det_db_thresh=____,      # 二值化阈值
    det_db_box_thresh=____,  # 框过滤阈值
    det_db_unclip_ratio=____, # 文本框扩展系数
    max_text_length=____,    # 最大文本长度
    det_db_score_mode=____,  # 得分计算模式

    # ===== 识别模型参数 =====
    # TODO: 识别参数调优
    rec_batch_num=____,      # 识别batch大小
    max_text_length=____,    # 最大文本长度
    rec_score_thres=____,    # 识别置信度阈值

    # ===== 方向分类参数 =====
    use_angle_cls=____,      # 启用方向分类
    cls_batch_num=____,    # 分类batch大小
    cls_thresh=____,        # 分类置信度阈值

    # ===== 输出控制 =====
    drop_score=____,       # 过滤低置信度结果
    show_log=____,         # 显示日志
    use_pdserving=____,    # 使用PaddleServing
    use_visual_backbone=____, # 可视化骨干网络
)
    \end{lstlisting}

    \begin{alertblock}{AI参数调优助手}
        不知道参数怎么设?\textbf{问AI:}
        "PaddleOCR在生产环境中GPU推理,det_db_thresh设为多少合适?"
    \end{alertblock}
\end{frame}

\begin{frame}[fragile]{自定义训练:PaddleOCR模型微调(70\%脚手架)}
    \begin{lstlisting}[basicstyle=\ttfamily\scriptsize]
# TODO: 使用AI助手完成PaddleOCR自定义训练
# 提示词:"PaddleOCR自定义训练完整流程,包括数据准备、
# 配置文件修改、训练启动、模型评估"

# ===== 步骤1: 数据准备 =====
# TODO: 准备训练数据(提示词:"PaddleOCR训练数据格式要求")
# 文本检测数据格式:
# img_001.jpg\t[[x1,y1],[x2,y2],[x3,y3],[x4,y4], "文字内容"]

# TODO: 划分训练集和验证集
import os
import shutil
import random

def prepare_dataset(src_dir, dst_dir, train_ratio=0.8):
    """准备PaddleOCR训练数据集"""
    # TODO: 使用AI助手补全代码
    # 提示词:"Python划分训练集验证集,复制图像和标注文件"

    images = [f for f in os.listdir(src_dir) if f.endswith(('.jpg', '.png'))]
    random.shuffle(images)

    # TODO: 按比例划分
    split_idx = int(len(images) * ____)
    train_images = images[____]
    val_images = images[____]

    # TODO: 复制文件到目标目录
    for img in train_images:
        shutil.copy(os.path.join(src_dir, img), ____)
        # TODO: 复制对应的标注文件
        ____

    return len(train_images), len(val_images)

# ===== 步骤2: 配置文件修改 =====
# TODO: 修改PaddleOCR配置文件(提示词:"PaddleOCR配置文件各参数含义")
config_content = '''
# 检测模型配置
global:
  debug: false
  use_gpu: ____  # TODO: 是否使用GPU
  epoch_num: ____  # TODO: 训练轮数
  log_smooth_window: 20
  print_batch_step: 10
  save_model_dir: ./output/det_db/
  save_epoch_step: 100
  eval_batch_step: [0, 2000]
  cal_metric_during_train: false
  pretrained_model: null
  checkpoints: null
  save_inference_dir: null
  use_visualdl: false
  infer_img: doc/imgs_en/img_10.jpg

# TODO: 更多配置参数...
'''

# ===== 步骤3: 启动训练 =====
# TODO: 使用PaddleOCR训练脚本
# 命令行:python tools/train.py -c configs/det/det_db_mv3.yml

def start_training(config_path, use_gpu=True, num_epochs=100):
    """启动PaddleOCR训练"""
    import subprocess

    cmd = [
        'python', 'tools/train.py',
        '-c', config_path,
        # TODO: 添加更多参数
    ]

    if use_gpu:
        cmd.append('--use_gpu')

    # TODO: 启动训练进程
    process = subprocess.____(____)

    return process

# ===== 步骤4: 模型评估 =====
# TODO: 评估训练好的模型
# 命令行:python tools/eval.py -c configs/det/det_db_mv3.yml -o Global.checkpoints=./output/det_db/best_accuracy

# TODO: 导出推理模型
# 命令行:python tools/export_model.py -c configs/det/det_db_mv3.yml -o Global.pretrained_model=./output/det_db/best_accuracy Global.save_inference_dir=./inference/det_db
    \end{lstlisting}

    \begin{alertblock}{AI辅助训练提示}
        训练遇到NaN loss?\textbf{问AI:}
        "PaddleOCR训练时loss变成NaN,可能是什么原因,如何调试?"
    \end{alertblock}
\end{frame}
