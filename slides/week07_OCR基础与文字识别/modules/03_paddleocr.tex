%===========================================================
% modules/03_paddleocr.tex - PaddleOCR快速上手
%===========================================================

\section{PaddleOCR快速上手}

\begin{frame}[fragile]{PaddleOCR安装(70\%脚手架)}
    \textbf{任务:}使用AI助手补全PaddleOCR安装命令

    \textbf{提示词(Prompts):}
    \begin{itemize}
        \item "How to install PaddlePaddle via pip"
        \item "How to install PaddleOCR"
        \item "pip install opencv-python and pillow"
    \end{itemize}

    \begin{lstlisting}
# TODO: Install PaddlePaddle
pip install _____________

# TODO: Install PaddleOCR
pip install _____________

# TODO: Install image processing dependencies
pip install _____________
    \end{lstlisting}

    \begin{alertblock}{注意}
        \begin{itemize}
            \item 首次使用会自动下载模型
            \item 需要联网
            \item 首次运行较慢
        \end{itemize}
    \end{alertblock}
\end{frame}

\begin{frame}[fragile]{基础使用(70\%脚手架)}
    \begin{lstlisting}[basicstyle=\ttfamily\scriptsize]
from paddleocr import PaddleOCR

# TODO: 使用AI助手(如Cursor的Ctrl+K)补全初始化参数
# 提示词:"配置PaddleOCR,需要支持中文识别和方向分类,用于试卷文字识别"
ocr = PaddleOCR(
    # TODO: 语言设置(提示词:"PaddleOCR lang参数说明")
    lang='____',  # 如 'ch', 'en', 'ch_en'

    # TODO: 启用方向分类(提示词:"PaddleOCR use_angle_cls参数说明")
    use_angle_cls=____  # True/False
)

# TODO: 识别图像(提示词:"PaddleOCR ocr方法参数说明")
result = ocr.ocr('____', cls=____)

# TODO: 解析识别结果(提示词:"PaddleOCR结果格式说明,如何提取文字和置信度")
for line in result[0]:
    text = line[____][____]      # 提取文字内容
    confidence = line[____][____]  # 提取置信度
    print(f"{text} ({confidence:.4f})")
    \end{lstlisting}

    \begin{alertblock}{AI辅助编程提示}
        使用Cursor/Claude Code时,可以这样Prompt:
        \begin{itemize}
            \item "PaddleOCR初始化需要哪些参数?用于中文试卷识别。"
            \item "ocr.ocr()方法的返回值是什么格式?"
            \item "如何从PaddleOCR结果中提取文字和置信度?"
        \end{itemize}
    \end{alertblock}
\end{frame}

\begin{frame}{结果格式}
    \textbf{返回格式:}
    \begin{verbatim}
[
  [
    [[x1,y1], [x2,y2], [x3,y3], [x4,y4]],  # 文字框坐标
    ('文字内容', 0.98)                        # 文字+置信度
  ],
  ...
]
    \end{verbatim}

    \vspace{0.2cm}

    \textbf{字段说明:}
    \begin{itemize}
        \item 坐标:文字框的四个角点
        \item 文字内容:识别出的文本
        \item 置信度:0-1之间,越接近1越准确
    \end{itemize}
\end{frame}
