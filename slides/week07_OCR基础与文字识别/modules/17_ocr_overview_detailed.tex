%===========================================================
% modules/17_ocr_overview_detailed.tex - OCR技术概述(详细)
%===========================================================

\section{OCR技术概述(详细)}

\begin{frame}{OCR技术发展历程}
    \begin{columns}
        \column{0.5\textwidth}
        \textbf{第一代:模板匹配时代(1960s-1980s)}
        \begin{itemize}
            \item 基于字符模板匹配
            \item 只能识别特定字体
            \item 对噪声敏感
            \item 代表:IBM 1287
        \end{itemize}

        \vspace{0.3cm}

        \textbf{第二代:特征提取时代(1980s-2000s)}
        \begin{itemize}
            \item 基于结构/统计特征
            \item 支持多字体识别
            \item 使用ML分类器(SVM、KNN)
            \item 代表:Tesseract(早期版本)
        \end{itemize}

        \vspace{0.3cm}

        \textbf{第三代:深度学习时代(2012-至今)}
        \begin{itemize}
            \item 基于CNN/RNN/Transformer
            \item 端到端识别
            \item 支持场景文字
            \item 代表:CRNN、PaddleOCR、TrOCR
        \end{itemize}

        \column{0.5\textwidth}
        \textbf{OCR技术发展里程碑:}

        \vspace{0.3cm}

        \begin{table}
            \centering
            \small
            \begin{tabular}{lll}
                \toprule
                \textbf{年份} & \textbf{里程碑} & \textbf{影响} \\
                \midrule
                1914 & 第一台OCR设备 & 起源 \\
                1965 & IBM 1287 & 商用化 \\
                1974 & Kurzweil OCR & 多字体支持 \\
                1985 & Tesseract诞生 & 开源化 \\
                1995 & OCR标准化 & 行业规范 \\
                2006 & Deep Learning兴起 & 技术革命 \\
                2012 & AlexNet突破 & 深度学习OCR \\
                2015 & CRNN提出 & 序列OCR \\
                2017 & Transformer提出 & 注意力机制 \\
                2020 & TrOCR发布 & 端到端Transformer OCR \\
                2021 & PaddleOCR 2.0 & 产业级OCR \\
                2023 & GPT-4V & 多模态OCR \\
                \bottomrule
            \end{tabular}
        \end{table}

        \vspace{0.3cm}

        \textbf{OCR技术发展趋势:}
        \begin{itemize}
            \item \textbf{通用化}:从印刷体到手写体、场景文字
            \item \textbf{多语言}:支持更多语种和混合文本
            \item \textbf{端到端}:从流水线到一体化模型
            \item \textbf{大模型}:基于Transformer的OCR大模型
            \item \textbf{多模态}:结合视觉-语言理解
        \end{itemize}
    \end{columns}
\end{frame}
