%===========================================================
% modules/11_dl_ocr.tex - 深度学习OCR模型实战
%===========================================================

\section{深度学习OCR模型实战}

\begin{frame}{深度学习OCR架构演进}
    \begin{columns}
        \column{0.5\textwidth}
        \textbf{OCR架构三代演进:}
        \begin{enumerate}
            \item \textbf{第一代:CNN+全连接}
            \begin{itemize}
                \item 单字识别
                \item 固定输入尺寸
                \item LeNet-5, AlexNet
            \end{itemize}
            \item \textbf{第二代:CNN+RNN+CTC}
            \begin{itemize}
                \item 序列识别
                \item 端到端训练
                \item CRNN, CTPN
            \end{itemize}
            \item \textbf{第三代:Transformer}
            \begin{itemize}
                \item 自注意力机制
                \item 视觉-语言预训练
                \item TrOCR, SRN
            \end{itemize}
        \end{enumerate}

        \column{0.5\textwidth}
        \begin{center}
            \begin{tikzpicture}[
                node distance=0.8cm,
                box/.style={rectangle, draw=blue!60, fill=blue!10, rounded corners, minimum width=4cm, minimum height=0.6cm, font=\scriptsize},
                arrow/.style={->, thick, blue!60}
            ]
                \node[box] (cnn) {CNN特征提取};
                \node[box, below=of cnn] (rnn) {RNN序列建模};
                \node[box, below=of rnn] (ctc) {CTC解码};
                \node[box, below=of ctc] (out) {文本输出};

                \draw[arrow] (cnn) -- (rnn);
                \draw[arrow] (rnn) -- (ctc);
                \draw[arrow] (ctc) -- (out);
            \end{tikzpicture}

            \vspace{0.5cm}

            \textit{\footnotesize CRNN架构流程图}
        \end{center}
    \end{columns}
\end{frame}

\begin{frame}[fragile]{PyTorch实现CRNN模型(70\%脚手架)}
    \begin{lstlisting}[basicstyle=\ttfamily\scriptsize]
import torch
import torch.nn as nn

class CRNN(nn.Module):
    """CRNN: CNN + RNN + CTC for OCR"""

    def __init__(self, img_height, nc, nclass, nh):
        super(CRNN, self).__init__()

        # TODO: 使用AI助手完成CNN特征提取层
        # 提示词:"PyTorch实现VGG风格的CNN特征提取,
        # 用于OCR图像特征提取,包含卷积层、BatchNorm、ReLU、MaxPool"

        self.cnn = nn.Sequential(
            # TODO: 第1层卷积
            nn.Conv2d(nc, 64, kernel_size=____, padding=____),
            nn.BatchNorm2d(____),
            nn.ReLU(____),
            nn.MaxPool2d(____, ____),

            # TODO: 第2层卷积
            nn.Conv2d(____, 128, kernel_size=____, padding=____),
            ____,  # TODO: 添加BatchNorm和ReLU
            nn.MaxPool2d(____, ____),

            # TODO: 继续添加更多卷积层...
        )

        # TODO: 计算CNN输出尺寸
        # 提示词:"计算CNN输出尺寸用于确定RNN输入维度"
        self.rnn_input_size = self._calculate_rnn_input(img_height)

        # TODO: 使用AI助手完成RNN层
        # 提示词:"PyTorch实现双向LSTM用于OCR序列建模"
        self.rnn = nn.LSTM(
            input_size=self.rnn_input_size,
            hidden_size=____,      # TODO: 隐藏层大小
            num_layers=____,       # TODO: LSTM层数
            bidirectional=____,    # TODO: 是否双向
            batch_first=____       # TODO: batch维度位置
        )

        # TODO: 完成输出层
        self.fc = nn.Linear(____, ____)  # TODO: 输入和输出维度

    def forward(self, x):
        # TODO: 使用AI助手完成前向传播
        # 提示词:"PyTorch实现CRNN前向传播流程"

        # CNN特征提取
        conv = self.cnn(x)
        batch, channel, height, width = conv.size()

        # TODO: 调整维度为RNN输入格式
        conv = conv.view(batch, ____, ____)  # (batch, width, channel*height)
        conv = conv.permute(____, ____, ____)  # (width, batch, features)

        # RNN序列建模
        recurrent, _ = self.rnn(conv)

        # TODO: 全连接层输出
        T, b, h = recurrent.size()
        t_rec = recurrent.view(T * b, h)
        output = self.fc(t_rec)
        output = output.view(T, b, -1)

        return output
    \end{lstlisting}
\end{frame}

\begin{frame}[fragile]{CTC Loss与解码(70\%脚手架)}
    \begin{lstlisting}[basicstyle=\ttfamily\scriptsize]
import torch
import torch.nn as nn

class CTCLossWrapper(nn.Module):
    """CTC Loss for OCR"""

    def __init__(self, blank=0, reduction='mean'):
        super().__init__()
        # TODO: 使用AI助手完成CTC Loss
        # 提示词:"PyTorch CTCLoss参数说明和使用方法"
        self.ctc_loss = nn.CTCLoss(
            blank=____,           # TODO: 空白符索引
            reduction=____,       # TODO: 缩减方式
            zero_infinity=____    # TODO: 是否将无穷大损失置零
        )

    def forward(self, logits, targets, input_lengths, target_lengths):
        """
        Args:
            logits: (T, N, C) - 模型输出
            targets: (N, S) - 目标序列
            input_lengths: (N,) - 输入序列长度
            target_lengths: (N,) - 目标序列长度
        """
        # TODO: 计算CTC Loss
        # 提示词:"PyTorch CTCLoss输入格式要求"
        log_probs = torch.log_softmax(____, dim=____)
        loss = self.ctc_loss(____, ____, ____, ____)
        return loss


def ctc_greedy_decoder(logits, blank=0):
    """CTC贪心解码"""
    # TODO: 使用AI助手完成CTC解码
    # 提示词:"CTC贪心解码算法实现步骤"

    # 获取最大概率的类别
    preds = torch.argmax(____, dim=____)  # (T, N)

    decoded = []
    for batch_idx in range(preds.size(1)):
        seq = preds[:, batch_idx].tolist()

        # TODO: 合并重复字符
        collapsed = []
        for i, char in enumerate(seq):
            if ____ and ____ != ____:  # TODO: 非空且不等于前一个
                collapsed.append(____)

        decoded.append(collapsed)

    return decoded


def ctc_beam_search_decoder(logits, beam_width=5, blank=0):
    """CTC束搜索解码(高级)"""
    # TODO: 使用AI助手实现束搜索
    # 提示词:"CTC束搜索解码算法实现"

    # 初始化beam
    beams = [(____, 0)]  # (序列, 分数)

    for t in range(logits.size(0)):
        candidates = []

        for seq, score in beams:
            for c in range(logits.size(1)):
                # TODO: 计算新的序列和分数
                new_seq = ____(seq, c)
                new_score = ____ + ____(logits[t, c])
                candidates.append((____, ____))

        # TODO: 保留top-k
        beams = sorted(candidates, key=lambda x: x[1], reverse=True)[:____]

    return beams[0]
    \end{lstlisting}
\end{frame}
