%===========================================================
% modules/18_ocr_classification.tex - OCR技术分类
%===========================================================

\section{OCR技术分类}

\begin{frame}{OCR技术多维分类体系}
    \begin{columns}
        \column{0.5\textwidth}
        \textbf{按技术路线分类:}
        \begin{itemize}
            \item \textbf{传统OCR}
            \begin{itemize}
                \item 模板匹配
                \item 特征工程
                \item 机器学习分类器
            \end{itemize}
            \item \textbf{深度学习OCR}
            \begin{itemize}
                \item CNN特征提取
                \item RNN序列建模
                \item Attention机制
                \item Transformer架构
            \end{itemize}
        \end{itemize}

        \vspace{0.3cm}

        \textbf{按输入类型分类:}
        \begin{itemize}
            \item \textbf{扫描文档OCR}
            \begin{itemize}
                \item 印刷体
                \item 手写体
                \item 表格
            \end{itemize}
            \item \textbf{场景文字OCR}
            \begin{itemize}
                \item 自然场景
                \item 街景文字
                \item 商品包装
            \end{itemize}
            \item \textbf{视频OCR}
            \begin{itemize}
                \item 字幕识别
                \item 实时识别
            \end{itemize}
        \end{itemize}

        \column{0.5\textwidth}
        \textbf{按识别对象分类:}
        \begin{itemize}
            \item \textbf{单字识别 vs 整行识别}
            \begin{itemize}
                \item 单字:独立分类
                \item 整行:序列建模
            \end{itemize}
            \item \textbf{印刷体 vs 手写体}
            \begin{itemize}
                \item 印刷体:规则、规范
                \item 手写体:个性化、变化大
            \end{itemize}
            \item \textbf{单语言 vs 多语言}
            \begin{itemize}
                \item 单语言:专用模型
                \item 多语言:统一模型
            \end{itemize}
        \end{itemize}

        \vspace{0.3cm}

        \textbf{按应用场景分类:}
        \begin{itemize}
            \item \textbf{文档数字化}
            \begin{itemize}
                \item 书籍扫描
                \item 档案数字化
                \item 票据处理
            \end{itemize}
            \item \textbf{证件识别}
            \begin{itemize}
                \item 身份证
                \item 银行卡
                \item 驾驶证
            \end{itemize}
            \item \textbf{移动应用}
            \begin{itemize}
                \item 拍照翻译
                \item 名片识别
                \item 文档扫描APP
            \end{itemize}
        \end{itemize}
    \end{columns}
\end{frame}

\begin{frame}{传统OCR vs 深度学习OCR对比}
    \begin{table}
        \centering
        \small
        \begin{tabular}{p{3cm}p{5cm}p{5cm}}
            \toprule
            \textbf{对比维度} & \textbf{传统OCR} & \textbf{深度学习OCR} \\
            \midrule
            \textbf{特征提取} &
            手工设计特征(HOG、LBP、投影) &
            自动学习特征(CNN) \\
            \midrule
            \textbf{模型设计} &
            分模块设计(检测、分割、识别) &
            端到端训练 & \\
            \midrule
            \textbf{数据依赖} &
            小样本即可训练 &
            需要大量标注数据 & \\
            \midrule
            \textbf{泛化能力} &
            对复杂场景适应性差 &
            强泛化能力,适应多场景 & \\
            \midrule
            \textbf{识别精度} &
            印刷体:高(>95%)
            手写体:中(70-85%) &
            印刷体:极高(>99%)
            手写体:高(85-95%) & \\
            \midrule
            \textbf{计算效率} &
            CPU即可实时处理 &
            通常需要GPU加速 & \\
            \midrule
            \textbf{可解释性} &
            强,每一步可分析 &
            弱,黑盒模型 & \\
            \midrule
            \textbf{维护成本} &
            高,需调多个模块参数 &
            中,主要调网络结构和超参 & \\
            \bottomrule
        \end{tabular}
    \end{table}
\end{frame}
