%===========================================================
% modules/08_traditional_ocr.tex - 传统OCR识别实战
%===========================================================

\section{传统OCR识别实战}

\begin{frame}{传统OCR识别流程概述}
    \begin{columns}
        \column{0.5\textwidth}
        \textbf{传统OCR四步走:}
        \begin{enumerate}
            \item \textbf{预处理}:图像增强、二值化、去噪
            \item \textbf{版面分析}:文字区域定位、分栏
            \item \textbf{字符分割}:行分割、字分割
            \item \textbf{字符识别}:特征提取、分类器
        \end{enumerate}

        \column{0.5\textwidth}
        \textbf{各阶段关键技术:}
        \begin{itemize}
            \item 预处理:灰度化、二值化、倾斜校正
            \item 版面分析:投影分析、连通域分析
            \item 分割:投影分割、连通域分割
            \item 识别:模板匹配、特征分类
        \end{itemize}
    \end{columns}
\end{frame}

\begin{frame}{图像预处理实战(70\%脚手架)}
    \begin{lstlisting}[basicstyle=\ttfamily\scriptsize]
import cv2
import numpy as np

def traditional_ocr_preprocess(image_path):
    """传统OCR图像预处理"""
    # TODO: 使用AI助手补全代码
    # 提示词:"OpenCV读取图像并转为灰度图"

    # TODO: 读取图像
    image = cv2.____(image_path)

    # TODO: 转为灰度图
    gray = cv2.cvtColor(____, ____)

    # TODO: 二值化(提示词:"Otsu自动阈值二值化OpenCV")
    _, binary = cv2.threshold(____, 0, 255, ____)

    # TODO: 去噪(提示词:"OpenCV形态学开运算去除噪点")
    kernel = np.ones((____, ____), np.uint8)
    denoised = cv2.morphologyEx(____, _____, kernel)

    return denoised
    \end{lstlisting}

    \begin{alertblock}{AI辅助提示}
        不知道cv2函数的参数?\textbf{选中函数名,按Ctrl+K问AI:}
        "cv2.morphologyEx的参数含义和常用取值是什么?"
    \end{alertblock}
\end{frame}

\begin{frame}{版面分析与文字区域定位(70\%脚手架)}
    \begin{lstlisting}[basicstyle=\ttfamily\scriptsize]
def layout_analysis(binary_image):
    """版面分析:定位文字区域"""
    # TODO: 使用AI助手补全代码
    # 提示词:"OpenCV查找轮廓定位文字区域"

    # TODO: 查找轮廓(提示词:"cv2.findContours查找外轮廓")
    contours, _ = cv2.findContours(
        _____,
        cv2._____,  # TODO: 只检测外轮廓
        cv2._____    # TODO: 简单近似方法
    )

    text_regions = []
    for contour in contours:
        # TODO: 获取边界框(提示词:"cv2.boundingRect获取轮廓边界框")
        x, y, w, h = cv2._____(contour)

        # TODO: 过滤条件(提示词:"根据宽高比和面积过滤非文字区域")
        if _____ and _____:  # TODO: 设置合理阈值
            text_regions.append((x, y, w, h))

    return text_regions
    \end{lstlisting}

    \begin{alertblock}{AI辅助调试}
        轮廓检测结果不对?\textbf{把代码和图像一起发给AI:}
        "这段代码检测文字区域效果很差,能帮我分析原因并提供改进建议吗?"
    \end{alertblock}
\end{frame}

\begin{frame}{字符分割与识别(70\%脚手架)}
    \begin{lstlisting}[basicstyle=\ttfamily\scriptsize]
def segment_and_recognize(text_region, binary_image):
    """字符分割与识别"""
    # TODO: 使用AI助手补全代码
    # 提示词:"OpenCV投影法分割文字行和字符"

    x, y, w, h = text_region

    # TODO: 提取区域图像
    region = binary_image[y:y+h, _____]

    # TODO: 水平投影分割行(提示词:"numpy计算水平投影分割文字行")
    h_projection = np._____(_____, axis=1)
    row_ranges = _____  # TODO: 根据投影值找出文字行范围

    characters = []
    for row_start, row_end in row_ranges:
        row_image = region[row_start:row_end, :]

        # TODO: 垂直投影分割字符(提示词:"垂直投影分割单个字符")
        v_projection = np._____(_____, axis=0)
        char_ranges = _____

        for char_start, char_end in char_ranges:
            char_image = row_image[:, char_start:char_end]
            characters.append(char_image)

    return characters
    \end{lstlisting}
\end{frame}
