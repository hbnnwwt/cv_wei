%===========================================================
% modules/01_intro.tex - 课程介绍与导入
%===========================================================

\section{课程导入}

\begin{frame}{从上周说起}
    \textbf{回顾第6周:判断题识别与符号匹配}
    \begin{itemize}
        \item 学会了检测判断题的对勾和叉号
        \item 了解了符号匹配和字符识别的区别
    \end{itemize}

    \vspace{0.5cm}

    \textbf{本周新问题:}
    \begin{alertblock}{挑战}
        试卷上还有大量印刷文字:标题、题号、选项文字……

        \textbf{怎么让机器"读懂"这些文字?}
    \end{alertblock}
\end{frame}

\begin{frame}{本周故事线}
    \begin{center}
        \begin{tikzpicture}[
            node distance=1.2cm,
            box/.style={rectangle, draw=blue!60, fill=blue!10, rounded corners, minimum width=3.5cm, minimum height=0.8cm, text centered, font=\small}
        ]
            \node[box] (n1) {1. OCR是什么?};
            \node[box, below=1.2cm of n1] (n2) {2. PaddleOCR安装与使用};
            \node[box, below=1.2cm of n2] (n3) {3. 识别试卷文字};
            \node[box, below=1.2cm of n3] (n4) {4. 优化识别效果};
            \node[box, below=1.2cm of n4] (n5) {5. 实战作业};

            \draw[->, thick, blue!60] (n1) -- (n2);
            \draw[->, thick, blue!60] (n2) -- (n3);
            \draw[->, thick, blue!60] (n3) -- (n4);
            \draw[->, thick, blue!60] (n4) -- (n5);
        \end{tikzpicture}
    \end{center}
\end{frame}

\begin{frame}{学习目标}
    \begin{columns}
        \column{0.5\textwidth}
        \textbf{知识目标:}
        \begin{itemize}
            \item 理解OCR技术原理
            \item 了解OCR发展历程
            \item 掌握PaddleOCR基本使用
        \end{itemize}

        \vspace{0.3cm}

        \textbf{技能目标:}
        \begin{itemize}
            \item 能安装配置PaddleOCR
            \item 能识别图像中的文字
            \item 能提取标题和题号
        \end{itemize}

        \column{0.5\textwidth}
        \textbf{素质目标:}
        \begin{itemize}
            \item 培养跨学科学习能力
            \item 锻炼动手实践能力
            \item 提升团队协作意识
        \end{itemize}

        \vspace{0.5cm}

        \begin{block}{预期成果}
            完成一个能识别试卷印刷文字的程序
        \end{block}
    \end{columns}
\end{frame}
