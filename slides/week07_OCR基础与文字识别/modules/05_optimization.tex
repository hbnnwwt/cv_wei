%===========================================================
% modules/05_optimization.tex - OCR优化技巧
%===========================================================

\section{OCR优化}

\begin{frame}[fragile]{图像预处理优化}
    \textbf{专门的OCR预处理(70\%脚手架):}
    \begin{lstlisting}
def preprocess_for_ocr(image):
    """OCR优化预处理"""
    # TODO: 使用AI助手补全代码
    # 提示词:"如何将BGR图像转为灰度图用于OCR预处理"

    # TODO: 转灰度
    gray = cv2.cvtColor(____, ____)

    # TODO: 去噪(提示词:"OpenCV非局部均值去噪方法")
    denoised = cv2.____(gray)

    # TODO: 二值化(提示词:"OTSU自动阈值二值化")
    _, binary = cv2.threshold(
        ____, 0, 255,
        ____ + ____(____)
    )

    return binary
    \end{lstlisting}

    \begin{alertblock}{AI辅助编程提示}
        不知道cv2函数的参数?\textbf{选中函数名,按Ctrl+K问AI:}
        "cv2.fastNlMeansDenoising的参数是什么?"
    \end{alertblock}
\end{frame}

\begin{frame}[fragile]{识别参数优化(70\%脚手架)}
    \textbf{PaddleOCR初始化参数:}
    \begin{lstlisting}
# TODO: 使用AI助手配置PaddleOCR参数
# 提示词:"配置PaddleOCR参数,需要:1)中文识别 2)方向分类
# 3)检测阈值0.3 4)置信度阈值0.5"

ocr = PaddleOCR(
    # TODO: 语言设置(提示词:"PaddleOCR中文语言代码")
    lang=____,

    # TODO: 方向分类(提示词:"PaddleOCR启用方向分类参数")
    use_angle_cls=____,

    # TODO: 检测阈值(默认0.3,越小检测越严格)
    det_db_thresh=____,
    det_db_box_thresh=____,

    # TODO: 识别置信度阈值(默认0.5)
    rec_score_thres=____,

    # 性能参数(可选)
    use_gpu=____,        # 是否使用GPU
    show_log=____        # 是否显示日志
)
    \end{lstlisting}

    \begin{alertblock}{AI辅助配置提示}
        不知道参数值?\textbf{选中PaddleOCR,按Ctrl+K问AI:}
        "PaddleOCR的det_db_thresh参数是什么意思,推荐值是多少?"
    \end{alertblock}
\end{frame}

\begin{frame}{常见优化技巧}
    \begin{columns}
        \column{0.5\textwidth}
        \textbf{1. 图像层面}
        \begin{itemize}
            \item 提高扫描分辨率(300dpi以上)
            \item 校正图像倾斜角度
            \item 去除噪点和干扰线
            \item 适当增强对比度
        \end{itemize}

        \vspace{0.3cm}

        \textbf{2. 预处理层面}
        \begin{itemize}
            \item 先进行版面分析
            \item 按区域分别识别
            \item 文字区域精确定位
        \end{itemize}

        \column{0.5\textwidth}
        \textbf{3. 后处理层面}
        \begin{itemize}
            \item 根据置信度过滤
            \item 利用上下文纠错
            \item 正则表达式匹配
            \item 词典校验
        \end{itemize}

        \vspace{0.3cm}

        \begin{block}{核心原则}
            OCR是系统工程,\textbf{预处理 > 算法 > 后处理}
        \end{block}
    \end{columns}
\end{frame}
