%===========================================================
% modules/06_quiz.tex - 课堂思考与测验
%===========================================================

\section{课堂思考}

\begin{frame}{课堂思考题}
    \begin{block}{问题1:OCR技术}
        \begin{itemize}
            \item 为什么深度学习让OCR准确率大幅提升?
            \item Tesseract和PaddleOCR各有什么优势?
        \end{itemize}
    \end{block}

    \vspace{0.3cm}

    \begin{block}{问题2:文字识别}
        \begin{itemize}
            \item OCR识别结果中"confidence"是什么意思?
            \item 如何处理OCR识别错误的文字?
        \end{itemize}
    \end{block}
\end{frame}

\section{课后作业}

\begin{frame}{课后作业}
    \begin{block}{题目}
        用OCR识别试卷中的印刷文字
    \end{block}

    \textbf{要求:}
    \begin{enumerate}
        \item 安装配置PaddleOCR
        \item 识别试卷标题和题号
        \item 提取所有文字内容
        \item 可视化标注识别结果
    \end{enumerate}

    \vspace{0.2cm}

    \textbf{评分标准:}
    \begin{itemize}
        \item 环境配置:20分
        \item 识别效果:40分
        \item 信息提取:25分
        \item 可视化:15分
    \end{itemize}
\end{frame}

\section{下节预告}

\begin{frame}{下节预告}
    \begin{center}
        \Large \textbf{第8周:手写简答题识别}

        \vspace{0.5cm}

        \normalsize
        故事问题:\textcolor{blue}{能看懂学生写的答案吗?}

        \vspace{0.3cm}

        你将学会:
        \begin{itemize}
            \item 手写识别的挑战
            \item TrOCR/PaddleOCR手写模型
            \item 手写文字识别
        \end{itemize}
    \end{center}
\end{frame}

\begin{frame}
    \begin{center}
        \Huge \textbf{谢谢!}
    \end{center}
\end{frame}
