%===========================================================
% modules/13_traditional_theory.tex - 传统OCR识别流程理论
%===========================================================

\section{传统OCR识别流程理论}

\begin{frame}{传统OCR四步流程详解}
    \begin{columns}
        \column{0.5\textwidth}
        \textbf{第一步:图像预处理}
        \begin{itemize}
            \item 灰度化:彩色图转灰度图
            \item 二值化:Otsu自适应阈值
            \item 去噪:形态学操作、滤波
            \item 倾斜校正:霍夫变换、投影法
        \end{itemize}

        \vspace{0.3cm}

        \textbf{第二步:版面分析}
        \begin{itemize}
            \item 连通域分析
            \item 投影分析
            \item 分栏检测
            \item 阅读顺序确定
        \end{itemize}

        \column{0.5\textwidth}
        \textbf{第三步:字符分割}
        \begin{itemize}
            \item 行分割:水平投影
            \item 字分割:垂直投影
            \item 粘连字符处理
            \item 断字合并
        \end{itemize}

        \vspace{0.3cm}

        \textbf{第四步:字符识别}
        \begin{itemize}
            \item 特征提取:HOG、LBP、投影
            \item 分类器:模板匹配、KNN、SVM
            \item 后处理:语言模型、词典校正
        \end{itemize}
    \end{columns}
\end{frame}

\begin{frame}{图像预处理技术详解}
    \begin{columns}
        \column{0.5\textwidth}
        \textbf{1. 灰度化方法:}
        \begin{itemize}
            \item 加权平均法:$Gray = 0.299R + 0.587G + 0.114B$
            \item 最大值法:$Gray = max(R, G, B)$
            \item 平均值法:$Gray = (R + G + B) / 3$
        \end{itemize}

        \vspace{0.3cm}

        \textbf{2. 二值化方法:}
        \begin{itemize}
            \item 全局阈值:固定阈值分割
            \item Otsu算法:自动计算最优阈值
            \item 自适应阈值:局部阈值处理
        \end{itemize}

        \column{0.5\textwidth}
        \textbf{3. 倾斜校正方法:}
        \begin{itemize}
            \item 投影法:寻找最小投影方差角度
            \item 霍夫变换:检测直线角度
            \item 傅里叶变换:频域分析
        \end{itemize}

        \vspace{0.3cm}

        \textbf{4. 去噪方法:}
        \begin{itemize}
            \item 中值滤波:去除椒盐噪声
            \item 高斯滤波:平滑处理
            \item 形态学操作:开运算、闭运算
        \end{itemize}
    \end{columns}
\end{frame}

\begin{frame}{特征提取与分类器设计}
    \begin{columns}
        \column{0.5\textwidth}
        \textbf{特征提取方法:}

        \vspace{0.2cm}

        \textbf{1. 统计特征:}
        \begin{itemize}
            \item 投影直方图
            \item 轮廓特征
            \item 网格特征
            \item Zernike矩
        \end{itemize}

        \vspace{0.2cm}

        \textbf{2. 结构特征:}
        \begin{itemize}
            \item 端点、交叉点
            \item 笔画方向
            \item 环、洞特征
            \item 拓扑结构
        \end{itemize}

        \vspace{0.2cm}

        \textbf{3. 变换域特征:}
        \begin{itemize}
            \item 傅里叶描述子
            \item 小波特征
            \item HOG特征
            \item LBP特征
        \end{itemize}

        \column{0.5\textwidth}
        \textbf{分类器设计:}

        \vspace{0.2cm}

        \textbf{1. 模板匹配:}
        \begin{itemize}
            \item 最近邻匹配
            \item 相关性匹配
            \item 归一化匹配
        \end{itemize}

        \vspace{0.2cm}

        \textbf{2. K近邻(KNN):}
        \begin{itemize}
            \item 距离度量:欧氏距离、马氏距离
            \item K值选择
            \item 加权KNN
        \end{itemize}

        \vspace{0.2cm}

        \textbf{3. 支持向量机(SVM):}
        \begin{itemize}
            \item 核函数选择
            \item 参数优化
            \item 多分类策略
        \end{itemize}

        \vspace{0.2cm}

        \textbf{4. 神经网络:}
        \begin{itemize}
            \item MLP多层感知机
            \item 卷积神经网络
            \item 训练策略
        \end{itemize}
    \end{columns}
\end{frame}
