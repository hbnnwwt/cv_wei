%===========================================================
% modules/16_ctc.tex - 序列识别与CTC
%===========================================================

\section{序列识别与CTC}

\begin{frame}{CTC(Connectionist Temporal Classification)}
    \begin{columns}
        \column{0.5\textwidth}
        \textbf{为什么需要CTC?}
        \begin{itemize}
            \item 输入序列长度 $T$ 与输出序列长度 $L$ 不一致
            \item 对齐方式未知($T \neq L$)
            \item 无法直接使用交叉熵损失
            \item 需要端到端训练
        \end{itemize}

        \vspace{0.3cm}

        \textbf{CTC核心思想:}
        \begin{enumerate}
            \item 引入空白符 $\epsilon$(blank)
            \item 允许重复字符和空白
            \item 所有可能的对齐路径
            \item 路径概率求和
        \end{enumerate}

        \column{0.5\textwidth}
        \textbf{CTC对齐示例:}

        目标序列:"cat"

        输入长度:$T=6$

        可能的对齐路径:
        \begin{itemize}
            \item c-c-a-a-t-t
            \item c-$\u0007epsilon$-a-t-$epsilon$-t
            \item $epsilon$-c-a-a-$epsilon$-t
            \item ...(所有可能的路径)
        \end{itemize}

        \vspace{0.3cm}

        \textbf{CTC输出处理:}

        原始输出:"c-$epsilon$-a-a-t-$epsilon$"

        合并重复:"c-$epsilon$-a-t-$epsilon$"

        移除空白:"cat"

        \vspace{0.3cm}

        \textbf{CTC的优势:}
        \begin{itemize}
            \item 无需对齐标注
            \item 端到端训练
            \item 处理变长序列
            \item 网络输出与文本长度解耦
        \end{itemize}
    \end{columns}
\end{frame}

\begin{frame}{CTC数学原理与算法}
    \begin{columns}
        \column{0.5\textwidth}
        \textbf{CTC损失函数:}

        对于目标序列 $y = (y_1, y_2, ..., y_L)$,CTC损失定义为:

        \begin{equation*}
            \mathcal{L}_{CTC} = -\log P(y|x) = -\log \sum_{\pi \in \mathcal{B}^{-1}(y)} P(\pi|x)
        \end{equation*}

        其中:
        \begin{itemize}
            \item $\pi$:CTC路径(包含空白符的序列)
            \item $\mathcal{B}^{-1}(y)$:映射到目标序列 $y$ 的所有路径集合
            \item $P(\pi|x) = \prod_{t=1}^{T} p_t(\pi_t|x)$:路径概率
        \end{itemize}

        \vspace{0.3cm}

        \textbf{前向-后向算法:}

        为了高效计算CTC损失,使用动态规划的前向-后向算法。

        定义前向变量 $\alpha_t(s)$:在时刻 $t$ 到达状态 $s$ 的所有路径概率之和。

        定义后向变量 $\beta_t(s)$:从时刻 $t$ 的状态 $s$ 到序列结束的所有路径概率之和。

        则:
        \begin{equation*}
            P(y|x) = \sum_{s} \alpha_t(s) \cdot \beta_t(s)
        \end{equation*}

        \column{0.5\textwidth}
        \textbf{CTC解码策略:}

        \vspace{0.2cm}

        \textbf{1. 贪心解码(Greedy Decoding):}
        \begin{itemize}
            \item 每时刻选择概率最大的标签
            \item 合并重复字符,移除空白
            \item 简单快速,但可能不是最优路径
        \end{itemize}

        \begin{equation*}
            \pi^* = \arg\max_{\pi} \prod_{t=1}^{T} p_t(\pi_t|x)
        \end{equation*}

        \vspace{0.2cm}

        \textbf{2. 束搜索(Beam Search):}
        \begin{itemize}
            \item 维护Top-k个候选路径
            \item 考虑更全局的路径组合
            \item 精度更高,计算量更大
        \end{itemize}

        \begin{equation*}
            \mathcal{B} = \text{top-}k \{ P(\pi|x) : \pi \in \Pi \}
        \end{equation*}

        \vspace{0.2cm}

        \textbf{3. 前缀束搜索(Prefix Beam Search):}
        \begin{itemize}
            \item 在标签级别进行束搜索
            \item 更高效,实际应用广泛
        \end{itemize}
    \end{columns}
\end{frame}
