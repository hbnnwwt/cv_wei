%===========================================================
% modules/15_text_detection.tex - 文字检测技术
%===========================================================

\section{文字检测技术}

\begin{frame}{文字检测技术概述}
    \begin{columns}
        \column{0.5\textwidth}
        \textbf{文字检测 vs 目标检测:}
        \begin{itemize}
            \item 文字:长宽比极端、任意方向
            \item 目标:形状规则、方向固定
            \item 文字:密集排列、大小不一
            \item 目标:稀疏分布、尺度变化
        \end{itemize}

        \vspace{0.3cm}

        \textbf{文字检测的挑战:}
        \begin{itemize}
            \item 多方向、多尺度文字
            \item 弯曲文字、艺术字体
            \item 复杂背景、低质量图像
            \item 密集文字、重叠区域
        \end{itemize}

        \column{0.5\textwidth}
        \textbf{文字检测方法分类:}

        \vspace{0.2cm}

        \textbf{1. 基于回归的方法:}
        \begin{itemize}
            \item TextBoxes系列
            \item EAST
            \item 直接回归文字框坐标
        \end{itemize}

        \vspace{0.2cm}

        \textbf{2. 基于分割的方法:}
        \begin{itemize}
            \item PixelLink
            \item PSENet
            \item DBNet
            \item 像素级文字区域分割
        \end{itemize}

        \vspace{0.2cm}

        \textbf{3. 混合方法:}
        \begin{itemize}
            \item 检测+分割结合
            \item 多任务联合训练
        \end{itemize}
    \end{columns}
\end{frame}

\begin{frame}{EAST:基于回归的文字检测}
    \begin{columns}
        \column{0.5\textwidth}
        \textbf{EAST核心思想:}
        \begin{itemize}
            \item 全卷积网络(FCN)
            \item 像素级预测
            \item 直接回归文字框
            \item 非极大值抑制(NMS)
        \end{itemize}

        \vspace{0.3cm}

        \textbf{EAST网络结构:}
        \begin{enumerate}
            \item \textbf{特征提取}:PVANet / ResNet / VGG
            \item \textbf{特征融合}:U-Net结构,多尺度融合
            \item \textbf{输出层}:
            \begin{itemize}
                \item Score map:文字/非文字概率
                \item RBOX:旋转矩形框 $(d_1, d_2, d_3, d_4, \theta)$
                \item QUAD:四边形框 $(x_1, y_1, ..., x_4, y_4)$
            \end{itemize}
        \end{enumerate}

        \column{0.5\textwidth}
        \textbf{EAST损失函数:}

        \vspace{0.2cm}

        总损失:$L = L_s + \lambda_g L_g$

        \vspace{0.2cm}

        \textbf{1. 分类损失(Score Map):}
        \begin{equation*}
            L_s = -\frac{1}{N} \sum_{i=1}^{N} [y_i \log(\hat{y}_i) + (1-y_i) \log(1-\hat{y}_i)]
        \end{equation*}

        \vspace{0.2cm}

        \textbf{2. 几何损失(Geometry):}

        对于RBOX:
        \begin{equation*}
            L_g = L_{iou} + L_{\theta}
        \end{equation*}

        其中:
        \begin{itemize}
            \item $L_{iou}$:IoU损失(预测框与真实框的重叠度)
            \item $L_{\theta}$:角度损失,$L_{\theta} = 1 - \cos(\theta_{pred} - \theta_{gt})$
        \end{itemize}

        对于QUAD:
        \begin{equation*}
            L_g = \frac{1}{8} \sum_{i=1}^{4} |x_i^{pred} - x_i^{gt}| + |y_i^{pred} - y_i^{gt}|
        \end{equation*}

        \vspace{0.3cm}

        \textbf{EAST优势:}
        \begin{itemize}
            \item 端到端训练,无需候选框
            \item 速度快:支持实时检测
            \item 支持多方向文字
            \item 精度高:在ICDAR数据集表现优秀
        \end{itemize}
    \end{columns}
\end{frame}

\begin{frame}{DBNet:基于可微二值化的文字检测}
    \begin{columns}
        \column{0.5\textwidth}
        \textbf{DBNet核心创新:}
        \begin{itemize}
            \item 可微二值化(Differentiable Binarization)
            \item 自适应阈值
            \item 轻量化网络
            \item 高精度+高效率
        \end{itemize}

        \vspace{0.3cm}

        \textbf{传统二值化 vs DBNet:}

        \vspace{0.2cm}

        \textbf{传统方法:}
        \begin{equation*}
            B_{i,j} = \begin{cases} 1 & \text{if } P_{i,j} \geq t \\ 0 & \text{otherwise} \end{cases}
        \end{equation*}
        其中$t$是固定阈值,\textbf{不可微}。

        \vspace{0.2cm}

        \textbf{DBNet可微二值化:}
        \begin{equation*}
            B_{i,j} = \frac{1}{1 + e^{-k(P_{i,j} - T_{i,j})}}
        \end{equation*}
        其中:
        \begin{itemize}
            \item $P_{i,j}$:概率图(probability map)
            \item $T_{i,j}$:自适应阈值图(threshold map)
            \item $k$:放大因子(默认50)
        \end{itemize}

        \textbf{优势}:可微分,可以端到端训练。

        \column{0.5\textwidth}
        \textbf{DBNet网络结构:}

        \vspace{0.2cm}

        \begin{enumerate}
            \item \textbf{Backbone}:ResNet-18/50 或 MobileNetV3
            \item \textbf{FPN}(特征金字塔网络):
            \begin{itemize}
                \item 融合多尺度特征
                \item 上采样 + 横向连接
            \end{itemize}
            \item \textbf{Head}:
            \begin{itemize}
                \item Probability Map Head:预测文字区域概率
                \item Threshold Map Head:预测自适应阈值
                \item Binary Map Head:二值化结果
            \end{itemize}
        \end{enumerate}

        \vspace{0.3cm}

        \textbf{DBNet损失函数:}
        \begin{equation*}
            L = L_s + \alpha \times L_b + \beta \times L_t
        \end{equation*}

        其中:
        \begin{itemize}
            \item $L_s$:Probability Map的BCE损失
            \item $L_b$:Binary Map的BCE损失
            \item $L_t$:Threshold Map的L1损失
            \item $\alpha$, $\beta$:权重系数
        \end{itemize}

        \vspace{0.3cm}

        \textbf{DBNet优势:}
        \begin{itemize}
            \item 速度快:轻量级 backbone
            \item 精度高:可微二值化
            \item 鲁棒性强:自适应阈值
            \item 易于部署:推理简单
        \end{itemize}
    \end{columns}
\end{frame}
