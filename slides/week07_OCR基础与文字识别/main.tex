%===========================================================
% main.tex - 主控文件
%===========================================================
\documentclass[aspectratio=169, 12pt]{beamer}

% 导入配置文件
%===========================================================
% preamble.tex - Beamer 配置文件
%===========================================================

% 中文支持
\usepackage[UTF8]{ctex}

% 图形与表格
\usepackage{graphicx}
\usepackage{booktabs}

% 颜色与图形(必须在listings之前加载)
\usepackage{xcolor}
\usepackage{tikz}
\usetikzlibrary{shapes, arrows.meta, positioning}

% 数学公式
\usepackage{amsmath}
\usepackage{amssymb}

% 代码高亮
\usepackage{listings}

\lstset{
    language=Python,
    basicstyle=\small\ttfamily,
    keywordstyle=\color{blue},
    commentstyle=\color{green!60!black},
    stringstyle=\color{orange},
    breaklines=true,
    showstringspaces=false,
    keepspaces=true
}

% 超链接
\usepackage{hyperref}

%===========================================================
% 主题设置
%===========================================================
\usetheme{Madrid}
\usecolortheme{whale}
\usefonttheme{professionalfonts}

%===========================================================
% 课程信息
%===========================================================
\title[核心开发与调试]{第10周:核心开发与调试}
\subtitle{让系统真正跑起来}
\author{北京石油化工学院\textbackslash 人工智能研究院\textbackslash 王文通}
\institute{通选课}
\date{2025-2026 学年}

%===========================================================
% 自定义命令
%===========================================================
% 高亮命令
\newcommand{\highlight}[1]{\textcolor{red}{\textbf{#1}}}


% 学校信息(包含 logo 图片)
\institute{%
    \raisebox{-0.5cm}{\includegraphics[height=1.2cm]{../name.png}}\hspace{0.5cm}%
    \raisebox{-0.5cm}{\includegraphics[height=1.2cm]{../xiaohui.png}}\hspace{0.3cm}%
    \begin{minipage}{6cm}
        \centering
        \textbf{北京石油化工学院}\\
        \textit{人工智能研究院}
    \end{minipage}
}

\begin{document}

%===========================================================
% 标题页与目录
%===========================================================
\begin{frame}
    \titlepage
\end{frame}

\section*{课程概览}
\begin{frame}{课程概览}
    \begin{columns}
        \column{0.45\textwidth}
        \textbf{本周内容:}
        \begin{itemize}
            \item OCR技术概述
            \item PaddleOCR快速上手
            \item 试卷文字识别实战
            \item OCR优化技巧
            \item 课堂思考与作业
        \end{itemize}

        \column{0.55\textwidth}
        \textbf{学期项目:AI阅卷助手}
        \begin{enumerate}
            \item 图像采集与预处理
            \item 答题卡定位(Timing Marks)
            \item 填涂检测与识别
            \item 手写文字 OCR
            \item 成绩统计与输出
        \end{enumerate}
    \end{columns}
\end{frame}

\begin{frame}{本周时间分配(160分钟 = 3学时)}
    \begin{columns}
        \column{0.5\textwidth}
        \textbf{第1学时(45分钟):}
        \begin{itemize}
            \item[00:00-00:15] 理论讲解:OCR技术概述(15min)
            \item[00:15-00:35] 实践:PaddleOCR安装与使用(20min)
            \item[00:35-00:45] 讨论与答疑(10min)
        \end{itemize}

        \textbf{第2学时(45分钟):}
        \begin{itemize}
            \item[00:45-01:05] Live Coding:试卷文字识别(20min)
            \item[01:05-01:30] 实践:标题和题号提取(25min)
        \end{itemize}

        \column{0.5\textwidth}
        \textbf{第3学时(70分钟):}
        \begin{itemize}
            \item[01:30-01:50] 理论讲解:OCR优化技巧(20min)
            \item[01:50-02:15] 实践:图像预处理优化(25min)
            \item[02:15-02:35] 互动测验(20min)
            \item[02:35-02:45] 总结与作业(10min)
        \end{itemize}

        \vspace{0.5cm}
        \begin{alertblock}{时间控制提示}
        如果进度落后,建议跳过"挑战任务"
        \end{alertblock}
    \end{columns}
\end{frame}

\begin{frame}{本周分组策略}
    \textbf{分组原则:}
    \begin{itemize}
        \item 每4人为一组
        \item 确保不同专业背景混合
        \item 建议包含:理工科、文科、无编程基础、有编程基础
    \end{itemize}

    \vspace{0.3cm}

    \textbf{角色分工:}
    \begin{table}
        \centering
        \small
        \begin{tabular}{lp{6cm}l}
            \toprule
            \textbf{角色} & \textbf{职责} & \textbf{适合} \\
            \midrule
            组长 & 统筹协调、进度管理 & 组织能力强的 \\
            算法实现者 & 实现OCR代码、图像处理 & 有编程基础的 \\
            参数调优者 & 调整识别参数、优化效果 & 细心负责的 \\
            测试者 & 收集测试用例、报告问题 & 细心负责的 \\
            \bottomrule
        \end{tabular}
    \end{table}

    \vspace{0.3cm}

    \begin{block}{本周协作任务}
        用PaddleOCR识别试卷中的印刷文字,提取标题和题号
    \end{block}
\end{frame}

\begin{frame}{预备知识检查}
    \textbf{在开始学习OCR之前,请确认你已掌握以下基础知识:}

    \vspace{0.3cm}

    \begin{columns}
        \column{0.5\textwidth}
        \textbf{Python基础:}
        \begin{itemize}
            \item 函数定义与调用
            \item 列表和字典操作
            \item 基本文件I/O
        \end{itemize}

        \vspace{0.2cm}

        \textbf{OpenCV基础:}
        \begin{itemize}
            \item 图像读取与显示
            \item 颜色空间转换(BGR→灰度)
            \item 基本图像操作
        \end{itemize}

        \column{0.5\textwidth}
        \textbf{NumPy基础:}
        \begin{itemize}
            \item 数组创建与索引
            \item 基本数学运算
            \item 数组形状操作
        \end{itemize}

        \vspace{0.2cm}

        \textbf{正则表达式基础:}
        \begin{itemize}
            \item 基本匹配模式
            \item 字符类和量词
            \item re模块基本使用
        \end{itemize}
    \end{columns}

    \vspace{0.3cm}

    \begin{alertblock}{发现自己有知识缺口?}
        \begin{itemize}
            \item 课前:花15分钟快速浏览相关教程
            \item 课中:向组内同学请教
            \item 课后:使用AI助手(如Cursor)随时提问
        \end{itemize}
    \end{alertblock}
\end{frame}

\begin{frame}{智能阅卷系统全景图}
    \textbf{本周OCR模块在整个系统中的位置和关联:}

    \vspace{0.3cm}

    \begin{center}
        \begin{tikzpicture}[
            node distance=0.3cm,
            box/.style={rectangle, draw=blue!60, fill=blue!5, rounded corners, minimum width=2.5cm, minimum height=0.8cm, font=\small, align=center},
            current/.style={rectangle, draw=red!60, fill=red!10, rounded corners, minimum width=2.5cm, minimum height=0.8cm, font=\small, align=center, very thick},
            arrow/.style={->, thick}
        ]
            % 行1:输入
            \node[box] (input) {图像采集\\(相机/扫描仪)};

            % 行2:预处理
            \node[box, below=0.5cm of input] (pre) {图像预处理\\(去噪/增强)};

            % 行3:定位
            \node[box, below=0.5cm of pre] (loc) {答题卡定位\\(Timing Marks)};

            % 行4:识别
            \node[box, below left=0.5cm and 0.3cm of loc] (choice) {选择题\\识别};
            \node[box, below=0.5cm of loc] (tf) {判断题\\识别};
            \node[current, below right=0.5cm and 0.3cm of loc] (ocr) {印刷文字\\OCR};

            % 行5:高级
            \node[box, dashed, below=0.5cm of ocr] (hand) {手写文字\\识别(下周)};

            % 行6:输出
            \node[box, below=0.5cm of hand] (stat) {成绩统计\\与输出};

            % 连接箭头
            \draw[arrow] (input) -- (pre);
            \draw[arrow] (pre) -- (loc);
            \draw[arrow] (loc) -- (choice);
            \draw[arrow] (loc) -- (tf);
            \draw[arrow] (loc) -- (ocr);
            \draw[arrow] (choice) |- (stat);
            \draw[arrow] (tf) -- (stat);
            \draw[arrow] (ocr) -- (hand);
            \draw[arrow] (hand) -- (stat);
        \end{tikzpicture}
    \end{center}

    \vspace{0.2cm}

    \begin{block}{本周任务的核心价值}
        印刷文字OCR是手写识别的\textbf{前置基础}——只有先准确识别印刷的"标准文字",才能为识别手写"非标准文字"奠定算法基础和数据对比基准。
    \end{block}
\end{frame}

\begin{frame}{预备知识(课前5分钟视频)}
    \textbf{必须观看的视频内容:}
    \begin{columns}
        \column{0.5\textwidth}
        \textbf{视频1:OCR基本流程(2分钟)}
        \begin{enumerate}
            \item 输入:图像采集
            \item 检测:文字区域定位
            \item 识别:字符分类
            \item 输出:文本结果
        \end{enumerate}

        \vspace{0.3cm}

        \textbf{视频2:深度学习入门(3分钟)}
        \begin{itemize}
            \item 神经网络:模拟人脑神经元
            \item 卷积运算:提取图像特征
            \item 训练与推理:模型学习与使用
        \end{itemize}

        \column{0.5\textwidth}
        \textbf{关键概念速记:}
        \begin{description}
            \item[CNN] 卷积神经网络,用于提取图像特征
            \item[RNN] 循环神经网络,处理序列数据
            \item[CTC] 连接时序分类,解决序列对齐问题
            \item[Transformer] 自注意力架构,当前最先进方法
        \end{description}
    \end{columns}

    \vspace{0.5cm}

    \begin{alertblock}{课前测试}
        观看视频后,回答:
        \begin{itemize}
            \item OCR的四个主要步骤是什么?
            \item CNN和RNN分别用于什么任务?
        \end{itemize}
    \end{alertblock}
\end{frame}

\begin{frame}{多屏协同设计}
    \textbf{本课程采用多屏协同教学方式:}

    \vspace{0.3cm}

    \begin{columns}
        \column{0.5\textwidth}
        \textbf{主屏(左侧):理论讲解}
        \begin{itemize}
            \item PPT幻灯片
            \item 概念和原理讲解
            \item 图像和代码展示
            \item 互动测验
        \end{itemize}

        \column{0.5\textwidth}
        \textbf{侧屏(右侧):实时演示}
        \begin{itemize}
            \item PaddleOCR代码实时演示
            \item 文字识别效果展示
            \item 参数调整实时反馈
            \item 调试过程展示
        \end{itemize}
    \end{columns}

    \vspace{0.5cm}

    \begin{block}{移动设备互动}
        使用手机参与互动测验(问卷星)
    \end{block}
\end{frame}

%===========================================================
% 教学模块
%===========================================================
%===========================================================
% modules/01_intro.tex - 课程介绍与导入
%===========================================================

\section{课程导入}

\begin{frame}{从上周说起}
    \textbf{回顾第6周:判断题识别与符号匹配}
    \begin{itemize}
        \item 学会了检测判断题的对勾和叉号
        \item 了解了符号匹配和字符识别的区别
    \end{itemize}

    \vspace{0.5cm}

    \textbf{本周新问题:}
    \begin{alertblock}{挑战}
        试卷上还有大量印刷文字:标题、题号、选项文字……

        \textbf{怎么让机器"读懂"这些文字?}
    \end{alertblock}
\end{frame}

\begin{frame}{本周故事线}
    \begin{center}
        \begin{tikzpicture}[
            node distance=1.2cm,
            box/.style={rectangle, draw=blue!60, fill=blue!10, rounded corners, minimum width=3.5cm, minimum height=0.8cm, text centered, font=\small}
        ]
            \node[box] (n1) {1. OCR是什么?};
            \node[box, below=1.2cm of n1] (n2) {2. PaddleOCR安装与使用};
            \node[box, below=1.2cm of n2] (n3) {3. 识别试卷文字};
            \node[box, below=1.2cm of n3] (n4) {4. 优化识别效果};
            \node[box, below=1.2cm of n4] (n5) {5. 实战作业};

            \draw[->, thick, blue!60] (n1) -- (n2);
            \draw[->, thick, blue!60] (n2) -- (n3);
            \draw[->, thick, blue!60] (n3) -- (n4);
            \draw[->, thick, blue!60] (n4) -- (n5);
        \end{tikzpicture}
    \end{center}
\end{frame}

\begin{frame}{学习目标}
    \begin{columns}
        \column{0.5\textwidth}
        \textbf{知识目标:}
        \begin{itemize}
            \item 理解OCR技术原理
            \item 了解OCR发展历程
            \item 掌握PaddleOCR基本使用
        \end{itemize}

        \vspace{0.3cm}

        \textbf{技能目标:}
        \begin{itemize}
            \item 能安装配置PaddleOCR
            \item 能识别图像中的文字
            \item 能提取标题和题号
        \end{itemize}

        \column{0.5\textwidth}
        \textbf{素质目标:}
        \begin{itemize}
            \item 培养跨学科学习能力
            \item 锻炼动手实践能力
            \item 提升团队协作意识
        \end{itemize}

        \vspace{0.5cm}

        \begin{block}{预期成果}
            完成一个能识别试卷印刷文字的程序
        \end{block}
    \end{columns}
\end{frame}

%===========================================================
% modules/02_ocr_overview.tex - OCR技术概述
%===========================================================

\section{OCR技术概述}

\begin{frame}{什么是OCR?}
    \begin{block}{OCR:Optical Character Recognition}
        光学字符识别 —— 将图像中的文字转换为计算机可编辑的文本
    \end{block}

    \vspace{0.3cm}

    \textbf{应用场景:}
    \begin{itemize}
        \item 文档数字化
        \item 车牌识别
        \item 证件识别
        \item \textbf{试卷文字识别}
    \end{itemize}
\end{frame}

\begin{frame}{OCR技术发展}
    \begin{table}
        \centering
        \begin{tabular}{lp{6cm}l}
            \toprule
            \textbf{阶段} & \textbf{技术} & \textbf{准确率} \\
            \midrule
            传统方法 & 模板匹配、特征提取 & 60-70\% \\
            机器学习 & SVM、HMM & 80-85\% \\
            深度学习 & CNN、RNN & 90-92\% \\
            Transformer & TrOCR、Donut & 95\%+ \\
            \bottomrule
        \end{tabular}
    \end{table}

    \textbf{当前:} 端到端大模型时代
\end{frame}

\begin{frame}{主流OCR工具对比}
    \begin{table}
        \centering
        \small
        \begin{tabular}{lp{4cm}lp{3cm}}
            \toprule
            \textbf{工具} & \textbf{优点} & \textbf{缺点} & \textbf{适用} \\
            \midrule
            Tesseract & 开源免费 & 中文准确率一般 & 英文文档 \\
            PaddleOCR & 中文准确率高 & 需要下载模型 & \textbf{中文场景} \\
            EasyOCR & 多语言支持 & 速度较慢 & 多语言混合 \\
            TrOCR & 准确率最高 & 资源占用大 & 高精度场景 \\
            \bottomrule
        \end{tabular}
    \end{table}

    \textbf{本课程推荐:} PaddleOCR(中文友好、易用)
\end{frame}

%===========================================================
% modules/03_paddleocr.tex - PaddleOCR快速上手
%===========================================================

\section{PaddleOCR快速上手}

\begin{frame}[fragile]{PaddleOCR安装(70\%脚手架)}
    \textbf{任务:}使用AI助手补全PaddleOCR安装命令

    \textbf{提示词(Prompts):}
    \begin{itemize}
        \item "How to install PaddlePaddle via pip"
        \item "How to install PaddleOCR"
        \item "pip install opencv-python and pillow"
    \end{itemize}

    \begin{lstlisting}
# TODO: Install PaddlePaddle
pip install _____________

# TODO: Install PaddleOCR
pip install _____________

# TODO: Install image processing dependencies
pip install _____________
    \end{lstlisting}

    \begin{alertblock}{注意}
        \begin{itemize}
            \item 首次使用会自动下载模型
            \item 需要联网
            \item 首次运行较慢
        \end{itemize}
    \end{alertblock}
\end{frame}

\begin{frame}[fragile]{基础使用(70\%脚手架)}
    \begin{lstlisting}[basicstyle=\ttfamily\scriptsize]
from paddleocr import PaddleOCR

# TODO: 使用AI助手(如Cursor的Ctrl+K)补全初始化参数
# 提示词:"配置PaddleOCR,需要支持中文识别和方向分类,用于试卷文字识别"
ocr = PaddleOCR(
    # TODO: 语言设置(提示词:"PaddleOCR lang参数说明")
    lang='____',  # 如 'ch', 'en', 'ch_en'

    # TODO: 启用方向分类(提示词:"PaddleOCR use_angle_cls参数说明")
    use_angle_cls=____  # True/False
)

# TODO: 识别图像(提示词:"PaddleOCR ocr方法参数说明")
result = ocr.ocr('____', cls=____)

# TODO: 解析识别结果(提示词:"PaddleOCR结果格式说明,如何提取文字和置信度")
for line in result[0]:
    text = line[____][____]      # 提取文字内容
    confidence = line[____][____]  # 提取置信度
    print(f"{text} ({confidence:.4f})")
    \end{lstlisting}

    \begin{alertblock}{AI辅助编程提示}
        使用Cursor/Claude Code时,可以这样Prompt:
        \begin{itemize}
            \item "PaddleOCR初始化需要哪些参数?用于中文试卷识别。"
            \item "ocr.ocr()方法的返回值是什么格式?"
            \item "如何从PaddleOCR结果中提取文字和置信度?"
        \end{itemize}
    \end{alertblock}
\end{frame}

\begin{frame}{结果格式}
    \textbf{返回格式:}
    \begin{verbatim}
[
  [
    [[x1,y1], [x2,y2], [x3,y3], [x4,y4]],  # 文字框坐标
    ('文字内容', 0.98)                        # 文字+置信度
  ],
  ...
]
    \end{verbatim}

    \vspace{0.2cm}

    \textbf{字段说明:}
    \begin{itemize}
        \item 坐标:文字框的四个角点
        \item 文字内容:识别出的文本
        \item 置信度:0-1之间,越接近1越准确
    \end{itemize}
\end{frame}

%=============================================================================
% 模块四:核心算法与实战演练
%=============================================================================

\section{图像滤镜原理}

\begin{frame}[fragile]{滤镜 1:灰度化 (Grayscale)}
	\textbf{为什么要灰度化?}
	\begin{itemize}
		\item 减少计算量(数据量降至 1/3)
		\item 识别试卷上的文字,颜色信息通常是不必要的
	\end{itemize}
	\textbf{原理:} $Gray = R \times 0.299 + G \times 0.587 + B \times 0.114$
	\\ (为什么绿色权重最高?因为人眼对绿色最敏感。)

	\begin{lstlisting}[language=Python, basicstyle=\ttfamily\small]
# 方法1:OpenCV 函数
gray = cv2.cvtColor(img, cv2.COLOR_BGR2GRAY)

# 方法2:手动计算(不推荐)
gray = 0.299 * r + 0.587 * g + 0.114 * b
\end{lstlisting}
\end{frame}

\begin{frame}[fragile]{滤镜 2:反色 (Inversion)}
	\textbf{原理:} $NewValue = 255 - OldValue$
	\begin{itemize}
		\item 黑色 (0) $\to$ 白色 (255)
		\item 白色 (255) $\to$ 黑色 (0)
	\end{itemize}
	\textbf{应用:} 增强暗背景下的试卷特征,或者扫描负片。

	\begin{lstlisting}[language=Python, basicstyle=\ttfamily\small]
# 方法1:NumPy 运算
inverted = 255 - img

# 方法2:OpenCV 位运算
inverted = cv2.bitwise_not(img)

# 方法3:NumPy 按位取反
inverted = np.bitwise_not(img)
\end{lstlisting}
\end{frame}

\begin{frame}[fragile]{滤镜 3:亮度调整与"溢出"陷阱}
	\textbf{错误做法:} \texttt{img + 50}
	\\ 如果像素值是 220,加 50 变成 270。而在 \texttt{uint8} 类型下,270 会变成 \highlight{14} (截断/绕回),导致图像出现难看的噪点。

	\begin{lstlisting}[title={安全写法}]
# 使用 numpy 的 clip 函数限制范围
bright_img = np.clip(img.astype(np.int32) + 50, 0, 255).astype(np.uint8)

# 或者使用 OpenCV 内置函数(推荐,速度更快)
bright_img = cv2.add(img, np.array([50.0]))
\end{lstlisting}
\end{frame}

% -----------------------------------------------------------------------------
% 代码实战环节
% -----------------------------------------------------------------------------

\section{代码实战}

\begin{frame}[fragile]{代码实战(1/5):图像翻转与旋转}
	\textbf{场景:}阅卷时试卷可能被倒置,需要自动旋转

	\begin{columns}
		\column{0.5\textwidth}
		\begin{lstlisting}[language=Python, basicstyle=\ttfamily\tiny]
import cv2
import numpy as np

img = cv2.imread('exam.jpg')

# 方法1:NumPy 数组切片
# 垂直翻转(上下颠倒)
flip_v = img[::-1, :, :]

# 水平翻转(左右颠倒)
flip_h = img[:, ::-1, :]

# 水平+垂直翻转(旋转180度)
flip_both = img[::-1, ::-1, :]
\end{lstlisting}

		\column{0.5\textwidth}
		\begin{lstlisting}[language=Python, basicstyle=\ttfamily\tiny]
# 方法2:OpenCV 函数(推荐)
flip_v = cv2.flip(img, 0)      # 垂直
flip_h = cv2.flip(img, 1)      # 水平
flip_both = cv2.flip(img, -1)  # 两者

# 显示对比
cv2.imshow('Original', img)
cv2.imshow('Flip V', flip_v)
cv2.imshow('Flip H', flip_h)
cv2.waitKey(0)
\end{lstlisting}
	\end{columns}

	\vspace{0.2cm}
	\textbf{性能对比:} NumPy 切片比 cv2.flip 快约 20\%,但 cv2.flip 更易读
\end{frame}

\begin{frame}[fragile]{代码实战(2/5):提取答题卡区域(ROI)}
	\textbf{场景:}从整张试卷中提取答题卡区域

	\begin{columns}
		\column{0.5\textwidth}
		\begin{lstlisting}[language=Python, basicstyle=\ttfamily\tiny]
import cv2
import numpy as np

exam = cv2.imread('exam.jpg')
h, w = exam.shape[:2]

# 假设答题卡在右下角
# 坐标:从宽度的60%到末尾,高度的50%到末尾
x1, x2 = int(w * 0.6), w
y1, y2 = int(h * 0.5), h

# 提取 ROI
roi = exam[y1:y2, x1:x2]

# 保存 ROI
cv2.imwrite('answer_sheet.jpg', roi)

print("原图大小:", exam.shape)
print("ROI 大小:", roi.shape)
\end{lstlisting}

		\column{0.5\textwidth}
		\textbf{坐标系统回顾:}
		\begin{itemize}
			\item 原点在左上角 (0, 0)
			\item \texttt{img[y1:y2, x1:x2]}
			\item y 是行(高度),x 是列(宽度)
		\end{itemize}

		\vspace{0.3cm}
		\begin{center}
			\begin{tikzpicture}[scale=0.4]
				\draw[thick, fill=blue!10] (0,0) rectangle (6,4);
				\node at (3,2) {整张试卷};
				\draw[thick, fill=red!30] (3.5,0) rectangle (6,2);
				\node at (4.75,1) {\tiny ROI};
				\draw[->] (3.5,2) -- (3.5,3) node[above] {\tiny y1};
				\draw[->] (3.5,0) -- (3.5,-1) node[below] {\tiny y2};
				\draw[->] (3.5,1) -- (2.5,1) node[left] {\tiny x1};
				\draw[->] (6,1) -- (7,1) node[right] {\tiny x2};
			\end{tikzpicture}
		\end{center}
	\end{columns}
\end{frame}

\begin{frame}[fragile]{代码实战(3/5):通道分离与合并}
	\textbf{场景:}提取特定颜色通道

	\begin{columns}
		\column{0.5\textwidth}
		\begin{lstlisting}[language=Python, basicstyle=\ttfamily\tiny]
import cv2

img = cv2.imread('exam.jpg')

# 方法1:使用 split 函数
b, g, r = cv2.split(img)

# 只保留红色通道,其他设为0
zeros = np.zeros_like(b)
img_r = cv2.merge([zeros, zeros, r])
\end{lstlisting}

		\vspace{0.2cm}
		\begin{lstlisting}[language=Python, basicstyle=\ttfamily\tiny]
# 方法2:直接索引(更快)
img_r = img.copy()
img_r[:, :, 0] = 0  # B通道
img_r[:, :, 1] = 0  # G通道
# R通道保持不变

cv2.imshow('Red Only', img_r)
cv2.waitKey(0)
\end{lstlisting}

		\column{0.5\textwidth}
		\textbf{通道顺序:}
		\begin{itemize}
			\item OpenCV: \textbf{BGR}
			\item matplotlib: \textbf{RGB}
			\item PIL: \textbf{RGB}
		\end{itemize}

		\vspace{0.3cm}
		\begin{alertblock}{常见错误}
			使用 \texttt{plt.imshow(img)} 显示 OpenCV 图像时,颜色会异常!
		\end{alertblock}

		\vspace{0.2cm}
		\textbf{解决方案:}
		\begin{lstlisting}[language=Python, basicstyle=\ttfamily\tiny]
img_rgb = cv2.cvtColor(img, cv2.COLOR_BGR2RGB)
plt.imshow(img_rgb)
\end{lstlisting}
	\end{columns}
\end{frame}

\begin{frame}[fragile]{代码实战(4/5):阅卷系统核心代码}
	\textbf{场景:}检测答题卡填涂位置

	\begin{columns}
		\column{0.5\textwidth}
		\begin{lstlisting}[language=Python, basicstyle=\ttfamily\tiny]
import cv2
import numpy as np

# 1. 读取答题卡区域
roi = cv2.imread('answer_sheet.jpg',
                 cv2.IMREAD_GRAYSCALE)

# 2. 二值化
_, binary = cv2.threshold(roi, 127, 255,
                          cv2.THRESH_BINARY)

# 3. 定义选项位置
positions = [
    (100, 100, 120, 120),  # A
    (100, 130, 120, 150),  # B
    (100, 160, 120, 180),  # C
    (100, 190, 120, 200)   # D
]
\end{lstlisting}

		\column{0.5\textwidth}
		\begin{lstlisting}[language=Python, basicstyle=\ttfamily\tiny]
# 4. 检测每个选项是否被填涂
answers = []
for (x1, y1, x2, y2) in positions:
    option = binary[y1:y2, x1:x2]

    # 计算黑色像素比例
    black_pixels = np.sum(option == 0)
    total_pixels = option.size
    ratio = black_pixels / total_pixels

    # 判断是否填涂(阈值30%)
    if ratio > 0.3:
        answers.append('填涂')
    else:
        answers.append('未填')

print(answers)
\end{lstlisting}
	\end{columns}

	\vspace{0.2cm}
	\textbf{核心思想:}填涂区域黑色像素占比显著高于未填涂区域
\end{frame}

\begin{frame}[fragile]{代码实战(5/5):图像增强对比}
	\textbf{场景:}答题卡光照不均,需要增强对比度

	\begin{columns}
		\column{0.5\textwidth}
		\begin{lstlisting}[language=Python, basicstyle=\ttfamily\tiny]
import cv2
import numpy as np

img = cv2.imread('exam.jpg')

# 方法1:线性对比度调整
# new = alpha * old + beta
enhanced = cv2.convertScaleAbs(
    img, alpha=1.5, beta=30
)
\end{lstlisting}

		\vspace{0.2cm}
		\begin{lstlisting}[language=Python, basicstyle=\ttfamily\tiny]
# 方法2:直方图均衡化
gray = cv2.cvtColor(img, cv2.COLOR_BGR2GRAY)
equalized = cv2.equalizeHist(gray)
\end{lstlisting}

		\column{0.5\textwidth}
		\begin{lstlisting}[language=Python, basicstyle=\ttfamily\tiny]
# 方法3:CLAHE(自适应)
clahe = cv2.createCLAHE(
    clipLimit=2.0,
    tileGridSize=(8,8)
)
enhanced_clahe = clahe.apply(gray)
\end{lstlisting}

		\vspace{0.2cm}
		\textbf{效果对比:}
		\begin{itemize}
			\item \textbf{线性调整}:简单但效果有限
			\item \textbf{直方图均衡化}:全局优化
			\item \textbf{CLAHE}:局部自适应,效果最好
		\end{itemize}

		\vspace{0.2cm}
		\textbf{阅卷推荐:} CLAHE 适合光照不均场景
	\end{columns}
\end{frame}

% -----------------------------------------------------------------------------
% 完整阅卷系统Live Coding
% -----------------------------------------------------------------------------

\begin{frame}[fragile]{Live Coding:完整的阅卷预处理流程}
	\textbf{目标:} 从照片到可识别的图像

	\begin{columns}
		\column{0.5\textwidth}
		\begin{lstlisting}[language=Python, basicstyle=\ttfamily\tiny]
def preprocess_exam(image_path):
    """试卷预处理完整流程"""

    # 1. 读取图像(支持中文路径)
    img = imread_chinese(image_path)

    # 2. 转为灰度
    gray = cv2.cvtColor(img,
                       cv2.COLOR_BGR2GRAY)

    # 3. 去噪
    denoised = cv2.GaussianBlur(
        gray, (5, 5), 0)

    # 4. 对比度增强(CLAHE)
    clahe = cv2.createCLAHE(2.0, (8, 8))
    enhanced = clahe.apply(denoised)

    # 5. 二值化
    binary = cv2.adaptiveThreshold(
        enhanced, 255,
        cv2.ADAPTIVE_THRESH_GAUSSIAN_C,
        cv2.THRESH_BINARY, 11, 2)

    return img, gray, enhanced, binary

# 使用
img, gray, enhanced, binary = \
    preprocess_exam('exam.jpg')
		\end{lstlisting}

		\column{0.5\textwidth}
		\textbf{流程图:}
		\begin{center}
			\begin{tikzpicture}[scale=0.6, node distance=0.8cm]
				\node[draw, rounded corners] (1) {原图};
				\node[draw, rounded corners, below of=1] (2) {灰度};
				\node[draw, rounded corners, below of=2] (3) {去噪};
				\node[draw, rounded corners, below of=3] (4) {增强};
				\node[draw, rounded corners, below of=4] (5) {二值};

				\draw[->] (1) -- (2);
				\draw[->] (2) -- (3);
				\draw[->] (3) -- (4);
				\draw[->] (4) -- (5);
			\end{tikzpicture}
		\end{center}

		\vspace{0.2cm}
		\textbf{展示结果:}
		\begin{itemize}
			\item 原始照片
			\item 预处理后图像
			\item 处理时间对比
		\end{itemize}
	\end{columns}
\end{frame}

\begin{frame}[fragile]{Live Coding:阅卷系统核心检测}
	\textbf{功能1:填涂检测}
	\begin{lstlisting}[language=Python, basicstyle=\ttfamily\tiny]
def detect_bubble(binary, position):
    """检测单个气泡的填涂状态"""
    x1, y1, x2, y2 = position

    # 提取气泡区域
    bubble = binary[y1:y2, x1:x2]

    # 计算填涂密度
    black_pixels = np.sum(bubble == 0)
    total_pixels = bubble.size
    fill_ratio = black_pixels / total_pixels

    # 判断状态
    if fill_ratio > 0.6:
        return 'filled'
    elif fill_ratio < 0.2:
        return 'empty'
    else:
        return 'uncertain'
	\end{lstlisting}

	\vspace{0.2cm}
	\textbf{功能2:多选检测与警告}
	\begin{lstlisting}[language=Python, basicstyle=\ttfamily\tiny]
def detect_multiple_choice(binary, positions):
    """检测多选并警告"""
    results = []
    for pos in positions:
        state = detect_bubble(binary, pos)
        results.append(state)

    # 统计填涂数量
    filled_count = sum(1 for r in results if r == 'filled')

    if filled_count > 1:
        print(f"警告:检测到多选({filled_count}个选项)")

    return results
	\end{lstlisting}
\end{frame}

\begin{frame}[fragile]{Live Coding:图像质量检测函数}
	\textbf{目标:} 自动判断试卷照片是否适合识别

	\begin{columns}
		\column{0.5\textwidth}
		\begin{lstlisting}[language=Python, basicstyle=\ttfamily\tiny]
def check_image_quality(img):
    """检测图像质量"""

    h, w = img.shape[:2]

    # 1. 分辨率检查
    if min(h, w) < 1000:
        return False, "分辨率过低"

    # 2. 曝光检查
    gray = cv2.cvtColor(img,
                       cv2.COLOR_BGR2GRAY)
    mean_brightness = np.mean(gray)

    if mean_brightness < 80:
        return False, "曝光不足"
    elif mean_brightness > 200:
        return False, "过曝"

    # 3. 清晰度检查
    laplacian_var = cv2.Laplacian(
        gray, cv2.CV_64F
    ).var()

    if laplacian_var < 100:
        return False, "图像模糊"

    return True, "质量合格"
		\end{lstlisting}

		\column{0.5\textwidth}
		\textbf{使用示例:}
		\begin{lstlisting}[language=Python, basicstyle=\ttfamily\tiny]
img = imread_chinese('exam.jpg')

is_good, msg = check_image_quality(img)

if is_good:
    print(f"图像质量:{msg}")
    # 继续处理
    result = process_image(img)
else:
    print(f"图像质量:{msg}")
    print("提示用户重新拍照")
		\end{lstlisting}

		\vspace{0.2cm}
		\textbf{质量标准:}
		\begin{itemize}
			\item 分辨率:≥1000px
			\item 曝光:80-200
			\item 清晰度:Laplacian方差 ≥100
		\end{itemize}
	\end{columns}
\end{frame}

\begin{frame}[fragile]{Live Coding:批量处理与结果输出}
	\textbf{批量处理函数:}
	\begin{lstlisting}[language=Python, basicstyle=\ttfamily\tiny]
import os
import json

def batch_process_exams(folder_path, output_path):
    """批量处理试卷"""
    results = []

    for filename in os.listdir(folder_path):
        if not filename.endswith(('.jpg', '.png')):
            continue

        input_path = os.path.join(folder_path, filename)

        # 1. 质量检查
        is_good, msg = check_image_quality(
            imread_chinese(input_path))
        if not is_good:
            print(f"X {filename}: {msg}")
            continue

        # 2. 预处理
        img, gray, enhanced, binary = \
            preprocess_exam(input_path)

        # 3. 检测答题
        answers = detect_all_answers(binary)

        # 4. 评分
        score, details = grade_answers(answers)

        # 5. 保存结果
        result = {
            'filename': filename,
            'score': score,
            'details': details,
            'quality': is_good
        }
        results.append(result)

        print(f"OK {filename}: {score}分")

    # 保存到JSON
    with open(output_path, 'w', encoding='utf-8') as f:
        json.dump(results, f, ensure_ascii=False, indent=2)

    return results
	\end{lstlisting}
\end{frame}

%===========================================================
% modules/05_optimization.tex - OCR优化技巧
%===========================================================

\section{OCR优化}

\begin{frame}[fragile]{图像预处理优化}
    \textbf{专门的OCR预处理(70\%脚手架):}
    \begin{lstlisting}
def preprocess_for_ocr(image):
    """OCR优化预处理"""
    # TODO: 使用AI助手补全代码
    # 提示词:"如何将BGR图像转为灰度图用于OCR预处理"

    # TODO: 转灰度
    gray = cv2.cvtColor(____, ____)

    # TODO: 去噪(提示词:"OpenCV非局部均值去噪方法")
    denoised = cv2.____(gray)

    # TODO: 二值化(提示词:"OTSU自动阈值二值化")
    _, binary = cv2.threshold(
        ____, 0, 255,
        ____ + ____(____)
    )

    return binary
    \end{lstlisting}

    \begin{alertblock}{AI辅助编程提示}
        不知道cv2函数的参数?\textbf{选中函数名,按Ctrl+K问AI:}
        "cv2.fastNlMeansDenoising的参数是什么?"
    \end{alertblock}
\end{frame}

\begin{frame}[fragile]{识别参数优化(70\%脚手架)}
    \textbf{PaddleOCR初始化参数:}
    \begin{lstlisting}
# TODO: 使用AI助手配置PaddleOCR参数
# 提示词:"配置PaddleOCR参数,需要:1)中文识别 2)方向分类
# 3)检测阈值0.3 4)置信度阈值0.5"

ocr = PaddleOCR(
    # TODO: 语言设置(提示词:"PaddleOCR中文语言代码")
    lang=____,

    # TODO: 方向分类(提示词:"PaddleOCR启用方向分类参数")
    use_angle_cls=____,

    # TODO: 检测阈值(默认0.3,越小检测越严格)
    det_db_thresh=____,
    det_db_box_thresh=____,

    # TODO: 识别置信度阈值(默认0.5)
    rec_score_thres=____,

    # 性能参数(可选)
    use_gpu=____,        # 是否使用GPU
    show_log=____        # 是否显示日志
)
    \end{lstlisting}

    \begin{alertblock}{AI辅助配置提示}
        不知道参数值?\textbf{选中PaddleOCR,按Ctrl+K问AI:}
        "PaddleOCR的det_db_thresh参数是什么意思,推荐值是多少?"
    \end{alertblock}
\end{frame}

\begin{frame}{常见优化技巧}
    \begin{columns}
        \column{0.5\textwidth}
        \textbf{1. 图像层面}
        \begin{itemize}
            \item 提高扫描分辨率(300dpi以上)
            \item 校正图像倾斜角度
            \item 去除噪点和干扰线
            \item 适当增强对比度
        \end{itemize}

        \vspace{0.3cm}

        \textbf{2. 预处理层面}
        \begin{itemize}
            \item 先进行版面分析
            \item 按区域分别识别
            \item 文字区域精确定位
        \end{itemize}

        \column{0.5\textwidth}
        \textbf{3. 后处理层面}
        \begin{itemize}
            \item 根据置信度过滤
            \item 利用上下文纠错
            \item 正则表达式匹配
            \item 词典校验
        \end{itemize}

        \vspace{0.3cm}

        \begin{block}{核心原则}
            OCR是系统工程,\textbf{预处理 > 算法 > 后处理}
        \end{block}
    \end{columns}
\end{frame}

%=============================================================================
% 模块六:互动测验与Live Coding
%=============================================================================

\section{互动测验}

\begin{frame}[fragile]{小测验时间(1):NumPy 综合测试}
	\begin{block}{问题}
		给定形状为 (100, 100, 3) 的彩色图像,如何提取中心 50×50 的红色通道值?
		\begin{enumerate}[(A)]
			\item \texttt{img[25:75, 25:75, 0]}
			\item \texttt{img[25:75, 25:75, 2]}
			\item \texttt{img[50:100, 50:100, 2]}
			\item \texttt{img[25:75, 25:75]}
		\end{enumerate}
	\end{block}
	\pause
	\textbf{答案:} \highlight{B. img[25:75, 25:75, 2]} \\
	\vspace{0.2cm}
	\textbf{解析:}
	\begin{itemize}
		\item 100×100 图像的中心 50×50:索引从 25 到 75
		\item OpenCV 中 BGR 顺序,红色通道索引为 2
		\item \texttt{img[y1:y2, x1:x2, channel]} 格式
	\end{itemize}
\end{frame}

% -----------------------------------------------------------------------------
% Live Coding 闪电编程任务
% -----------------------------------------------------------------------------

\begin{frame}[fragile]{Live Coding(1/3):5分钟闪电编程任务}
	\begin{alertblock}{编程挑战}
		给定一张"脏"试卷图像(含噪声、光照不均)\\
		\textbf{任务:}在 5 分钟内,用 \highlight{3 行代码} 提取出学号区的均值
	\end{alertblock}

	\begin{columns}
		\column{0.5\textwidth}
		\textbf{提示:}
		\begin{enumerate}
			\item 读取图像(已有)
			\item 切片学号区
			\item 计算均值
		\end{enumerate}

		\vspace{0.3cm}
		\textbf{学号区位置:}
		\begin{itemize}
			\item 假设在左上角
			\item 坐标范围:\texttt{[50:150, 100:300]}
			\item 高度:100px,宽度:200px
		\end{itemize}

		\column{0.5\textwidth}
		\begin{lstlisting}[language=Python, basicstyle=\ttfamily\tiny]
# 给定代码(不要修改)
import cv2
import numpy as np

img = cv2.imread('dirty_exam.jpg')

# ===== 你的任务:补全这3行 =====
# 第1行:灰度化
gray = ???

# 第2行:切片学号区
id_region = ???

# 第3行:计算均值
mean_value = ???

# =============================

print(f"学号区均值: {mean_value}")
\end{lstlisting}
	\end{columns}
\end{frame}

\begin{frame}[fragile]{Live Coding(2/3):参考答案与解析}
	\textbf{参考答案:}

	\begin{lstlisting}[language=Python, basicstyle=\ttfamily\tiny]
import cv2
import numpy as np

img = cv2.imread('dirty_exam.jpg')

# 第1行:灰度化
gray = cv2.cvtColor(img, cv2.COLOR_BGR2GRAY)

# 第2行:切片学号区
id_region = gray[50:150, 100:300]

# 第3行:计算均值
mean_value = np.mean(id_region)

print(f"学号区均值: {mean_value:.2f}")
\end{lstlisting}

	\vspace{0.3cm}
	\textbf{代码解析:}
	\begin{itemize}
		\item \textbf{灰度化}:\texttt{cv2.cvtColor(img, cv2.COLOR\_BGR2GRAY)} 将三维彩色图转为二维灰度图
		\item \textbf{切片}:\texttt{gray[y1:y2, x1:x2]},y在前,x在后,提取高度方向 [50:150],宽度方向 [100:300]
		\item \textbf{均值}:\texttt{np.mean(array)} 返回所有像素的平均亮度,可用于判断学号区是否填涂
	\end{itemize}
\end{frame}

\begin{frame}[fragile]{Live Coding(3/3):即时反馈与扩展}
	\textbf{课堂互动:}分享你的切片坐标

	\begin{columns}
		\column{0.5\textwidth}
		\textbf{学生分享环节:}
		\begin{itemize}
			\item 你用的是哪个坐标范围?
			\item \texttt{img[50:150, 100:300]}?
			\item 还是 \texttt{img[0:100, 0:200]}?
			\item 坐标不同,结果如何?
		\end{itemize}

		\vspace{0.3cm}
		\textbf{观察要点:}
		\begin{enumerate}
			\item 不同坐标范围,均值会不同
			\item 越亮的区域,均值越大
			\item 越暗的区域,均值越小
		\end{enumerate}

		\column{0.5\textwidth}
		\textbf{进阶挑战(可选):}
		\begin{lstlisting}[language=Python, basicstyle=\ttfamily\tiny]
# 挑战1:计算标准差
std_value = np.std(id_region)
print(f"标准差: {std_value:.2f}")

# 挑战2:判断是否填涂
if mean_value < 100:
    print("学号区可能已填涂")
else:
    print("学号区未填涂")

# 挑战3:可视化切片
cv2.imshow('学号区', id_region)
cv2.waitKey(0)
\end{lstlisting}

		\vspace{0.2cm}
		\textbf{阅卷应用:}
		\begin{itemize}
			\item 均值:判断填涂密度
			\item 标准差:判断填涂均匀性
			\item 阈值:自动识别填涂状态
		\end{itemize}
	\end{columns}
\end{frame}

\begin{frame}[fragile]{小测验时间(5):综合应用}
	\begin{block}{问题}
		在阅卷系统中,要将答题卡的填涂区域(黑色)从白色纸张中分离出来,应该使用哪种阈值类型?
		\begin{enumerate}[(A)]
			\item \texttt{cv2.THRESH\_BINARY}
			\item \texttt{cv2.THRESH\_BINARY\_INV}
			\item \texttt{cv2.THRESH\_TRUNC}
			\item \texttt{cv2.THRESH\_TOZERO}
		\end{enumerate}
	\end{block}
	\pause
	\textbf{答案:} \highlight{A. cv2.THRESH\_BINARY} \\
	\vspace{0.2cm}
	\textbf{解析:}
	\begin{itemize}
		\item 填涂区域是黑色(低像素值)
		\item 纸张是白色(高像素值)
		\item \texttt{THRESH\_BINARY} 会将低于阈值的设为 0(黑),高于阈值的设为 255(白)
		\item 正好分离填涂和纸张
	\end{itemize}
\end{frame}

%===========================================================
% modules/07_debug_helper.tex - AI调试助手
%===========================================================

\section{AI调试助手}

\begin{frame}{遇到问题?使用AI调试助手}
    \textbf{常见错误与AI调试策略:}
    \begin{itemize}
        \item \textbf{安装失败}:
        \begin{itemize}
            \item 提示AI:"paddlepaddle安装报错:[粘贴错误信息],提供替代方案"
            \item AI会建议:使用清华镜像源、检查Python版本等
        \end{itemize}

        \item \textbf{识别为空}:
        \begin{itemize}
            \item 提示AI:"PaddleOCR返回空结果,检查图像路径[路径],代码如下:[代码]"
            \item AI会检查:路径格式、图像是否存在、参数配置
        \end{itemize}

        \item \textbf{中文乱码}:
        \begin{itemize}
            \item 提示AI:"PaddleOCR中文识别乱码,代码如下:[代码]"
            \item AI会建议:确认lang='ch'、检查编码、版本兼容性
        \end{itemize}

        \item \textbf{置信度太低}:
        \begin{itemize}
            \item 提示AI:"OCR识别置信度太低(平均0.3),如何优化?"
            \item AI会建议:图像预处理、参数调优、分辨率提升
        \end{itemize}
    \end{itemize}
\end{frame}

\begin{frame}{AI辅助调试三部曲}
    \begin{alertblock}{AI辅助调试三部曲}
        \begin{enumerate}
            \item \textbf{粘贴完整错误}:Traceback、警告信息、异常堆栈
            \item \textbf{说明环境信息}:Python版本、操作系统、依赖包版本
            \item \textbf{提供可复现代码}:最小化代码片段 + 数据样本
        \end{enumerate}
    \end{alertblock}

    \vspace{0.3cm}

    \textbf{Prompt模板:}
    \begin{lstlisting}[basicstyle=\ttfamily\scriptsize]
# 环境信息
Python: 3.10.11
OS: Windows 11
PaddleOCR: 2.7.0

# 错误信息
[粘贴完整的Traceback]

# 代码片段
[粘贴最小可复现代码]

# 数据样本
[提供测试图像或文字描述]
    \end{lstlisting}
\end{frame}

\begin{frame}{使用AI辅助学习PaddleOCR}
    \textbf{推荐场景:}
    \begin{itemize}
        \item 安装失败时:让AI检查环境兼容性
        \item 参数配置时:让AI解释各参数含义
        \item 识别结果解析时:让AI说明数据结构
        \item 性能优化时:让AI提出改进建议
    \end{itemize}

    \vspace{0.3cm}

    \textbf{Cursor快捷键:}
    \begin{table}
        \centering
        \small
        \begin{tabular}{ll}
            \toprule
            \textbf{快捷键} & \textbf{功能} \\
            \midrule
            Ctrl+K & 原地编辑选中代码 \\
            Ctrl+L & 打开AI对话侧边栏 \\
            Ctrl+I & AI生成代码 \\
            \bottomrule
        \end{tabular}
    \end{table}
\end{frame}

%===========================================================
% modules/08_traditional_ocr.tex - 传统OCR识别实战
%===========================================================

\section{传统OCR识别实战}

\begin{frame}{传统OCR识别流程概述}
    \begin{columns}
        \column{0.5\textwidth}
        \textbf{传统OCR四步走:}
        \begin{enumerate}
            \item \textbf{预处理}:图像增强、二值化、去噪
            \item \textbf{版面分析}:文字区域定位、分栏
            \item \textbf{字符分割}:行分割、字分割
            \item \textbf{字符识别}:特征提取、分类器
        \end{enumerate}

        \column{0.5\textwidth}
        \textbf{各阶段关键技术:}
        \begin{itemize}
            \item 预处理:灰度化、二值化、倾斜校正
            \item 版面分析:投影分析、连通域分析
            \item 分割:投影分割、连通域分割
            \item 识别:模板匹配、特征分类
        \end{itemize}
    \end{columns}
\end{frame}

\begin{frame}{图像预处理实战(70\%脚手架)}
    \begin{lstlisting}[basicstyle=\ttfamily\scriptsize]
import cv2
import numpy as np

def traditional_ocr_preprocess(image_path):
    """传统OCR图像预处理"""
    # TODO: 使用AI助手补全代码
    # 提示词:"OpenCV读取图像并转为灰度图"

    # TODO: 读取图像
    image = cv2.____(image_path)

    # TODO: 转为灰度图
    gray = cv2.cvtColor(____, ____)

    # TODO: 二值化(提示词:"Otsu自动阈值二值化OpenCV")
    _, binary = cv2.threshold(____, 0, 255, ____)

    # TODO: 去噪(提示词:"OpenCV形态学开运算去除噪点")
    kernel = np.ones((____, ____), np.uint8)
    denoised = cv2.morphologyEx(____, _____, kernel)

    return denoised
    \end{lstlisting}

    \begin{alertblock}{AI辅助提示}
        不知道cv2函数的参数?\textbf{选中函数名,按Ctrl+K问AI:}
        "cv2.morphologyEx的参数含义和常用取值是什么?"
    \end{alertblock}
\end{frame}

\begin{frame}{版面分析与文字区域定位(70\%脚手架)}
    \begin{lstlisting}[basicstyle=\ttfamily\scriptsize]
def layout_analysis(binary_image):
    """版面分析:定位文字区域"""
    # TODO: 使用AI助手补全代码
    # 提示词:"OpenCV查找轮廓定位文字区域"

    # TODO: 查找轮廓(提示词:"cv2.findContours查找外轮廓")
    contours, _ = cv2.findContours(
        _____,
        cv2._____,  # TODO: 只检测外轮廓
        cv2._____    # TODO: 简单近似方法
    )

    text_regions = []
    for contour in contours:
        # TODO: 获取边界框(提示词:"cv2.boundingRect获取轮廓边界框")
        x, y, w, h = cv2._____(contour)

        # TODO: 过滤条件(提示词:"根据宽高比和面积过滤非文字区域")
        if _____ and _____:  # TODO: 设置合理阈值
            text_regions.append((x, y, w, h))

    return text_regions
    \end{lstlisting}

    \begin{alertblock}{AI辅助调试}
        轮廓检测结果不对?\textbf{把代码和图像一起发给AI:}
        "这段代码检测文字区域效果很差,能帮我分析原因并提供改进建议吗?"
    \end{alertblock}
\end{frame}

\begin{frame}{字符分割与识别(70\%脚手架)}
    \begin{lstlisting}[basicstyle=\ttfamily\scriptsize]
def segment_and_recognize(text_region, binary_image):
    """字符分割与识别"""
    # TODO: 使用AI助手补全代码
    # 提示词:"OpenCV投影法分割文字行和字符"

    x, y, w, h = text_region

    # TODO: 提取区域图像
    region = binary_image[y:y+h, _____]

    # TODO: 水平投影分割行(提示词:"numpy计算水平投影分割文字行")
    h_projection = np._____(_____, axis=1)
    row_ranges = _____  # TODO: 根据投影值找出文字行范围

    characters = []
    for row_start, row_end in row_ranges:
        row_image = region[row_start:row_end, :]

        # TODO: 垂直投影分割字符(提示词:"垂直投影分割单个字符")
        v_projection = np._____(_____, axis=0)
        char_ranges = _____

        for char_start, char_end in char_ranges:
            char_image = row_image[:, char_start:char_end]
            characters.append(char_image)

    return characters
    \end{lstlisting}
\end{frame}

%===========================================================
% modules/10_paddleocr_advanced.tex - PaddleOCR高级实战
%===========================================================

\section{PaddleOCR高级实战}

\begin{frame}{PaddleOCR架构解析}
    \begin{columns}
        \column{0.5\textwidth}
        \textbf{PP-OCR三大模块:}
        \begin{enumerate}
            \item \textbf{文本检测(DBNet)}
            \begin{itemize}
                \item 可微二值化网络
                \item 轻量级骨干网络
                \item 实时文字定位
            \end{itemize}
            \item \textbf{方向分类(MobileNet)}
            \begin{itemize}
                \item 0/90/180/270度分类
                \item 轻量级分类器
                \item 提高识别准确率
            \end{itemize}
            \item \textbf{文本识别(SVTR)}
            \begin{itemize}
                \item 视觉Transformer
                \item 视觉-语言预训练
                \item 中英文高精度识别
            \end{itemize}
        \end{enumerate}

        \column{0.5\textwidth}
        \textbf{数据处理流程:}
        \begin{center}
            \begin{tikzpicture}[
                node distance=0.4cm,
                box/.style={rectangle, draw=blue!60, fill=blue!10, rounded corners, minimum width=3cm, minimum height=0.5cm, font=\scriptsize},
                arrow/.style={->, thick}
            ]
                \node[box] (img) {输入图像};
                \node[box, below=of img] (det) {文本检测 (DBNet)};
                \node[box, below=of det] (crop) {文字区域裁剪};
                \node[box, below=of crop] (cls) {方向分类 (MobileNet)};
                \node[box, below=of cls] (rec) {文本识别 (SVTR)};
                \node[box, below=of rec] (out) {输出结果};

                \draw[arrow] (img) -- (det);
                \draw[arrow] (det) -- (crop);
                \draw[arrow] (crop) -- (cls);
                \draw[arrow] (cls) -- (rec);
                \draw[arrow] (rec) -- (out);
            \end{tikzpicture}
        \end{center}
    \end{columns}
\end{frame}

\begin{frame}[fragile]{PaddleOCR模型配置与优化(70\%脚手架)}
    \begin{lstlisting}[basicstyle=\ttfamily\scriptsize]
from paddleocr import PaddleOCR

# TODO: 使用AI助手配置高性能OCR
# 提示词:"配置PaddleOCR高性能模式,需要GPU支持、
# 多线程处理、大模型版本,用于生产环境"

ocr = PaddleOCR(
    # ===== 基础配置 =====
    # TODO: 语言设置(支持多语言组合)
    lang=____,  # 如 'ch', 'en', 'ch_en', 'japan', 'korean'

    # TODO: 模型版本(提示词:"PaddleOCR模型版本选择")
    det_model_dir=____,  # 检测模型路径
    rec_model_dir=____,  # 识别模型路径
    cls_model_dir=____,  # 方向分类模型路径

    # ===== 性能优化配置 =====
    # TODO: GPU配置(提示词:"PaddleOCR GPU加速配置参数")
    use_gpu=____,
    gpu_mem=____,  # GPU显存限制

    # TODO: 多线程配置
    use_tensorrt=____,  # TensorRT加速
    use_fp16=____,      # 半精度推理
    enable_mkldnn=____, # MKL-DNN加速
    cpu_threads=____,   # CPU线程数

    # ===== 检测模型参数 =====
    # TODO: 检测参数调优(提示词:"DBNet检测参数调优")
    det_db_thresh=____,      # 二值化阈值
    det_db_box_thresh=____,  # 框过滤阈值
    det_db_unclip_ratio=____, # 文本框扩展系数
    max_text_length=____,    # 最大文本长度
    det_db_score_mode=____,  # 得分计算模式

    # ===== 识别模型参数 =====
    # TODO: 识别参数调优
    rec_batch_num=____,      # 识别batch大小
    max_text_length=____,    # 最大文本长度
    rec_score_thres=____,    # 识别置信度阈值

    # ===== 方向分类参数 =====
    use_angle_cls=____,      # 启用方向分类
    cls_batch_num=____,    # 分类batch大小
    cls_thresh=____,        # 分类置信度阈值

    # ===== 输出控制 =====
    drop_score=____,       # 过滤低置信度结果
    show_log=____,         # 显示日志
    use_pdserving=____,    # 使用PaddleServing
    use_visual_backbone=____, # 可视化骨干网络
)
    \end{lstlisting}

    \begin{alertblock}{AI参数调优助手}
        不知道参数怎么设?\textbf{问AI:}
        "PaddleOCR在生产环境中GPU推理,det_db_thresh设为多少合适?"
    \end{alertblock}
\end{frame}

\begin{frame}[fragile]{自定义训练:PaddleOCR模型微调(70\%脚手架)}
    \begin{lstlisting}[basicstyle=\ttfamily\scriptsize]
# TODO: 使用AI助手完成PaddleOCR自定义训练
# 提示词:"PaddleOCR自定义训练完整流程,包括数据准备、
# 配置文件修改、训练启动、模型评估"

# ===== 步骤1: 数据准备 =====
# TODO: 准备训练数据(提示词:"PaddleOCR训练数据格式要求")
# 文本检测数据格式:
# img_001.jpg\t[[x1,y1],[x2,y2],[x3,y3],[x4,y4], "文字内容"]

# TODO: 划分训练集和验证集
import os
import shutil
import random

def prepare_dataset(src_dir, dst_dir, train_ratio=0.8):
    """准备PaddleOCR训练数据集"""
    # TODO: 使用AI助手补全代码
    # 提示词:"Python划分训练集验证集,复制图像和标注文件"

    images = [f for f in os.listdir(src_dir) if f.endswith(('.jpg', '.png'))]
    random.shuffle(images)

    # TODO: 按比例划分
    split_idx = int(len(images) * ____)
    train_images = images[____]
    val_images = images[____]

    # TODO: 复制文件到目标目录
    for img in train_images:
        shutil.copy(os.path.join(src_dir, img), ____)
        # TODO: 复制对应的标注文件
        ____

    return len(train_images), len(val_images)

# ===== 步骤2: 配置文件修改 =====
# TODO: 修改PaddleOCR配置文件(提示词:"PaddleOCR配置文件各参数含义")
config_content = '''
# 检测模型配置
global:
  debug: false
  use_gpu: ____  # TODO: 是否使用GPU
  epoch_num: ____  # TODO: 训练轮数
  log_smooth_window: 20
  print_batch_step: 10
  save_model_dir: ./output/det_db/
  save_epoch_step: 100
  eval_batch_step: [0, 2000]
  cal_metric_during_train: false
  pretrained_model: null
  checkpoints: null
  save_inference_dir: null
  use_visualdl: false
  infer_img: doc/imgs_en/img_10.jpg

# TODO: 更多配置参数...
'''

# ===== 步骤3: 启动训练 =====
# TODO: 使用PaddleOCR训练脚本
# 命令行:python tools/train.py -c configs/det/det_db_mv3.yml

def start_training(config_path, use_gpu=True, num_epochs=100):
    """启动PaddleOCR训练"""
    import subprocess

    cmd = [
        'python', 'tools/train.py',
        '-c', config_path,
        # TODO: 添加更多参数
    ]

    if use_gpu:
        cmd.append('--use_gpu')

    # TODO: 启动训练进程
    process = subprocess.____(____)

    return process

# ===== 步骤4: 模型评估 =====
# TODO: 评估训练好的模型
# 命令行:python tools/eval.py -c configs/det/det_db_mv3.yml -o Global.checkpoints=./output/det_db/best_accuracy

# TODO: 导出推理模型
# 命令行:python tools/export_model.py -c configs/det/det_db_mv3.yml -o Global.pretrained_model=./output/det_db/best_accuracy Global.save_inference_dir=./inference/det_db
    \end{lstlisting}

    \begin{alertblock}{AI辅助训练提示}
        训练遇到NaN loss?\textbf{问AI:}
        "PaddleOCR训练时loss变成NaN,可能是什么原因,如何调试?"
    \end{alertblock}
\end{frame}

%===========================================================
% modules/11_dl_ocr.tex - 深度学习OCR模型实战
%===========================================================

\section{深度学习OCR模型实战}

\begin{frame}{深度学习OCR架构演进}
    \begin{columns}
        \column{0.5\textwidth}
        \textbf{OCR架构三代演进:}
        \begin{enumerate}
            \item \textbf{第一代:CNN+全连接}
            \begin{itemize}
                \item 单字识别
                \item 固定输入尺寸
                \item LeNet-5, AlexNet
            \end{itemize}
            \item \textbf{第二代:CNN+RNN+CTC}
            \begin{itemize}
                \item 序列识别
                \item 端到端训练
                \item CRNN, CTPN
            \end{itemize}
            \item \textbf{第三代:Transformer}
            \begin{itemize}
                \item 自注意力机制
                \item 视觉-语言预训练
                \item TrOCR, SRN
            \end{itemize}
        \end{enumerate}

        \column{0.5\textwidth}
        \begin{center}
            \begin{tikzpicture}[
                node distance=0.8cm,
                box/.style={rectangle, draw=blue!60, fill=blue!10, rounded corners, minimum width=4cm, minimum height=0.6cm, font=\scriptsize},
                arrow/.style={->, thick, blue!60}
            ]
                \node[box] (cnn) {CNN特征提取};
                \node[box, below=of cnn] (rnn) {RNN序列建模};
                \node[box, below=of rnn] (ctc) {CTC解码};
                \node[box, below=of ctc] (out) {文本输出};

                \draw[arrow] (cnn) -- (rnn);
                \draw[arrow] (rnn) -- (ctc);
                \draw[arrow] (ctc) -- (out);
            \end{tikzpicture}

            \vspace{0.5cm}

            \textit{\footnotesize CRNN架构流程图}
        \end{center}
    \end{columns}
\end{frame}

\begin{frame}[fragile]{PyTorch实现CRNN模型(70\%脚手架)}
    \begin{lstlisting}[basicstyle=\ttfamily\scriptsize]
import torch
import torch.nn as nn

class CRNN(nn.Module):
    """CRNN: CNN + RNN + CTC for OCR"""

    def __init__(self, img_height, nc, nclass, nh):
        super(CRNN, self).__init__()

        # TODO: 使用AI助手完成CNN特征提取层
        # 提示词:"PyTorch实现VGG风格的CNN特征提取,
        # 用于OCR图像特征提取,包含卷积层、BatchNorm、ReLU、MaxPool"

        self.cnn = nn.Sequential(
            # TODO: 第1层卷积
            nn.Conv2d(nc, 64, kernel_size=____, padding=____),
            nn.BatchNorm2d(____),
            nn.ReLU(____),
            nn.MaxPool2d(____, ____),

            # TODO: 第2层卷积
            nn.Conv2d(____, 128, kernel_size=____, padding=____),
            ____,  # TODO: 添加BatchNorm和ReLU
            nn.MaxPool2d(____, ____),

            # TODO: 继续添加更多卷积层...
        )

        # TODO: 计算CNN输出尺寸
        # 提示词:"计算CNN输出尺寸用于确定RNN输入维度"
        self.rnn_input_size = self._calculate_rnn_input(img_height)

        # TODO: 使用AI助手完成RNN层
        # 提示词:"PyTorch实现双向LSTM用于OCR序列建模"
        self.rnn = nn.LSTM(
            input_size=self.rnn_input_size,
            hidden_size=____,      # TODO: 隐藏层大小
            num_layers=____,       # TODO: LSTM层数
            bidirectional=____,    # TODO: 是否双向
            batch_first=____       # TODO: batch维度位置
        )

        # TODO: 完成输出层
        self.fc = nn.Linear(____, ____)  # TODO: 输入和输出维度

    def forward(self, x):
        # TODO: 使用AI助手完成前向传播
        # 提示词:"PyTorch实现CRNN前向传播流程"

        # CNN特征提取
        conv = self.cnn(x)
        batch, channel, height, width = conv.size()

        # TODO: 调整维度为RNN输入格式
        conv = conv.view(batch, ____, ____)  # (batch, width, channel*height)
        conv = conv.permute(____, ____, ____)  # (width, batch, features)

        # RNN序列建模
        recurrent, _ = self.rnn(conv)

        # TODO: 全连接层输出
        T, b, h = recurrent.size()
        t_rec = recurrent.view(T * b, h)
        output = self.fc(t_rec)
        output = output.view(T, b, -1)

        return output
    \end{lstlisting}
\end{frame}

\begin{frame}[fragile]{CTC Loss与解码(70\%脚手架)}
    \begin{lstlisting}[basicstyle=\ttfamily\scriptsize]
import torch
import torch.nn as nn

class CTCLossWrapper(nn.Module):
    """CTC Loss for OCR"""

    def __init__(self, blank=0, reduction='mean'):
        super().__init__()
        # TODO: 使用AI助手完成CTC Loss
        # 提示词:"PyTorch CTCLoss参数说明和使用方法"
        self.ctc_loss = nn.CTCLoss(
            blank=____,           # TODO: 空白符索引
            reduction=____,       # TODO: 缩减方式
            zero_infinity=____    # TODO: 是否将无穷大损失置零
        )

    def forward(self, logits, targets, input_lengths, target_lengths):
        """
        Args:
            logits: (T, N, C) - 模型输出
            targets: (N, S) - 目标序列
            input_lengths: (N,) - 输入序列长度
            target_lengths: (N,) - 目标序列长度
        """
        # TODO: 计算CTC Loss
        # 提示词:"PyTorch CTCLoss输入格式要求"
        log_probs = torch.log_softmax(____, dim=____)
        loss = self.ctc_loss(____, ____, ____, ____)
        return loss


def ctc_greedy_decoder(logits, blank=0):
    """CTC贪心解码"""
    # TODO: 使用AI助手完成CTC解码
    # 提示词:"CTC贪心解码算法实现步骤"

    # 获取最大概率的类别
    preds = torch.argmax(____, dim=____)  # (T, N)

    decoded = []
    for batch_idx in range(preds.size(1)):
        seq = preds[:, batch_idx].tolist()

        # TODO: 合并重复字符
        collapsed = []
        for i, char in enumerate(seq):
            if ____ and ____ != ____:  # TODO: 非空且不等于前一个
                collapsed.append(____)

        decoded.append(collapsed)

    return decoded


def ctc_beam_search_decoder(logits, beam_width=5, blank=0):
    """CTC束搜索解码(高级)"""
    # TODO: 使用AI助手实现束搜索
    # 提示词:"CTC束搜索解码算法实现"

    # 初始化beam
    beams = [(____, 0)]  # (序列, 分数)

    for t in range(logits.size(0)):
        candidates = []

        for seq, score in beams:
            for c in range(logits.size(1)):
                # TODO: 计算新的序列和分数
                new_seq = ____(seq, c)
                new_score = ____ + ____(logits[t, c])
                candidates.append((____, ____))

        # TODO: 保留top-k
        beams = sorted(candidates, key=lambda x: x[1], reverse=True)[:____]

    return beams[0]
    \end{lstlisting}
\end{frame}

%===========================================================
% modules/13_traditional_theory.tex - 传统OCR识别流程理论
%===========================================================

\section{传统OCR识别流程理论}

\begin{frame}{传统OCR四步流程详解}
    \begin{columns}
        \column{0.5\textwidth}
        \textbf{第一步:图像预处理}
        \begin{itemize}
            \item 灰度化:彩色图转灰度图
            \item 二值化:Otsu自适应阈值
            \item 去噪:形态学操作、滤波
            \item 倾斜校正:霍夫变换、投影法
        \end{itemize}

        \vspace{0.3cm}

        \textbf{第二步:版面分析}
        \begin{itemize}
            \item 连通域分析
            \item 投影分析
            \item 分栏检测
            \item 阅读顺序确定
        \end{itemize}

        \column{0.5\textwidth}
        \textbf{第三步:字符分割}
        \begin{itemize}
            \item 行分割:水平投影
            \item 字分割:垂直投影
            \item 粘连字符处理
            \item 断字合并
        \end{itemize}

        \vspace{0.3cm}

        \textbf{第四步:字符识别}
        \begin{itemize}
            \item 特征提取:HOG、LBP、投影
            \item 分类器:模板匹配、KNN、SVM
            \item 后处理:语言模型、词典校正
        \end{itemize}
    \end{columns}
\end{frame}

\begin{frame}{图像预处理技术详解}
    \begin{columns}
        \column{0.5\textwidth}
        \textbf{1. 灰度化方法:}
        \begin{itemize}
            \item 加权平均法:$Gray = 0.299R + 0.587G + 0.114B$
            \item 最大值法:$Gray = max(R, G, B)$
            \item 平均值法:$Gray = (R + G + B) / 3$
        \end{itemize}

        \vspace{0.3cm}

        \textbf{2. 二值化方法:}
        \begin{itemize}
            \item 全局阈值:固定阈值分割
            \item Otsu算法:自动计算最优阈值
            \item 自适应阈值:局部阈值处理
        \end{itemize}

        \column{0.5\textwidth}
        \textbf{3. 倾斜校正方法:}
        \begin{itemize}
            \item 投影法:寻找最小投影方差角度
            \item 霍夫变换:检测直线角度
            \item 傅里叶变换:频域分析
        \end{itemize}

        \vspace{0.3cm}

        \textbf{4. 去噪方法:}
        \begin{itemize}
            \item 中值滤波:去除椒盐噪声
            \item 高斯滤波:平滑处理
            \item 形态学操作:开运算、闭运算
        \end{itemize}
    \end{columns}
\end{frame}

\begin{frame}{特征提取与分类器设计}
    \begin{columns}
        \column{0.5\textwidth}
        \textbf{特征提取方法:}

        \vspace{0.2cm}

        \textbf{1. 统计特征:}
        \begin{itemize}
            \item 投影直方图
            \item 轮廓特征
            \item 网格特征
            \item Zernike矩
        \end{itemize}

        \vspace{0.2cm}

        \textbf{2. 结构特征:}
        \begin{itemize}
            \item 端点、交叉点
            \item 笔画方向
            \item 环、洞特征
            \item 拓扑结构
        \end{itemize}

        \vspace{0.2cm}

        \textbf{3. 变换域特征:}
        \begin{itemize}
            \item 傅里叶描述子
            \item 小波特征
            \item HOG特征
            \item LBP特征
        \end{itemize}

        \column{0.5\textwidth}
        \textbf{分类器设计:}

        \vspace{0.2cm}

        \textbf{1. 模板匹配:}
        \begin{itemize}
            \item 最近邻匹配
            \item 相关性匹配
            \item 归一化匹配
        \end{itemize}

        \vspace{0.2cm}

        \textbf{2. K近邻(KNN):}
        \begin{itemize}
            \item 距离度量:欧氏距离、马氏距离
            \item K值选择
            \item 加权KNN
        \end{itemize}

        \vspace{0.2cm}

        \textbf{3. 支持向量机(SVM):}
        \begin{itemize}
            \item 核函数选择
            \item 参数优化
            \item 多分类策略
        \end{itemize}

        \vspace{0.2cm}

        \textbf{4. 神经网络:}
        \begin{itemize}
            \item MLP多层感知机
            \item 卷积神经网络
            \item 训练策略
        \end{itemize}
    \end{columns}
\end{frame}

%===========================================================
% modules/14_dl_theory.tex - 深度学习OCR原理
%===========================================================

\section{深度学习OCR原理}

\begin{frame}{深度学习OCR架构演进}
    \begin{columns}
        \column{0.5\textwidth}
        \textbf{第一代:CNN单字识别}
        \begin{itemize}
            \item LeNet-5 (1998)
            \item AlexNet (2012)
            \item VGGNet (2014)
            \item ResNet (2015)
        \end{itemize}

        \textbf{特点:}
        \begin{itemize}
            \item 固定输入尺寸
            \item 单字分类
            \item 需字符分割
        \end{itemize}

        \column{0.5\textwidth}
        \textbf{第二代:CNN+RNN+CTC}
        \begin{itemize}
            \item CRNN (2015)
            \item CTPN (2016)
            \item SegLink (2017)
        \end{itemize}

        \textbf{特点:}
        \begin{itemize}
            \item 序列建模
            \item 端到端训练
            \item 无需字符分割
        \end{itemize}

        \vspace{0.3cm}

        \textbf{第三代:Transformer}
        \begin{itemize}
            \item TrOCR (2021)
            \item SRN (2020)
            \item Vision-Language预训练
        \end{itemize}
    \end{columns}
\end{frame}

\begin{frame}{CNN特征提取原理}
    \begin{columns}
        \column{0.5\textwidth}
        \textbf{卷积层(Convolution):}
        \begin{itemize}
            \item 局部连接:感受野
            \item 权值共享:平移不变性
            \item 特征图:$H \times W \times C$
        \end{itemize}

        \vspace{0.2cm}

        \textbf{池化层(Pooling):}
        \begin{itemize}
            \item 最大池化:保留显著特征
            \item 平均池化:保留背景信息
            \item 下采样:降低维度
        \end{itemize}

        \vspace{0.2cm}

        \textbf{激活函数:}
        \begin{itemize}
            \item ReLU:$f(x) = \max(0, x)$
            \item Sigmoid:$f(x) = \frac{1}{1+e^{-x}}$
            \item Tanh:$f(x) = \frac{e^x - e^{-x}}{e^x + e^{-x}}$
        \end{itemize}

        \column{0.5\textwidth}
        \textbf{OCR中的CNN架构:}

        \vspace{0.2cm}

        \textbf{VGG-style(CRNN):}
        \begin{lstlisting}[basicstyle=\ttfamily\tiny]
Conv(64) -> MaxPool
Conv(128) -> MaxPool
Conv(256) -> Conv(256) -> MaxPool
Conv(512) -> Conv(512) -> MaxPool
Conv(512) -> Conv(512) -> MaxPool
        \end{lstlisting}

        \vspace{0.2cm}

        \textbf{ResNet-style:}
        \begin{itemize}
            \item 残差连接:$y = F(x) + x$
            \item 解决梯度消失
            \item 支持更深网络
        \end{itemize}

        \vspace{0.2cm}

        \textbf{MobileNet-style:}
        \begin{itemize}
            \item 深度可分离卷积
            \item 轻量级设计
            \item 移动端部署
        \end{itemize}
    \end{columns}
\end{frame}

\begin{frame}{RNN序列建模原理}
    \begin{columns}
        \column{0.5\textwidth}
        \textbf{为什么OCR需要RNN?}
        \begin{itemize}
            \item 文字是序列数据
            \item 上下文关联性强
            \item 变长输入输出
            \item 字符间依赖关系
        \end{itemize}

        \vspace{0.3cm}

        \textbf{标准RNN的局限:}
        \begin{itemize}
            \item 梯度消失/爆炸
            \item 长程依赖困难
            \item 信息传递衰减
        \end{itemize}

        \column{0.5\textwidth}
        \textbf{LSTM(长短期记忆网络):}

        \vspace{0.2cm}

        LSTM通过门控机制解决RNN的局限:

        \vspace{0.2cm}

        \textbf{1. 遗忘门(Forget Gate):}
        \begin{equation*}
            f_t = \sigma(W_f \cdot [h_{t-1}, x_t] + b_f)
        \end{equation*}
        决定丢弃什么信息

        \vspace{0.2cm}

        \textbf{2. 输入门(Input Gate):}
        \begin{align*}
            i_t &= \sigma(W_i \cdot [h_{t-1}, x_t] + b_i) \\
            \tilde{C}_t &= \tanh(W_C \cdot [h_{t-1}, x_t] + b_C)
        \end{align*}
        决定存储什么新信息

        \vspace{0.2cm}

        \textbf{3. 细胞状态更新:}
        \begin{equation*}
            C_t = f_t \odot C_{t-1} + i_t \odot \tilde{C}_t
        \end{equation*}

        \vspace{0.2cm}

        \textbf{4. 输出门(Output Gate):}
        \begin{align*}
            o_t &= \sigma(W_o \cdot [h_{t-1}, x_t] + b_o) \\
            h_t &= o_t \odot \tanh(C_t)
        \end{align*}
        决定输出什么信息
    \end{columns}
\end{frame}

\begin{frame}{Transformer for OCR}
    \begin{columns}
        \column{0.5\textwidth}
        \textbf{为什么Transformer适合OCR?}
        \begin{itemize}
            \item 全局依赖建模
            \item 并行计算高效
            \item 长程依赖捕获
            \item 可解释性强
        \end{itemize}

        \vspace{0.3cm}

        \textbf{OCR中的Transformer架构:}
        \begin{enumerate}
            \item \textbf{Encoder-Only}:
            \begin{itemize}
                \item ViT-style
                \item 图像分块编码
                \item 单阶段识别
            \end{itemize}
            \item \textbf{Encoder-Decoder}:
            \begin{itemize}
                \item TrOCR
                \item 图像编码+文本解码
                \item 自回归生成
            \end{itemize}
        \end{enumerate}

        \column{0.5\textwidth}
        \textbf{TrOCR(Transformer OCR):}

        \vspace{0.2cm}

        \textbf{架构:}
        \begin{lstlisting}[basicstyle=\ttfamily\tiny]
# Image Encoder (BEiT/DeiT)
Image Patch Embedding
+ Position Embedding
+ Transformer Encoder Layers
-> Image Features

# Text Decoder (GPT-style)
Token Embedding
+ Position Embedding
+ Causal Mask
+ Transformer Decoder Layers
+ Linear + Softmax
-> Text Tokens
        \end{lstlisting}

        \vspace{0.2cm}

        \textbf{训练策略:}
        \begin{enumerate}
            \item \textbf{Pre-training}:
            \begin{itemize}
                \item 大规模合成数据
                \item 掩码图像建模
                \item 视觉-语言预训练
            \end{itemize}
            \item \textbf{Fine-tuning}:
            \begin{itemize}
                \item 真实场景数据
                \item 特定领域适应
                \item 小样本学习
            \end{itemize}
        \end{enumerate}

        \vspace{0.2cm}

        \textbf{Transformer OCR优势:}
        \begin{itemize}
            \item 端到端训练,无需组件
            \item 预训练知识迁移
            \item 多语言支持
            \item 手写识别效果好
        \end{itemize}
    \end{columns}
\end{frame}

%===========================================================
% modules/15_text_detection.tex - 文字检测技术
%===========================================================

\section{文字检测技术}

\begin{frame}{文字检测技术概述}
    \begin{columns}
        \column{0.5\textwidth}
        \textbf{文字检测 vs 目标检测:}
        \begin{itemize}
            \item 文字:长宽比极端、任意方向
            \item 目标:形状规则、方向固定
            \item 文字:密集排列、大小不一
            \item 目标:稀疏分布、尺度变化
        \end{itemize}

        \vspace{0.3cm}

        \textbf{文字检测的挑战:}
        \begin{itemize}
            \item 多方向、多尺度文字
            \item 弯曲文字、艺术字体
            \item 复杂背景、低质量图像
            \item 密集文字、重叠区域
        \end{itemize}

        \column{0.5\textwidth}
        \textbf{文字检测方法分类:}

        \vspace{0.2cm}

        \textbf{1. 基于回归的方法:}
        \begin{itemize}
            \item TextBoxes系列
            \item EAST
            \item 直接回归文字框坐标
        \end{itemize}

        \vspace{0.2cm}

        \textbf{2. 基于分割的方法:}
        \begin{itemize}
            \item PixelLink
            \item PSENet
            \item DBNet
            \item 像素级文字区域分割
        \end{itemize}

        \vspace{0.2cm}

        \textbf{3. 混合方法:}
        \begin{itemize}
            \item 检测+分割结合
            \item 多任务联合训练
        \end{itemize}
    \end{columns}
\end{frame}

\begin{frame}{EAST:基于回归的文字检测}
    \begin{columns}
        \column{0.5\textwidth}
        \textbf{EAST核心思想:}
        \begin{itemize}
            \item 全卷积网络(FCN)
            \item 像素级预测
            \item 直接回归文字框
            \item 非极大值抑制(NMS)
        \end{itemize}

        \vspace{0.3cm}

        \textbf{EAST网络结构:}
        \begin{enumerate}
            \item \textbf{特征提取}:PVANet / ResNet / VGG
            \item \textbf{特征融合}:U-Net结构,多尺度融合
            \item \textbf{输出层}:
            \begin{itemize}
                \item Score map:文字/非文字概率
                \item RBOX:旋转矩形框 $(d_1, d_2, d_3, d_4, \theta)$
                \item QUAD:四边形框 $(x_1, y_1, ..., x_4, y_4)$
            \end{itemize}
        \end{enumerate}

        \column{0.5\textwidth}
        \textbf{EAST损失函数:}

        \vspace{0.2cm}

        总损失:$L = L_s + \lambda_g L_g$

        \vspace{0.2cm}

        \textbf{1. 分类损失(Score Map):}
        \begin{equation*}
            L_s = -\frac{1}{N} \sum_{i=1}^{N} [y_i \log(\hat{y}_i) + (1-y_i) \log(1-\hat{y}_i)]
        \end{equation*}

        \vspace{0.2cm}

        \textbf{2. 几何损失(Geometry):}

        对于RBOX:
        \begin{equation*}
            L_g = L_{iou} + L_{\theta}
        \end{equation*}

        其中:
        \begin{itemize}
            \item $L_{iou}$:IoU损失(预测框与真实框的重叠度)
            \item $L_{\theta}$:角度损失,$L_{\theta} = 1 - \cos(\theta_{pred} - \theta_{gt})$
        \end{itemize}

        对于QUAD:
        \begin{equation*}
            L_g = \frac{1}{8} \sum_{i=1}^{4} |x_i^{pred} - x_i^{gt}| + |y_i^{pred} - y_i^{gt}|
        \end{equation*}

        \vspace{0.3cm}

        \textbf{EAST优势:}
        \begin{itemize}
            \item 端到端训练,无需候选框
            \item 速度快:支持实时检测
            \item 支持多方向文字
            \item 精度高:在ICDAR数据集表现优秀
        \end{itemize}
    \end{columns}
\end{frame}

\begin{frame}{DBNet:基于可微二值化的文字检测}
    \begin{columns}
        \column{0.5\textwidth}
        \textbf{DBNet核心创新:}
        \begin{itemize}
            \item 可微二值化(Differentiable Binarization)
            \item 自适应阈值
            \item 轻量化网络
            \item 高精度+高效率
        \end{itemize}

        \vspace{0.3cm}

        \textbf{传统二值化 vs DBNet:}

        \vspace{0.2cm}

        \textbf{传统方法:}
        \begin{equation*}
            B_{i,j} = \begin{cases} 1 & \text{if } P_{i,j} \geq t \\ 0 & \text{otherwise} \end{cases}
        \end{equation*}
        其中$t$是固定阈值,\textbf{不可微}。

        \vspace{0.2cm}

        \textbf{DBNet可微二值化:}
        \begin{equation*}
            B_{i,j} = \frac{1}{1 + e^{-k(P_{i,j} - T_{i,j})}}
        \end{equation*}
        其中:
        \begin{itemize}
            \item $P_{i,j}$:概率图(probability map)
            \item $T_{i,j}$:自适应阈值图(threshold map)
            \item $k$:放大因子(默认50)
        \end{itemize}

        \textbf{优势}:可微分,可以端到端训练。

        \column{0.5\textwidth}
        \textbf{DBNet网络结构:}

        \vspace{0.2cm}

        \begin{enumerate}
            \item \textbf{Backbone}:ResNet-18/50 或 MobileNetV3
            \item \textbf{FPN}(特征金字塔网络):
            \begin{itemize}
                \item 融合多尺度特征
                \item 上采样 + 横向连接
            \end{itemize}
            \item \textbf{Head}:
            \begin{itemize}
                \item Probability Map Head:预测文字区域概率
                \item Threshold Map Head:预测自适应阈值
                \item Binary Map Head:二值化结果
            \end{itemize}
        \end{enumerate}

        \vspace{0.3cm}

        \textbf{DBNet损失函数:}
        \begin{equation*}
            L = L_s + \alpha \times L_b + \beta \times L_t
        \end{equation*}

        其中:
        \begin{itemize}
            \item $L_s$:Probability Map的BCE损失
            \item $L_b$:Binary Map的BCE损失
            \item $L_t$:Threshold Map的L1损失
            \item $\alpha$, $\beta$:权重系数
        \end{itemize}

        \vspace{0.3cm}

        \textbf{DBNet优势:}
        \begin{itemize}
            \item 速度快:轻量级 backbone
            \item 精度高:可微二值化
            \item 鲁棒性强:自适应阈值
            \item 易于部署:推理简单
        \end{itemize}
    \end{columns}
\end{frame}

%===========================================================
% modules/16_ctc.tex - 序列识别与CTC
%===========================================================

\section{序列识别与CTC}

\begin{frame}{CTC(Connectionist Temporal Classification)}
    \begin{columns}
        \column{0.5\textwidth}
        \textbf{为什么需要CTC?}
        \begin{itemize}
            \item 输入序列长度 $T$ 与输出序列长度 $L$ 不一致
            \item 对齐方式未知($T \neq L$)
            \item 无法直接使用交叉熵损失
            \item 需要端到端训练
        \end{itemize}

        \vspace{0.3cm}

        \textbf{CTC核心思想:}
        \begin{enumerate}
            \item 引入空白符 $\epsilon$(blank)
            \item 允许重复字符和空白
            \item 所有可能的对齐路径
            \item 路径概率求和
        \end{enumerate}

        \column{0.5\textwidth}
        \textbf{CTC对齐示例:}

        目标序列:"cat"

        输入长度:$T=6$

        可能的对齐路径:
        \begin{itemize}
            \item c-c-a-a-t-t
            \item c-$\u0007epsilon$-a-t-$epsilon$-t
            \item $epsilon$-c-a-a-$epsilon$-t
            \item ...(所有可能的路径)
        \end{itemize}

        \vspace{0.3cm}

        \textbf{CTC输出处理:}

        原始输出:"c-$epsilon$-a-a-t-$epsilon$"

        合并重复:"c-$epsilon$-a-t-$epsilon$"

        移除空白:"cat"

        \vspace{0.3cm}

        \textbf{CTC的优势:}
        \begin{itemize}
            \item 无需对齐标注
            \item 端到端训练
            \item 处理变长序列
            \item 网络输出与文本长度解耦
        \end{itemize}
    \end{columns}
\end{frame}

\begin{frame}{CTC数学原理与算法}
    \begin{columns}
        \column{0.5\textwidth}
        \textbf{CTC损失函数:}

        对于目标序列 $y = (y_1, y_2, ..., y_L)$,CTC损失定义为:

        \begin{equation*}
            \mathcal{L}_{CTC} = -\log P(y|x) = -\log \sum_{\pi \in \mathcal{B}^{-1}(y)} P(\pi|x)
        \end{equation*}

        其中:
        \begin{itemize}
            \item $\pi$:CTC路径(包含空白符的序列)
            \item $\mathcal{B}^{-1}(y)$:映射到目标序列 $y$ 的所有路径集合
            \item $P(\pi|x) = \prod_{t=1}^{T} p_t(\pi_t|x)$:路径概率
        \end{itemize}

        \vspace{0.3cm}

        \textbf{前向-后向算法:}

        为了高效计算CTC损失,使用动态规划的前向-后向算法。

        定义前向变量 $\alpha_t(s)$:在时刻 $t$ 到达状态 $s$ 的所有路径概率之和。

        定义后向变量 $\beta_t(s)$:从时刻 $t$ 的状态 $s$ 到序列结束的所有路径概率之和。

        则:
        \begin{equation*}
            P(y|x) = \sum_{s} \alpha_t(s) \cdot \beta_t(s)
        \end{equation*}

        \column{0.5\textwidth}
        \textbf{CTC解码策略:}

        \vspace{0.2cm}

        \textbf{1. 贪心解码(Greedy Decoding):}
        \begin{itemize}
            \item 每时刻选择概率最大的标签
            \item 合并重复字符,移除空白
            \item 简单快速,但可能不是最优路径
        \end{itemize}

        \begin{equation*}
            \pi^* = \arg\max_{\pi} \prod_{t=1}^{T} p_t(\pi_t|x)
        \end{equation*}

        \vspace{0.2cm}

        \textbf{2. 束搜索(Beam Search):}
        \begin{itemize}
            \item 维护Top-k个候选路径
            \item 考虑更全局的路径组合
            \item 精度更高,计算量更大
        \end{itemize}

        \begin{equation*}
            \mathcal{B} = \text{top-}k \{ P(\pi|x) : \pi \in \Pi \}
        \end{equation*}

        \vspace{0.2cm}

        \textbf{3. 前缀束搜索(Prefix Beam Search):}
        \begin{itemize}
            \item 在标签级别进行束搜索
            \item 更高效,实际应用广泛
        \end{itemize}
    \end{columns}
\end{frame}

%===========================================================
% modules/17_ocr_overview_detailed.tex - OCR技术概述(详细)
%===========================================================

\section{OCR技术概述(详细)}

\begin{frame}{OCR技术发展历程}
    \begin{columns}
        \column{0.5\textwidth}
        \textbf{第一代:模板匹配时代(1960s-1980s)}
        \begin{itemize}
            \item 基于字符模板匹配
            \item 只能识别特定字体
            \item 对噪声敏感
            \item 代表:IBM 1287
        \end{itemize}

        \vspace{0.3cm}

        \textbf{第二代:特征提取时代(1980s-2000s)}
        \begin{itemize}
            \item 基于结构/统计特征
            \item 支持多字体识别
            \item 使用ML分类器(SVM、KNN)
            \item 代表:Tesseract(早期版本)
        \end{itemize}

        \vspace{0.3cm}

        \textbf{第三代:深度学习时代(2012-至今)}
        \begin{itemize}
            \item 基于CNN/RNN/Transformer
            \item 端到端识别
            \item 支持场景文字
            \item 代表:CRNN、PaddleOCR、TrOCR
        \end{itemize}

        \column{0.5\textwidth}
        \textbf{OCR技术发展里程碑:}

        \vspace{0.3cm}

        \begin{table}
            \centering
            \small
            \begin{tabular}{lll}
                \toprule
                \textbf{年份} & \textbf{里程碑} & \textbf{影响} \\
                \midrule
                1914 & 第一台OCR设备 & 起源 \\
                1965 & IBM 1287 & 商用化 \\
                1974 & Kurzweil OCR & 多字体支持 \\
                1985 & Tesseract诞生 & 开源化 \\
                1995 & OCR标准化 & 行业规范 \\
                2006 & Deep Learning兴起 & 技术革命 \\
                2012 & AlexNet突破 & 深度学习OCR \\
                2015 & CRNN提出 & 序列OCR \\
                2017 & Transformer提出 & 注意力机制 \\
                2020 & TrOCR发布 & 端到端Transformer OCR \\
                2021 & PaddleOCR 2.0 & 产业级OCR \\
                2023 & GPT-4V & 多模态OCR \\
                \bottomrule
            \end{tabular}
        \end{table}

        \vspace{0.3cm}

        \textbf{OCR技术发展趋势:}
        \begin{itemize}
            \item \textbf{通用化}:从印刷体到手写体、场景文字
            \item \textbf{多语言}:支持更多语种和混合文本
            \item \textbf{端到端}:从流水线到一体化模型
            \item \textbf{大模型}:基于Transformer的OCR大模型
            \item \textbf{多模态}:结合视觉-语言理解
        \end{itemize}
    \end{columns}
\end{frame}

%===========================================================
% modules/18_ocr_classification.tex - OCR技术分类
%===========================================================

\section{OCR技术分类}

\begin{frame}{OCR技术多维分类体系}
    \begin{columns}
        \column{0.5\textwidth}
        \textbf{按技术路线分类:}
        \begin{itemize}
            \item \textbf{传统OCR}
            \begin{itemize}
                \item 模板匹配
                \item 特征工程
                \item 机器学习分类器
            \end{itemize}
            \item \textbf{深度学习OCR}
            \begin{itemize}
                \item CNN特征提取
                \item RNN序列建模
                \item Attention机制
                \item Transformer架构
            \end{itemize}
        \end{itemize}

        \vspace{0.3cm}

        \textbf{按输入类型分类:}
        \begin{itemize}
            \item \textbf{扫描文档OCR}
            \begin{itemize}
                \item 印刷体
                \item 手写体
                \item 表格
            \end{itemize}
            \item \textbf{场景文字OCR}
            \begin{itemize}
                \item 自然场景
                \item 街景文字
                \item 商品包装
            \end{itemize}
            \item \textbf{视频OCR}
            \begin{itemize}
                \item 字幕识别
                \item 实时识别
            \end{itemize}
        \end{itemize}

        \column{0.5\textwidth}
        \textbf{按识别对象分类:}
        \begin{itemize}
            \item \textbf{单字识别 vs 整行识别}
            \begin{itemize}
                \item 单字:独立分类
                \item 整行:序列建模
            \end{itemize}
            \item \textbf{印刷体 vs 手写体}
            \begin{itemize}
                \item 印刷体:规则、规范
                \item 手写体:个性化、变化大
            \end{itemize}
            \item \textbf{单语言 vs 多语言}
            \begin{itemize}
                \item 单语言:专用模型
                \item 多语言:统一模型
            \end{itemize}
        \end{itemize}

        \vspace{0.3cm}

        \textbf{按应用场景分类:}
        \begin{itemize}
            \item \textbf{文档数字化}
            \begin{itemize}
                \item 书籍扫描
                \item 档案数字化
                \item 票据处理
            \end{itemize}
            \item \textbf{证件识别}
            \begin{itemize}
                \item 身份证
                \item 银行卡
                \item 驾驶证
            \end{itemize}
            \item \textbf{移动应用}
            \begin{itemize}
                \item 拍照翻译
                \item 名片识别
                \item 文档扫描APP
            \end{itemize}
        \end{itemize}
    \end{columns}
\end{frame}

\begin{frame}{传统OCR vs 深度学习OCR对比}
    \begin{table}
        \centering
        \small
        \begin{tabular}{p{3cm}p{5cm}p{5cm}}
            \toprule
            \textbf{对比维度} & \textbf{传统OCR} & \textbf{深度学习OCR} \\
            \midrule
            \textbf{特征提取} &
            手工设计特征(HOG、LBP、投影) &
            自动学习特征(CNN) \\
            \midrule
            \textbf{模型设计} &
            分模块设计(检测、分割、识别) &
            端到端训练 & \\
            \midrule
            \textbf{数据依赖} &
            小样本即可训练 &
            需要大量标注数据 & \\
            \midrule
            \textbf{泛化能力} &
            对复杂场景适应性差 &
            强泛化能力,适应多场景 & \\
            \midrule
            \textbf{识别精度} &
            印刷体:高(>95%)
            手写体:中(70-85%) &
            印刷体:极高(>99%)
            手写体:高(85-95%) & \\
            \midrule
            \textbf{计算效率} &
            CPU即可实时处理 &
            通常需要GPU加速 & \\
            \midrule
            \textbf{可解释性} &
            强,每一步可分析 &
            弱,黑盒模型 & \\
            \midrule
            \textbf{维护成本} &
            高,需调多个模块参数 &
            中,主要调网络结构和超参 & \\
            \bottomrule
        \end{tabular}
    \end{table}
\end{frame}

%===========================================================
% modules/19_applications.tex - OCR应用场景
%===========================================================

\section{OCR应用场景}

\begin{frame}{OCR核心应用场景}
    \begin{columns}
        \column{0.5\textwidth}
        \textbf{1. 文档数字化}
        \begin{itemize}
            \item 书籍扫描与电子化
            \item 档案数字化管理
            \item 历史文献保护
            \item 图书馆数字资源建设
        \end{itemize}

        \vspace{0.3cm}

        \textbf{2. 证件识别}
        \begin{itemize}
            \item 身份证识别
            \item 银行卡识别
            \item 驾驶证/行驶证
            \item 护照识别
            \item 名片识别
        \end{itemize}

        \vspace{0.3cm}

        \textbf{3. 票据处理}
        \begin{itemize}
            \item 发票识别与验真
            \item 收据/小票识别
            \item 快递单据处理
            \item 财务报表录入
        \end{itemize}

        \column{0.5\textwidth}
        \textbf{4. 场景文字识别}
        \begin{itemize}
            \item 街景文字识别
            \item 路标/广告牌识别
            \item 商品包装文字
            \item 车牌识别
            \item 工业场景文字
        \end{itemize}

        \vspace{0.3cm}

        \textbf{5. 手写识别}
        \begin{itemize}
            \item 手写笔记数字化
            \item 表单手写内容
            \item 签名识别
            \item 邮件地址识别
            \item 教育作业批改
        \end{itemize}

        \vspace{0.3cm}

        \textbf{6. 智能应用}
        \begin{itemize}
            \item 拍照翻译
            \item 视觉搜索引擎
            \item 辅助阅读(视障)
            \item 智能阅卷
            \item 内容审核
        \end{itemize}
    \end{columns}
\end{frame}

\begin{frame}{行业OCR应用案例}
    \begin{table}
        \centering
        \small
        \begin{tabular}{p{2.5cm}p{4cm}p{5cm}}
            \toprule
            \textbf{行业} & \textbf{典型应用} & \textbf{技术特点} \\
            \midrule
            \textbf{金融服务} &
            支票识别、发票验真、卡证识别 &
            高精度、安全性、合规性 \\
            \midrule
            \textbf{医疗健康} &
            病历数字化、处方识别、报告录入 &
            手写体识别、医学术语、隐私保护 \\
            \midrule
            \textbf{教育培训} &
            智能阅卷、公式识别、笔记数字化 &
            手写识别、公式处理、批量处理 \\
            \midrule
            \textbf{物流快递} &
            面单识别、地址提取、分拣自动化 &
            实时处理、复杂背景、抗干扰 \\
            \midrule
            \textbf{零售电商} &
            商品标签、价格识别、小票处理 &
            多语言、复杂排版、移动端 \\
            \midrule
            \textbf{政府公共} &
            档案数字化、身份证识别、车牌识别 &
            大规模、标准化、高可靠 \\
            \midrule
            \textbf{交通出行} &
            车牌识别、路标识别、证件检查 &
            实时性、户外环境、移动场景 \\
            \bottomrule
        \end{tabular}
    \end{table}
\end{frame}

\begin{frame}{OCR技术选型指南}
    \begin{columns}
        \column{0.5\textwidth}
        \textbf{按场景选择OCR方案:}

        \vspace{0.3cm}

        \textbf{1. 印刷体文档识别}
        \begin{itemize}
            \item 推荐:Tesseract、PaddleOCR
            \item 特点:成熟稳定、多语言
            \item 适合:书籍、档案、报告
        \end{itemize}

        \vspace{0.2cm}

        \textbf{2. 场景文字识别}
        \begin{itemize}
            \item 推荐:PaddleOCR、MMOCR
            \item 特点:检测+识别端到端
            \item 适合:街景、商品、广告牌
        \end{itemize}

        \vspace{0.2cm}

        \textbf{3. 手写体识别}
        \begin{itemize}
            \item 推荐:PaddleOCR手写模型、TrOCR
            \item 特点:针对手写优化
            \item 适合:表单、笔记、作业
        \end{itemize}

        \column{0.5\textwidth}
        \textbf{4. 证件/票据识别}
        \begin{itemize}
            \item 推荐:商业OCR API
            \item 特点:高精度、结构化输出
            \item 适合:身份证、发票、银行卡
        \end{itemize}

        \vspace{0.2cm}

        \textbf{5. 实时视频OCR}
        \begin{itemize}
            \item 推荐:轻量级模型+优化
            \item 特点:速度快、资源占用低
            \item 适合:车牌识别、实时翻译
        \end{itemize}

        \vspace{0.3cm}

        \textbf{选型决策树:}

        \vspace{0.2cm}

        \begin{center}
            \begin{tikzpicture}[
                node distance=0.3cm,
                box/.style={rectangle, draw=blue!60, fill=blue!10, rounded corners, minimum width=2.5cm, minimum height=0.4cm, font=\tiny, align=center},
                arrow/.style={->, thick, font=\tiny}
            ]
                \node[box] (start) {开始选型};
                \node[box, below=0.5cm of start] (q1) {场景\\文字?};
                \node[box, below left=0.5cm and 0.3cm of q1] (scene) {PaddleOCR\\MMOCR};
                \node[box, below right=0.5cm and 0.3cm of q1] (q2) {手写体?};
                \node[box, below left=0.5cm and 0.3cm of q2] (hand) {TrOCR\\手写模型};
                \node[box, below right=0.5cm and 0.3cm of q2] (q3) {实时性?};
                \node[box, below left=0.5cm and 0.3cm of q3] (fast) {轻量模型\\PP-OCR};
                \node[box, below right=0.5cm and 0.3cm of q3] (normal) {Tesseract\\标准模型};

                \draw[arrow] (start) -- (q1);
                \draw[arrow] (q1) -- node[left, font=\tiny] {是} (scene);
                \draw[arrow] (q1) -- node[right, font=\tiny] {否} (q2);
                \draw[arrow] (q2) -- node[left, font=\tiny] {是} (hand);
                \draw[arrow] (q2) -- node[right, font=\tiny] {否} (q3);
                \draw[arrow] (q3) -- node[left, font=\tiny] {是} (fast);
                \draw[arrow] (q3) -- node[right, font=\tiny] {否} (normal);
            \end{tikzpicture}
        \end{center}
    \end{columns}
\end{frame}


%===========================================================
% 总结与作业
%===========================================================
%=============================================================================
% 总结与作业
%=============================================================================

\section{完整流程演示}

\begin{frame}[fragile]{阅卷系统完整流程(1/3):图像预处理}
    \textbf{目标:}将拍摄的试卷图像转换为适合处理的格式

    \begin{lstlisting}
import cv2
import numpy as np

# 1. 读取图像
img = cv2.imread('exam_paper.jpg')

# 2. 灰度化
gray = cv2.cvtColor(img, cv2.COLOR_BGR2GRAY)

# 3. 去噪
denoised = cv2.GaussianBlur(gray, (5, 5), 0)

# 4. 对比度增强
clahe = cv2.createCLAHE(clipLimit=2.0, tileGridSize=(8,8))
enhanced = clahe.apply(denoised)

# 5. 二值化
binary = cv2.adaptiveThreshold(enhanced, 255,
    cv2.ADAPTIVE_THRESH_GAUSSIAN_C,
    cv2.THRESH_BINARY, 11, 2)
    \end{lstlisting}

    \textbf{预处理步骤:}
    \begin{enumerate}
        \item \textbf{灰度化}:去除颜色信息,减少数据量
        \item \textbf{去噪}:去除扫描噪点
        \item \textbf{对比度增强}:CLAHE 自适应处理
        \item \textbf{二值化}:分离填涂和纸张
    \end{enumerate}
\end{frame}

\begin{frame}[fragile]{阅卷系统完整流程(2/3):定位答题卡}
    \textbf{目标:}在试卷中找到答题卡区域

    \begin{lstlisting}
# 1. 边缘检测
edges = cv2.Canny(binary, 50, 150)

# 2. 查找轮廓
contours, _ = cv2.findContours(edges, cv2.RETR_EXTERNAL, cv2.CHAIN_APPROX_SIMPLE)

# 3. 筛选矩形轮廓
for cnt in contours:
    peri = cv2.arcLength(cnt, True)
    approx = cv2.approxPolyDP(cnt, 0.02 * peri, True)
    if len(approx) == 4:
        x, y, w, h = cv2.boundingRect(cnt)
        answer_sheet = binary[y:y+h, x:x+w]
        break
    \end{lstlisting}
\end{frame}

\begin{frame}[fragile]{阅卷系统完整流程(3/3):识别填涂}
    \textbf{目标:}识别每个选项是否被填涂

    \begin{lstlisting}
# 定义每个选项的位置
options = [
    (50, 30, 80, 60),   # 第1题 A
    (90, 30, 120, 60),  # 第1题 B
]

results = []
for (x1, y1, x2, y2) in options:
    option = answer_sheet[y1:y2, x1:x2]
    black_ratio = np.sum(option == 0) / option.size
    is_filled = black_ratio > 0.3  # 阈值30%
    results.append(is_filled)
    \end{lstlisting}

    \textbf{识别算法优化:}
    \begin{itemize}
        \item 阈值 30\% 是经验值,可根据实际情况调整
        \item 检测是否多选,给出警告
        \item 未填涂记录,后续人工复核
    \end{itemize}
\end{frame}

% -----------------------------------------------------------------------------
% 互动测验
% -----------------------------------------------------------------------------

\section{互动测验}

\begin{frame}[fragile]{快速问答环节}
    \begin{columns}
        \column{0.5\textwidth}
        \textbf{问题1:OpenCV默认读取的彩色图像是什么顺序?}
        \begin{itemize}
            \item[A] RGB
            \item[B] \highlight{BGR}(正确)
            \item[C] HSV
            \item[D] LAB
        \end{itemize}

        \vspace{0.3cm}
        \textbf{问题2:如何判断一个图像是否读取成功?}
        \begin{itemize}
            \item[A] if img != None
            \item[B] \highlight{if img is not None}(正确)
            \item[C] if img.exists()
            \item[D] if len(img) > 0
        \end{itemize}

        \column{0.5\textwidth}
        \textbf{问题3:uint8类型的像素值范围是?}
        \begin{itemize}
            \item[A] 0-1023
            \item[B] \highlight{0-255}(正确)
            \item[C] -128-127
            \item[D] 0-65535
        \end{itemize}

        \vspace{0.3cm}
        \textbf{问题4:图像像素相加时,如何避免溢出?}
        \begin{itemize}
            \item[A] 直接相加
            \item[B] \highlight{cv2.add() 或 np.clip()}(正确)
            \item[C] 转为float后相加
            \item[D] 无需处理
        \end{itemize}
    \end{columns}

    \vspace{0.5cm}
    \begin{center}
        \highlight{正确率:\uncover<2->{\textbf{100\%} 🎉}}
    \end{center}
\end{frame}

\begin{frame}[fragile]{小测验时间(2):代码找错挑战}
    \textbf{找出以下代码中的3个错误:}

    \begin{lstlisting}
import cv2
import numpy as np

# 读取图像
img = cv2.imread('张三试卷.jpg')  # 错误1

# 亮度增加50
bright_img = img + 50  # 错误2

# 显示
plt.imshow(img)  # 错误3
plt.show()
    \end{lstlisting}

    \vspace{0.3cm}
    \begin{block}{答案揭晓}
        \begin{enumerate}
            \item \textbf{错误1:} 中文路径问题。需要使用\texttt{imread\_chinese()}函数
            \item \textbf{错误2:} 直接相加会导致溢出。应该使用\texttt{cv2.add(img, np.array([50.0]))}
            \item \textbf{错误3:} OpenCV是BGR,matplotlib是RGB。应该先转换\texttt{img = cv2.cvtColor(img, cv2.COLOR\_BGR2RGB)}
        \end{enumerate}
    \end{block}

    \vspace{0.3cm}
    \begin{center}
        \textbf{修正后的代码:}
        \begin{lstlisting}
img = imread_chinese('张三试卷.jpg')
bright_img = cv2.add(img, np.array([50.0]))
img_rgb = cv2.cvtColor(img, cv2.COLOR_BGR2RGB)
plt.imshow(img_rgb)
        \end{lstlisting}
    \end{center}
\end{frame}

% -----------------------------------------------------------------------------
% 课后作业
% -----------------------------------------------------------------------------

\section{课后作业}

\begin{frame}[fragile]{课后作业:我的第一个图像处理器}
    \begin{exampleblock}{作业要求}
        编写一个 Python 脚本,读取一张照片并生成一张包含 4 张子图的对比图:
        \begin{enumerate}
            \item 原图
            \item 灰度图
            \item 亮度增强后的图
            \item 反色后的图
        \end{enumerate}
    \end{exampleblock}
    \textbf{提交方式:} 截图 + 代码
\end{frame}

\begin{frame}{知识点网络与下周预告}
    \begin{columns}
        \column{0.6\textwidth}
        \textbf{本周核心知识点:}
        \begin{itemize}
            \item 计算机视觉基本概念
            \item 图像的数字表示(矩阵、RGB)
            \item OpenCV基础操作
            \item 图像预处理(灰度、二值化、去噪)
            \item 阅卷系统入门
        \end{itemize}

        \vspace{0.3cm}
        \textbf{下周预告(Week 2):AI辅助编程工具实战}
        \begin{itemize}
            \item \textbf{ChatGPT/Claude}:学习编程的AI助手
            \item \textbf{Prompt工程}:如何让AI帮我们写代码
            \item \textbf{实战演练}:用AI辅助实现人脸检测
        \end{itemize}

        \column{0.4\textwidth}
        \begin{block}{跨周链接}
            \begin{itemize}
                \item Week 1:图像基础 ⚙️
                \item Week 2:AI工具 🤖
                \item Week 3:图像预处理(深度) 🖼️
                \item Week 4:版面分析 📄
                \item Week 5:选择题识别 ⭕
            \end{itemize}
        \end{block}

        \vspace{0.3cm}
        \begin{alertblock}{重点提示}
        Week 2我们将学习如何用\textbf{AI工具}来加速Week 1学到的OpenCV代码开发!
        \end{alertblock}
    \end{columns}
\end{frame}

\begin{frame}{总结与问答}
    \begin{center}
        \Huge \textbf{Q \& A}
        \vspace{1cm}
        \Large 准备好进入计算机视觉的世界了吗?
    \end{center}
\end{frame}


\end{document}
