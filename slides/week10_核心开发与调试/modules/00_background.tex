%===========================================================
% 模块00:背景知识铺垫 (10-12页)
%===========================================================

\section{背景知识}

%----------------------------------------------------------
% 前置知识检查(新增)
%----------------------------------------------------------

\begin{frame}{前置知识自检}
    \textbf{学习本课程前,请确认已掌握以下知识:}

    \begin{columns}
        \begin{column}{0.48\textwidth}
            \textbf{Python基础}
            \begin{itemize}
                \item 变量、数据类型、运算符
                \item 条件判断(if/elif/else)
                \item 循环结构(for/while)
                \item 函数定义与调用
                \item 类与对象基础
            \end{itemize}

            \vspace{0.3cm}

            \textbf{文件操作}
            \begin{itemize}
                \item 读取/写入文本文件
                \item 路径处理(os.path)
            \end{itemize}
        \end{column}
        \begin{column}{0.48\textwidth}
            \textbf{版本控制基础}
            \begin{itemize}
                \item Git基本概念(仓库、提交)
                \item 基本命令(add/commit/push)
                \item 分支概念(可选)
            \end{itemize}

            \vspace{0.3cm}

            \textbf{数学基础}
            \begin{itemize}
                \item 矩阵/数组基本概念
                \item 百分比计算
                \item 集合论基础(可选)
            \end{itemize}
        \end{column}
    \end{columns}

    \vspace{0.3cm}

    \begin{alertblock}{如需补充学习}
        \begin{itemize}
            \item Python入门:\href{https://www.liaoxuefeng.com/wiki/1016959663602400}{廖雪峰Python教程}
            \item Git入门:\href{https://git-scm.com/book/zh/v2}{Pro Git中文版}
        \end{itemize}
    \end{alertblock}
\end{frame}

\begin{frame}{本课程学习路径}
    \textbf{三条学习路径,适合不同基础的你:}

    \begin{center}
        \begin{tikzpicture}[
            node distance=0.6cm,
            path/.style={draw, fill=blue!10, minimum width=2.2cm, minimum height=0.7cm, align=center, font=\small},
            arrow/.style={-Stealth, thick}
        ]
            % 三条路径
            \node[path, fill=green!20] (observe) {观察者路径\\基础60分};
            \node[path, fill=yellow!20, right=0.8cm of observe] (use) {使用者路径\\进阶+20分};
            \node[path, fill=red!20, right=0.8cm of use] (create) {创造者路径\\挑战+20分};

            % 适用人群标签
            \node[below=0.2cm of observe, font=\fontsize{8}{9.6}\selectfont] {适合:编程基础较弱};
            \node[below=0.2cm of use, font=\fontsize{8}{9.6}\selectfont] {适合:有编程基础};
            \node[below=0.2cm of create, font=\fontsize{8}{9.6}\selectfont] {适合:编程能力较强};

            % 学习建议框
            \node[below=1.8cm of observe, draw, fill=blue!5, minimum width=10cm, minimum height=1.2cm, align=left, font=\small] {
                \textbf{学习建议:}
                \begin{itemize}
                    \item \textbf{观察者}:重点学习模块1-2,直接运行代码示例
                    \item \textbf{使用者}:完成全部模块,重点实践模块3-4
                    \item \textbf{创造者}:深入学习全部内容,完成模块4挑战任务
                \end{itemize}
            };
        \end{tikzpicture}
    \end{center}
\end{frame}

%----------------------------------------------------------
\subsection{本周预习资源}
%----------------------------------------------------------

\begin{frame}{本周预习资源}
    \begin{block}{课前5分钟视频(必须观看)}
        \begin{itemize}
            \item \textbf{视频1:} 软件工程基础(模块化与接口设计)
            \begin{itemize}
                \item 什么是模块化?为什么要模块化?
                \item 接口设计的基本原则
            \end{itemize}
            \item \textbf{视频2:} 团队协作基础(Git工作流与代码规范)
            \begin{itemize}
                \item Git基本概念:仓库、分支、提交
                \item 代码规范的重要性(PEP 8简介)
            \end{itemize}
        \end{itemize}
    \end{block}

    \vspace{0.3cm}

    \textbf{预习目标:}
    \begin{itemize}
        \item 理解模块化设计的好处(高内聚低耦合)
        \item 掌握Git基本工作流(add/commit/push/pull)
        \item 了解代码规范的重要性(PEP 8)
    \end{itemize}

    \vspace{0.3cm}

    \textbf{预习检查:}
    \begin{itemize}
        \item[$\square$] 能说出模块化的三个好处
        \item[$\square$] 能完成一次Git提交
        \item[$\square$] 能识别代码中的常见规范问题
    \end{itemize}
\end{frame}

%----------------------------------------------------------
\subsection{软件开发流程}
%----------------------------------------------------------

\begin{frame}{软件开发生命周期(SDLC)}
    \begin{columns}
        \begin{column}{0.5\textwidth}
            \textbf{经典阶段:}
            \begin{enumerate}
                \item \textbf{需求分析}:明确系统功能与性能要求
                \item \textbf{系统设计}:架构设计与模块划分
                \item \textbf{编码实现}:将设计转化为代码
                \item \textbf{测试验证}:确保质量符合预期
                \item \textbf{部署运维}:交付并持续维护
            \end{enumerate}
        \end{column}
        \begin{column}{0.5\textwidth}
            \begin{tikzpicture}[
                node distance=0.8cm,
                box/.style={draw, fill=blue!20, minimum width=3cm, minimum height=0.6cm, align=center, font=\small}
            ]
                \node[box] (req) {需求分析};
                \node[box, below=of req] (des) {系统设计};
                \node[box, below=of des] (cod) {编码实现};
                \node[box, below=of cod] (tes) {测试验证};
                \node[box, below=of tes] (dep) {部署运维};

                \draw[-Stealth, thick] (req) -- (des);
                \draw[-Stealth, thick] (des) -- (cod);
                \draw[-Stealth, thick] (cod) -- (tes);
                \draw[-Stealth, thick] (tes) -- (dep);

                \draw[-Stealth, thick, dashed, color=red] (dep.east) to[bend right=60] (req.east);
                \node[right=0.2cm of dep, font=\tiny, color=red] {迭代};
            \end{tikzpicture}
        \end{column}
    \end{columns}
\end{frame}

\begin{frame}{开发模型对比}
    \begin{columns}
        \begin{column}{0.48\textwidth}
            \textbf{瀑布模型}
            \begin{itemize}
                \item 阶段清晰,顺序执行
                \item 文档驱动,易于管理
                \item 变更成本高
                \item 适合需求明确的项目
            \end{itemize}

            \vspace{0.3cm}

            \textbf{适用场景:}
            \begin{itemize}
                \item 政府、金融项目
                \item 安全关键系统
            \end{itemize}
        \end{column}
        \begin{column}{0.48\textwidth}
            \textbf{敏捷开发}
            \begin{itemize}
                \item 迭代增量,快速交付
                \item 拥抱变化,持续反馈
                \item 轻量文档,重视协作
                \item 适合需求易变项目
            \end{itemize}

            \vspace{0.3cm}

            \textbf{适用场景:}
            \begin{itemize}
                \item 互联网产品
                \item 创新实验项目
            \end{itemize}
        \end{column}
    \end{columns}
\end{frame}

\begin{frame}{敏捷开发核心实践}
    \textbf{Scrum框架:}
    \vspace{0.3cm}

    \begin{center}
        \begin{tikzpicture}[
            node distance=0.3cm,
            box/.style={draw, fill=green!20, minimum width=2.5cm, minimum height=0.7cm, align=center, font=\footnotesize}
        ]
            \node[box] (backlog) {产品\\待办列表};
            \node[box, right=of backlog] (sprint) {Sprint\\待办列表};
            \node[box, right=of sprint] (work) {Sprint\\执行};
            \node[box, right=of work] (review) {评审与\\回顾};

            \draw[-Stealth, thick] (backlog) -- (sprint);
            \draw[-Stealth, thick] (sprint) -- (work);
            \draw[-Stealth, thick] (work) -- (review);
            \draw[-Stealth, thick, dashed] (review.south) to[bend right=45] (backlog.south);
        \end{tikzpicture}
    \end{center}

    \vspace{0.3cm}

    \textbf{关键概念:}
    \begin{itemize}
        \item \textbf{Sprint}:2-4周的迭代周期
        \item \textbf{每日站会}:15分钟同步进展
        \item \textbf{迭代评审}:演示成果,收集反馈
        \item \textbf{回顾会议}:持续改进流程
    \end{itemize}
\end{frame}

\begin{frame}{持续集成与持续部署(CI/CD)}
    \textbf{核心理念:自动化一切可以自动化的}

    \vspace{0.3cm}

    \begin{center}
        \begin{tikzpicture}[
            node distance=0.5cm,
            box/.style={draw, fill=green!20, minimum width=2cm, minimum height=0.8cm, align=center, font=\small},
            arrow/.style={-Stealth, thick}
        ]
            \node[box] (code) {代码提交};
            \node[box, right=of code] (build) {自动构建};
            \node[box, right=of build] (test) {自动测试};
            \node[box, right=of test] (deploy) {自动部署};

            \draw[arrow] (code) -- (build);
            \draw[arrow] (build) -- (test);
            \draw[arrow] (test) -- (deploy);

            \node[below=0.3cm of build, font=\tiny] {编译、打包};
            \node[below=0.3cm of test, font=\tiny] {单元/集成测试};
            \node[below=0.3cm of deploy, font=\tiny] {发布上线};
        \end{tikzpicture}
    \end{center}

    \vspace{0.3cm}

    \textbf{关键实践:}
    \begin{itemize}
        \item \textbf{版本控制}:Git管理代码,分支策略规范
        \item \textbf{自动化构建}:Jenkins、GitHub Actions、GitLab CI
        \item \textbf{自动化测试}:每次提交触发测试套件
        \item \textbf{环境一致性}:Docker容器化部署
    \end{itemize}
\end{frame}

\begin{frame}{版本控制最佳实践}
    \textbf{Git分支策略:}
    \vspace{0.3cm}

    \begin{columns}
        \begin{column}{0.48\textwidth}
            \textbf{Git Flow模型:}
            \begin{itemize}
                \item \texttt{main/master}:生产环境
                \item \texttt{develop}:开发集成分支
                \item \texttt{feature/*}:功能开发分支
                \item \texttt{release/*}:发布准备分支
                \item \texttt{hotfix/*}:紧急修复分支
            \end{itemize}
        \end{column}
        \begin{column}{0.48\textwidth}
            \textbf{Commit Message规范:}
            \begin{itemize}
                \item \texttt{feat:} 新功能
                \item \texttt{fix:} 修复bug
                \item \texttt{docs:} 文档更新
                \item \texttt{refactor:} 重构
                \item \texttt{test:} 测试相关
                \item \texttt{chore:} 构建/工具
            \end{itemize}
        \end{column}
    \end{columns}

    \vspace{0.5cm}

    \begin{exampleblock}{示例:规范的Commit}
        \texttt{feat(answer-sheet): 添加答题卡区域检测算法}\\
        \texttt{- 实现基于轮廓的区域分割}\\
        \texttt{- 支持A4、B5多种纸张尺寸}
    \end{exampleblock}
\end{frame}

%----------------------------------------------------------
\subsection{代码质量管理}
%----------------------------------------------------------

\begin{frame}{代码规范与风格}
    \textbf{为什么需要代码规范?}

    \begin{itemize}
        \item \textbf{可读性}:代码被阅读的次数远多于编写
        \item \textbf{可维护性}:团队协作用统一"语言"
        \item \textbf{减少错误}:规范规避常见陷阱
    \end{itemize}

    \vspace{0.3cm}

    \textbf{Python代码规范(PEP 8):}
    \begin{itemize}
        \item 缩进:4个空格(禁用Tab)
        \item 行长度:每行不超过79字符
        \item 命名:函数/变量用小写下划线,类用大驼峰
        \item 空行:类和函数间空两行,方法间空一行
    \end{itemize}

    \vspace{0.3cm}

    \textbf{命名规范示例:}
    \begin{itemize}
        \item \texttt{class AnswerSheetDetector:} (类名大驼峰)
        \item \texttt{def detect\_regions(image):} (函数名小写+下划线)
        \item \texttt{MAX\_SCORE = 100} (常量大写)
    \end{itemize}
\end{frame}

\begin{frame}{代码审查(Code Review)}
    \begin{columns}
        \begin{column}{0.5\textwidth}
            \textbf{审查内容:}
            \begin{itemize}
                \item 代码正确性
                \item 逻辑清晰度
                \item 性能影响
                \item 安全漏洞
                \item 测试覆盖
                \item 文档完整性
            \end{itemize}
        \end{column}
        \begin{column}{0.5\textwidth}
            \textbf{审查原则:}
            \begin{itemize}
                \item 对事不对人
                \item 小步快跑(PR要小)
                \item 及时响应(24小时内)
                \item 记录决策(为什么修改)
            \end{itemize}
        \end{column}
    \end{columns}

    \vspace{0.3cm}

    \begin{alertblock}{代码审查清单示例}
        \begin{itemize}
            \item[$\square$] 是否处理了边界情况?
            \item[$\square$] 是否有适当的错误处理?
            \item[$\square$] 命名是否清晰表达意图?
            \item[$\square$] 是否有重复代码可以提取?
            \item[$\square$] 是否添加了必要的测试?
        \end{itemize}
    \end{alertblock}
\end{frame}

\begin{frame}{技术债务管理}
    \textbf{什么是技术债务?}

    \begin{quote}
        "为了快速交付而采取的非最优技术方案,未来需要额外成本来偿还。" —— Ward Cunningham
    \end{quote}

    \vspace{0.3cm}

    \textbf{技术债务类型:}
    \begin{itemize}
        \item \textbf{有意债务}:战略性地选择快速方案,计划偿还
        \item \textbf{无意债务}:由于缺乏经验或知识而引入
        \item \textbf{过期债务}:原本合理的方案随时间变得过时
    \end{itemize}

    \vspace{0.3cm}

    \textbf{管理策略:}
    \begin{itemize}
        \item 债务可视化:记录和追踪技术债务
        \item 定期偿还:每个Sprint预留20\%时间重构
        \item 防止累积:代码审查时识别新增债务
    \end{itemize}
\end{frame}

\begin{frame}{重构与优化}
    \textbf{重构定义:}
    \begin{quote}
        "在不改变代码外部行为的前提下,改善其内部结构。" —— Martin Fowler
    \end{quote}

    \vspace{0.3cm}

    \textbf{重构时机:}
    \begin{itemize}
        \item \textbf{规则一}:添加功能时
        \item \textbf{规则二}:修复bug时
        \item \textbf{规则三}:代码审查时
    \end{itemize}

    \vspace{0.3cm}

    \textbf{常见重构技术:}
    \begin{columns}
        \begin{column}{0.48\textwidth}
            \begin{itemize}
                \item 提取函数(Extract Method)
                \item 内联函数(Inline Method)
                \item 提取类(Extract Class)
                \item 搬移函数(Move Method)
            \end{itemize}
        \end{column}
        \begin{column}{0.48\textwidth}
            \begin{itemize}
                \item 重命名(Rename)
                \item 引入参数对象
                \item 替换魔法数字
                \item 简化条件表达式
            \end{itemize}
        \end{column}
    \end{columns}
\end{frame}

%----------------------------------------------------------
\subsection{调试与排错}
%----------------------------------------------------------

\begin{frame}{调试的基本概念}
    \textbf{调试(Debugging)}:识别、定位和修复软件缺陷的过程

    \vspace{0.3cm}

    \begin{columns}
        \begin{column}{0.48\textwidth}
            \textbf{常见错误类型:}
            \begin{itemize}
                \item \textbf{语法错误}:编译/解析失败
                \item \textbf{运行时错误}:异常、崩溃
                \item \textbf{逻辑错误}:结果不正确
                \item \textbf{性能错误}:运行过慢
                \item \textbf{并发错误}:时序相关、难复现
            \end{itemize}
        \end{column}
        \begin{column}{0.48\textwidth}
            \textbf{调试的本质:}
            \begin{itemize}
                \item 建立假设
                \item 收集证据
                \item 验证假设
                \item 修正理解
            \end{itemize}
        \end{column}
    \end{columns}
\end{frame}

\begin{frame}{调试策略与方法}
    \textbf{系统化调试方法:}

    \vspace{0.3cm}

    \begin{enumerate}
        \item \textbf{复现问题}
            \begin{itemize}
                \item 找到稳定复现问题的步骤
                \item 最小化复现案例(Minimal Reproduction)
            \end{itemize}

        \item \textbf{定位根因}
            \begin{itemize}
                \item 二分法缩小范围
                \item 使用调试器或日志追踪
            \end{itemize}

        \item \textbf{验证假设}
            \begin{itemize}
                \item 临时修改测试假设
                \item 添加断言验证状态
            \end{itemize}

        \item \textbf{修复并验证}
            \begin{itemize}
                \item 修复问题
                \item 确保修复不引入新问题
            \end{itemize}
    \end{enumerate}
\end{frame}

\begin{frame}{日志与监控}
    \textbf{日志记录的重要性:}

    \begin{itemize}
        \item 生产环境调试的唯一手段
        \item 系统行为的历史记录
        \item 问题诊断的关键证据
    \end{itemize}

    \vspace{0.3cm}

    \textbf{日志级别:}
    \begin{table}
        \centering
        \small
        \begin{tabular}{lll}
            \toprule
            \textbf{级别} & \textbf{用途} & \textbf{示例} \\
            \midrule
            DEBUG & 详细调试信息 & 变量值、执行路径 \\
            INFO & 正常操作记录 & 服务启动、配置加载 \\
            WARNING & 警告,可能有问题 & 资源即将耗尽 \\
            ERROR & 错误,功能受影响 & 文件读取失败 \\
            CRITICAL & 严重错误,系统崩溃 & 数据库连接断开 \\
            \bottomrule
        \end{tabular}
    \end{table}

    \vspace{0.2cm}

    \begin{block}{日志最佳实践}
        \begin{itemize}
            \item 使用结构化格式(JSON)
            \item 包含关键上下文信息(时间、用户、请求ID)
            \item 避免敏感信息(密码、Token)
            \item 设置合理的日志轮转策略
        \end{itemize}
    \end{block}
\end{frame}

\begin{frame}{性能监控与分析}
    \textbf{关键性能指标(KPI):}

    \begin{columns}
        \begin{column}{0.48\textwidth}
            \textbf{响应时间}
            \begin{itemize}
                \item 平均响应时间
                \item P95/P99延迟
                \item 超时率
            \end{itemize}

            \vspace{0.2cm}

            \textbf{吞吐量}
            \begin{itemize}
                \item QPS/TPS
                \item 并发用户数
                \item 峰值处理能力
            \end{itemize}
        \end{column}
        \begin{column}{0.48\textwidth}
            \textbf{资源使用}
            \begin{itemize}
                \item CPU使用率
                \item 内存占用
                \item 磁盘IO
                \item 网络流量
            \end{itemize}

            \vspace{0.2cm}

            \textbf{错误率}
            \begin{itemize}
                \item 异常比例
                \item 失败请求率
                \item 服务可用性
            \end{itemize}
        \end{column}
    \end{columns}

    \vspace{0.3cm}

    \textbf{监控工具链:}
    \begin{itemize}
        \item Prometheus + Grafana:指标采集与可视化
        \item ELK Stack:日志分析与搜索
        \item Jaeger/Zipkin:分布式追踪
    \end{itemize}
\end{frame}

%----------------------------------------------------------
% 本模块要点速查(新增)
%----------------------------------------------------------

\subsection{本模块要点速查}

\begin{frame}{模块00-背景知识:要点速查}
    \begin{block}{核心概念}
        \begin{itemize}
            \item \textbf{SDLC}:软件开发生命周期(需求→设计→编码→测试→部署)
            \item \textbf{敏捷开发}:迭代增量,快速反馈(2-4周Sprint)
            \item \textbf{CI/CD}:持续集成与持续部署(自动化构建、测试、发布)
            \item \textbf{Code Review}:代码审查,对事不对人,小步快跑
            \item \textbf{重构}:不改变外部行为,改善内部结构
        \end{itemize}
    \end{block}

    \begin{alertblock}{关键公式}
        \begin{itemize}
            \item \textbf{高内聚} = 模块内元素紧密相关
            \item \textbf{低耦合} = 模块间依赖最小
        \end{itemize}
    \end{alertblock}

    \begin{exampleblock}{下一步学习}
        进入核心理论模块,学习模块化设计和面向对象编程
    \end{exampleblock}
\end{frame}
