%===========================================================
% 补充模块:渐进式学习增强内容
%===========================================================

%----------------------------------------------------------
% 前置知识检查(在00_background.tex开头插入)
%----------------------------------------------------------

\begin{frame}{前置知识自检}
    \textbf{学习本课程前,请确认已掌握以下知识:}

    \begin{columns}
        \begin{column}{0.48\textwidth}
            \textbf{Python基础}
            \begin{itemize}
                \item 变量、数据类型、运算符
                \item 条件判断(if/elif/else)
                \item 循环结构(for/while)
                \item 函数定义与调用
                \item 类与对象基础
            \end{itemize}

            \vspace{0.3cm}

            \textbf{文件操作}
            \begin{itemize}
                \item 读取/写入文本文件
                \item 路径处理(os.path)
            \end{itemize}
        \end{column}
        \begin{column}{0.48\textwidth}
            \textbf{版本控制基础}
            \begin{itemize}
                \item Git基本概念(仓库、提交)
                \item 基本命令(add/commit/push)
                \item 分支概念(可选)
            \end{itemize}

            \vspace{0.3cm}

            \textbf{数学基础}
            \begin{itemize}
                \item 矩阵/数组基本概念
                \item 百分比计算
                \item 集合论基础(可选)
            \end{itemize}
        \end{column}
    \end{columns}

    \vspace{0.3cm}

    \begin{alertblock}{如需补充学习}
        \begin{itemize}
            \item Python入门:\href{https://www.liaoxuefeng.com/wiki/1016959663602400}{廖雪峰Python教程}
            \item Git入门:\href{https://git-scm.com/book/zh/v2}{Pro Git中文版}
        \end{itemize}
    \end{alertblock}
\end{frame}

%----------------------------------------------------------
% 学习路径导航
%----------------------------------------------------------

\begin{frame}{本课程学习路径}
    \textbf{三条学习路径,适合不同基础的你:}

    \begin{center}
        \begin{tikzpicture}[
            node distance=0.6cm,
            path/.style={draw, fill=blue!10, minimum width=2.2cm, minimum height=0.8cm, align=center, font=\small},
            arrow/.style={-Stealth, thick}
        ]
            % 三条路径
            \node[path, fill=green!20] (observe) {观察者路径\\基础60分};
            \node[path, fill=yellow!20, right=0.8cm of observe] (use) {使用者路径\\进阶+20分};
            \node[path, fill=red!20, right=0.8cm of use] (create) {创造者路径\\挑战+20分};

            % 适用人群标签
            \node[below=0.2cm of observe, font=\fontsize{8}{9}\selectfont] {适合:编程基础较弱};
            \node[below=0.2cm of use, font=\fontsize{8}{9}\selectfont] {适合:有编程基础};
            \node[below=0.2cm of create, font=\fontsize{8}{9}\selectfont] {适合:编程能力较强};

            % 学习建议框
            \node[below=1.8cm of observe, draw, fill=blue!5, minimum width=10cm, minimum height=1.2cm, align=left, font=\small] {
                \textbf{学习建议:}
                \begin{itemize}
                    \item \textbf{观察者}:重点学习模块1-2,直接运行代码示例
                    \item \textbf{使用者}:完成全部模块,重点实践模块3-4
                    \item \textbf{创造者}:深入学习全部内容,完成模块4挑战任务
                \end{itemize}
            };
        \end{tikzpicture}
    \end{center}
\end{frame}

%----------------------------------------------------------
% 术语表(在核心理论模块末尾插入)
%----------------------------------------------------------

\begin{frame}{核心术语速查}
    \begin{table}
        \centering
        \small
        \resizebox{\linewidth}{!}{
        \begin{tabular}{lp{6cm}}
            \toprule
            \textbf{术语} & \textbf{含义} \\
            \midrule
            \textbf{模块化} & 将系统分解为独立、可复用的部分 \\
            \textbf{重构} & 不改变功能,优化代码内部结构 \\
            \textbf{高内聚低耦合} & 模块内部紧密相关,模块间依赖最小 \\
            \textbf{TDD} & 测试驱动开发:先写测试,再写代码 \\
            \textbf{Mock} & 模拟对象,隔离外部依赖进行测试 \\
            \textbf{Code Smell} & 代码中的设计问题指示信号 \\
            \textbf{断点调试} & 在指定行暂停,检查程序状态 \\
            \textbf{覆盖率} & 测试用例覆盖代码的比例 \\
            \bottomrule
        \end{tabular}
        }
    \end{table}
\end{frame}

%----------------------------------------------------------
% 非计算机专业补充说明框
%----------------------------------------------------------

\begin{frame}{概念补充:什么是重构?}
    \begin{columns}
        \begin{column}{0.48\textwidth}
            \textbf{生活类比}

            \begin{block}{整理房间}
                \begin{itemize}
                    \item 房间很乱(代码可读性差)
                    \item 不改变房间里的东西(功能不变)
                    \item 只是重新整理收纳(优化结构)
                    \item 目的是更容易找到东西(更容易维护)
                \end{itemize}
            \end{block}
        \end{column}
        \begin{column}{0.48\textwidth}
            \textbf{代码示例}

            \begin{block}{重构前}
                \lstinline|def f(x):|\\
                \lstinline|# 100行代码|
            \end{block}

            \begin{block}{重构后}
                \lstinline|def process():|\\
                \lstinline|load()|\\
                \lstinline|process_data()|\\
                \lstinline|save()|
            \end{block}
        \end{column}
    \end{columns}
\end{frame}

\begin{frame}{概念补充:什么是测试驱动开发?}
    \textbf{想象学做菜的过程:}

    \begin{center}
        \begin{tikzpicture}[
            node distance=0.5cm,
            box/.style={draw, fill=blue!10, minimum width=3cm, minimum height=1cm, align=center, font=\small},
            arrow/.style={-Stealth, thick}
        ]
            \node[box, fill=red!20] (red) {1. 先定标准\\(写测试)};
            \node[box, fill=green!20, right=of red] (green) {2. 做最简单能通过的\\(写最少代码)};
            \node[box, fill=yellow!20, right=of green] (refactor) {3. 改进做法\\(重构优化)};

            \draw[arrow] (red) -- (green);
            \draw[arrow] (green) -- (refactor);
            \draw[arrow, dashed] (refactor.south) to[bend right=45] (red.south);
        \end{tikzpicture}
    \end{center}

    \begin{block}{好处}
        \begin{itemize}
            \item 确保做出来的东西符合要求(测试通过)
            \item 避免过度设计(只写最少代码)
            \item 持续改进质量(每次都优化一点)
        \end{itemize}
    \end{block}
\end{frame}

%----------------------------------------------------------
% 课堂Quiz(在案例模块增加)
%----------------------------------------------------------

\begin{frame}{课堂Quiz 1:识别Code Smell}
    \textbf{以下函数有什么问题?(单选)}

    \begin{columns}
        \begin{column}{0.55\textwidth}
            \begin{block}{代码}
\begin{verbatim}
def p(img, thresh=127, mode=1,
      debug=False):
    # 处理图片
    # 50行代码...
    # 如果调试模式
    if debug:
        print("处理中...")
    return result
\end{verbatim}
            \end{block}
        \end{column}
        \begin{column}{0.4\textwidth}
            \begin{block}{选项}
                A. 语法错误

                B. 参数太多

                C. 函数名不规范

                D. 以上都是
            \end{block}

            \vspace{0.5cm}

            \textbf{答案:D}\\$\newline$
            (函数名p应改为process_image,参数过多应封装为配置类)
        \end{column}
    \end{columns}
\end{frame}

\begin{frame}{课堂Quiz 2:异常处理}
    \textbf{以下异常处理方式最好的是?}

    \begin{columns}
        \begin{column}{0.48\textwidth}
            \textbf{选项A}
\begin{Shaded}
try:
    f()
except:
    pass
\end{Shaded}
        \end{column}
        \begin{column}{0.48\textwidth}
            \textbf{选项B}
\begin{Shaded}
try:
    f()
except ValueError as e:
    logger.error(e)
    raise
except Exception:
    logger.exception()
\end{Shaded}
        \end{column}
    \end{columns}

    \begin{block}{}
        \textbf{答案:B}\\$\newline$
        原因:1. 不捕获所有异常(避免掩盖bug)\\
        2. 记录错误信息\\
        3. 重新抛出异常让上层处理
    \end{block}
\end{frame}

%----------------------------------------------------------
% 现场练习设计(在Live Coding模块增加)
%----------------------------------------------------------

\begin{frame}{现场练习:识别Bug}
    \textbf{任务}:找出以下代码中的Bug(3分钟)

    \begin{columns}
        \begin{column}{0.5\textwidth}
\begin{Shaded}
def calculate_density(regions):
    results = []
    for region in regions:
        if region is not None:
            density = sum(region) / len(region)
            results.append(density)
    return results

# 使用
img = None
result = calculate_density([img])
print(result)
\end{Shaded}
        \end{column}
        \begin{column}{0.5\textwidth}
            \textbf{提示}:运行这段代码会发生什么?

            \vspace{0.3cm}

            \textbf{Bug分析}:
            \begin{itemize}
                \item None对象无法参与运算
                \item 应在循环内检查,而非循环前
                \item 缺少异常处理
            \end{itemize}
        \end{column}
    \end{columns}
\end{frame}

\begin{frame}{现场练习:完善代码}
    \textbf{任务}:完善以下代码,添加错误处理(5分钟)

    \begin{block}{待完善代码}
\begin{Shaded}
def recognize_choice(image, options):
    results = {}
    for option in options:
        region = image[option]
        density = np.sum(region > 127) / region.size
        results[option] = density
    return results
\end{Shaded}
    \end{block}

    \begin{columns}
        \begin{column}{0.48\textwidth}
            \textbf{需要处理的情况:}
            \begin{itemize}
                \item image为None
                \item option不在image中
                \item region为空
            \end{itemize}
        \end{column}
        \begin{column}{0.48\textwidth}
            \textbf{参考解答要点:}
\begin{Shaded}
if image is None:
    raise ValueError(...)
if option not in image:
    raise KeyError(...)
if region.size == 0:
    continue  # 或return {}
\end{Shaded}
        \end{column}
    \end{columns}
\end{frame}

%----------------------------------------------------------
% 要点速查页模板(每个模块末尾使用)
%----------------------------------------------------------

\subsection{本模块要点速查}

\begin{frame}{模块00-背景知识:要点速查}
    \begin{block}{核心概念}
        \begin{itemize}
            \item \textbf{SDLC}:软件开发生命周期(需求→设计→编码→测试→部署)
            \item \textbf{敏捷开发}:迭代增量,快速反馈(2-4周Sprint)
            \item \textbf{CI/CD}:持续集成与持续部署(自动化构建、测试、发布)
            \item \textbf{Code Review}:代码审查,对事不对人,小步快跑
            \item \textbf{重构}:不改变外部行为,改善内部结构
        \end{itemize}
    \end{block}

    \begin{alertblock}{关键公式}
        \begin{itemize}
            \item \textbf{高内聚} = 模块内元素紧密相关
            \item \textbf{低耦合} = 模块间依赖最小
        \end{itemize}
    \end{alertblock}

    \begin{exampleblock}{下一步学习}
        进入核心理论模块,学习模块化设计和面向对象编程
    \end{exampleblock}
\end{frame}

\begin{frame}{模块01-核心理论:要点速查}
    \begin{block}{设计原则}
        \begin{itemize}
            \item \textbf{SOLID原则}:单一职责、开闭原则、里氏替换、接口隔离、依赖倒置
            \item \textbf{封装}:隐藏实现细节,暴露必要接口
            \item \textbf{继承与多态}:代码复用,灵活扩展
            \item \textbf{纯函数}:相同输入→相同输出,无副作用
        \end{itemize}
    \end{block}

    \begin{block}{调试流程}
        \texttt{发现问题 → 复现问题 → 建立假设 → 验证假设 → 修复问题}
    \end{block}

    \begin{block}{测试金字塔}
        大量单元测试(70\%) + 适量集成测试(20\%) + 少量E2E测试(10\%)
    \end{block}
\end{frame}

\begin{frame}{模块02-工具环境:要点速查}
    \begin{columns}
        \begin{column}{0.48\textwidth}
            \textbf{开发工具选择}
            \begin{itemize}
                \item 大型项目:PyCharm
                \item 多语言/轻量:VS Code
                \item 数据分析:Jupyter
            \end{itemize}
        \end{column}
        \begin{column}{0.48\textwidth}
            \textbf{代码质量工具}
            \begin{itemize}
                \item 检查:flake8、pylint
                \item 格式化:black
                \item 类型检查:mypy
            \end{itemize}
        \end{column}
    \end{columns}

    \begin{block}{AI编程工具}
        \begin{itemize}
            \item Cursor:AI原生IDE(Cmd+L对话)
            \item Claude Code:命令行助手
            \item Copilot:代码补全
        \end{itemize}
    \end{block}
\end{frame}

\begin{frame}{模块03-Live Coding:要点速查}
    \begin{block}{重构技术}
        \begin{itemize}
            \item 提取函数(长函数→多个小函数)
            \item 提取类(职责分离)
            \item 参数对象(过多参数→配置类)
            \item 替换魔法数字(有意义的常量)
        \end{itemize}
    \end{block}

    \begin{block}{异常处理策略}
        \begin{itemize}
            \item 自定义异常层次
            \item 捕获底层异常→转换业务异常
            \item 防御式编程:前置条件检查
        \end{itemize}
    \end{block}

    \begin{block}{TDD循环}
        \texttt{红灯(测试失败) → 绿灯(最少代码) → 重构(优化)}
    \end{block}
\end{frame}
