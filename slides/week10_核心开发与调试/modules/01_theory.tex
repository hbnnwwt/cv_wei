%===========================================================
% 模块01:核心理论与原理 (14-18页)
%===========================================================

\section{核心理论与原理}

%----------------------------------------------------------
\subsection{代码组织与架构}
%----------------------------------------------------------

\begin{frame}{模块化设计原则}
    \textbf{为什么需要模块化?}

    \begin{itemize}
        \item \textbf{降低复杂度}:分而治之,理解局部而非整体
        \item \textbf{提高复用性}:独立模块可在不同项目使用
        \item \textbf{便于维护}:修改局部不影响全局
        \item \textbf{支持并行开发}:不同开发者负责不同模块
    \end{itemize}

    \begin{block}{高内聚低耦合}
        \textbf{高内聚}:模块内部元素紧密相关,职责单一

        \textbf{低耦合}:模块之间依赖最小化,接口清晰
    \end{block}
\end{frame}

\begin{frame}{智能阅卷系统的模块划分}
    \textbf{系统模块结构:}

    \begin{center}
        \begin{tikzpicture}[
            node distance=0.5cm,
            box/.style={draw, fill=blue!15, minimum width=2.8cm, minimum height=0.6cm, align=center, font=\small}
        ]
            \node[box, fill=green!30] (main) {AutoGradingSystem\\主程序};
            \node[box, below=0.3cm of main] (pre) {ImagePreprocessor\\图像预处理};
            \node[box, right=0.3cm of pre] (det) {RegionDetector\\区域检测};
            \node[box, left=0.3cm of pre] (rec) {ChoiceRecognizer\\选择识别};
            \node[box, below=0.3cm of det] (calc) {GradeCalculator\\评分计算};

            \draw[-Stealth, thick] (main) -- (pre);
            \draw[-Stealth, thick] (pre) -- (det);
            \draw[-Stealth, thick] (pre) -- (rec);
            \draw[-Stealth, thick] (det) -- (calc);
        \end{tikzpicture}
    \end{center}

    \textbf{模块间关系:}
    \begin{itemize}
        \item 主程序调用各模块
        \item 预处理 $\to$ 区域检测 $\to$ 识别 $\to$ 评分
        \item 各模块独立,便于单独开发和测试
    \end{itemize}
\end{frame}

\begin{frame}{接口设计原则}
    \textbf{良好的接口应该:}

    \begin{enumerate}
        \item \textbf{简洁明了}:只做一件事,命名清晰
        \item \textbf{隐藏实现}:调用者无需知道内部细节
        \item \textbf{稳定可靠}:接口定义一旦发布尽量少变更
        \item \textbf{完备一致}:错误处理、边界情况都有定义
    \end{enumerate}

    \begin{exampleblock}{接口示例:选择题识别}
        \begin{itemize}
            \item[] \texttt{def recognize\_choice(image, options, threshold=0.5):}
            \item[] \hspace{1em} \texttt{"""识别选择题答案"""}
            \item[] \hspace{1em} \texttt{Args:}
            \item[] \hspace{2em} \texttt{image: 题目区域图像}
            \item[] \hspace{2em} \texttt{options: 选项位置列表 ['A', 'B', 'C', 'D']}
            \item[] \hspace{2em} \texttt{threshold: 填涂判定阈值}
            \item[] \hspace{1em} \texttt{Returns: str: 识别结果,如 'A' 或 None}
        \end{itemize}
    \end{exampleblock}
\end{frame}

\begin{frame}{面向对象设计:封装}
    \textbf{封装(Encapsulation)}:隐藏内部实现,暴露必要接口

    \vspace{0.3cm}

    \begin{columns}
        \begin{column}{0.48\textwidth}
            \textbf{封装的好处:}
            \begin{itemize}
                \item 保护内部状态不被意外修改
                \item 降低模块间的耦合
                \item 便于修改内部实现
                \item 提供清晰的使用方式
            \end{itemize}
        \end{column}
        \begin{column}{0.48\textwidth}
            \textbf{Python中的封装:}
            \begin{itemize}
                \item \texttt{\_var}:建议私有(约定)
                \item \texttt{\_\_var}:名称改写(伪私有)
                \item \texttt{property}:属性访问控制
            \end{itemize}
        \end{column}
    \end{columns}

    \vspace{0.3cm}

    \begin{block}{示例:AnswerSheet类}
        \texttt{class AnswerSheet:}\\
        \texttt{\ \ \ \ def \_\_init\_\_(self):}\\
        \texttt{\ \ \ \ \ \ \ self.\_\_image = None  \# 私有属性}\\
        \texttt{\ \ \ \ \ \ \ self.\_threshold = 0.5}\\
        \texttt{\ \ \ \ @property}\\
        \texttt{\ \ \ \ def image(self):}\\
        \texttt{\ \ \ \ \ \ \ return self.\_\_image}
    \end{block}
\end{frame}

\begin{frame}{面向对象设计:继承与多态}
    \textbf{继承(Inheritance):}
    \begin{itemize}
        \item 代码复用,提取公共行为
        \item 建立类型层次关系
        \item 支持扩展和定制
    \end{itemize}

    \textbf{多态(Polymorphism):}
    \begin{itemize}
        \item 同一接口,不同实现
        \item 运行时动态绑定
        \item 提高代码灵活性
    \end{itemize}

    \begin{exampleblock}{识别器基类}
        \texttt{class BaseRecognizer:}\\
        \texttt{\ \ \ \ def recognize(self, image):}\\
        \texttt{\ \ \ \ \ \ \ raise NotImplementedError}\\
        \texttt{class OMRRecognizer(BaseRecognizer):}\\
        \texttt{\ \ \ \ def recognize(self, image):}\\
        \texttt{\ \ \ \ \ \ \ \# OMR识别实现}\\
        \texttt{\ \ \ \ \ \ \ pass}
    \end{exampleblock}
\end{frame}

\begin{frame}{SOLID设计原则}
    \textbf{面向对象设计的五大原则:}

    \begin{description}
        \item[S - 单一职责原则] 类只负责一项功能
        \item[O - 开闭原则] 对扩展开放,对修改封闭
        \item[L - 里氏替换原则] 子类可替换父类使用
        \item[I - 接口隔离原则] 客户端不依赖不需要的接口
        \item[D - 依赖倒置原则] 依赖抽象而非具体实现
    \end{description}

    \vspace{0.3cm}

    \begin{alertblock}{应用示例:阅卷系统}
        \begin{itemize}
            \item \textbf{S}:ImagePreprocessor只处理图像,不负责识别
            \item \textbf{O}:新增题型识别器,无需修改主程序
            \item \textbf{D}:依赖BaseRecognizer接口,不依赖具体实现
        \end{itemize}
    \end{alertblock}
\end{frame}

\begin{frame}{函数式编程思想}
    \textbf{核心概念:}

    \begin{columns}
        \begin{column}{0.48\textwidth}
            \textbf{纯函数}
            \begin{itemize}
                \item 相同输入永远得到相同输出
                \item 无副作用(不修改外部状态)
                \item 便于测试和推理
            \end{itemize}

            \vspace{0.2cm}

            \textbf{高阶函数}
            \begin{itemize}
                \item 函数可作为参数传递
                \item 函数可返回函数
                \item \texttt{map}, \texttt{filter}, \texttt{reduce}
            \end{itemize}
        \end{column}
        \begin{column}{0.48\textwidth}
            \textbf{不可变数据}
            \begin{itemize}
                \item 数据创建后不可修改
                \item 避免共享状态问题
                \item 简化并发编程
            \end{itemize}

            \vspace{0.2cm}

            \textbf{函数组合}
            \begin{itemize}
                \item 小函数组合成复杂功能
                \item 管道式数据处理
                \item \texttt{result = f(g(h(x)))}
            \end{itemize}
        \end{column}
    \end{columns}

    \vspace{0.3cm}

    \begin{block}{Python示例}
        \texttt{images = [load(p) for p in paths]  \# 列表推导}\\
        \texttt{regions = map(detect, images)  \# 高阶函数}\\
        \texttt{valid = filter(is\_valid, regions)  \# 纯函数过滤}
    \end{block}
\end{frame}

%----------------------------------------------------------
\subsection{错误处理与异常}
%----------------------------------------------------------

\begin{frame}{Python异常层次结构}
    \textbf{标准异常类型:}

    \begin{columns}
        \begin{column}{0.48\textwidth}
            \textbf{常见异常:}
            \begin{itemize}
                \item \texttt{Exception}:所有异常的基类
                \item \texttt{ValueError}:值不正确
                \item \texttt{TypeError}:类型错误
                \item \texttt{FileNotFoundError}:文件不存在
                \item \texttt{KeyError} / \texttt{IndexError}:访问错误
            \end{itemize}
        \end{column}
        \begin{column}{0.48\textwidth}
            \textbf{自定义异常:}
            \begin{itemize}
                \item[] \texttt{class GradingError(Exception):}
                \item[] \hspace{1em} \texttt{pass}
                \item[] \texttt{class ImageLoadError(GradingError):}
                \item[] \hspace{1em} \texttt{pass}
                \item[] \texttt{class RecognitionError(GradingError):}
                \item[] \hspace{1em} \texttt{pass}
            \end{itemize}
        \end{column}
    \end{columns}

    \vspace{0.3cm}

    \begin{exampleblock}{异常捕获层次}
        \texttt{try:}\\
        \texttt{\ \ \ \ process\_image(path)}\\
        \texttt{except ImageLoadError as e:}\\
        \texttt{\ \ \ \ logger.error(f"加载失败: \{e\}")}\\
        \texttt{except GradingError as e:}\\
        \texttt{\ \ \ \ logger.error(f"处理错误: \{e\}")}\\
        \texttt{except Exception as e:}\\
        \texttt{\ \ \ \ logger.exception("未知错误")}
    \end{exampleblock}
\end{frame}

\begin{frame}{错误处理策略}
    \textbf{防御式编程:}
    \begin{itemize}
        \item 提前验证输入条件
        \item 断言不变式
        \item 失败快速(Fail Fast)
    \end{itemize}

    \textbf{异常转换与封装:}
    \begin{itemize}
        \item 捕获底层异常,转换为业务异常
        \item 保留原始异常信息(异常链)
        \item 提供更有意义的错误消息
    \end{itemize}

    \begin{block}{异常转换示例}
        \texttt{try:}\\
        \texttt{\ \ \ \ img = cv2.imread(path)}\\
        \texttt{\ \ \ \ if img is None:}\\
        \texttt{\ \ \ \ \ \ \ raise ImageLoadError(f"无法加载: \{path\}")}\\
        \texttt{except cv2.error as e:}\\
        \texttt{\ \ \ \ raise ImageLoadError(f"OpenCV错误") from e}
    \end{block}
\end{frame}

\begin{frame}{日志记录最佳实践}
    \textbf{结构化日志配置:}

    \begin{itemize}
        \item[] \texttt{import logging}
        \item[] \texttt{logging.basicConfig(level=logging.INFO)}
        \item[] \texttt{logger = logging.getLogger(\_\_name\_\_)}
    \end{itemize}

    \vspace{0.5cm}

    \textbf{日志记录建议:}
    \begin{itemize}
        \item DEBUG:详细变量值、执行路径
        \item INFO:重要操作里程碑
        \item WARNING:可恢复的异常情况
        \item ERROR:功能受影响的错误
    \end{itemize}
\end{frame}

%----------------------------------------------------------
\subsection{调试技术与工具}
%----------------------------------------------------------

\begin{frame}{断点调试技巧}
    \textbf{断点类型:}

    \begin{columns}
        \begin{column}{0.48\textwidth}
            \textbf{行断点}
            \begin{itemize}
                \item 最常用,执行到指定行暂停
                \item IDE中点击行号设置
                \item 临时查看程序状态
            \end{itemize}

            \vspace{0.2cm}

            \textbf{条件断点}
            \begin{itemize}
                \item 满足条件时才暂停
                \item 右键断点 $\to$ 编辑条件
                \item 示例:\texttt{i == 100}
            \end{itemize}
        \end{column}
        \begin{column}{0.48\textwidth}
            \textbf{异常断点}
            \begin{itemize}
                \item 发生异常时自动暂停
                \item 捕获未被处理的异常
                \item 快速定位异常源头
            \end{itemize}

            \vspace{0.2cm}

            \textbf{日志断点}
            \begin{itemize}
                \item 不暂停,输出日志
                \item 用于性能敏感代码
                \item 避免中断程序执行
            \end{itemize}
        \end{column}
    \end{columns}

    \vspace{0.3cm}

    \begin{block}{Python代码中断点}
        \texttt{import pdb; pdb.set\_trace()  \# 设置断点}\\
        \texttt{breakpoint()  \# Python 3.7+ 更简洁的写法}
    \end{block}
\end{frame}

\begin{frame}{单步执行与变量查看}
    \textbf{调试操作:}

    \begin{columns}
        \begin{column}{0.48\textwidth}
            \begin{description}
                \item[Resume] 继续执行到下一断点
                \item[Step Over] 执行当前行,不进入函数
                \item[Step Into] 进入函数内部
                \item[Step Out] 跳出当前函数
            \end{description}
        \end{column}
        \begin{column}{0.48\textwidth}
            \textbf{变量查看:}
            \begin{itemize}
                \item Variables面板:所有局部变量
                \item Watches:自定义表达式
                \item 鼠标悬停:快速查看
                \item 控制台:执行表达式
            \end{itemize}
        \end{column}
    \end{columns}

    \vspace{0.3cm}

    \begin{alertblock}{调试技巧}
        \begin{itemize}
            \item 从预期结果倒推:哪里可能导致这个结果?
            \item 检查中间状态:变量是否符合预期?
            \item 修改变量值:测试修复是否有效
            \item 使用条件断点:跳过不感兴趣的循环
        \end{itemize}
    \end{alertblock}
\end{frame}

\begin{frame}{调试流程可视化}
    \begin{center}
        \begin{tikzpicture}[
            node distance=1.2cm,
            phase/.style={draw, fill=blue!10, minimum width=2.2cm, minimum height=0.9cm, align=center, font=\small},
            arrow/.style={-Stealth, thick}
        ]
            % 调试流程
            \node[phase, fill=red!20] (problem) {发现问题};
            \node[phase, right=of problem] (reproduce) {复现问题};
            \node[phase, right=of reproduce] (hypothesis) {建立假设};
            \node[phase, right=of hypothesis] (verify) {验证假设};
            \node[phase, fill=green!20, right=of verify] (fix) {修复问题};

            % 箭头
            \draw[arrow] (problem) -- (reproduce);
            \draw[arrow] (reproduce) -- (hypothesis);
            \draw[arrow] (hypothesis) -- (verify);
            \draw[arrow, dashed] (verify) to[bend right=45] (hypothesis);
            \draw[arrow] (verify) -- (fix);

            % 标注
            \node[below=0.15cm of reproduce, font=\tiny] {最小复现案例};
            \node[below=0.15cm of hypothesis, font=\tiny] {二分法缩小范围};
            \node[below=0.15cm of verify, font=\tiny] {日志/断点/变量};
        \end{tikzpicture}
    \end{center}

    \vspace{0.3cm}

    \textbf{调试本质:科学排查流程}
    \begin{enumerate}
        \item \textbf{建立假设}:基于理解,推测可能出错的位置
        \item \textbf{收集证据}:通过日志、断点、变量值获取数据
        \item \textbf{验证假设}:检查证据是否支持假设
        \item \textbf{修正理解}:根据结果调整假设,循环直到找到根因
    \end{enumerate}

    \vspace{0.2cm}

    \begin{block}{关键洞察}
        调试不是乱试,而是\textbf{科学实验}——假设、验证、迭代的过程
    \end{block}
\end{frame}

\begin{frame}{性能分析工具}
    \textbf{cProfile - CPU性能分析:}

    \begin{itemize}
        \item[] \texttt{import cProfile}
        \item[] \texttt{cProfile.run('process\_image(path)')}
    \end{itemize}

    \textbf{line\_profiler - 逐行分析:}

    \begin{itemize}
        \item[] \texttt{@profile}
        \item[] \texttt{def process(image):}
        \item[] \hspace{1em} \texttt{result = detect(image)  \# 显示此行耗时}
        \end{itemize}

    \vspace{0.2cm}

    \textbf{memory\_profiler - 内存分析:}

    \begin{itemize}
        \item[] \texttt{@profile}
        \item[] \texttt{def load\_images(paths):}
        \item[] \hspace{1em} \texttt{images = [load(p) for p in paths]}
        \end{itemize}

    \vspace{0.2cm}

    \begin{block}{运行命令}
        \texttt{kernprof -l -v script.py  \# line\_profiler}\\
        \texttt{python -m memory\_profiler script.py  \# memory\_profiler}
    \end{block}
\end{frame}

%----------------------------------------------------------
\subsection{测试与质量保证}
%----------------------------------------------------------

\begin{frame}{测试金字塔}
    \textbf{测试类型层次:}

    \begin{center}
        \begin{tikzpicture}
            % 金字塔
            \draw[fill=green!30] (0,0) -- (4,0) -- (2,3) -- cycle;
            \node at (2,0.5) {单元测试};
            \node[font=\footnotesize] at (2,0.9) {大量、快速、稳定};

            \draw[fill=yellow!30] (0.5,0.7) -- (3.5,0.7) -- (2,2.7) -- cycle;
            \node at (2,1.3) {集成测试};
            \node[font=\footnotesize] at (2,1.7) {中等数量};

            \draw[fill=red!30] (1,1.4) -- (3,1.4) -- (2,2.4) -- cycle;
            \node at (2,1.8) {端到端测试};
            \node[font=\footnotesize] at (2,2.1) {少量、慢};
        \end{tikzpicture}
    \end{center}

    \textbf{测试策略:}
    \begin{itemize}
        \item 底层测试多,高层测试少
        \item 单元测试运行快,反馈及时
        \item E2E测试保证关键路径正确
    \end{itemize}
\end{frame}

\begin{frame}{测试驱动开发(TDD)}
    \textbf{TDD循环:红-绿-重构}

    \begin{center}
        \begin{tikzpicture}[
            node distance=1.5cm,
            box/.style={draw, fill=blue!10, minimum width=3cm, minimum height=0.8cm, align=center}
        ]
            \node[box, fill=red!20] (red) {\textbf{红灯}\\写失败的测试};
            \node[box, fill=green!20, right=of red] (green) {\textbf{绿灯}\\写最少代码通过};
            \node[box, fill=yellow!20, right=of green] (refactor) {\textbf{重构}\\优化代码质量};

            \draw[-Stealth, thick] (red) -- (green);
            \draw[-Stealth, thick] (green) -- (refactor);
            \draw[-Stealth, thick, dashed] (refactor.south) to[bend right=45] (red.south);
        \end{tikzpicture}
    \end{center}

    \textbf{TDD的好处:}
    \begin{itemize}
        \item 保证代码可测试
        \item 充当使用文档
        \item 减少bug数量
        \item 提高设计质量
    \end{itemize}
\end{frame}

\begin{frame}{pytest高级特性}
    \textbf{Fixture机制:}

    \begin{itemize}
        \item[] \texttt{@pytest.fixture}
        \item[] \texttt{def sample\_image():}
        \item[] \hspace{1em} \texttt{return create\_test\_image()}
        \item[] \texttt{def test\_recognize(sample\_image):}
        \item[] \hspace{1em} \texttt{result = recognize(sample\_image)}
        \item[] \hspace{1em} \texttt{assert result == 'A'}
    \end{itemize}

    \vspace{0.2cm}

    \textbf{参数化测试:}

    \begin{itemize}
        \item[] \texttt{@pytest.mark.parametrize("input,expected", [}
        \item[] \hspace{1em} \texttt{(filled\_img, 'A'),}
        \item[] \hspace{1em} \texttt{(empty\_img, None),}
        \item[] \hspace{1em} \texttt{(partial\_img, None)}
        \item[] \texttt{])}
        \item[] \texttt{def test\_recognize(input, expected):}
        \item[] \hspace{1em} \texttt{assert recognize(input) == expected}
    \end{itemize}
\end{frame}

\begin{frame}{Mock与测试隔离}
    \textbf{为什么要Mock?}

    \begin{itemize}
        \item 隔离外部依赖(文件、网络、数据库)
        \item 加速测试执行
        \item 模拟异常情况
        \item 确保测试可重复
    \end{itemize}

    \begin{exampleblock}{Mock示例}
        \texttt{from unittest.mock import patch, Mock}\\
        \texttt{@patch('cv2.imread')}\\
        \texttt{def test\_load(mock\_imread):}\\
        \texttt{\ \ \ \ mock\_imread.return\_value = test\_img}\\
        \texttt{\ \ \ \ result = load('any\_path.jpg')}\\
        \texttt{\ \ \ \ assert result is not None}\\
        \texttt{\ \ \ \ mock\_imread.assert\_called\_once()}
    \end{exampleblock}
\end{frame}

\begin{frame}{测试覆盖率}
    \textbf{覆盖率指标:}

    \begin{columns}
        \begin{column}{0.48\textwidth}
            \textbf{语句覆盖}
            \begin{itemize}
                \item 执行到的代码行比例
                \item 基础指标
            \end{itemize}

            \textbf{分支覆盖}
            \begin{itemize}
                \item if/else各分支执行情况
                \item 更严格
            \end{itemize}
        \end{column}
        \begin{column}{0.48\textwidth}
            \textbf{使用coverage.py:}
            \begin{itemize}
                \item[] \texttt{pip install coverage pytest-cov}
                \item[] \texttt{pytest --cov=module --cov-report=html}
                \item[] \texttt{open htmlcov/index.html}
            \end{itemize}
        \end{column}
    \end{columns}

    \vspace{0.3cm}

    \begin{alertblock}{覆盖率建议}
        \begin{itemize}
            \item 核心业务逻辑:$>80\%$
            \item 工具类:$>90\%$
            \item 注意:高覆盖率不等于高质量
            \item 关注边界条件和异常路径
        \end{itemize}
    \end{alertblock}
\end{frame}

%----------------------------------------------------------
% 本模块要点速查(新增)
%----------------------------------------------------------

\subsection{本模块要点速查}

\begin{frame}{模块01-核心理论:要点速查}
    \begin{block}{设计原则}
        \begin{itemize}
            \item \textbf{SOLID原则}:单一职责、开闭原则、里氏替换、接口隔离、依赖倒置
            \item \textbf{封装}:隐藏实现细节,暴露必要接口
            \item \textbf{继承与多态}:代码复用,灵活扩展
            \item \textbf{纯函数}:相同输入→相同输出,无副作用
        \end{itemize}
    \end{block}

    \begin{block}{调试流程}
        \texttt{发现问题 → 复现问题 → 建立假设 → 验证假设 → 修复问题}
    \end{block}

    \begin{block}{测试金字塔}
        \texttt{大量单元测试(70\%) + 适量集成测试(20\%) + 少量E2E测试(10\%)}
    \end{block}
\end{frame}

%----------------------------------------------------------
% 概念补充:重构/TDD通俗解释(新增)
%----------------------------------------------------------

\subsection{概念补充}

\begin{frame}{概念补充:什么是重构?}
    \begin{columns}
        \begin{column}{0.48\textwidth}
            \textbf{生活类比}

            \begin{block}{整理房间}
                \begin{itemize}
                    \item 房间很乱(代码可读性差)
                    \item 不改变房间里的东西(功能不变)
                    \item 只是重新整理收纳(优化结构)
                    \item 目的是更容易找到东西(更容易维护)
                \end{itemize}
            \end{block}
        \end{column}
        \begin{column}{0.48\textwidth}
            \textbf{代码示例}

            \begin{block}{重构前}
                \texttt{def f(x):}\\
                \texttt{\ \ \ \# 100行代码}
            \end{block}

            \begin{block}{重构后}
                \texttt{def process():}\\
                \texttt{\ \ \ load()}\\
                \texttt{\ \ \ process\_data()}\\
                \texttt{\ \ \ save()}
            \end{block}
        \end{column}
    \end{columns}
\end{frame}

\begin{frame}{概念补充:什么是测试驱动开发?}
    \textbf{想象学做菜的过程:}

    \begin{center}
        \begin{tikzpicture}[
            node distance=0.5cm,
            box/.style={draw, fill=blue!10, minimum width=3cm, minimum height=1cm, align=center, font=\small},
            arrow/.style={-Stealth, thick}
        ]
            \node[box, fill=red!20] (red) {1. 先定标准\\(写测试)};
            \node[box, fill=green!20, right=of red] (green) {2. 做最简单能通过的\\(写最少代码)};
            \node[box, fill=yellow!20, right=of green] (refactor) {3. 改进做法\\(重构优化)};

            \draw[arrow] (red) -- (green);
            \draw[arrow] (green) -- (refactor);
            \draw[arrow, dashed] (refactor.south) to[bend right=45] (red.south);
        \end{tikzpicture}
    \end{center}

    \begin{block}{好处}
        \begin{itemize}
            \item 确保做出来的东西符合要求(测试通过)
            \item 避免过度设计(只写最少代码)
            \item 持续改进质量(每次都优化一点)
        \end{itemize}
    \end{block}
\end{frame}
