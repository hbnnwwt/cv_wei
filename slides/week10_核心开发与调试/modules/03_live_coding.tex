%===========================================================
% 模块03:Live Coding实战 (20-25页)
%===========================================================

\section{Live Coding实战}

%----------------------------------------------------------
\subsection{代码组织与重构实战}
%----------------------------------------------------------

\begin{frame}[fragile]{识别Code Smells}
    \textbf{什么是Code Smell?}
    \begin{itemize}
        \item 代码中潜在的糟糕设计指示
        \item 不是bug,但可能导致问题
        \item 需要重构的信号
    \end{itemize}

    \begin{columns}
        \begin{column}{0.48\textwidth}
            \textbf{常见Code Smells:}
            \begin{itemize}
                \item 长函数(Long Method)
                \item 大类(Large Class)
                \item 重复代码(Duplicated Code)
                \item 过长参数列表(Long Parameter List)
                \item 发散式变化(Divergent Change)
                \item 特性依恋(Feature Envy)
            \end{itemize}
        \end{column}
        \begin{column}{0.48\textwidth}
            \begin{block}{示例:长函数}
\begin{verbatim}
def process_answer_sheet(path):
    # 200行代码...
    image = load(path)
    gray = cv2.cvtColor(...)
    binary = cv2.threshold(...)
    contours = cv2.findContours(...)
    # ...更多代码
\end{verbatim}
            \end{block}
        \end{column}
    \end{columns}
\end{frame}

\begin{frame}[fragile]{重构:提取函数}
    \textbf{重构前:}
\begin{verbatim}
def process_answer_sheet(path):
    image = cv2.imread(path)
    gray = cv2.cvtColor(image, cv2.COLOR_BGR2GRAY)
    binary = cv2.threshold(gray, 127, 255, cv2.THRESH_BINARY)[1]
    contours = cv2.findContours(binary, cv2.RETR_EXTERNAL, cv2.CHAIN_APPROX_SIMPLE)[0]
    # ... 混杂的处理逻辑
\end{verbatim}

    \textbf{重构后:}
\begin{verbatim}
def process_answer_sheet(path):
    image = load_image(path)
    binary = binarize(image)
    regions = find_regions(binary)
    return recognize(regions)

def load_image(path):
    img = cv2.imread(path)
    if img is None:
        raise ImageLoadError(f"无法加载: {path}")
    return img

def binarize(image):
    gray = cv2.cvtColor(image, cv2.COLOR_BGR2GRAY)
    return cv2.threshold(gray, 127, 255, cv2.THRESH_BINARY)[1]
\end{verbatim}
\end{frame}

\begin{frame}[fragile]{重构:提取类}
    \textbf{重构前:所有逻辑在一个类中}
\begin{verbatim}
class AnswerSheetProcessor:
    def process(self, path):
        # 加载
        # 预处理
        # 检测
        # 识别
        # 评分
        # 全部在这里...
\end{verbatim}

    \textbf{重构后:职责分离}
\begin{verbatim}
class ImagePreprocessor:
    def load(self, path): pass
    def normalize(self, image): pass
    def enhance(self, image): pass

class RegionDetector:
    def find_answers(self, image): pass
    def locate_students(self, image): pass

class ChoiceRecognizer:
    def recognize(self, region): pass

# 主类变为协调器
class AnswerSheetProcessor:
    def __init__(self):
        self.prep = ImagePreprocessor()
        self.detector = RegionDetector()
        self.recognizer = ChoiceRecognizer()
\end{verbatim}
\end{frame}

\begin{frame}[fragile]{重构:引入参数对象}
    \textbf{重构前:参数过多}
\begin{verbatim}
def recognize(image, x, y, width, height, threshold,
              options, algorithm, debug_mode):
    # 9个参数,难以理解
    pass

# 调用时容易出错
result = recognize(img, 100, 200, 50, 30, 0.5,
                   ['A','B','C','D'], 'omr', True)
\end{verbatim}

    \textbf{重构后:参数对象}
\begin{verbatim}
@dataclass
class RecognitionConfig:
    region: Rectangle
    threshold: float = 0.5
    options: List[str] = field(default_factory=lambda: ['A','B','C','D'])
    algorithm: str = 'omr'
    debug: bool = False

def recognize(image, config: RecognitionConfig):
    # 清晰明了
    pass

# 调用时清晰
config = RecognitionConfig(region=Rectangle(100, 200, 50, 30))
result = recognize(img, config)
\end{verbatim}
\end{frame}

\begin{frame}[fragile]{重构:替换魔法数字}
    \textbf{重构前:代码中有未命名的常量}
\begin{verbatim}
if density > 0.7:
    answer = 'A'
elif density > 0.5:
    answer = 'B'
elif density > 0.3:
    answer = 'C'
else:
    answer = 'D'

for i in range(100):
    process(i)
\end{verbatim}

    \textbf{重构后:使用有意义的常量}
\begin{verbatim}
# 常量定义
THRESHOLD_HIGH = 0.7
THRESHOLD_MEDIUM = 0.5
THRESHOLD_LOW = 0.3
MAX_ITERATIONS = 100

# 使用
if density > THRESHOLD_HIGH:
    answer = 'A'
elif density > THRESHOLD_MEDIUM:
    answer = 'B'

for i in range(MAX_ITERATIONS):
    process(i)
\end{verbatim}
\end{frame}

\begin{frame}[fragile]{模块化重构实战}
    \textbf{单体结构(避免):}
\begin{verbatim}
# 所有功能在一个文件中
# grading_system.py (1000行)
class AutoGradingSystem:
    def load(self): pass
    def preprocess(self): pass
    def detect(self): pass
    def recognize(self): pass
    def grade(self): pass
    def export(self): pass
    # ... 更多方法
\end{verbatim}

    \textbf{模块化结构(推荐):}
\begin{verbatim}
# 清晰的模块划分
grading_system/
├── __init__.py
├── preprocessor/
│   ├── __init__.py
│   ├── loader.py
│   └── enhancer.py
├── detector/
│   ├── __init__.py
│   └── region_detector.py
├── recognizer/
│   ├── __init__.py
│   ├── omr.py
│   └── handwriting.py
└── grader/
    ├── __init__.py
    └── calculator.py
\end{verbatim}
\end{frame}

%----------------------------------------------------------
\subsection{错误处理与日志实战}
%----------------------------------------------------------

\begin{frame}[fragile]{自定义异常层次}
    \textbf{定义异常体系:}
\begin{verbatim}
# 基类异常
class GradingError(Exception):
    """阅卷系统基础异常"""
    pass

# 具体异常
class ImageLoadError(GradingError):
    """图像加载失败"""
    pass

class RegionDetectionError(GradingError):
    """区域检测失败"""
    pass

class RecognitionError(GradingError):
    """识别失败"""
    pass

class ConfigError(GradingError):
    """配置错误"""
    pass
\end{verbatim}

    \textbf{使用自定义异常:}
\begin{verbatim}
def load_image(path):
    img = cv2.imread(path)
    if img is None:
        raise ImageLoadError(f"无法加载图像: {path}")
    return img
\end{verbatim}
\end{frame}

\begin{frame}[fragile]{异常捕获策略}
    \textbf{策略一:在合适层级捕获异常}
\begin{verbatim}
# 低层函数:抛出异常
def load_image(path):
    img = cv2.imread(path)
    if img is None:
        raise ImageLoadError(f"无法加载: {path}")
    return img

# 中层函数:转换异常
def process_sheet(path):
    try:
        img = load_image(path)
    except ImageLoadError as e:
        logger.error(f"加载失败: {e}")
        return None
    # 继续处理...

# 高层函数:捕获所有
def main():
    try:
        result = process_sheet(path)
    except GradingError as e:
        handle_error(e)
\end{verbatim}
\end{frame}

\begin{frame}[fragile]{防御式编程实战}
    \textbf{输入验证:}
\begin{verbatim}
def recognize_choice(image, options, threshold=0.5):
    # 输入验证
    if image is None:
        raise ValueError("图像不能为空")
    if image.size == 0:
        raise ValueError("图像尺寸无效")
    if not options:
        raise ValueError("选项列表不能为空")
    if not 0 < threshold < 1:
        raise ValueError(f"阈值必须在0-1之间,实际: {threshold}")

    # 业务逻辑
    result = _do_recognize(image, options, threshold)
    return result
\end{verbatim}

    \textbf{断言使用:}
\begin{verbatim}
def calculate_score(answers, correct_answers):
    assert len(answers) == len(correct_answers), \
        f"答案数量不匹配: {len(answers)} vs {len(correct_answers)}"

    score = sum(a == c for a, c in zip(answers, correct_answers))
    return score
\end{verbatim}
\end{frame}

\begin{frame}[fragile]{日志系统配置}
    \textbf{完整日志配置:}
\begin{verbatim}
import logging
from logging.handlers import RotatingFileHandler

def setup_logging(name):
    logger = logging.getLogger(name)
    logger.setLevel(logging.DEBUG)

    # 文件处理器(带轮转)
    file_handler = RotatingFileHandler(
        'grading.log', maxBytes=10*1024*1024, backupCount=5
    )
    file_handler.setLevel(logging.INFO)

    # 控制台处理器
    console_handler = logging.StreamHandler()
    console_handler.setLevel(logging.WARNING)

    # 格式化
    formatter = logging.Formatter(
        '%(asctime)s - %(name)s - %(levelname)s - %(message)s'
    )
    file_handler.setFormatter(formatter)
    console_handler.setFormatter(formatter)

    logger.addHandler(file_handler)
    logger.addHandler(console_handler)

    return logger
\end{verbatim}
\end{frame}

\begin{frame}[fragile]{结构化日志输出}
    \textbf{结构化日志:}
\begin{verbatim}
import json
import logging

class StructuredLogger:
    def __init__(self, name):
        self.logger = logging.getLogger(name)

    def log_event(self, level, event, **context):
        log_data = {
            'event': event,
            'timestamp': datetime.now().isoformat(),
            **context
        }
        self.logger.log(level, json.dumps(log_data))

    def info(self, event, **kwargs):
        self.log_event(logging.INFO, event, **kwargs)

# 使用
logger = StructuredLogger('grading')
logger.info('image_loaded', path='test.jpg', size=(800, 600))
logger.info('region_detected', count=5, areas=[100, 200, 300])
\end{verbatim}

    \textbf{日志输出示例:}
\begin{verbatim}
{"event": "image_loaded", "timestamp": "2025-02-06T10:30:00",
 "path": "test.jpg", "size": [800, 600]}
\end{verbatim}
\end{frame}

%----------------------------------------------------------
\subsection{调试技术实战}
%----------------------------------------------------------

\begin{frame}[fragile]{条件断点实战}
    \textbf{场景:循环中只关心特定情况}

    \begin{columns}
        \begin{column}{0.48\textwidth}
            \textbf{问题代码:}
\begin{verbatim}
results = []
for i in range(10000):
    region = detect_region(i)
    result = recognize(region)
    results.append(result)
    # 只有i=5000时出错
\end{verbatim}
        \end{column}
        \begin{column}{0.48\textwidth}
            \textbf{条件断点设置:}
\begin{verbatim}
# 在IDE中设置:
# 行号: "result = recognize(region)"
# 条件: i == 5000

# 或使用代码:
for i in range(10000):
    region = detect_region(i)
    if i == 5000:
        breakpoint()  # 只在这里暂停
    result = recognize(region)
\end{verbatim}
        \end{column}
    \end{columns}

    \vspace{0.3cm}

    \textbf{复杂条件示例:}
\begin{verbatim}
# 只在密度异常时暂停
if density < 0.3 or density > 0.9:
    breakpoint()

# 只在特定数据时暂停
if user_id == 'problematic_user_123':
    breakpoint()
\end{verbatim}
\end{frame}

\begin{frame}[fragile]{异常断点使用}
    \textbf{自动在异常时暂停:}

    \begin{columns}
        \begin{column}{0.48\textwidth}
            \textbf{IDE设置:}
            \begin{itemize}
                \item PyCharm:Run $\to$ View Breakpoints $\to$ Python Exception
                \item VS Code:调试面板 $\to$ Breakpoints $\to$ 异常
            \end{itemize}

            \vspace{0.2cm}

            \textbf{选择捕获:}
            \begin{itemize}
                \item Raise:抛出异常时
                \item Caught:捕获异常时
            \end{itemize}
        \end{column}
        \begin{column}{0.48\textwidth}
            \textbf{代码中的异常捕获:}
\begin{verbatim}
# 在可疑代码块外
try:
    complex_operation()
except Exception as e:
    # 自动暂停并查看
    breakpoint()
    raise  # 重新抛出
\end{verbatim}
        \end{column}
    \end{columns}

    \vspace{0.3cm}

    \begin{alertblock}{使用场景}
        \begin{itemize}
            \item 代码中某个位置抛出异常,但不知道原因
            \item 异常被上层捕获,难以定位源头
            \item 第三方库中抛出的异常
        \end{itemize}
    \end{alertblock}
\end{frame}

\begin{frame}[fragile]{二分法定位问题}
    \textbf{场景:长流程中某处出错}

\begin{verbatim}
def process_exam(image_path):
    # 第一步
    image = load_image(image_path)
    print(f"[1] 图像加载: {image.shape if image is not None else 'None'}")

    # 第二步
    preprocessed = preprocess(image)
    print(f"[2] 预处理: {preprocessed.shape}")

    # 第三步
    regions = detect_regions(preprocessed)
    print(f"[3] 检测到区域: {len(regions)}")

    # 第四步
    answers = recognize(regions)
    print(f"[4] 识别结果: {answers}")

    return answers
\end{verbatim}

    \textbf{定位技巧:}
    \begin{itemize}
        \item 如果[2]输出正常但[3]失败,问题在detect\_regions
        \item 使用二分法快速缩小问题范围
        \item 在关键步骤添加检查点
    \end{itemize}
\end{frame}

\begin{frame}[fragile]{性能分析与优化}
    \textbf{使用cProfile分析性能:}
\begin{verbatim}
import cProfile
import pstats

def profile_process():
    profiler = cProfile.Profile()
    profiler.enable()

    # 执行要分析的代码
    result = process_exam("test.jpg")

    profiler.disable()

    # 输出统计
    stats = pstats.Stats(profiler)
    stats.sort_stats('cumulative')
    stats.print_stats(20)  # 打印前20个最耗时的函数

profile_process()
\end{verbatim}

    \textbf{典型输出:}
\begin{verbatim}
   ncalls  tottime  percall  cumtime  percall filename:lineno(function)
      100    0.003    0.000    0.003    0.000 detector.py:45(detect)
        1    0.002    0.002    0.015    0.015 main.py:10(process)
     1000    0.001    0.000    0.001    0.000 recognizer.py:23(recognize)
\end{verbatim}
\end{frame}

%----------------------------------------------------------
\subsection{AI辅助调试实战}
%----------------------------------------------------------

\begin{frame}[fragile]{用AI理解复杂代码}
    \textbf{场景:不理解OpenCV函数参数含义}

    \begin{columns}
        \begin{column}{0.48\textwidth}
            \textbf{向AI提问(RTF框架):}
\begin{verbatim}
角色:OpenCV专家
任务:解释这个函数的参数
格式:分点说明,举例说明

cv2.findContours(
    binary, cv2.RETR_EXTERNAL,
    cv2.CHAIN_APPROX_SIMPLE
)
\end{verbatim}
        \end{column}
        \begin{column}{0.48\textwidth}
            \textbf{AI会解释:}
            \begin{itemize}
                \item \textbf{binary}: 输入的二值图像
                \item \textbf{RETR\_EXTERNAL}: 只检测外轮廓
                \item \textbf{CHAIN\_APPROX\_SIMPLE}: 压缩轮廓存储
            \end{itemize}

            \vspace{0.2cm}

            \textbf{AI还会提供:}
            \begin{itemize}
                \item 返回值说明(轮廓列表、层次结构)
                \item 使用示例代码
                \item 常见错误及解决方案
            \end{itemize}
        \end{column}
    \end{columns}
\end{frame}

\begin{frame}[fragile]{用AI生成测试用例}
    \textbf{场景:为填涂识别函数生成测试}

    \begin{columns}
        \begin{column}{0.48\textwidth}
            \textbf{代码函数:}
\begin{verbatim}
def recognize_filled(image):
    gray = cv2.cvtColor(image,
        cv2.COLOR_BGR2GRAY)
    density = np.sum(gray > 127) /
               gray.size
    return density > 0.5
\end{verbatim}
        \end{column}
        \begin{column}{0.48\textwidth}
            \textbf{Prompt模板:}
\begin{verbatim}
为这个函数生成pytest测试用例
覆盖以下场景:
1. 填涂区域(应返回True)
2. 空白区域(应返回False)
3. 部分填涂(边界情况)
4. 空图像(异常处理)
\end{verbatim}
        \end{column}
    \end{columns}

    \vspace{0.2cm}

    \textbf{AI生成的测试:}
\begin{verbatim}
@pytest.mark.parametrize("image,expected", [
    (create_filled_img(), True),
    (create_empty_img(), False),
    (create_partial_img(0.6), True),
    (create_partial_img(0.4), False),
])
def test_recognize_filled(image, expected):
    assert recognize_filled(image) == expected

def test_empty_image():
    with pytest.raises(ValueError):
        recognize_filled(None)
\end{verbatim}
\end{frame}

\begin{frame}[fragile]{用AI辅助重构}
    \textbf{场景:识别长函数中的Code Smell}

    \begin{columns}
        \begin{column}{0.48\textwidth}
            \textbf{原始代码(有坏味道):}
\begin{verbatim}
def process_sheet(path):
    img = cv2.imread(path)
    gray = cv2.cvtColor(img,
        cv2.COLOR_BGR2GRAY)
    binary = cv2.threshold(
        gray, 127, 255,
        cv2.THRESH_BINARY)[1]
    # ...100+行混杂逻辑
\end{verbatim}
        \end{column}
        \begin{column}{0.48\textwidth}
            \textbf{向AI提问:}
\begin{verbatim}
这段代码有什么Code Smell?
请重构为更清晰的模块化结构
要求:
1. 提取独立函数
2. 添加文档字符串
3. 处理错误情况
4. 使用类型注解
\end{verbatim}
        \end{column}
    \end{columns}

    \vspace{0.2cm}

    \textbf{AI重构建议:}
\begin{itemize}
    \item 提取 \texttt{load\_image()} 函数
    \item 提取 \texttt{binarize()} 函数
    \item 提取 \texttt{find\_regions()} 函数
    \item 使用异常处理而非返回None
    \item 添加类型注解提高可读性
\end{itemize}
\end{frame}

\begin{frame}[fragile]{AI辅助错误诊断}
    \textbf{场景:遇到难以理解的错误信息}

    \textbf{错误信息:}
\begin{verbatim}
OpenCV Error: Assertion failed
((npoints == prevpoints.size()) && ...
\end{verbatim}

    \begin{columns}
        \begin{column}{0.48\textwidth}
            \textbf{有效的Prompt:}
\begin{verbatim}
我遇到了OpenCV错误:
[粘贴完整错误信息和相关代码]

请告诉我:
1. 这个错误的可能原因是什么?
2. 如何定位具体问题?
3. 有哪些解决方案?
\end{verbatim}
        \end{column}
        \begin{column}{0.48\textwidth}
            \textbf{AI诊断流程:}
            \begin{enumerate}
                \item 分析错误类型(断言失败)
                \item 指出可能原因(点集数量变化)
                \item 提供检查步骤(打印变量值)
                \item 给出修复方案(检查预处理逻辑)
            \end{enumerate}
        \end{column}
    \end{columns}
\end{frame}

\begin{frame}{AI辅助调试最佳实践}
    \textbf{何时使用AI辅助:}

    \begin{table}
        \centering
        \small
        \begin{tabular}{lp{8cm}}
            \toprule
            \textbf{场景} & \textbf{最佳实践} \\
            \midrule
            理解API文档 & 提供"角色+任务+格式"的Prompt \\
            生成测试代码 & 指定测试框架和覆盖场景 \\
            重构建议 & 说明具体重构目标和约束 \\
            错误诊断 & 提供完整错误信息和上下文 \\
            性能优化 & 描述当前性能和优化目标 \\
            \bottomrule
        \end{tabular}
    \end{table}

    \textbf{Prompt工程技巧:}
    \begin{itemize}
        \item \textbf{RTF}:Role(角色)、Task(任务)、Format(格式)
        \item \textbf{上下文}:提供足够的代码和错误信息
        \item \textbf{约束}:明确输出格式和具体要求
        \item \textbf{迭代}:根据结果逐步细化Prompt
    \end{itemize}
\end{frame}

%----------------------------------------------------------
\subsection{测试驱动开发实战}
%----------------------------------------------------------

\begin{frame}[fragile]{TDD循环实践}
    \textbf{场景:实现填涂识别功能}

    \begin{columns}
        \begin{column}{0.48\textwidth}
            \textbf{第一步:红灯(写失败的测试)}
\begin{verbatim}
def test_filled_recognized():
    image = create_filled_image()
    result = recognize_filled(image)
    assert result == True
    # 运行:失败!函数不存在
\end{verbatim}
        \end{column}
        \begin{column}{0.48\textwidth}
            \textbf{第二步:绿灯(最少代码通过)}
\begin{verbatim}
def recognize_filled(image):
    # 最简单的实现
    return True
    # 测试通过!
\end{verbatim}
        \end{column}
    \end{columns}

    \vspace{0.2cm}

    \textbf{第三步:重构(优化实现)}
\begin{verbatim}
# 添加更多测试
def test_empty_not_recognized():
    image = create_empty_image()
    assert recognize_filled(image) == False

# 重构实现
def recognize_filled(image):
    gray = cv2.cvtColor(image, cv2.COLOR_BGR2GRAY)
    density = np.sum(gray > 127) / gray.size
    return density > 0.5
\end{verbatim}
\end{frame}

\begin{frame}[fragile]{参数化测试}
    \textbf{减少重复代码:}
\begin{verbatim}
# 传统方式:重复
def test_threshold_0_3():
    assert recognize(img, 0.3) == 'A'

def test_threshold_0_5():
    assert recognize(img, 0.5) == 'A'

def test_threshold_0_7():
    assert recognize(img, 0.7) == None

# 参数化:简洁
@pytest.mark.parametrize("threshold,expected", [
    (0.3, 'A'),
    (0.5, 'A'),
    (0.7, None),
])
def test_recognize_threshold(threshold, expected):
    assert recognize(test_img, threshold) == expected
\end{verbatim}

    \textbf{多参数组合:}
\begin{verbatim}
@pytest.mark.parametrize("image,threshold,expected", [
    (filled_img, 0.5, 'A'),
    (empty_img, 0.5, None),
    (partial_img, 0.3, 'A'),
    (partial_img, 0.7, None),
])
def test_recognize_variations(image, threshold, expected):
    assert recognize(image, threshold) == expected
\end{verbatim}
\end{frame}

\begin{frame}[fragile]{Mock实战:隔离外部依赖}
    \textbf{场景:测试图像加载,但不依赖真实文件}

    \begin{columns}
        \begin{column}{0.48\textwidth}
            \textbf{问题:测试依赖文件系统}
\begin{verbatim}
def test_load_image():
    img = load_image("test.jpg")
    # 需要真实文件
    # 文件可能不存在
    # 测试不稳定
\end{verbatim}
        \end{column}
        \begin{column}{0.48\textwidth}
            \textbf{解决:Mock cv2.imread}
\begin{verbatim}
from unittest.mock import patch

@patch('cv2.imread')
def test_load_image(mock_imread):
    # 设置返回值
    mock_imread.return_value = fake_image

    # 测试
    result = load_image("any_path.jpg")

    # 验证
    assert result is not None
    mock_imread.assert_called_once_with(
        "any_path.jpg"
    )
\end{verbatim}
        \end{column}
    \end{columns}

    \vspace{0.2cm}

    \textbf{Mock异常情况:}
\begin{verbatim}
@patch('cv2.imread')
def test_load_failure(mock_imread):
    mock_imread.return_value = None  # 模拟加载失败

    with pytest.raises(ImageLoadError):
        load_image("invalid.jpg")
\end{verbatim}
\end{frame}

\begin{frame}[fragile]{测试夹具(Fixture)}
    \textbf{复用测试数据:}
\begin{verbatim}
@pytest.fixture
def sample_image():
    """创建标准测试图像"""
    return np.ones((100, 100, 3), dtype=np.uint8) * 255

@pytest.fixture
def sample_regions():
    """创建标准测试区域"""
    return [
        Rectangle(10, 10, 20, 20),
        Rectangle(40, 10, 20, 20),
        Rectangle(70, 10, 20, 20),
    ]

def test_detect_with_sample(sample_image):
    regions = detect(sample_image)
    assert len(regions) > 0

def test_recognize_with_regions(sample_regions):
    for region in sample_regions:
        result = recognize(region)
        assert result is not None
\end{verbatim}

    \textbf{带清理的Fixture:}
\begin{verbatim}
@pytest.fixture
def temp_file():
    file = tempfile.NamedTemporaryFile(delete=False)
    yield file.name
    file.close()  # 清理
    os.unlink(file.name)
\end{verbatim}
\end{frame}

%----------------------------------------------------------
\subsection{综合实战:完整系统开发}
%----------------------------------------------------------

\begin{frame}{系统架构实现}
    \textbf{智能阅卷系统架构:}

    \begin{center}
        \begin{tikzpicture}[
            node distance=0.8cm,
            box/.style={draw, fill=blue!15, minimum width=2.5cm, minimum height=0.7cm, align=center, font=\small},
            arrow/.style={-Stealth, thick}
        ]
            % 顶层
            \node[box, fill=green!30] (app) {Application\\应用层};
            \node[box, below=of app] (svc) {GradingService\\服务层};
            \node[box, below=of svc] (infra) {Infrastructure\\基础设施层};

            % 第二层
            \node[box, left=2cm of svc] (prep) {Preprocessor\\预处理};
            \node[box, right=2cm of svc] (rec) {Recognizer\\识别};

            % 箭头
            \draw[arrow] (app) -- (svc);
            \draw[arrow] (svc) -- (infra);
            \draw[arrow, dashed] (svc) -- (prep);
            \draw[arrow, dashed] (svc) -- (rec);
        \end{tikzpicture}
    \end{center}

    \textbf{开发顺序:}
    \begin{enumerate}
        \item 先实现基础设施层(图像加载、文件操作)
        \item 再实现核心功能层(预处理、检测、识别)
        \item 最后实现服务层和应用层(业务编排)
    \end{enumerate}
\end{frame}

\begin{frame}[fragile]{核心功能实现:预处理模块}
\begin{verbatim}
class ImagePreprocessor:
    """图像预处理模块"""

    def __init__(self, config: PreprocessConfig):
        self.config = config
        self.logger = setup_logging(self.__class__.__name__)

    def load(self, path: str) -> np.ndarray:
        """加载图像"""
        self.logger.info(f"加载图像: {path}")
        img = cv2.imread(path)
        if img is None:
            raise ImageLoadError(f"无法加载: {path}")
        return img

    def normalize(self, image: np.ndarray) -> np.ndarray:
        """归一化处理"""
        normalized = cv2.normalize(
            image, None, 0, 255,
            cv2.NORM_MINMAX
        )
        self.logger.debug(f"归一化: {image.shape}")
        return normalized

    def enhance(self, image: np.ndarray) -> np.ndarray:
        """图像增强"""
        # 去噪
        denoised = cv2.fastNlMeansDenoisingColored(image)
        # 对比度增强
        lab = cv2.cvtColor(denoised, cv2.COLOR_BGR2LAB)
        l, a, b = cv2.split(lab)
        l = cv2.createCLAHE(clipLimit=2.0).apply(l)
        enhanced = cv2.cvtColor(cv2.merge([l,a,b]), cv2.COLOR_LAB2BGR)
        return enhanced
\end{verbatim}
\end{frame}

\begin{frame}[fragile]{核心功能实现:识别模块}
\begin{verbatim}
class ChoiceRecognizer:
    """选择题识别模块"""

    def __init__(self, config: RecognitionConfig):
        self.config = config
        self.logger = setup_logging(self.__class__.__name__)

    def recognize(self, image: np.ndarray,
                  options: List[str]) -> Optional[str]:
        """识别选择题答案"""
        if image is None:
            raise ValueError("图像不能为空")

        self.logger.debug(f"识别选项: {options}")

        results = {}
        for idx, option in enumerate(options):
            region = self._extract_region(image, idx)
            density = self._calculate_density(region)
            results[option] = density
            self.logger.debug(f"{option}: {density:.3f}")

        # 返回密度最高的选项
        max_option = max(results.items(), key=lambda x: x[1])
        if max_option[1] > self.config.threshold:
            return max_option[0]
        return None

    def _calculate_density(self, image: np.ndarray) -> float:
        """计算填涂密度"""
        gray = cv2.cvtColor(image, cv2.COLOR_BGR2GRAY)
        binary = cv2.threshold(gray, 127, 255,
                              cv2.THRESH_BINARY)[1]
        return np.sum(binary == 0) / binary.size
\end{verbatim}
\end{frame}

\begin{frame}[fragile]{系统组装与集成}
\begin{verbatim}
class GradingService:
    """阅卷服务"""

    def __init__(self, config: SystemConfig):
        self.prep = ImagePreprocessor(config.preprocess)
        self.detector = RegionDetector(config.detect)
        self.recognizer = ChoiceRecognizer(config.recognize)
        self.calculator = GradeCalculator(config.grade)
        self.logger = setup_logging(self.__class__.__name__)

    def process(self, image_path: str) -> GradingResult:
        """处理答题卡"""
        try:
            # 加载
            image = self.prep.load(image_path)

            # 预处理
            normalized = self.prep.normalize(image)
            enhanced = self.prep.enhance(normalized)

            # 检测
            regions = self.detector.find_regions(enhanced)
            self.logger.info(f"检测到 {len(regions)} 个区域")

            # 识别
            answers = {}
            for qid, region in regions.items():
                result = self.recognizer.recognize(region, ['A','B','C','D'])
                answers[qid] = result

            # 评分
            score = self.calculator.calculate(answers)

            return GradingResult(answers=answers, score=score)

        except GradingError as e:
            self.logger.error(f"处理失败: {e}")
            raise
\end{verbatim}
\end{frame}

\begin{frame}{单元测试与集成测试}
    \textbf{单元测试:}
    \begin{itemize}
        \item 测试各模块独立功能
        \item 使用Mock隔离依赖
        \item 快速、稳定、可重复
    \end{itemize}

    \textbf{集成测试:}
    \begin{itemize}
        \item 测试模块间协作
        \item 验证接口契约
        \item 使用真实或测试数据
    \end{itemize}

    \begin{block}{测试策略}
        \begin{itemize}
            \item \textbf{Preprocessor}: 单元测试,Mock cv2函数
            \item \textbf{Recognizer}: 单元测试 + 参数化测试
            \item \textbf{GradingService}: 集成测试,验证完整流程
            \item \textbf{端到端}: 测试真实图像文件处理
        \end{itemize}
    \end{block}
\end{frame}

%----------------------------------------------------------
% 本模块要点速查(新增)
%----------------------------------------------------------

\subsection{本模块要点速查}

\begin{frame}{模块03-Live Coding:要点速查}
    \begin{block}{重构技术}
        \begin{itemize}
            \item 提取函数(长函数→多个小函数)
            \item 提取类(职责分离)
            \item 参数对象(过多参数→配置类)
            \item 替换魔法数字(有意义的常量)
        \end{itemize}
    \end{block}

    \begin{block}{异常处理策略}
        \begin{itemize}
            \item 自定义异常层次
            \item 捕获底层异常→转换业务异常
            \item 防御式编程:前置条件检查
        \end{itemize}
    \end{block}

    \begin{block}{TDD循环}
        \texttt{红灯(测试失败) → 绿灯(最少代码) → 重构(优化)}
    \end{block}
\end{frame}

%----------------------------------------------------------
% 现场练习设计(新增)
%----------------------------------------------------------

\subsection{现场练习}

\begin{frame}{现场练习:识别Bug}
    \textbf{任务}:找出以下代码中的Bug(3分钟)

    \begin{columns}
        \begin{column}{0.5\textwidth}
\begin{Shaded}
def calculate_density(regions):
    results = []
    for region in regions:
        if region is not None:
            density = sum(region) / len(region)
            results.append(density)
    return results

# 使用
img = None
result = calculate_density([img])
print(result)
\end{Shaded}
        \end{column}
        \begin{column}{0.5\textwidth}
            \textbf{提示}:运行这段代码会发生什么?

            \vspace{0.3cm}

            \textbf{Bug分析}:
            \begin{itemize}
                \item None对象无法参与运算
                \item 应在循环内检查,而非循环前
                \item 缺少异常处理
            \end{itemize}
        \end{column}
    \end{columns}
\end{frame}

\begin{frame}{现场练习:完善代码}
    \textbf{任务}:完善以下代码,添加错误处理(5分钟)

    \begin{block}{待完善代码}
\begin{Shaded}
def recognize_choice(image, options):
    results = {}
    for option in options:
        region = image[option]
        density = np.sum(region > 127) / region.size
        results[option] = density
    return results
\end{Shaded}
    \end{block}

    \begin{columns}
        \begin{column}{0.48\textwidth}
            \textbf{需要处理的情况:}
            \begin{itemize}
                \item image为None
                \item option不在image中
                \item region为空
            \end{itemize}
        \end{column}
        \begin{column}{0.48\textwidth}
            \textbf{参考解答要点:}
\begin{Shaded}
if image is None:
    raise ValueError(...)
if option not in image:
    raise KeyError(...)
if region.size == 0:
    continue  # 或return {}
\end{Shaded}
        \end{column}
    \end{columns}
\end{frame}
