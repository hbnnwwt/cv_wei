%===========================================================
% 模块02:工具与环境介绍 (5-8页)
%===========================================================

\section{工具与环境}

%----------------------------------------------------------
\subsection{Python开发工具}
%----------------------------------------------------------

\begin{frame}{IDE选择与对比}
    \begin{columns}
        \begin{column}{0.32\textwidth}
            \textbf{PyCharm}
            \begin{itemize}
                \item 专业Python IDE
                \item 智能代码补全
                \item 强大的调试器
                \item 集成测试工具
                \item 专业版收费
            \end{itemize}
        \end{column}
        \begin{column}{0.32\textwidth}
            \textbf{VS Code}
            \begin{itemize}
                \item 轻量编辑器
                \item 插件生态丰富
                \item Python扩展强大
                \item 免费开源
                \item 多语言支持
            \end{itemize}
        \end{column}
        \begin{column}{0.32\textwidth}
            \textbf{Jupyter}
            \begin{itemize}
                \item 交互式环境
                \item 适合数据探索
                \item 可视化集成
                \item 文档混排
                \item 教学演示友好
            \end{itemize}
        \end{column}
    \end{columns}

    \vspace{0.5cm}

    \begin{block}{推荐选择}
        \begin{itemize}
            \item 大型项目开发:PyCharm(专业版)
            \item 日常/多语言开发:VS Code
            \item 数据分析/教学:Jupyter Notebook
        \end{itemize}
    \end{block}
\end{frame}

\begin{frame}[fragile]{调试工具详解}
    \textbf{Python内置调试器pdb:}

    \begin{lstlisting}[basicstyle=\ttfamily\scriptsize]
import pdb

def process_image(image_path):
    image = load_image(image_path)
    pdb.set_trace()  \#设置断点
    result = analyze(image)
    return result

\#常用pdb命令
\#n - 下一行(Step Over)
\#s - 进入函数(Step Into)
\#c - 继续执行
\#p var - 打印变量
\#q - 退出调试
    \end{lstlisting}

    \textbf{ipdb增强版:}
    \begin{itemize}
        \item 语法高亮
        \item Tab自动补全
        \item 更好的交互体验
    \end{itemize}
\end{frame}

\begin{frame}[fragile]{性能分析工具}
    \textbf{cProfile - 标准库性能分析:}

    \begin{lstlisting}[basicstyle=\ttfamily\scriptsize]
import cProfile
import pstats

\#分析代码
profiler = cProfile.Profile()
profiler.enable()

\#要分析的代码
process_batch(images)

profiler.disable()
stats = pstats.Stats(profiler)
stats.sort_stats('cumulative')
stats.print_stats(20)  \#打印前20个热点
    \end{lstlisting}

    \vspace{0.3cm}

    \textbf{line\_profiler - 逐行分析:}
    \begin{itemize}
        \item 精确到每行代码的耗时
        \item 使用\texttt{@profile}装饰器标记
        \item 适合定位性能瓶颈
    \end{itemize}
\end{frame}

\begin{frame}[fragile]{代码质量工具}
    \begin{columns}
        \begin{column}{0.48\textwidth}
            \textbf{静态检查工具}
            \begin{itemize}
                \item \textbf{pylint}:全面代码检查
                \item \textbf{flake8}:PEP 8 + 复杂度检查
                \item \textbf{mypy}:静态类型检查
                \item \textbf{bandit}:安全漏洞扫描
            \end{itemize}
        \end{column}
        \begin{column}{0.48\textwidth}
            \textbf{代码格式化工具}
            \begin{itemize}
                \item \textbf{black}:严格格式化,无配置
                \item \textbf{autopep8}:PEP 8自动修复
                \item \textbf{isort}:import排序
                \item \textbf{yapf}:Google出品,可配置
            \end{itemize}
        \end{column}
    \end{columns}

    \vspace{0.5cm}

    \begin{lstlisting}[basicstyle=\ttfamily\scriptsize]
\#使用示例
$ flake8 myproject/          \#检查代码风格
$ black myproject/           \#自动格式化
$ mypy myproject/            \#类型检查
$ bandit -r myproject/       \#安全扫描
    \end{lstlisting}
\end{frame}

%----------------------------------------------------------
\subsection{测试框架与工具}
%----------------------------------------------------------

\begin{frame}{pytest测试框架}
    \textbf{为什么选择pytest?}

    \begin{itemize}
        \item 简洁的断言语法(无需\texttt{assertEqual})
        \item 强大的Fixture机制
        \item 丰富的插件生态
        \item 详细的失败报告
    \end{itemize}

    \vspace{0.3cm}

    \textbf{unittest风格:}
    \begin{itemize}
        \item[] \texttt{class TestRecognizer(unittest.TestCase):}
        \item[] \hspace{1em} \texttt{def test(self):}
        \item[] \hspace{2em} \texttt{assert result == 'A'}
    \end{itemize}

    \textbf{pytest风格:}
    \begin{itemize}
        \item[] \texttt{def test\_recognize():}
        \item[] \hspace{1em} \texttt{assert result == 'A'}
    \end{itemize}
\end{frame}

\begin{frame}[fragile]{Mock与测试隔离}
    \textbf{为什么要Mock?}

    \begin{itemize}
        \item 隔离被测代码,避免外部依赖
        \item 控制测试环境,确保可重复
        \item 模拟异常情况,提高覆盖率
        \item 加速测试执行(避免真实IO)
    \end{itemize}

    \vspace{0.3cm}

    \textbf{Mock对象示例:}
    \begin{itemize}
        \item[] \texttt{from unittest.mock import Mock}
        \item[] \texttt{mock = Mock()}
        \item[] \texttt{mock.method.return\_value = 42}
        \item[] \texttt{assert mock.method() == 42}
    \end{itemize}

    \textbf{Patch装饰器:}
    \begin{itemize}
        \item[] \texttt{@patch('module.function')}
        \item[] \texttt{def test(mocked\_func):}
        \item[] \hspace{1em} \texttt{mocked\_func.return\_value = 42}
    \end{itemize}
\end{frame}

\begin{frame}{测试覆盖率}
    \textbf{coverage.py使用:}

    \begin{itemize}
        \item[] \texttt{\$ pip install coverage pytest-cov}
        \item[] \texttt{\$ coverage run -m pytest}
        \item[] \texttt{\$ coverage report}
        \item[] \texttt{\$ coverage html}
        \item[] \texttt{\$ open htmlcov/index.html}
    \end{itemize}

    \vspace{0.3cm}

    \textbf{覆盖率目标:}
    \begin{itemize}
        \item 核心逻辑:80\%以上
        \item 工具类:90\%以上
        \item 但高覆盖率不等于高质量,关键是有效测试
    \end{itemize}
\end{frame}

%----------------------------------------------------------
\subsection{版本控制与协作}
%----------------------------------------------------------

\begin{frame}{Git高级操作}
    \textbf{常用高级命令:}

    \begin{itemize}
        \item \textbf{交互式暂存:} \texttt{git add -p}
        \item \textbf{修改提交:} \texttt{git commit --amend}
        \item \textbf{储存修改:} \texttt{git stash} / \texttt{git stash pop}
        \item \textbf{变基整理:} \texttt{git rebase -i HEAD\textasciitilde{}3}
        \item \textbf{二分调试:} \texttt{git bisect start}
    \end{itemize}
\end{frame}

\begin{frame}{Git工作流}
    \begin{columns}
        \begin{column}{0.48\textwidth}
            \textbf{Feature Branch工作流}
            \begin{itemize}
                \item 主分支保持稳定
                \item 每个功能新建分支
                \item 完成后Code Review合并
                \item 适合大多数团队
            \end{itemize}
        \end{column}
        \begin{column}{0.48\textwidth}
            \textbf{Forking工作流}
            \begin{itemize}
                \item 开发者fork仓库
                \item 在自己的仓库开发
                \item 通过Pull Request贡献
                \item 适合开源项目
            \end{itemize}
        \end{column}
    \end{columns}

    \vspace{0.5cm}

    \textbf{Commit Message规范:}
    \begin{itemize}
        \item \texttt{feat:} 新功能
        \item \texttt{fix:} 修复bug
        \item \texttt{docs:} 文档更新
        \item \texttt{refactor:} 重构
        \item \texttt{test:} 测试相关
        \item \texttt{chore:} 构建/工具
    \end{itemize}
\end{frame}

%----------------------------------------------------------
\subsection{AI编程工具}
%----------------------------------------------------------

\begin{frame}{AI编程工具概览}
    \textbf{AI改变编程方式:}

    \vspace{0.3cm}

    \begin{columns}
        \begin{column}{0.48\textwidth}
            \textbf{传统编程流程:}
            \begin{enumerate}
                \item 查阅文档
                \item 编写代码
                \item 调试测试
                \item 优化重构
            \end{enumerate}
        \end{column}
        \begin{column}{0.48\textwidth}
            \textbf{AI辅助编程流程:}
            \begin{enumerate}
                \item 描述需求
                \item AI生成代码
                \item 验证测试
                \item AI辅助优化
            \end{enumerate}
        \end{column}
    \end{columns}

    \vspace{0.5cm}

    \textbf{主流AI编程工具:}
    \begin{itemize}
        \item \textbf{Cursor}:AI原生IDE,内置GPT-4
        \item \textbf{Claude Code}:Anthropic官方CLI工具
        \item \textbf{GitHub Copilot}:代码自动补全
    \end{itemize}
\end{frame}

\begin{frame}[fragile]{Cursor:AI原生IDE}
    \textbf{核心特性:}
    \begin{itemize}
        \item \textbf{AI Chat}:内置对话界面,无需切换窗口
        \item \textbf{代码生成}:根据描述生成完整函数
        \item \textbf{代码解释}:选中代码即可获得详细解释
        \item \textbf{重构建议}:自动识别Code Smell并优化
    \end{itemize}

    \vspace{0.3cm}

    \begin{columns}
        \begin{column}{0.48\textwidth}
            \textbf{快捷键:}
            \begin{itemize}
                \item \texttt{Cmd+L}:打开AI Chat
                \item \texttt{Cmd+K}:生成/编辑代码
                \item \texttt{Cmd+I}:询问当前文件
            \end{itemize}
        \end{column}
        \begin{column}{0.48\textwidth}
            \textbf{适用场景:}
            \begin{itemize}
                \item 快速原型开发
                \item 代码理解学习
                \item 重构优化代码
            \end{itemize}
        \end{column}
    \end{columns}
\end{frame}

\begin{frame}[fragile]{Claude Code:命令行AI助手}
    \textbf{Claude Code特点:}

    \vspace{0.3cm}

    \begin{columns}
        \begin{column}{0.48\textwidth}
            \textbf{优势:}
            \begin{itemize}
                \item 长上下文窗口(200K tokens)
                \item 精准代码理解
                \item 多文件编辑能力
                \item 命令行无缝集成
            \end{itemize}
        \end{column}
        \begin{column}{0.48\textwidth}
            \textbf{使用方式:}
\begin{verbatim}
\#安装
npm install -g @anthropic-ai/claude-code

\#使用
claude  \#启动交互会话
\end{verbatim}
        \end{column}
    \end{columns}

    \vspace{0.3cm}

    \textbf{实战示例:}
\begin{verbatim}
\#用户:解释这个检测填涂的函数
[选中代码]

\#Claude:这个函数通过灰度化和阈值判断...
\#- 先转为灰度图
\#- 计算像素密度
\#- 返回是否填涂
\end{verbatim}
\end{frame}

\begin{frame}[fragile]{GitHub Copilot:智能代码补全}
    \textbf{Copilot工作原理:}
    \begin{itemize}
        \item 基于OpenAI Codex模型
        \item 从GitHub公开代码学习
        \item 根据上下文自动补全
    \end{itemize}

    \vspace{0.3cm}

    \textbf{使用场景:}
    \begin{columns}
        \begin{column}{0.48\textwidth}
            \textbf{最佳场景:}
            \begin{itemize}
                \item 编写样板代码
                \item 生成测试用例
                \item API调用示例
                \item 正则表达式
            \end{itemize}
        \end{column}
        \begin{column}{0.48\textwidth}
            \textbf{使用技巧:}
            \begin{itemize}
                \item 写注释描述意图
                \item 函数命名要清晰
                \item Tab接受建议
                \item Esc忽略建议
            \end{itemize}
        \end{column}
    \end{columns}

    \vspace{0.3cm}

    \begin{exampleblock}{Copilot示例}
\begin{verbatim}
\#注释:计算答题卡填涂区域密度
def calculate_density(image):
    \#Copilot自动补全以下代码
    gray = cv2.cvtColor(image, cv2.COLOR_BGR2GRAY)
    return np.sum(gray > 127) / gray.size
\end{verbatim}
    \end{exampleblock}
\end{frame}

\begin{frame}[fragile]{AI辅助代码审查}
    \textbf{用AI进行Code Review:}

    \vspace{0.3cm}

    \begin{columns}
        \begin{column}{0.48\textwidth}
            \textbf{Prompt模板:}
\begin{verbatim}
请审查这段代码的质量:

1. 代码规范问题
2. 潜在bug
3. 性能优化建议
4. 重构建议

[粘贴代码]
\end{verbatim}
        \end{column}
        \begin{column}{0.48\textwidth}
            \textbf{AI会检查:}
            \begin{itemize}
                \item 命名是否规范
                \item 是否有重复代码
                \item 错误处理是否完善
                \item 边界条件是否考虑
                \item 性能是否可优化
            \end{itemize}
        \end{column}
    \end{columns}

    \vspace{0.3cm}

    \textbf{实战案例:}
\begin{verbatim}
\#原始代码
def proc(img,p):
    return cv2.threshold(img,p,255,1)[1]

\#AI建议:
\#1. 函数名太短,应改为 process_image
\#2. 参数p应改为 threshold_value
\#3. 应添加参数验证
\#4. 应添加文档字符串
\end{verbatim}
\end{frame}

\begin{frame}[fragile]{AI辅助调试实战}
    \textbf{场景:代码运行报错,看不懂错误信息}

    \vspace{0.2cm}

    \textbf{错误信息:}
\begin{verbatim}
cv2.error: OpenCV(4.8.0) :-1: error: (-5:Bad argument)
in function 'threshold'
> Overwhelming requirement: (img.depth() == CV_8U ||
 img.depth() == CV_32F)
\end{verbatim}

    \vspace{0.2cm}

    \textbf{向AI提问:}
\begin{verbatim}
这段OpenCV代码报错,是什么原因?如何修复?

[粘贴错误信息和相关代码]
\end{verbatim}

    \vspace{0.2cm}

    \textbf{AI诊断:}
    \begin{itemize}
        \item \textbf{问题}:图像深度不符合要求
        \item \textbf{原因}:threshold要求CV\_8U或CV\_32F类型
        \item \textbf{修复}:添加类型转换 \texttt{img.astype(np.uint8)}
    \end{itemize}
\end{frame}

%----------------------------------------------------------
% 本模块要点速查(新增)
%----------------------------------------------------------

\subsection{本模块要点速查}

\begin{frame}{模块02-工具环境:要点速查}
    \begin{columns}
        \begin{column}{0.48\textwidth}
            \textbf{开发工具选择}
            \begin{itemize}
                \item 大型项目:PyCharm
                \item 多语言/轻量:VS Code
                \item 数据分析:Jupyter
            \end{itemize}
        \end{column}
        \begin{column}{0.48\textwidth}
            \textbf{代码质量工具}
            \begin{itemize}
                \item 检查:flake8、pylint
                \item 格式化:black
                \item 类型检查:mypy
            \end{itemize}
        \end{column}
    \end{columns}

    \begin{block}{AI编程工具}
        \begin{itemize}
            \item Cursor:AI原生IDE(\texttt{Cmd+L}对话)
            \item Claude Code:命令行助手
            \item Copilot:代码补全
        \end{itemize}
    \end{block}
\end{frame}
