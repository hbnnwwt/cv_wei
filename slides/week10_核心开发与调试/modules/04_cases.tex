%===========================================================
% 模块04:案例分析与互动 (5-8页)
%===========================================================

\section{案例分析与互动}

%----------------------------------------------------------
\subsection{真实系统开发案例}
%----------------------------------------------------------

\begin{frame}{案例1:小型阅卷系统开发}
    \textbf{项目背景:}
    \begin{itemize}
        \item 范围:单班级,30-50份试卷
        \item 题型:选择题(20题)+ 判断题(10题)
        \item 时间:2周开发周期
        \item 团队:3名学生
    \end{itemize}

    \vspace{0.3cm}

    \textbf{架构设计:}
    \begin{itemize}
        \item 单机版Python应用
        \item OpenCV + NumPy处理图像
        \item CSV格式存储结果
        \item 命令行界面
    \end{itemize}

    \vspace{0.3cm}

    \textbf{经验总结:}
    \begin{itemize}
        \item 先做原型验证核心算法
        \item 用真实试卷数据测试
        \item 预留参数调整接口
    \end{itemize}
\end{frame}

\begin{frame}{案例2:企业级阅卷系统开发}
    \textbf{项目背景:}
    \begin{itemize}
        \item 范围:全校考试,单次10,000+份试卷
        \item 题型:选择题、判断题、填空题、简答题
        \item 时间:3个月开发周期
        \item 团队:8人(前后端、算法、测试)
    \end{itemize}

    \vspace{0.3cm}

    \textbf{架构设计:}
    \begin{itemize}
        \item 微服务架构(识别服务、评分服务、存储服务)
        \item 消息队列处理批量任务
        \item 分布式存储(图片、结果)
        \item Web管理界面 + REST API
    \end{itemize}
\end{frame}

\begin{frame}{案例对比与启示}
    \begin{table}
        \centering
        \small
        \begin{tabular}{lcc}
            \toprule
            \textbf{维度} & \textbf{小型系统} & \textbf{企业级系统} \\
            \midrule
            架构复杂度 & 单体应用 & 微服务 \\
            开发周期 & 2周 & 3个月 \\
            团队规模 & 3人 & 8人 \\
            代码规范 & 基本规范 & 严格规范+审查 \\
            测试覆盖 & 关键功能测试 & 全面测试+自动化 \\
            部署方式 & 手动运行 & CI/CD自动化 \\
            监控运维 & 日志查看 & 完整监控告警 \\
            \bottomrule
        \end{tabular}
    \end{table}

    \vspace{0.3cm}

    \textbf{关键启示:}
    \begin{itemize}
        \item 规模决定架构,不要过度设计
        \item 小型项目也要考虑扩展性
        \item 自动化是规模化的关键
    \end{itemize}
\end{frame}

%----------------------------------------------------------
\subsection{常见问题与解决方案}
%----------------------------------------------------------

\begin{frame}[fragile]{性能瓶颈诊断}
    \textbf{问题:处理一张试卷需要30秒}

    \vspace{0.3cm}

    \textbf{诊断过程:}
    \begin{enumerate}
        \item \textbf{定位瓶颈}:使用cProfile发现75\%时间在OCR
        \item \textbf{分析原因}:每次调用都重新加载模型
        \item \textbf{解决方案}:模型单例化,只加载一次
        \item \textbf{优化效果}:处理时间从30s降到3s
    \end{enumerate}

    \vspace{0.3cm}

    \begin{lstlisting}[basicstyle=\ttfamily\scriptsize]
# 优化前:每次创建新实例
def recognize_text(image):
    ocr = PaddleOCR()  # 耗时操作
    return ocr.ocr(image)

# 优化后:单例模式
_ocr_instance = None
def get_ocr():
    global _ocr_instance
    if _ocr_instance is None:
        _ocr_instance = PaddleOCR()
    return _ocr_instance
    \end{lstlisting}
\end{frame}

\begin{frame}[fragile]{内存泄漏排查}
    \textbf{问题:批量处理时内存持续增长,最终崩溃}

    \vspace{0.3cm}

    \textbf{诊断过程:}
    \begin{enumerate}
        \item 使用\texttt{tracemalloc}跟踪内存分配
        \item 发现图像数组未释放
        \item 确认是循环中加载了大量高分辨率图片
    \end{enumerate}

    \vspace{0.3cm}

    \textbf{解决方案:}
    \begin{lstlisting}[basicstyle=\ttfamily\scriptsize]
def process_batch(image_paths):
    for path in image_paths:
        # 使用上下文管理器确保资源释放
        with ImageProcessor(path) as processor:
            result = processor.process()
            save_result(result)
        # 显式垃圾回收
        if processed_count % 100 == 0:
            gc.collect()
    \end{lstlisting}
\end{frame}

\begin{frame}[fragile]{并发访问问题}
    \textbf{问题:多线程处理时出现随机错误}

    \vspace{0.3cm}

    \textbf{根本原因:}
    \begin{itemize}
        \item OpenCV部分函数非线程安全
        \item 共享状态被多个线程修改
        \item 资源竞争导致数据损坏
    \end{itemize}

    \vspace{0.3cm}

    \textbf{解决方案:}
    \begin{lstlisting}[basicstyle=\ttfamily\scriptsize]
from concurrent.futures import ProcessPoolExecutor
import multiprocessing as mp

def process_images_parallel(image_paths, max_workers=4):
    """使用进程池而非线程池"""
    # OpenCV更适合多进程
    with ProcessPoolExecutor(max_workers=max_workers) as executor:
        results = list(executor.map(process_single, image_paths))
    return results

# 或者使用队列+工作进程模式
def worker(input_queue, output_queue):
    while True:
        task = input_queue.get()
        if task is None:
            break
        result = process_single(task)
        output_queue.put(result)
    \end{lstlisting}
\end{frame}

%----------------------------------------------------------
\subsection{课堂互动与Quiz}
%----------------------------------------------------------

\begin{frame}[fragile]{代码重构挑战}
    \textbf{挑战1:改进这个函数}

    \begin{lstlisting}[basicstyle=\ttfamily\scriptsize]
def process(data):
    if data is not None:
        if len(data) > 0:
            result = []
            for i in range(len(data)):
                if data[i] > 0:
                    result.append(data[i] * 2)
            if len(result) > 0:
                return sum(result) / len(result)
    return 0
    \end{lstlisting}

    \vspace{0.3cm}

    \textbf{问题:}
    \begin{enumerate}
        \item 这段代码有哪些坏味道?
        \item 如何重构使其更清晰?
    \end{enumerate}
\end{frame}

\begin{frame}[fragile]{Bug诊断Quiz}
    \textbf{下面的代码有什么问题?}

    \begin{lstlisting}[basicstyle=\ttfamily\scriptsize]
class Recognizer:
    def __init__(self):
        self.threshold = 0.5
        self.results = []  # 注意这里!

    def recognize(self, image):
        result = self._process(image)
        self.results.append(result)
        return result

# 使用
rec1 = Recognizer()
rec2 = Recognizer()
rec1.recognize(img1)
print(rec2.results)  # 竟然也有数据?!
    \end{lstlisting}

    \vspace{0.3cm}

    \textbf{答案:} 可变默认参数陷阱!\texttt{self.results = []} 在类定义时创建,所有实例共享同一列表。
\end{frame}

\begin{frame}[fragile]{最佳实践投票}
    \textbf{以下哪种做法更好?}

    \vspace{0.3cm}

    \textbf{选项A:}
    \begin{lstlisting}[basicstyle=\ttfamily\scriptsize]
def process(image):
    try:
        return recognize(image)
    except:
        return None
    \end{lstlisting}

    \vspace{0.3cm}

    \textbf{选项B:}
    \begin{lstlisting}[basicstyle=\ttfamily\scriptsize]
def process(image):
    if image is None:
        raise ValueError("图像不能为空")
    try:
        return recognize(image)
    except RecognitionError as e:
        logger.error("识别失败: %s", e)
        raise ProcessingError("处理失败") from e
    \end{lstlisting}
\end{frame}

\begin{frame}{经验分享讨论}
    \textbf{讨论话题:}

    \begin{enumerate}
        \item \textbf{最难忘的Bug}:你遇到最难排查的问题是什么?

        \item \textbf{重构经历}:是否有过"拯救烂代码"的经历?

        \item \textbf{调试技巧}:有什么独门调试技巧?

        \item \textbf{工具推荐}:最喜欢的开发工具是什么?
    \end{enumerate}

    \vspace{0.3cm}

    \textbf{分享格式:}
    \begin{itemize}
        \item 问题描述(1分钟)
        \item 解决过程(2分钟)
        \item 经验教训(1分钟)
    \end{itemize}
\end{frame}

%----------------------------------------------------------
% 课堂Quiz(新增)
%----------------------------------------------------------

\subsection{课堂Quiz}

\begin{frame}{课堂Quiz 1:识别Code Smell}
    \textbf{以下函数有什么问题?(单选)}

    \begin{columns}
        \begin{column}{0.55\textwidth}
            \begin{block}{代码}
\begin{Shaded}
def p(img, thresh=127, mode=1,
      debug=False):
    # 处理图片
    # 50行代码...
    # 如果调试模式
    if debug:
        print("处理中...")
    return result
\end{Shaded}
            \end{block}
        \end{column}
        \begin{column}{0.4\textwidth}
            \begin{block}{选项}
                A. 语法错误

                B. 参数太多

                C. 函数名不规范

                D. 以上都是
            \end{block}

            \\vspace{0.5cm}

            \textbf{答案:D}\\$\newline$
            (函数名p应改为process\_image,参数过多应封装为配置类)
        \end{column}
    \end{columns}
\end{frame}

\begin{frame}{课堂Quiz 2:异常处理}
    \textbf{以下异常处理方式最好的是?}

    \begin{columns}
        \begin{column}{0.48\textwidth}
            \textbf{选项A}
\begin{Shaded}
try:
    f()
except:
    pass
\end{Shaded}
        \end{column}
        \begin{column}{0.48\textwidth}
            \textbf{选项B}
\begin{Shaded}
try:
    f()
except ValueError as e:
    logger.error(e)
    raise
except Exception:
    logger.exception()
\end{Shaded}
        \end{column}
    \end{columns}

    \begin{block}{}
        \textbf{答案:B}\\$\newline$
        原因:1. 不捕获所有异常(避免掩盖bug)\\
        2. 记录错误信息\\
        3. 重新抛出异常让上层处理
    \end{block}
\end{frame}
