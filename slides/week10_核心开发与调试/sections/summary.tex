%===========================================================
% 总结与作业
%===========================================================

\section{总结与作业}

%----------------------------------------------------------
\subsection{知识点总结}
%----------------------------------------------------------

\begin{frame}{系统开发完整流程}
    \begin{center}
        \begin{tikzpicture}[
            node distance=0.8cm,
            phase/.style={draw, fill=blue!15, minimum width=2.8cm, minimum height=0.8cm, align=center, font=\small},
            deliverable/.style={font=\tiny, align=center}
        ]
            % 阶段
            \node[phase] (req) {需求分析};
            \node[phase, right=0.5cm of req] (design) {系统设计};
            \node[phase, right=0.5cm of design] (impl) {编码实现};
            \node[phase, right=0.5cm of impl] (test) {测试验证};
            \node[phase, right=0.5cm of test] (deploy) {部署运维};

            % 连接线
            \draw[-Stealth, thick] (req) -- (design);
            \draw[-Stealth, thick] (design) -- (impl);
            \draw[-Stealth, thick] (impl) -- (test);
            \draw[-Stealth, thick] (test) -- (deploy);

            % 产出物
            \node[deliverable, below=0.1cm of req] {需求文档\\用例图};
            \node[deliverable, below=0.1cm of design] {架构图\\接口设计};
            \node[deliverable, below=0.1cm of impl] {源代码\\注释文档};
            \node[deliverable, below=0.1cm of test] {测试报告\\缺陷列表};
            \node[deliverable, below=0.1cm of deploy] {运维手册\\监控告警};
        \end{tikzpicture}
    \end{center}

    \vspace{0.3cm}

    \textbf{核心原则:}
    \begin{itemize}
        \item 需求驱动,文档先行
        \item 迭代增量,持续交付
        \item 测试保护,重构无忧
    \end{itemize}
\end{frame}

\begin{frame}{调试技术速查表}
    \begin{columns}
        \begin{column}{0.48\textwidth}
            \textbf{问题定位方法}
            \begin{itemize}
                \item 二分法:快速缩小范围
                \item 假设验证:科学排查思路
                \item 最小复现:剥离干扰因素
                \item 日志追踪:生产环境必备
            \end{itemize}
        \end{column}
        \begin{column}{0.48\textwidth}
            \textbf{调试工具选择}
            \begin{itemize}
                \item IDE调试器:单步、断点、变量
                \item pdb/ipdb:命令行调试
                \item print/logging:简单直接
                \item 性能分析:cProfile、line\_profiler
            \end{itemize}
        \end{column}
    \end{columns}

    \vspace{0.3cm}

    \textbf{调试黄金法则:}
    \begin{enumerate}
        \item 先复现,再定位
        \item 先假设,再验证
        \item 先理解,再修复
        \item 先测试,再提交
    \end{enumerate}
\end{frame}

\begin{frame}{测试策略对比表}
    \begin{table}
        \centering
        \scriptsize
        \begin{tabular}{lccc}
            \toprule
            \textbf{测试类型} & \textbf{粒度} & \textbf{速度} & \textbf{范围} \\
            \midrule
            单元测试 & 函数/类 & 极快(毫秒) & 单个组件 \\
            集成测试 & 模块间 & 较快(秒) & 接口交互 \\
            端到端测试 & 完整流程 & 慢(分钟) & 用户场景 \\
            \bottomrule
        \end{table}

        \vspace{0.3cm}

        \textbf{测试金字塔原则:}
        \begin{itemize}
            \item 大量单元测试(70\%)
            \item 适量集成测试(20\%)
            \item 少量端到端测试(10\%)
        \end{itemize}
    \end{table}
\end{frame}

\begin{frame}{常见问题解决方案汇总}
    \begin{table}
        \centering
        \scriptsize
        \begin{tabular}{lp{6cm}}
            \toprule
            \textbf{问题类型} & \textbf{解决方案} \\
            \midrule
            性能瓶颈 & 分析热点、算法优化、缓存、异步处理 \\
            内存泄漏 & 检查循环引用、及时释放资源、gc监控 \\
            并发问题 & 锁机制、线程安全、进程隔离 \\
            识别不准 & 调参、数据增强、模型优化 \\
            模块集成失败 & 检查接口、版本兼容、依赖管理 \\
            测试不稳定 & 隔离依赖、Mock外部服务、固定随机种子 \\
            \bottomrule
        \end{tabular}
    \end{table}
\end{frame}

%----------------------------------------------------------
\subsection{课后作业}
%----------------------------------------------------------

\begin{frame}{任务层级(可选路径)}
    \textbf{三条路径,适合不同基础的同学:}

    \vspace{0.3cm}

    \begin{columns}
        \begin{column}{0.32\textwidth}
            \begin{block}{观察者路径\\(基础60分)}
                \begin{itemize}
                    \item 理解代码架构
                    \item 运行框架代码
                    \item 调整参数功能
                \end{itemize}
                \vspace{0.2cm}
                \textbf{适合:}\\编程基础较弱
            \end{block}
        \end{column}
        \begin{column}{0.32\textwidth}
            \begin{block}{使用者路径\\(进阶+20分)}
                \begin{itemize}
                    \item 补充模块代码
                    \item 编写单元测试
                    \item 优化准确率80\%+
                \end{itemize}
                \vspace{0.2cm}
                \textbf{适合:}\\有编程基础
            \end{block}
        \end{column}
        \begin{column}{0.32\textwidth}
            \begin{block}{创造者路径\\(挑战+20分)}
                \begin{itemize}
                    \item 设计新功能
                    \item AI辅助重构
                    \item 性能优化
                \end{itemize}
                \vspace{0.2cm}
                \textbf{适合:}编程能力较强
            \end{block}
        \end{column}
    \end{columns}
\end{frame}

\begin{frame}{本周核心任务}
    \begin{alertblock}{所有路径必须完成(P0任务)}
        \begin{enumerate}
            \item \textbf{完成模块开发}(至少选择题+判断题)
            \item \textbf{完成模块集成}
            \item \textbf{通过基本功能测试}
        \end{enumerate}
    \end{alertblock}

    \vspace{0.3cm}

    \textbf{各路径额外任务:}
    \begin{itemize}
        \item \textbf{观察者}:完成基础任务即可获得60分
        \item \textbf{使用者}:补充核心模块+测试,额外获得20分
        \item \textbf{创造者}:新功能+重构+优化,额外获得20分
    \end{itemize}
\end{frame}

\begin{frame}{作业详细要求}
    \textbf{代码质量要求:}
    \begin{itemize}
        \item 函数长度不超过50行
        \item 每个函数有清晰文档字符串
        \item 关键逻辑有注释说明
        \item 命名清晰,见名知意
    \end{itemize}

    \vspace{0.3cm}

    \textbf{测试要求:}
    \begin{itemize}
        \item 核心函数有单元测试
        \item 测试覆盖正常和异常情况
        \item 提供测试运行说明
    \end{itemize}

    \vspace{0.3cm}

    \textbf{提交格式:}
    \begin{itemize}
        \item 源码压缩包(包含README)
        \item 测试样例和预期结果
        \item 运行演示视频(可选加分)
    \end{itemize}
\end{frame}

\begin{frame}{评分标准}
    \begin{table}
        \centering
        \small
        \begin{tabular}{lc}
            \toprule
            \textbf{评分项} & \textbf{分值} \\
            \midrule
            功能完整性 & 40分 \\
            代码质量 & 20分 \\
            测试覆盖 & 15分 \\
            文档完整性 & 15分 \\
            演示效果 & 10分 \\
            \midrule
            \textbf{总分} & \textbf{100分} \\
            \bottomrule
        \end{table}

        \vspace{0.3cm}

        \textbf{加分项:}
        \begin{itemize}
            \item 使用TDD开发(+5分)
            \item 性能优化有数据支撑(+5分)
            \item 额外的错误处理(+3分)
        \end{itemize}
    \end{table}
\end{frame}

%----------------------------------------------------------
\subsection{延伸学习资源}
%----------------------------------------------------------

\begin{frame}{推荐学习资源}
    \textbf{书籍推荐:}
    \begin{itemize}
        \item 《代码大全》- Steve McConnell(软件工程圣经)
        \item 《重构》- Martin Fowler(重构技术权威)
        \item 《代码整洁之道》- Robert C. Martin(Clean Code)
        \item 《Python编程:从入门到实践》- 基础入门
    \end{itemize}

    \vspace{0.3cm}

    \textbf{在线资源:}
    \begin{itemize}
        \item Python官方文档:https://docs.python.org/zh-cn/3/
        \item pytest文档:https://docs.pytest.org/
        \item Real Python:https://realpython.com/(高质量教程)
        \item 廖雪峰Python教程(中文入门)
    \end{itemize}
\end{frame}

\begin{frame}{下节预告}
    \begin{center}
        \Large \textbf{第11周:成果展示与总结}

        \vspace{0.5cm}

        \normalsize
        每组5分钟演示 + 2分钟答辩

        \vspace{0.3cm}

        \textbf{展示内容:}
        \begin{itemize}
            \item 系统功能演示
            \item 技术架构介绍
            \item 开发过程分享
            \item 问题与解决方案
        \end{itemize}

        \vspace{0.3cm}

        \Large \textbf{准备好你们的展示!}
    \end{center}
\end{frame}

\begin{frame}
    \begin{center}
        \Huge \textbf{加油!}

        \vspace{1cm}

        \Large 最后一周冲刺

        \vspace{0.5cm}

        \normalsize
        让系统真正跑起来!
    \end{center}
\end{frame}
