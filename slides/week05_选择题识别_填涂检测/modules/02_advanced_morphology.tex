%===========================================================
% 02_advanced_morphology.tex - 高级形态学操作
%===========================================================
\section{高级形态学操作}

\begin{frame}{形态学梯度(Gradient)}
    \textbf{定义:} 膨胀图像减去腐蚀图像

    \vspace{0.3cm}

    \begin{columns}
        \column{0.5\textwidth}
        \textbf{数学表达:}
        \[ \text{gradient}(A) = A \oplus B - A \ominus B \]

        \textbf{效果:}
        \begin{itemize}
            \item 提取物体边界
            \item 突出边缘信息
            \item 保留细节特征
        \end{itemize}

        \column{0.5\textwidth}
        \begin{block}{OpenCV实现}
        \aihint{} "请用OpenCV实现形态学梯度,解释它如何提取边界"
        \begin{lstlisting}
# TODO: 可使用AI助手完成
gradient = cv2.morphologyEx(
    binary_image,
    cv2.MORPH_GRADIENT,
    kernel
)
        \end{lstlisting}
        \end{block}
    \end{columns}

    \vspace{0.3cm}

    \textbf{应用:} 检测填涂区域边界,辅助判断填涂完整度
\end{frame}

\begin{frame}{顶帽变换(Top Hat)}
    \textbf{定义:} 原图像减去开运算结果

    \vspace{0.3cm}

    \begin{columns}
        \column{0.5\textwidth}
        \textbf{数学表达:}
        \[ \text{Tophat}(A) = A - (A \circ B) \]

        \textbf{效果:}
        \begin{itemize}
            \item 提取比结构元素小的亮区域
            \item 去除不均匀光照
            \item 增强细小目标
        \end{itemize}

        \column{0.5\textwidth}
        \begin{block}{OpenCV实现}
        \aihint{} "请用OpenCV实现顶帽变换,解释它如何去除光照不均"
        \begin{lstlisting}
# TODO: 可使用AI助手完成
tophat = cv2.morphologyEx(
    image,
    cv2.MORPH_TOPHAT,
    kernel
)
        \end{lstlisting}
        \end{block}
    \end{columns}

    \vspace{0.3cm}

    \textbf{应用:} 去除答题卡背景光照不均,突出填涂区域
\end{frame}

\begin{frame}{黑帽变换(Black Hat)}
    \textbf{定义:} 闭运算结果减去原图像

    \vspace{0.3cm}

    \begin{columns}
        \column{0.5\textwidth}
        \textbf{数学表达:}
        \[ \text{Blackhat}(A) = (A \bullet B) - A \]

        \textbf{效果:}
        \begin{itemize}
            \item 提取比结构元素小的暗区域
            \item 检测孔洞和缺口
            \item 增强暗部细节
        \end{itemize}

        \column{0.5\textwidth}
        \begin{block}{OpenCV实现}
        \aihint{} "请用OpenCV实现黑帽变换,解释它如何检测孔洞"
        \begin{lstlisting}
# TODO: 可使用AI助手完成
blackhat = cv2.morphologyEx(
    image,
    cv2.MORPH_BLACKHAT,
    kernel
)
        \end{lstlisting}
        \end{block}
    \end{columns}

    \vspace{0.3cm}

    \textbf{应用:} 检测填涂区域的孔洞,辅助完整性判断
\end{frame}

\begin{frame}[fragile]{Live Coding:高级形态学操作对比}
    \begin{block}{AI辅助提示}
        \aihint{} "请实现高级形态学操作对比代码,包括梯度、顶帽、黑帽变换的可视化"
    \end{block}

    \begin{block}{任务:对比5种形态学操作的效果}
    \begin{lstlisting}[basicstyle=\ttfamily\scriptsize]
import cv2
import numpy as np
import matplotlib.pyplot as plt

# 读取填涂区域图像
img = cv2.imread("option_roi.png", cv2.IMREAD_GRAYSCALE)

# 创建结构元素
kernel = cv2.getStructuringElement(cv2.MORPH_RECT, (5, 5))

# TODO: 实现5种形态学操作
gradient = cv2.morphologyEx(img, cv2.MORPH_GRADIENT, kernel)
tophat = cv2.morphologyEx(img, cv2.MORPH_TOPHAT, kernel)
blackhat = cv2.morphologyEx(img, cv2.MORPH_BLACKHAT, kernel)

# 对比显示
titles = ["原图", "梯度", "顶帽", "黑帽"]
images = [img, gradient, tophat, blackhat]

# TODO: 用Matplotlib 2x2网格显示,观察差异
fig, axes = plt.subplots(2, 2, figsize=(10, 10))
for ax, img, title in zip(axes.ravel(), images, titles):
    ax.imshow(img, cmap='gray')
    ax.set_title(title)
    ax.axis('off')
plt.tight_layout()
plt.savefig("morphology_comparison.png")
    \end{lstlisting}
    \end{block}
\end{frame}

\begin{frame}{形态学操作选择决策树}
    \begin{center}
    \begin{tikzpicture}[
        node distance=0.8cm and 1cm,
        decision/.style={diamond, draw, fill=yellow!20, aspect=2},
        action/.style={rectangle, draw, fill=green!20},
        question/.style={rectangle, draw, fill=blue!10}
    ]
        \node[question] (problem) {问题类型?};
        \node[decision, below=of problem] (noise) {有噪点?};
        \node[decision, below left=of noise] (hole) {有孔洞?};
        \node[decision, below right=of noise] (light) {光照不均?};
        \node[action, below=of hole] (open) {开运算};
        \node[action, below=of hole] (close) {闭运算};
        \node[action, right=of close] (tophat) {顶帽变换};

        \draw[->] (problem) -- (noise);
        \draw[->] (noise) -- node[above left] {是} (hole);
        \draw[->] (noise) -- node[above right] {否} (light);
        \draw[->] (hole) -- node[left] {是} (close);
        \draw[->] (hole) -- node[left] {否} (open);
        \draw[->] (light) -- node[right] {是} (tophat);
    \end{tikzpicture}
    \end{center}

    \textbf{填涂检测推荐流程:}
    \begin{enumerate}
        \item 先用开运算去除噪点
        \item 再用闭运算填充孔洞
        \item 可选:顶帽变换去除光照不均
    \end{enumerate}
\end{frame}
