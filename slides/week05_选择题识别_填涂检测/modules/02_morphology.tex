%===========================================================
% 02_morphology.tex - 图像形态学处理
%===========================================================
\section{图像形态学处理}

\begin{frame}{为什么需要形态学处理?}
    \begin{columns}
        \column{0.5\textwidth}
        \textbf{填涂图像的常见问题:}
        \begin{itemize}
            \item 铅笔痕迹不连续(有孔洞)
            \item 填涂边缘不规则
            \item 纸张上有噪点/污渍
            \item 扫描时有颗粒噪声
        \end{itemize}

        \column{0.5\textwidth}
        \textbf{形态学处理的作用:}
        \begin{itemize}
            \item \textcolor{blue}{膨胀}:填充孔洞,连接断点
            \item \textcolor{red}{腐蚀}:去除噪点,分离目标
            \item \textcolor{green}{开运算}:先腐后胀,去噪保形
            \item \textcolor{orange}{闭运算}:先胀后腐,填充闭孔
        \end{itemize}
    \end{columns}
\end{frame}

\begin{frame}{结构元素(Structuring Element)}
    \textbf{结构元素:} 形态学操作的"探针",决定操作的范围和形状

    \vspace{0.3cm}

    \begin{columns}
        \column{0.5\textwidth}
        \textbf{常见形状:}
        \begin{itemize}
            \item 矩形(最常用)
            \item 十字形
            \item 椭圆形
        \end{itemize}

        \column{0.5\textwidth}
        \textbf{大小选择:}
        \begin{itemize}
            \item 太小:效果不明显
            \item 太大:过度处理
            \item 推荐:(3,3) 或 (5,5)
        \end{itemize}
    \end{columns}

    \vspace{0.3cm}

    \begin{block}{OpenCV创建结构元素}
    \texttt{kernel = cv2.getStructuringElement(shape, ksize)}
    \end{block}
\end{frame}

\begin{frame}{腐蚀(Erosion)}
    \textbf{原理:} 用结构元素"扫描"图像,如果结构元素完全包含在前景中,则保留中心像素

    \vspace{0.3cm}

    \begin{columns}
        \column{0.5\textwidth}
        \begin{itemize}
            \item \textbf{效果:} 前景区域\textcolor{red}{缩小},背景区域\textcolor{red}{扩大}
            \item \textbf{用途:}
                \begin{itemize}
                    \item 去除细小噪点
                    \item 分离连接的目标
                \end{itemize}
        \end{itemize}

        \column{0.5\textwidth}
        \begin{block}{OpenCV实现}
        \aihint{} "请用OpenCV实现图像腐蚀操作,kernel大小为3x3"
        \begin{lstlisting}
# TODO: 可使用AI助手完成以下代码
kernel = np.ones((3,3), np.uint8)
eroded = cv2.erode(
    binary_image,
    kernel,
    iterations=1
)
        \end{lstlisting}
        \end{block}
    \end{columns}
\end{frame}

\begin{frame}{膨胀(Dilation)}
    \textbf{原理:} 用结构元素"扫描"图像,如果结构元素接触到前景,则将中心像素设为前景

    \vspace{0.3cm}

    \begin{columns}
        \column{0.5\textwidth}
        \begin{itemize}
            \item \textbf{效果:} 前景区域\textcolor{blue}{扩大},背景区域\textcolor{blue}{缩小}
            \item \textbf{用途:}
                \begin{itemize}
                    \item 填充孔洞
                    \item 连接断开的区域
                \end{itemize}
        \end{itemize}

        \column{0.5\textwidth}
        \begin{block}{OpenCV实现}
        \aihint{} "请用OpenCV实现图像膨胀操作,解释它与腐蚀的区别"
        \begin{lstlisting}
# TODO: 可使用AI助手完成以下代码
kernel = np.ones((3,3), np.uint8)
dilated = cv2.dilate(
    binary_image,
    kernel,
    iterations=1
)
        \end{lstlisting}
        \end{block}
    \end{columns}
\end{frame}

\begin{frame}{开运算(Opening)}
    \textbf{定义:} 先腐蚀后膨胀

    \vspace{0.3cm}

    \begin{columns}
        \column{0.5\textwidth}
        \textbf{效果:}
        \begin{itemize}
            \item 去除细小噪点
            \item 平滑边界
            \item \textcolor{green}{基本保持目标大小}
        \end{itemize}

        \column{0.5\textwidth}
        \begin{block}{OpenCV实现}
        \aihint{} "请用OpenCV实现开运算(先腐后胀),解释它的应用场景"
        \begin{lstlisting}
# TODO: 可使用AI助手完成以下代码
opening = cv2.morphologyEx(
    binary_image,
    cv2.MORPH_OPEN,
    kernel
)
        \end{lstlisting}
        \end{block}
    \end{columns}

    \vspace{0.3cm}

    \textbf{应用场景:} 填涂区域周围有散落的铅笔粉尘时
\end{frame}

\begin{frame}{闭运算(Closing)}
    \textbf{定义:} 先膨胀后腐蚀

    \vspace{0.3cm}

    \begin{columns}
        \column{0.5\textwidth}
        \textbf{效果:}
        \begin{itemize}
            \item 填充内部孔洞
            \item 连接邻近区域
            \item \textcolor{orange}{基本保持目标大小}
        \end{itemize}

        \column{0.5\textwidth}
        \begin{block}{OpenCV实现}
        \aihint{} "请用OpenCV实现闭运算(先胀后腐),解释为什么它适合填充孔洞"
        \begin{lstlisting}
# TODO: 可使用AI助手完成以下代码
closing = cv2.morphologyEx(
    binary_image,
    cv2.MORPH_CLOSE,
    kernel
)
        \end{lstlisting}
        \end{block}
    \end{columns}

    \vspace{0.3cm}

    \textbf{应用场景:} \highlight{填涂不完整、有孔洞时(最常用!)}
\end{frame}

\begin{frame}[fragile]{Live Coding:形态学处理实战}
    \begin{block}{AI辅助提示}
        \aihint{} "请用Python和OpenCV实现形态学处理对比,包括腐蚀、膨胀、开运算、闭运算"
    \end{block}

    \begin{block}{任务:对比不同形态学操作的效果}
    \begin{lstlisting}[basicstyle=\ttfamily\scriptsize]
import cv2
import numpy as np

# 读取填涂区域图像
option_img = cv2.imread("option_roi.png", cv2.IMREAD_GRAYSCALE)

# 二值化
_, binary = cv2.threshold(option_img, 127, 255, cv2.THRESH_BINARY_INV)

# 创建结构元素
kernel = cv2.getStructuringElement(cv2.MORPH_RECT, (3, 3))

# TODO: 尝试不同的形态学操作
eroded = cv2.erode(binary, kernel, iterations=1)      # 腐蚀
dilated = cv2.dilate(binary, kernel, iterations=1)    # 膨胀
opening = cv2.morphologyEx(binary, cv2.MORPH_OPEN, kernel)  # 开运算
closing = cv2.morphologyEx(binary, cv2.MORPH_CLOSE, kernel)  # 闭运算

# 对比显示
results = [binary, eroded, dilated, opening, closing]
titles = ["原图", "腐蚀", "膨胀", "开运算", "闭运算"]
# TODO: 用Matplotlib并排显示,观察差异
    \end{lstlisting}
    \end{block}
\end{frame}

\begin{frame}{形态学处理对填涂检测的影响}
    \begin{columns}
        \column{0.5\textwidth}
        \textbf{问题:填涂有孔洞}
        \begin{itemize}
            \item 密度计算偏小
            \item 可能误判为未填涂
        \end{itemize}
        \vspace{0.2cm}
        \textbf{解决方案:闭运算}
        \begin{itemize}
            \item 填充内部孔洞
            \item 提高密度值
        \end{itemize}

        \column{0.5\textwidth}
        \textbf{问题:周围有噪点}
        \begin{itemize}
            \item 噪点被计入密度
            \item 可能误判为已填涂
        \end{itemize}
        \vspace{0.2cm}
        \textbf{解决方案:开运算}
        \begin{itemize}
            \item 去除细小噪点
            \item 降低干扰影响
        \end{itemize}
    \end{columns}

    \vspace{0.5cm}

    \begin{alertblock}{最佳实践}
    对于填涂检测,推荐使用\textbf{闭运算}填充孔洞,提高识别准确率
    \end{alertblock}
\end{frame}
