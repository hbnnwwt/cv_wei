%===========================================================
% 07_cases.tex - 真实案例分析
%===========================================================
\section{真实案例分析}

\begin{frame}{案例1:标准答题卡}
    \textbf{场景描述:}
    \begin{itemize}
        \item 标准答题卡,填涂完整
        \item 使用2B铅笔填涂
        \item 扫描质量良好
    \end{itemize}

    \vspace{0.3cm}

    \textbf{预期结果:}
    \begin{itemize}
        \item 识别准确率:\textcolor{green}{> 99\%}
        \item 密度值范围:0.5 - 0.8
        \item 形态学处理效果:不明显(但无害)
    \end{itemize}

    \vspace{0.3cm}

    \textbf{代码演示:}
    \begin{itemize}
        \item 直接使用基础算法即可
        \item 阈值设为0.3效果最佳
    \end{itemize}
\end{frame}

\begin{frame}{案例2:填涂质量较差}
    \textbf{场景描述:}
    \begin{itemize}
        \item 铅笔填涂很轻
        \item 填涂范围不完整
        \item 有擦除痕迹
    \end{itemize}

    \vspace{0.3cm}

    \textbf{挑战与解决方案:}
    \begin{table}
        \centering
        \begin{tabular}{l|l|l}
        \toprule
        挑战 & 原因 & 解决方案 \\
        \midrule
        密度值偏低 & 填涂轻 & 降低阈值或使用闭运算 \\
        有孔洞 & 填涂不连续 & 形态学闭运算填充 \\
        误判未填涂 & 密度低于阈值 & 自适应阈值策略 \\
        \bottomrule
        \end{tabular}
    \end{table}

    \textbf{代码优化:}
    \begin{itemize}
        \item 使用闭运算:\texttt{cv2.morphologyEx(..., cv2.MORPH\_CLOSE, kernel)}
        \item 自适应阈值:\texttt{threshold = np.mean(densities)}
    \end{itemize}
\end{frame}

\begin{frame}{案例3:有污损的答题卡}
    \textbf{场景描述:}
    \begin{itemize}
        \item 纸张有折痕
        \item 有墨水污渍
        \item 扫描有噪点
    \end{itemize}

    \vspace{0.3cm}

    \textbf{挑战与解决方案:}
    \begin{table}
        \centering
        \begin{tabular}{l|l|l}
        \toprule
        挑战 & 原因 & 解决方案 \\
        \midrule
        误判为填涂 & 污渍被计入 & 开运算去除小噪点 \\
        密度异常高 & 大面积污渍 & 面积筛选 + 形状检查 \\
        连通域过多 & 噪点干扰 & 连通域面积过滤 \\
        \bottomrule
        \end{tabular}
    \end{table}

    \textbf{代码优化:}
    \begin{itemize}
        \item 开运算去噪:\texttt{cv2.morphologyEx(..., cv2.MORPH\_OPEN, kernel)}
        \item 组合处理:先开运算去噪,再闭运算填充
    \end{itemize}
\end{frame}

\begin{frame}{案例4:多选题识别}
    \textbf{场景描述:}
    \begin{itemize}
        \item 多选题,可能有多个选项填涂
        \item 需要识别所有填涂的选项
    \end{itemize}

    \vspace{0.3cm}

    \textbf{算法调整:}
    \begin{block}{AI辅助提示}
        \aihint{} "请实现多选题识别函数,需要识别所有填涂的选项并返回答案列表"
    \end{block}

    \begin{block}{多选题识别}
    \begin{lstlisting}
# TODO: 可使用AI助手完成以下代码
def recognize_multiple_choice(densities, threshold=0.3):
    """识别多选题"""
    answers = []
    for i, density in enumerate(densities):
        if density >= threshold:
            answers.append(chr(ord('A') + i))
    return answers if answers else None

# 示例:输出 "AC" 表示选择了A和C
    \end{lstlisting}
    \end{block}

    \textbf{注意事项:}
    \begin{itemize}
        \item 多选题的阈值可能需要调高(避免误判)
        \item 需要考虑"全选"的特殊情况
    \end{itemize}
\end{frame}

\begin{frame}{案例5:倾斜扫描的答题卡}
    \textbf{场景描述:}
    \begin{itemize}
        \item 扫描/拍摄时答题卡倾斜
        \item 选项位置发生偏移
    \end{itemize}

    \vspace{0.3cm}

    \textbf{解决方案:几何校正}
    \begin{enumerate}
        \item \textbf{检测定位标记}(Timing Marks)
        \item \textbf{计算变换矩阵}(透视变换)
        \item \textbf{校正图像}
        \item \textbf{按固定位置提取选项}
    \end{enumerate}

    \textbf{代码提示:}
    \begin{itemize}
        \item 使用第4周学到的透视变换技术
        \item \aihint{} "如何用OpenCV实现文档倾斜校正?"
    \end{itemize}
\end{frame}

\begin{frame}{案例对比总结}
    \begin{table}
        \centering
        \small
        \begin{tabular}{l|c|c|c|c}
        \toprule
        案例 & 难度 & 准确率 & 关键技术 & 调优建议 \\
        \midrule
        标准答题卡 & ⭐ & >99\% & 基础算法 & 无需调优 \\
        填涂较轻 & ⭐⭐ & 85-95\% & 闭运算+自适应阈值 & 降低阈值 \\
        有污损 & ⭐⭐⭐ & 80-90\% & 开运算+连通域筛选 & 形态学组合 \\
        多选题 & ⭐⭐ & 90-95\% & 多选项检测 & 调高阈值 \\
        倾斜扫描 & ⭐⭐⭐⭐ & 70-85\% & 几何校正 & 预处理校正 \\
        \bottomrule
        \end{tabular}
    \end{table}

    \begin{alertblock}{关键经验}
    \begin{enumerate}
        \item \textbf{形态学预处理}是提升准确率的关键
        \item \textbf{自适应阈值}比固定阈值更鲁棒
        \item \textbf{几何校正}能解决大部分定位问题
    \end{enumerate}
    \end{alertblock}
\end{frame}

\begin{frame}{课堂讨论}
    \textbf{问题1:} 如果答题卡既有填涂较轻的区域,又有污损区域,如何同时处理?

    \vspace{0.2cm}

    \textbf{问题2:} 如何判断一个答题卡是否"未填涂任何选项"?

    \vspace{0.2cm}

    \textbf{问题3:} 实际应用中,如何处理"擦除不干净"的情况?

    \vspace{0.5cm}

    \begin{center}
        \textcolor{blue}{\textbf{[分组讨论 5分钟]}}
    \end{center}
\end{frame}
