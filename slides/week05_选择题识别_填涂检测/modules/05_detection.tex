%===========================================================
% 05_detection.tex - 填涂检测算法
%===========================================================
\section{填涂检测算法}

\begin{frame}{像素密度统计}
    \textbf{核心思想:} 填涂区域的深色像素占比高于空白区域

    \vspace{0.3cm}

    \begin{columns}
        \column{0.5\textwidth}
        \textbf{密度计算步骤:}
        \begin{enumerate}
            \item 提取选项ROI
            \item 二值化(反色)
            \item 统计白色像素数
            \item 除以总像素数
        \end{enumerate}

        \column{0.5\textwidth}
        \begin{block}{密度公式}
        \[ \text{density} = \frac{\text{填涂像素数}}{\text{总像素数}} \]
        \end{block}

        \textbf{典型值:}
        \begin{itemize}
            \item 未填涂:0.05 - 0.15
            \item 轻度填涂:0.20 - 0.40
            \item 完整填涂:0.50 - 0.80
        \end{itemize}
    \end{columns}
\end{frame}

\begin{frame}[fragile]{密度计算实现}
    \begin{block}{AI辅助提示}
        \aihint{} "请用Python和OpenCV实现填涂密度计算函数,包括二值化和像素计数"
    \end{block}

    \begin{block}{calculate\_density函数}
    \begin{lstlisting}[basicstyle=\ttfamily\scriptsize]
# TODO: 可使用AI助手完成以下代码
def calculate_density(option_roi, threshold=127):
    """
    计算填涂密度

    Args:
        option_roi: 选项区域图像(灰度图)
        threshold: 二值化阈值

    Returns:
        density: 密度值 [0, 1]
    """
    # 二值化(反色:填涂变白)
    _, binary = cv2.threshold(
        option_roi, threshold, 255,
        cv2.THRESH_BINARY_INV
    )

    # 统计白色像素
    non_zero = cv2.countNonZero(binary)

    # 计算密度
    total = option_roi.shape[0] * option_roi.shape[1]
    density = non_zero / total

    return density
    \end{lstlisting}
    \end{block}
\end{frame}

\begin{frame}{阈值确定方法}
    \begin{columns}
        \column{0.5\textwidth}
        \textbf{方法1:固定阈值}
        \begin{itemize}
            \item 经验值:0.2 - 0.4
            \item 简单但不适应场景
        \end{itemize}

        \begin{block}{示例}
        \texttt{THRESHOLD = 0.3}\\
        \texttt{if density > 0.3:}\\
        \texttt{~~填涂 = True}
        \end{block}

        \column{0.5\textwidth}
        \textbf{方法2:自适应阈值(推荐)}
        \begin{itemize}
            \item 基于所有选项的密度分布
            \item 自动确定最佳阈值
        \end{itemize}

        \begin{block}{示例}
        \texttt{threshold = np.mean(densities)}\\
        \texttt{threshold = np.median(densities)}
        \end{block}
    \end{columns}
\end{frame}

\begin{frame}{自适应阈值策略}
    \textbf{策略1:均值法}
    \[ \text{threshold} = \frac{1}{n}\sum_{i=1}^{n}\text{density}_i \]

    \textbf{策略2:Otsu思想}
    \begin{itemize}
        \item 寻找最大化类间方差的阈值
        \item 假设:填涂和未填涂呈双峰分布
    \end{itemize}

    \textbf{策略3:相对最大值}
    \begin{itemize}
        \item 单选题:选择密度最高的
        \item threshold = max(density) × 0.8
    \end{itemize}
\end{frame}

\begin{frame}[fragile]{单选题识别}
    \begin{block}{AI辅助提示}
        \aihint{} "请实现单选题识别函数,使用自适应阈值确定答案"
    \end{block}

    \begin{block}{recognize\_single\_choice函数}
    \begin{lstlisting}[basicstyle=\ttfamily\scriptsize]
# TODO: 可使用AI助手完成以下代码
def recognize_single_choice(question_img, option_positions):
    """识别单道选择题"""
    densities = []

    # 计算每个选项的密度
    for (x, y, w, h) in option_positions:
        roi = question_img[y:y+h, x:x+w]
        density = calculate_density(roi)
        densities.append(density)

    # 自适应确定阈值
    threshold = np.mean(densities) + 0.1

    # 找到密度最高的选项
    max_density = max(densities)

    if max_density >= threshold:
        answer_idx = densities.index(max_density)
        return chr(ord('A') + answer_idx)

    return None  # 未填涂
    \end{lstlisting}
    \end{block}
\end{frame}

\begin{frame}{形态学预处理对密度的影响}
    \begin{table}
        \centering
        \begin{tabular}{l|c|c|c}
        \toprule
        处理方式 & 填涂完整 & 填涂有孔 & 轻度填涂 \\
        \midrule
        不处理 & 0.65 & 0.42 & 0.25 \\
        闭运算 & \textcolor{green}{0.72} & \textcolor{green}{0.58} & \textcolor{green}{0.31} \\
        开运算 & 0.61 & 0.38 & 0.23 \\
        \bottomrule
        \end{tabular}
        \caption{形态学处理对密度的影响}
    \end{table}

    \begin{alertblock}{结论}
    闭运算能显著提高填涂区域的密度值,改善识别效果
    \end{alertblock}
\end{frame}

\begin{frame}[fragile]{Live Coding:填涂检测实战}
    \begin{block}{AI辅助提示}
        \aihint{} "请实现完整的填涂检测流程,包括形态学预处理、密度计算和阈值判断"
    \end{block}

    \begin{block}{任务:实现完整的填涂检测流程}
    \begin{lstlisting}[basicstyle=\ttfamily\scriptsize]
# TODO: 可使用AI助手完成以下代码
def detect_bubble(roi):
    """检测单个选项是否填涂"""
    # TODO: 形态学预处理(闭运算填充孔洞)
    kernel = cv2.getStructuringElement(cv2.MORPH_RECT, (3, 3))
    processed = cv2.morphologyEx(roi, cv2.MORPH_CLOSE, kernel)

    # TODO: 计算密度
    density = calculate_density(processed)

    # TODO: 判断是否填涂
    threshold = 0.3  # 或使用自适应阈值
    filled = density > threshold

    return filled, density

# 测试
for i, (x, y, w, h) in enumerate(option_positions):
    roi = image[y:y+h, x:x+w]
    filled, density = detect_bubble(roi)
    print(f"选项 {chr(65+i)}: 密度={density:.2f}, 填涂={filled}")
    \end{lstlisting}
    \end{block}
\end{frame}
