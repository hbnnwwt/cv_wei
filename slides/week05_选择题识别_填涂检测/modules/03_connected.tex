%===========================================================
% 03_connected.tex - 连通域分析
%===========================================================
\section{连通域分析}

\begin{frame}{什么是连通域?}
    \textbf{连通域(Connected Component):} 图像中相互连接的像素组成的区域

    \vspace{0.3cm}

    \begin{columns}
        \column{0.5\textwidth}
        \textbf{连通性定义:}
        \begin{itemize}
            \item \textbf{4-连通}:上下左右4个方向
            \item \textbf{8-连通}:包括对角线8个方向
        \end{itemize}

        \column{0.5\textwidth}
        \begin{center}
        \begin{tikzpicture}[scale=0.5]
            \draw[step=1cm,gray,very thin] (0,0) grid (4,4);
            \fill[blue] (1,1) rectangle (2,2);
            \fill[blue] (2,1) rectangle (3,2);
            \fill[blue] (1,2) rectangle (2,3);
            \node at (2,-0.5) {8-连通:1个区域};
        \end{tikzpicture}
        \end{center}
    \end{columns}
\end{frame}

\begin{frame}{连通域标记算法}
    \textbf{目标:} 给每个连通域分配唯一的标签(ID)

    \vspace{0.3cm}

    \textbf{两遍扫描算法:}
    \begin{enumerate}
        \item \textbf{第一遍}:从左到右、从上到下扫描,分配临时标签
        \item \textbf{第二遍}:合并等价标签,最终确定每个区域的ID
    \end{enumerate}

    \vspace{0.3cm}

    \begin{block}{OpenCV实现}
    \aihint{} "请用OpenCV的connectedComponentsWithStats函数实现连通域分析"
    \texttt{num\_labels, labels, stats, centroids = cv2.connectedComponentsWithStats(image)}
    \end{block}
\end{frame}

\begin{frame}{连通域特征提取}
    \textbf{OpenCV返回的统计信息(stats):}

    \begin{table}
        \centering
        \begin{tabular}{cll}
        \toprule
        索引 & 字段名 & 含义 \\
        \midrule
        0 & cv2.CC\_STAT\_LEFT & 外接矩形左上角x坐标 \\
        1 & cv2.CC\_STAT\_TOP & 外接矩形左上角y坐标 \\
        2 & cv2.CC\_STAT\_WIDTH & 外接矩形宽度 \\
        3 & cv2.CC\_STAT\_HEIGHT & 外接矩形高度 \\
        4 & cv2.CC\_STAT\_AREA & 连通域面积(像素数) \\
        \bottomrule
        \end{tabular}
    \end{table}

    \textbf{质心(centroids):} 每个连通域的中心点坐标
\end{frame}

\begin{frame}{连通域筛选策略}
    \textbf{如何找到填涂气泡?} 基于连通域特征筛选

    \begin{columns}
        \column{0.5\textwidth}
        \begin{block}{面积筛选}
        \begin{itemize}
            \item 去除过小的噪点
            \item 去除过大的背景区域
        \end{itemize}
        \texttt{min\_area < area < max\_area}
        \end{block}

        \column{0.5\textwidth}
        \begin{block}{形状筛选}
        \begin{itemize}
            \item 宽高比(气泡接近圆形)
            \item 矩形度(面积/外接矩形面积)
        \end{itemize}
        \texttt{0.7 < width/height < 1.3}
        \end{block}
    \end{columns}
\end{frame}

\begin{frame}[fragile]{Live Coding:连通域分析实战}
    \begin{block}{AI辅助提示}
        \aihint{} "请用OpenCV实现连通域分析,找出答题卡上所有气泡状的连通区域"
    \end{block}

    \begin{block}{任务:找出答题卡上的所有气泡}
    \begin{lstlisting}[basicstyle=\ttfamily\scriptsize]
# TODO: 可使用AI助手完成以下代码
import cv2
import numpy as np

# 读取答题卡图像
card = cv2.imread("answer_card.png", cv2.IMREAD_GRAYSCALE)

# 二值化
_, binary = cv2.threshold(card, 127, 255, cv2.THRESH_BINARY_INV)

# 连通域分析
num_labels, labels, stats, centroids = cv2.connectedComponentsWithStats(binary)

# TODO: 筛选出气泡状的连通域(可使用AI助手完成)
bubbles = []
for i in range(1, num_labels):  # 跳过背景(label=0)
    area = stats[i, cv2.CC_STAT_AREA]
    width = stats[i, cv2.CC_STAT_WIDTH]
    height = stats[i, cv2.CC_STAT_HEIGHT]

    # 筛选条件
    if 500 < area < 5000 and 0.7 < width/height < 1.3:
        bubbles.append((i, stats[i], centroids[i]))

print(f"找到 {len(bubbles)} 个气泡")
# TODO: 可视化:用不同颜色标记每个气泡
    \end{lstlisting}
    \end{block}
\end{frame}

\begin{frame}[fragile]{连通域可视化}
    \textbf{AI辅助提示:}
    \aihint{} "请用OpenCV和NumPy实现连通域彩色标注,为每个区域分配不同颜色"

    \textbf{彩色标注连通域:}
    \begin{lstlisting}
# 创建彩色图像
height, width = binary.shape
colored = np.zeros((height, width, 3), dtype=np.uint8)

# 为每个连通域分配随机颜色
np.random.seed(42)
colors = np.random.randint(0, 255, (num_labels, 3), dtype=np.uint8)

for i in range(1, num_labels):
    colored[labels == i] = colors[i]
    # 绘制外接矩形
    x, y, w, h = stats[i, :4]
    cv2.rectangle(colored, (x, y), (x+w, y+h), colors[i].tolist(), 2)

cv2.imwrite("labeled_components.png", colored)
    \end{lstlisting}
\end{frame}

\begin{frame}{连通域与填涂检测}
    \textbf{两种应用思路:}

    \begin{enumerate}
        \item \textbf{自动定位气泡}
            \begin{itemize}
                \item 用连通域分析找到所有气泡位置
                \item 适用于非标准答题卡
                \item 需要结合形状筛选
            \end{itemize}

        \item \textbf{验证填涂质量}
            \begin{itemize}
                \item 统计已知位置区域的连通域数量
                \item 单个连通域 = 填涂完整
                \item 多个连通域 = 填涂不完整或有噪点
            \end{itemize}
    \end{enumerate}
\end{frame}
