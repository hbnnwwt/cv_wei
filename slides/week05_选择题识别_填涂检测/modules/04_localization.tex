%===========================================================
% 04_localization.tex - 填涂区域定位
%===========================================================
\section{填涂区域定位}

\begin{frame}{定位方法对比}
    \begin{table}
        \centering
        \begin{tabular}{p{0.45\textwidth}|p{0.45\textwidth}}
        \toprule
        \textbf{方法1:固定位置法} & \textbf{方法2:轮廓检测法(推荐)} \\
        \midrule
        适合标准答题卡 & 自动适应不同版面 \\
        预先定义选项坐标 & 通过连通域分析找气泡 \\
        实现简单 & 实现复杂但灵活 \\
        怕扫描偏移 & 有一定鲁棒性 \\
        \bottomrule
        \end{tabular}
    \end{table}
\end{frame}

\begin{frame}{方法1:固定位置法}
    \textbf{思路:} 假设答题卡位置固定,直接用坐标裁剪

    \begin{block}{AI辅助提示}
        \aihint{} "请实现固定位置法定位选项,解释它的优缺点"
    \end{block}

    \begin{block}{实现}
    \begin{lstlisting}
# TODO: 可使用AI助手完成以下代码
# 预定义的选项位置 (x, y, width, height)
option_positions = [
    [(100, 200, 30, 30), (150, 200, 30, 30), ...],  # 第1题
    [(100, 250, 30, 30), (150, 250, 30, 30), ...],  # 第2题
    ...
]

# 提取第1题第A选项
x, y, w, h = option_positions[0][0]
roi = image[y:y+h, x:x+w]
    \end{lstlisting}
    \end{block}

    \begin{alertblock}{缺点}
    扫描/拍摄偏移会导致定位失败
    \end{alertblock}
\end{frame}

\begin{frame}{方法2:轮廓检测法}
    \textbf{思路:} 用轮廓检测找到所有气泡,按位置分组

    \begin{block}{实现流程}
    \begin{enumerate}
        \item 二值化图像
        \item 查找轮廓:\texttt{cv2.findContours}
        \item 筛选轮廓:面积、形状
        \item 按Y坐标分组(同一题)
        \item 按X坐标排序(A、B、C、D顺序)
    \end{enumerate}
    \end{block}
\end{frame}

\begin{frame}[fragile]{轮廓检测基础}
    \begin{block}{AI辅助提示}
        \aihint{} "请用OpenCV的findContours函数实现轮廓检测,解释RETR_EXTERNAL和CHAIN_APPROX_SIMPLE参数的含义"
    \end{block}

    \begin{block}{OpenCV轮廓查找}
    \begin{lstlisting}
# TODO: 可使用AI助手完成以下代码
# 二值化
_, binary = cv2.threshold(image, 127, 255, cv2.THRESH_BINARY_INV)

# 查找轮廓
contours, hierarchy = cv2.findContours(
    binary,
    cv2.RETR_EXTERNAL,  # 只检测外轮廓
    cv2.CHAIN_APPROX_SIMPLE  # 压缩轮廓
)

print(f"找到 {len(contours)} 个轮廓")
    \end{lstlisting}
    \end{block}
\end{frame}

\begin{frame}[fragile]{轮廓特征提取}
    \begin{block}{AI辅助提示}
        \aihint{} "请用OpenCV提取轮廓特征,包括外接矩形、面积、周长,并实现筛选条件"
    \end{block}

    \begin{block}{常用轮廓特征}
    \begin{lstlisting}
# TODO: 可使用AI助手完成以下代码
for contour in contours:
    # 外接矩形
    x, y, w, h = cv2.boundingRect(contour)

    # 面积
    area = cv2.contourArea(contour)

    # 周长
    perimeter = cv2.arcLength(contour, closed=True)

    # TODO: 筛选气泡状的轮廓(可使用AI助手完成)
    if 500 < area < 5000 and 0.7 < w/h < 1.3:
        bubbles.append((x, y, w, h))
    \end{lstlisting}
    \end{block}
\end{frame}

\begin{frame}[fragile]{Live Coding:选项区域定位}
    \begin{block}{AI辅助提示}
        \aihint{} "请实现locate_options函数,用轮廓检测定位答题卡上所有选项区域,并按位置分组"
    \end{block}

    \begin{block}{任务:定位答题卡上的所有选项}
    \begin{lstlisting}[basicstyle=\ttfamily\scriptsize]
def locate_options(image_path):
    """定位所有选项区域"""
    # 读取图像
    img = cv2.imread(image_path, cv2.IMREAD_GRAYSCALE)

    # TODO: 二值化
    _, binary = cv2.threshold(img, 127, 255, cv2.THRESH_BINARY_INV)

    # TODO: 查找轮廓
    contours, _ = cv2.findContours(binary, cv2.RETR_EXTERNAL, cv2.CHAIN_APPROX_SIMPLE)

    # TODO: 筛选气泡
    options = []
    for cnt in contours:
        x, y, w, h = cv2.boundingRect(cnt)
        area = cv2.contourArea(cnt)

        # 筛选条件:面积、形状
        if 500 < area < 5000 and 0.7 < w/h < 1.3:
            options.append((x, y, w, h))

    # TODO: 按位置分组(同一题的选项在一组)
    options.sort(key=lambda b: b[1])  # 按Y排序
    # TODO: 返回分组后的选项位置
    return options
    \end{lstlisting}
    \end{block}
\end{frame}
