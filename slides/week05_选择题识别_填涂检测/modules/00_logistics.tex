%===========================================================
% 00_logistics.tex - 课程后勤
%===========================================================
\section{课程准备}

\begin{frame}{三个理解层级:本周学习路径}
    \begin{columns}
        \column{0.33\textwidth}
        \textbf{基础概念:}
        \begin{itemize}
            \item 什么是OMR?
            \item 像素密度的概念
            \item 形态学基本运算
            \item 连通域的原理
        \end{itemize}

        \column{0.33\textwidth}
        \textbf{可视化演示:}
        \begin{itemize}
            \item 对比填涂与未填涂的密度差异
            \item 观察形态学处理效果
            \item 连通域标记可视化
            \item 调整阈值观察识别效果
        \end{itemize}

        \column{0.33\textwidth}
        \textbf{扩展应用:}
        \begin{itemize}
            \item 设计自适应OMR识别器
            \item 处理不同填涂程度答题卡
            \item 优化识别准确率
            \item 实现完整阅卷系统
        \end{itemize}
    \end{columns}

    \vspace{0.5cm}

    \begin{center}
        \textcolor{blue}{\textit{每个知识点都按照"理解→观察→创造"的层次递进}}
    \end{center}
\end{frame}

\begin{frame}{预备知识(课前5分钟视频)}
    \begin{block}{本周需要的前置知识}
        \begin{itemize}
            \item \textbf{二值图像的基本操作}
                \begin{itemize}
                    \item 什么是二值图像?
                    \item 阈值处理:全局阈值 vs 自适应阈值
                \end{itemize}
            \item \textbf{形态学运算原理}
                \begin{itemize}
                    \item 腐蚀(Erosion)与膨胀(Dilation)
                    \item 开运算与闭运算
                \end{itemize}
        \end{itemize}
    \end{block}

    \vspace{0.3cm}

    \begin{center}
        \textcolor{blue}{\textbf{[课前视频]}} 扫码观看5分钟预备知识视频
    \end{center}
\end{frame}

\begin{frame}{三种学习路径}
    \begin{columns}
        \column{0.33\textwidth}
        \begin{block}{观察者}
            \begin{itemize}
                \item 理解OMR原理
                \item 看教师演示
                \item 完成基础思考题
            \end{itemize}
            \textcolor{gray}{适合:无编程基础}
        \end{block}

        \column{0.33\textwidth}
        \begin{block}{使用者}
            \begin{itemize}
                \item 运行示例代码
                \item 调整参数观察效果
                \item 完成核心任务
            \end{itemize}
            \textcolor{blue}{推荐:大多数学生}
        \end{block}

        \column{0.33\textwidth}
        \begin{block}{创造者}
            \begin{itemize}
                \item 修改算法逻辑
                \item 优化阈值策略
                \item 实现挑战任务
            \end{itemize}
            \textcolor{green}{挑战:有编程基础}
        \end{block}
    \end{columns}
\end{frame}

\begin{frame}{分组与角色}
    \textbf{分组原则(4人/组):}
    \begin{itemize}
        \item 不同专业背景混合
        \item 至少1人有编程基础
    \end{itemize}

    \vspace{0.3cm}

    \textbf{角色分工:}
    \begin{description}
        \item[组长] 协调进度,分配任务
        \item[技术负责人] 把关代码质量
        \item[模块开发A] 负责形态学与定位
        \item[模块开发B] 负责检测与识别
    \end{description}
\end{frame}

\begin{frame}{AI辅助编程本周重点}
    \begin{block}{用AI理解复杂概念}
        \begin{itemize}
            \item \aihint{} "请用直观的比喻解释形态学腐蚀和膨胀"
            \item \aihint{} "什么是连通域?如何用图像找连通区域?"
        \end{itemize}
    \end{block}

    \begin{block}{用AI生成代码框架}
        \begin{itemize}
            \item \aihint{} "帮我创建一个ChoiceRecognizer类,包含calculate_density方法"
            \item \aihint{} "如何用OpenCV统计二值图像中的白色像素数量?"
        \end{itemize}
    \end{block}

    \begin{block}{AI调试技巧}
        \begin{enumerate}
            \item 贴出完整的错误信息(Traceback)
            \item 说明你的代码意图
            \item 请AI解释可能的原因
        \end{enumerate}
    \end{block}
\end{frame}

\begin{frame}{多屏协同设计}
    \textbf{教室布局与屏幕分工:}
    \begin{columns}
        \column{0.33\textwidth}
        \begin{block}{主屏(理论)}
            \begin{itemize}
                \item 核心概念讲解
                \item 算法原理展示
                \item 知识点梳理
            \end{itemize}
        \end{block}

        \column{0.33\textwidth}
        \begin{block}{侧屏(演示)}
            \begin{itemize}
                \item 实时代码演示
                \item AI辅助编程展示
                \item 处理效果对比
            \end{itemize}
        \end{block}

        \column{0.33\textwidth}
        \begin{block}{设备(互动)}
            \begin{itemize}
                \item Quiz实时答题
                \item 代码片段查看
                \item 小组讨论记录
            \end{itemize}
        \end{block}
    \end{columns}

    \vspace{0.5cm}

    \begin{alertblock}{动静结合原则}
    每个知识点:理论讲解(15min)$\to$ 代码演示(15min)$\to$ 实践操作(15min)
    \end{alertblock}
\end{frame}
