%===========================================================
% 01_omr_overview.tex - OMR技术概述
%===========================================================
\section{OMR技术概述}

\begin{frame}{什么是OMR?}
    \begin{block}{OMR:Optical Mark Recognition}
        光学标记识别 —— 通过检测填涂区域来识别答案
    \end{block}

    \vspace{0.5cm}

    \textbf{核心思想:} 填涂区域与空白区域的\textbf{像素密度差异}

    \begin{center}
    \begin{tikzpicture}
        \shade[ball color=black] (0,0) circle (0.5cm);
        \node at (0,-1) {填涂 $\to$ 密度高};
        \shade[ball color=white, draw=black] (3,0) circle (0.5cm);
        \node at (3,-1) {空白 $\to$ 密度低};
    \end{tikzpicture}
    \end{center}
\end{frame}

\begin{frame}{OMR vs OCR}
    \begin{table}
        \centering
        \begin{tabular}{l|l|l}
        \toprule
        特性 & OMR (光学标记识别) & OCR (光学字符识别) \\
        \midrule
        识别对象 & 填涂标记、气泡 & 文字、字符 \\
        判断依据 & 像素密度 & 特征匹配、深度学习 \\
        典型应用 & 标准化考试答题卡 & 文档数字化、车牌识别 \\
        技术复杂度 & 较低 & 较高 \\
        准确率 & 极高(99\%+) & 取决于场景 \\
        \bottomrule
        \end{tabular}
        \caption{OMR与OCR的对比}
    \end{table}
\end{frame}

\begin{frame}{OMR技术发展简史}
    \begin{itemize}
        \item[1930s] 第一台OMR机器用于考试评分
        \item[1960s] IBM推出大型OMR系统
        \item[1980s] 个人电脑+扫描仪实现桌面OMR
        \item[2000s] 数字相机+图像处理实现便携OMR
        \item[2020s] 手机拍照+AI实现智能OMR
    \end{itemize}

    \vspace{0.5cm}

    \textbf{本周目标:} 用Python+OpenCV实现一个基础的OMR系统
\end{frame}

\begin{frame}{标准化考试答题卡技术}
    \textbf{答题卡的关键元素:}

    \begin{columns}
        \column{0.5\textwidth}
        \begin{itemize}
            \item \textbf{定位标记(Timing \texttt{\_} Marks)}
                \begin{itemize}
                    \item 黑色方块/线条
                    \item 用于定位和校正
                \end{itemize}
            \item \textbf{选项气泡(Bubbles)}
                \begin{itemize}
                    \item 圆形或椭圆形
                    \item 规则排列
                \end{itemize}
        \end{itemize}

        \column{0.5\textwidth}
        \begin{itemize}
            \item \textbf{填涂区域}
                \begin{itemize}
                    \item 铅笔/碳素笔填涂
                    \item 深度影响识别
                \end{itemize}
            \item \textbf{导出块}
                \begin{itemize}
                    \item 考号填写区
                    \item 需特殊识别
                \end{itemize}
        \end{itemize}
    \end{columns}
\end{frame}

\begin{frame}{选择题识别的挑战}
    \begin{columns}
        \column{0.5\textwidth}
        \begin{alertblock}{填涂质量问题}
            \begin{itemize}
                \item 填涂深浅不一
                \item 填涂范围不完整
                \item 笔迹颜色差异
            \end{itemize}
        \end{alertblock}

        \begin{alertblock}{擦除与修改}
            \begin{itemize}
                \item 擦除不干净的痕迹
                \item 多次填涂的混乱
            \end{itemize}
        \end{alertblock}

        \column{0.5\textwidth}
        \begin{alertblock}{图像质量}
            \begin{itemize}
                \item 扫描/拍摄角度偏差
                \item 光照不均匀
                \item 纸张折痕/污损
            \end{itemize}
        \end{alertblock}

        \begin{alertblock}{多选与漏选}
            \begin{itemize}
                \item 多选项填涂(多选题)
                \item 所有选项未填(空题)
            \end{itemize}
        \end{alertblock}
    \end{columns}
\end{frame}

\begin{frame}{识别流程概述}
    \begin{enumerate}
        \item \textbf{图像预处理}:去噪、二值化、几何校正
        \item \textbf{选项区域定位}:找到每个选项的位置
        \item \textbf{提取选项图像}:裁剪出单个选项区域
        \item \textbf{形态学处理}:去除噪点、填充孔洞
        \item \textbf{统计像素密度}:计算深色像素占比
        \item \textbf{阈值判断}:密度超过阈值则认为已填涂
    \end{enumerate}

    \vspace{0.3cm}

    \begin{center}
        \begin{tikzpicture}[node distance=0.5cm]
            \node[draw, fill=blue!10, rectangle, rounded corners] (1) {预处理};
            \node[draw, fill=yellow!10, rectangle, rounded corners, right=of 1] (2) {定位};
            \node[draw, fill=green!10, rectangle, rounded corners, right=of 2] (3) {提取};
            \node[draw, fill=orange!10, rectangle, rounded corners, right=of 3] (4) {形态学};
            \node[draw, fill=red!10, rectangle, rounded corners, right=of 4] (5) {统计};
            \node[draw, fill=purple!10, rectangle, rounded corners, right=of 5] (6) {判断};

            \draw[->] (1) -- (2);
            \draw[->] (2) -- (3);
            \draw[->] (3) -- (4);
            \draw[->] (4) -- (5);
            \draw[->] (5) -- (6);
        \end{tikzpicture}
    \end{center}
\end{frame}

\begin{frame}{OMR技术在教育领域的应用}
    \textbf{标准化考试:}
    \begin{itemize}
        \item 高考:全国千万考生,选择题占比60-70\%
        \item 研究生考试:政治、英语等科目
        \item 英语四六级:每年超过1800万考生
        \item 公务员考试:行测部分全部为选择题
        \item 职业资格认证:医师、会计师、建造师等
    \end{itemize}

    \vspace{0.3cm}

    \textbf{在线教育:}
    \begin{itemize}
        \item 在线作业与测验
        \item 智慧课堂互动答题
        \item 企业培训考核
        \item 职业技能鉴定
    \end{itemize}

    \vspace{0.3cm}

    \textbf{市场规模:}
    \begin{itemize}
        \item 中国OMR市场年增长率约15\%
        \item 2025年市场规模预计超过50亿元
        \item 从硬件向软件+服务转型
    \end{itemize}
\end{frame}

\begin{frame}{答题卡的历史演进}
    \textbf{第一代:穿孔卡片(1930s-1960s)}
    \begin{itemize}
        \item IBM开发,用于人口普查和考试评分
        \item 需要专用读卡设备
        \item 速度慢、成本高
    \end{itemize}

    \vspace{0.2cm}

    \textbf{第二代:光标阅读卡(1960s-1990s)}
    \begin{itemize}
        \item 使用石墨铅笔填涂
        \item 光电传感器识别
        \item 速度快、准确率高
    \end{itemize}

    \vspace{0.2cm}

    \textbf{第三代:数字化OMR(1990s-2010s)}
    \begin{itemize}
        \item 扫描仪+计算机处理
        \item 图像处理算法优化
        \item 成本大幅降低
    \end{itemize}

    \vspace{0.2cm}

    \textbf{第四代:智能OMR(2010s-至今)}
    \begin{itemize}
        \item 手机拍照即可识别
        \item AI算法提升鲁棒性
        \item 云端实时处理
    \end{itemize}
\end{frame}

\begin{frame}{答题卡标准化设计原则}
    \textbf{设计原则:}
    \begin{enumerate}
        \item \textbf{唯一性}:每个填涂位置唯一对应一个选项
        \item \textbf{独立性}:各填涂区域互不干扰
        \item \textbf{可检测性}:提供定位标记便于自动识别
        \item \textbf{容错性}:允许一定程度的填涂偏差
    \end{enumerate}

    \vspace{0.3cm}

    \textbf{常见答题卡格式:}
    \begin{columns}
        \column{0.5\textwidth}
        \begin{itemize}
            \item \textbf{横版答题卡}
                \begin{itemize}
                    \item 左侧为选择题区域
                    \item 右侧为考号填涂区
                    \item 顶部为定位标记
                \end{itemize}
            \item \textbf{竖版答题卡}
                \begin{itemize}
                    \item 上部为选择题区域
                    \item 下部为姓名考号区
                    \item 四角定位标记
                \end{itemize}
        \end{itemize}

        \column{0.5\textwidth}
        \textbf{印刷质量要求:}
        \begin{itemize}
            \item 定位标记精度:±0.5mm
            \item 选项气泡位置:±1mm
            \item 线条粗细:0.8-1.2mm
            \item 颜色对比度:>70\%
        \end{itemize}
    \end{columns}
\end{frame}

\begin{frame}{国内外OMR系统发展现状}
    \textbf{国外主要厂商:}
    \begin{itemize}
        \item \textbf{Scantron}(美国):市场领导者,服务全球教育机构
        \item \textbf{Pearson VUE}(英国):考试评估综合服务
        \item \textbf{Datacard}(美国):身份证件+答题卡识别
    \end{itemize}

    \vspace{0.3cm}

    \textbf{国内主要厂商:}
    \begin{itemize}
        \item \textbf{海云天}:教育考试信息化龙头
        \item \textbf{鸥玛}:OMR设备制造商
        \item \textbf{新开普}:智慧考试综合解决方案
        \item \textbf{科大讯飞}:AI+OMR智能阅卷
    \end{itemize}

    \vspace{0.3cm}

    \textbf{技术趋势:}
    \begin{itemize}
        \item 硬件向软件服务转型
        \item 传统OMR向AI-OMR升级
        \item 离线处理向云端实时处理发展
    \end{itemize}
\end{frame}
