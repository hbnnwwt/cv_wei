%===========================================================
% 08_quiz.tex - 课堂Quiz
%===========================================================
\section{课堂Quiz}

\begin{frame}{Quiz 1:形态学操作选择}
    \begin{block}{问题}
    填涂区域有内部孔洞(铅笔痕迹不连续),应该使用哪种形态学操作?
    \end{block}

    \begin{columns}
        \column{0.25\textwidth}
        \begin{alertblock}{A. 腐蚀}
        会放大孔洞
        \end{alertblock}

        \column{0.25\textwidth}
        \begin{alertblock}{B. 膨胀}
        能填充但也会扩大边界
        \end{alertblock}

        \column{0.25\textwidth}
        \begin{exampleblock}{C. 闭运算 ✓}
        填充孔洞且保持大小
        \end{exampleblock}

        \column{0.25\textwidth}
        \begin{alertblock}{D. 开运算}
        用于去噪点
        \end{alertblock}
    \end{columns}
\end{frame}

\begin{frame}{Quiz 2:密度计算}
    \begin{block}{问题}
    一个30×30像素的选项ROI,二值化后有600个白色像素,密度是多少?
    \end{block}

    \begin{columns}
        \column{0.25\textwidth}
        \begin{alertblock}{A. 0.2}
        \end{alertblock}

        \column{0.25\textwidth}
        \begin{exampleblock}{B. 0.67 ✓}
        600/(30×30) = 0.67
        \end{exampleblock}

        \column{0.25\textwidth}
        \begin{alertblock}{C. 0.75}
        \end{alertblock}

        \column{0.25\textwidth}
        \begin{alertblock}{D. 0.8}
        \end{alertblock}
    \end{columns}
\end{frame}

\begin{frame}{Quiz 3:阈值策略}
    \begin{block}{问题}
    对于填涂质量差异大的答题卡,哪种阈值策略最合适?
    \end{block}

    \begin{columns}
        \column{0.33\textwidth}
        \begin{alertblock}{A. 固定0.2}
        太低,易误判
        \end{alertblock}

        \column{0.33\textwidth}
        \begin{exampleblock}{B. 自适应均值 ✓}
        根据数据动态调整
        \end{exampleblock}

        \column{0.33\textwidth}
        \begin{alertblock}{C. 固定0.5}
        太高,漏检填涂
        \end{alertblock}
    \end{columns}
\end{frame}

\begin{frame}{代码找茬挑战}
    \begin{block}{找出代码中的问题}
    \begin{lstlisting}
def calculate_density(roi):
    _, binary = cv2.threshold(roi, 127, 255, cv2.THRESH_BINARY)
    non_zero = cv2.countNonZero(binary)
    return non_zero / (roi.shape[0] * roi.shape[1])
    \end{lstlisting}
    \end{block}

    \textbf{问题:} 填涂区域应该是高密度(白色),但这里...

    \vspace{0.3cm}

    \textbf{答案:} 应该使用 \texttt{cv2.THRESH\_BINARY\_INV} 反色

    \vspace{0.3cm}

    \begin{center}
        \textcolor{blue}{\textbf{[AI提示]}} 用AI解释:为什么要用THRESH\_BINARY\_INV?
    \end{center}
\end{frame}

\begin{frame}{代码拼图挑战}
    \begin{block}{将打乱的代码复原(小组竞赛)}
    \begin{enumerate}
        \item \texttt{densities.append(density)}
        \item \texttt{for (x, y, w, h) in options:}
        \item \texttt{roi = image[y:y+h, x:x+w]}
        \item \texttt{density = calculate\_density(roi)}
    \end{enumerate}
    \end{block}

    \vspace{0.5cm}

    \textbf{正确顺序:} 2 → 3 → 4 → 1
\end{frame}

\begin{frame}{实战挑战}
    \begin{block}{任务:在10分钟内完成}
    \begin{itemize}
        \item \textbf{基础任务}:运行示例代码,识别标准答题卡
        \item \textbf{进阶任务}:调整参数,识别填涂较轻的答题卡
        \item \textbf{挑战任务}:实现自适应阈值,处理混合情况
    \end{itemize}
    \end{block}

    \vspace{0.3cm}

    \begin{center}
        \textbf{完成任务的小组获得 +2 分附加分!}
    \end{center}
\end{frame}

\begin{frame}{Quiz答案总结}
    \begin{table}
        \centering
        \begin{tabular}{cll}
        \toprule
        题号 & 正确答案 & 解析 \\
        \midrule
        Quiz 1 & C. 闭运算 & 填充内部孔洞 \\
        Quiz 2 & B. 0.67 & 600/900 = 0.67 \\
        Quiz 3 & B. 自适应均值 & 根据数据动态调整 \\
        代码找茬 & THRESH\_BINARY\_INV & 需要反色 \\
        代码拼图 & 2→3→4→1 & 先遍历再计算再添加 \\
        \bottomrule
        \end{tabular}
    \end{table}
\end{frame}
