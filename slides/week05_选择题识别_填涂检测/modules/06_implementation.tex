%===========================================================
% 06_implementation.tex - 完整实现
%===========================================================
\section{完整实现}

\begin{frame}{选择题识别器类设计}
    \textbf{类结构:}
    \begin{itemize}
        \item \texttt{ChoiceRecognizer}: 封装识别逻辑
        \item \texttt{recognize\_all()}: 批量识别所有选择题
        \item \texttt{calculate\_density()}: 计算密度
        \item \texttt{visualize()}: 可视化结果
    \end{itemize}

    \vspace{0.3cm}

    \textbf{数据流:}
    \begin{center}
        \begin{tikzpicture}[node distance=0.8cm]
            \node[draw, fill=blue!10] (input) {输入图像};
            \node[draw, fill=yellow!10, right=of input] (locate) {定位选项};
            \node[draw, fill=green!10, right=of locate] (detect) {检测填涂};
            \node[draw, fill=orange!10, right=of detect] (output) {输出结果};
            \draw[->] (input) -- (locate);
            \draw[->] (locate) -- (detect);
            \draw[->] (detect) -- (output);
        \end{tikzpicture}
    \end{center}
\end{frame}

\begin{frame}[fragile]{ChoiceRecognizer类}
    \begin{block}{AI辅助提示}
        \aihint{} "请帮我创建一个ChoiceRecognizer类,包含calculate_density、recognize_all和visualize方法"
    \end{block}

    \begin{block}{类定义}
    \begin{lstlisting}[basicstyle=\ttfamily\scriptsize]
# TODO: 可使用AI助手完成以下代码
class ChoiceRecognizer:
    """选择题识别器"""

    def __init__(self, threshold=0.3, use_morphology=True):
        """
        Args:
            threshold: 密度阈值
            use_morphology: 是否使用形态学预处理
        """
        self.threshold = threshold
        self.use_morphology = use_morphology
        self.option_labels = ['A', 'B', 'C', 'D']

    def calculate_density(self, roi):
        """计算填涂密度"""
        # TODO: 形态学预处理
        if self.use_morphology:
            kernel = cv2.getStructuringElement(cv2.MORPH_RECT, (3, 3))
            roi = cv2.morphologyEx(roi, cv2.MORPH_CLOSE, kernel)

        # TODO: 二值化并统计
        _, binary = cv2.threshold(roi, 127, 255, cv2.THRESH_BINARY_INV)
        non_zero = cv2.countNonZero(binary)
        total = roi.shape[0] * roi.shape[1]
        return non_zero / total
    \end{lstlisting}
    \end{block}
\end{frame}

\begin{frame}[fragile]{recognize\_all方法}
    \begin{block}{AI辅助提示}
        \aihint{} "请实现recognize_all方法,批量识别所有选择题并返回结果"
    \end{block}

    \begin{block}{批量识别实现}
    \begin{lstlisting}[basicstyle=\ttfamily\scriptsize]
    def recognize_all(self, image, question_groups):
        """
        识别所有选择题

        Args:
            image: 答题卡图像(灰度图)
            question_groups: 选项位置分组
                [[(x,y,w,h), ...], ...]  # 每个问题的选项

        Returns:
            results: 识别结果列表
        """
        results = []

        for i, option_positions in enumerate(question_groups):
            # 计算密度
            densities = []
            for (x, y, w, h) in option_positions:
                roi = image[y:y+h, x:x+w]
                density = self.calculate_density(roi)
                densities.append(density)

            # 找到填涂的选项
            max_density = max(densities)
            if max_density >= self.threshold:
                idx = densities.index(max_density)
                answer = self.option_labels[idx]
            else:
                answer = None

            results.append({
                'question': i + 1,
                'answer': answer,
                'confidence': max_density
            })

        return results
    \end{lstlisting}
    \end{block}
\end{frame}

\begin{frame}[fragile]{结果可视化}
    \begin{block}{AI辅助提示}
        \aihint{} "请实现visualize方法,用不同颜色标注识别结果并保存图像"
    \end{block}

    \begin{block}{visualize方法}
    \begin{lstlisting}[basicstyle=\ttfamily\scriptsize]
# TODO: 可使用AI助手完成以下代码
    def visualize(self, image, results, question_groups):
        """可视化识别结果"""
        output = cv2.cvtColor(image, cv2.COLOR_GRAY2BGR)

        for result, positions in zip(results, question_groups):
            q_num = result['question']
            answer = result['answer']
            confidence = result['confidence']

            # 标注问题编号
            x, y, _, _ = positions[0]
            cv2.putText(output, f"Q{q_num}: {answer}",
                       (x, y-10), cv2.FONT_HERSHEY_SIMPLEX,
                       0.5, (0, 255, 0), 2)

            # 标注选项状态
            for (x, y, w, h) in positions:
                # 绿色=选中, 红色=未选中
                cv2.rectangle(output, (x, y), (x+w, y+h),
                           (0, 255, 0), 2)

        return output
    \end{lstlisting}
    \end{block}
\end{frame}

\begin{frame}[fragile]{完整使用示例}
    \begin{block}{AI辅助提示}
        \aihint{} "请给我一个完整的OMR识别使用示例,从读取图像到输出结果"
    \end{block}

    \begin{block}{端到端使用}
    \begin{lstlisting}[basicstyle=\ttfamily\scriptsize]
import cv2

# 读取答题卡
image = cv2.imread("answer_card.png", cv2.IMREAD_GRAYSCALE)

# 创建识别器
recognizer = ChoiceRecognizer(threshold=0.3, use_morphology=True)

# 定位选项(使用之前的locate_options函数)
option_groups = locate_options(image)

# 识别所有选择题
results = recognizer.recognize_all(image, option_groups)

# 输出结果
for r in results:
    print(f"第{r['question']}题: {r['answer']} (置信度: {r['confidence']:.2f})")

# 可视化
output = recognizer.visualize(image, results, option_groups)
cv2.imwrite("result.png", output)
    \end{lstlisting}
    \end{block}
\end{frame}

\begin{frame}[fragile]{Live Coding:完整流程}
    \begin{block}{AI辅助提示}
        \aihint{} "请指导我从零开始构建一个完整的OMR识别系统,包括定位、检测、识别和可视化"
    \end{block}

    \begin{block}{挑战任务:从零构建识别系统}
    \begin{lstlisting}[basicstyle=\ttfamily\scriptsize]
# TODO: 任务清单
# 1. 读取答题卡图像
image = cv2.imread("answer_card.png", cv2.IMREAD_GRAYSCALE)

# 2. 定位所有选项区域
option_groups = locate_options(image)  # 你需要实现这个函数

# 3. 创建识别器
recognizer = ChoiceRecognizer(threshold=0.3)

# 4. 识别所有选择题
results = recognizer.recognize_all(image, option_groups)

# 5. 输出结果
for r in results:
    if r['answer']:
        print(f"第{r['question']}题: {r['answer']}")

# 6. 可视化结果
output = recognizer.visualize(image, results, option_groups)
cv2.imwrite("result.png", output)

# TODO: 挑战:实现自适应阈值
# TODO: 挑战:处理多选题情况
    \end{lstlisting}
    \end{block}
\end{frame}
