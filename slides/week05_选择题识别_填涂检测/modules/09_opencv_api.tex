%===========================================================
% 09_opencv_api.tex - OpenCV API工具介绍
%===========================================================
\section{OpenCV API工具介绍}

\begin{frame}{形态学处理API}
    \textbf{核心函数:}

    \begin{table}
        \centering
        \small
        \begin{tabular}{lp{5cm}l}
        \toprule
        函数 & 功能描述 & 使用场景 \\
        \midrule
        \texttt{cv2.erode()} & 腐蚀操作 & 去除噪点、分离目标 \\
        \texttt{cv2.dilate()} & 膨胀操作 & 填充孔洞、连接区域 \\
        \texttt{cv2.morphologyEx()} & 通用形态学操作 & 开/闭/梯度/顶帽/黑帽 \\
        \texttt{cv2.getStructuringElement()} & 创建结构元素 & 定义操作的探针形状 \\
        \bottomrule
        \end{tabular}
    \end{table}

    \vspace{0.3cm}

    \begin{block}{morphologyEx操作类型}
    \begin{itemize}
        \item \texttt{cv2.MORPH\_ERODE}:腐蚀
        \item \texttt{cv2.MORPH\_DILATE}:膨胀
        \item \texttt{cv2.MORPH\_OPEN}:开运算
        \item \texttt{cv2.MORPH\_CLOSE}:闭运算
        \item \texttt{cv2.MORPH\_GRADIENT}:梯度
        \item \texttt{cv2.MORPH\_TOPHAT}:顶帽
        \item \texttt{cv2.MORPH\_BLACKHAT}:黑帽
    \end{itemize}
    \end{block}
\end{frame}

\begin{frame}{连通域分析API}
    \textbf{核心函数:}

    \begin{table}
        \centering
        \small
        \begin{tabular}{lp{4cm}l}
        \toprule
        函数 & 功能描述 & 返回值 \\
        \midrule
        \texttt{cv2.connectedComponents()} & 连通域标记 & 标签数量、标签图像 \\
        \texttt{cv2.connectedComponentsWithStats()} & 带统计的标记 & +统计信息、质心 \\
        \texttt{cv2.findContours()} & 轮廓查找 & 轮廓列表、层次结构 \\
        \bottomrule
        \end{tabular}
    \end{table}

    \vspace{0.3cm}

    \textbf{统计信息索引(stats):}
    \begin{itemize}
        \item \texttt{cv2.CC\_STAT\_LEFT}:外接矩形左上角x
        \item \texttt{cv2.CC\_STAT\_TOP}:外接矩形左上角y
        \item \texttt{cv2.CC\_STAT\_WIDTH}:外接矩形宽度
        \item \texttt{cv2.CC\_STAT\_HEIGHT}:外接矩形高度
        \item \texttt{cv2.CC\_STAT\_AREA}:连通域面积
    \end{itemize}
\end{frame}

\begin{frame}{模板匹配API}
    \textbf{AI辅助提示:}
    \aihint{} "请解释OpenCV的matchTemplate函数如何工作,以及不同匹配方法的区别"

    \textbf{核心函数:}

    \begin{block}{cv2.matchTemplate()}
    \begin{lstlisting}
# TODO: 可使用AI助手完成以下代码
result = cv2.matchTemplate(
    image,      # 搜索图像
    template,   # 模板图像
    method,     # 匹配方法
    mask=None   # 掩码(可选)
)
    \end{lstlisting}
    \end{block}

    \textbf{匹配方法:}
    \begin{table}
        \centering
        \tiny
        \begin{tabular}{ll}
        \toprule
        方法 & 说明 \\
        \midrule
        \texttt{cv2.TM\_SQDIFF} & 平方差匹配(值越小越好) \\
        \texttt{cv2.TM\_SQDIFF\_NORMED} & 归一化平方差 \\
        \texttt{cv2.TM\_CCOEFF} & 相关系数 \\
        \texttt{cv2.TM\_CCOEFF\_NORMED} & 归一化相关系数(推荐) \\
        \texttt{cv2.TM\_CCORR} & 互相关 \\
        \bottomrule
        \end{tabular}
    \end{table}
\end{frame}

\begin{frame}{几何变换API}
    \textbf{AI辅助提示:}
    \aihint{} "请解释透视变换的原理,并用OpenCV实现答题卡倾斜校正"

    \textbf{透视变换:}

    \begin{block}{获取透视变换矩阵}
    \begin{lstlisting}
# TODO: 可使用AI助手完成以下代码
M = cv2.getPerspectiveTransform(
    src,  # 源图像四个点
    dst   # 目标图像四个点
)
    \end{lstlisting}
    \end{block}

    \begin{block}{执行透视变换}
    \begin{lstlisting}
# TODO: 可使用AI助手完成以下代码
warped = cv2.warpPerspective(
    image,    # 输入图像
    M,        # 变换矩阵
    dsize     # 输出图像大小 (width, height)
)
    \end{lstlisting}
    \end{block}

    \textbf{应用:} 答题卡倾斜校正、透视变形修复
\end{frame}

\begin{frame}[fragile]{Live Coding:OpenCV API综合使用}
    \begin{block}{AI辅助提示}
        \aihint{} "请指导我使用OpenCV的API完成答题卡定位,包括二值化、轮廓查找、气泡筛选"
    \end{block}

    \begin{block}{任务:使用API完成答题卡定位}
    \begin{lstlisting}[basicstyle=\ttfamily\scriptsize]
import cv2
import numpy as np

# 读取答题卡
card = cv2.imread("answer_card.png", cv2.IMREAD_GRAYSCALE)

# TODO: 二值化
_, binary = cv2.threshold(card, 127, 255, cv2.THRESH_BINARY_INV)

# TODO: 查找轮廓(定位选项气泡)
contours, hierarchy = cv2.findContours(
    binary,
    cv2.RETR_EXTERNAL,
    cv2.CHAIN_APPROX_SIMPLE
)

# TODO: 筛选气泡状轮廓
bubbles = []
for cnt in contours:
    x, y, w, h = cv2.boundingRect(cnt)
    area = cv2.contourArea(cnt)
    ratio = w / h if h > 0 else 0

    # 筛选条件:面积适中、接近圆形
    if 500 < area < 5000 and 0.7 < ratio < 1.3:
        bubbles.append((x, y, w, h))

print(f"找到 {len(bubbles)} 个选项气泡")
    \end{lstlisting}
    \end{block}
\end{frame}
