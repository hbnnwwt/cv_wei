\documentclass[aspectratio=169, 12pt]{beamer}
\usepackage[UTF8]{ctex}
\usepackage{graphicx}
\usepackage{booktabs}
\usepackage{listings}
\usepackage{xcolor}
\usepackage{tikz}
\usepackage{hyperref}

\usetheme{Madrid}
\usecolortheme{whale}
\usefonttheme{professionalfonts}

\lstset{
    language=Python,
    basicstyle=\ttfamily\small,
    keywordstyle=\color{blue},
    commentstyle=\color{green!60!black},
    stringstyle=\color{orange},
    breaklines=true,
    frame=single,
    showstringspaces=false,
    backgroundcolor=\color{gray!10}
}

\title[选择题识别(填涂检测)]{第5周:选择题识别(填涂检测)}
\subtitle{怎么知道选了A还是B?}
\author{计算机视觉课程组}
\institute{通选课}
\date{}

\begin{document}

\begin{frame}
    \titlepage
\end{frame}

\begin{frame}{课程概览}
    \tableofcontents
\end{frame}

\section{OMR原理}

\begin{frame}{什么是OMR?}
    \begin{block}{OMR:Optical Mark Recognition}
        光学标记识别 —— 通过检测填涂区域来识别答案
    \end{block}

    \vspace{0.5cm}

    \textbf{核心思想:} 填涂区域与空白区域的\textbf{像素密度差异}

    \begin{center}
        \begin{tikzpicture}
            \shade[ball color=black] (0,0) circle (0.5cm);
            \node at (0,-1) {填涂 $\to$ 密度高};
            \shade[ball color=white, draw=black] (3,0) circle (0.5cm);
            \node at (3,-1) {空白 $\to$ 密度低};
        \end{tikzpicture}
    \end{center}
\end{frame}

\begin{frame}{识别流程}
    \begin{enumerate}
        \item \textbf{定位选项区域}:找到每个选项的位置
        \item \textbf{提取选项图像}:裁剪出单个选项区域
        \item \textbf{二值化处理}:转换为黑白图像
        \item \textbf{统计像素密度}:计算深色像素占比
        \item \textbf{阈值判断}:密度超过阈值则认为已填涂
    \end{enumerate}

    \vspace{0.3cm}

    \begin{center}
        \begin{tikzpicture}
            \node[draw, fill=blue!10] (1) {定位};
            \node[draw, fill=yellow!10, right of=1] (2) {提取};
            \node[draw, fill=green!10, right of=2] (3) {二值化};
            \node[draw, fill=red!10, right of=3] (4) {统计};
            \node[draw, fill=purple!10, right of=4] (5) {判断};

            \draw[->] (1) -- (2);
            \draw[->] (2) -- (3);
            \draw[->] (3) -- (4);
            \draw[->] (4) -- (5);
        \end{tikzpicture}
    \end{center}
\end{frame}

\section{选项区域定位}

\begin{frame}{定位方法}
    \textbf{方法1:固定位置法}(简单但不灵活)
    \begin{itemize}
        \item 适合标准答题卡
        \item 预先定义选项坐标
    \end{itemize}

    \vspace{0.3cm}

    \textbf{方法2:轮廓检测法}(推荐)
    \begin{itemize}
        \item 通过连通域分析找到气泡
        \item 根据面积、形状筛选
        \item 自动适应不同版面
    \end{itemize}
\end{frame}

\section{填涂检测核心算法}

\begin{frame}{像素密度统计}
    \textbf{方法:非零像素计数}

    \begin{lstlisting}
def calculate_density(option_roi, threshold=127):
    """计算填涂密度"""
    # 二值化(反色:填涂变白)
    _, binary = cv2.threshold(
        option_roi, threshold, 255,
        cv2.THRESH_BINARY_INV
    )

    # 统计白色像素
    non_zero = cv2.countNonZero(binary)

    # 计算密度
    total = option_roi.shape[0] * option_roi.shape[1]
    density = non_zero / total

    return density
    \end{lstlisting}
\end{frame}

\begin{frame}{阈值确定}
    \textbf{固定阈值法:}
    \begin{itemize}
        \item 经验值:0.2 - 0.4
        \item 简单但不适应不同场景
    \end{itemize}

    \vspace{0.3cm}

    \textbf{自适应阈值法:}
    \begin{itemize}
        \item 基于所有选项的密度分布
        \item 使用Otsu思想确定最佳阈值
        \item 推荐使用
    \end{itemize}
\end{frame}

\section{单选题识别}

\begin{frame}[fragile]{单选题识别}
    \begin{lstlisting}[basicstyle=\ttfamily\scriptsize]
def recognize_single_choice(question_img, option_positions):
    """识别单道选择题"""
    densities = []

    # 计算每个选项的密度
    for (x, y, w, h) in option_positions:
        roi = question_img[y:y+h, x:x+w]
        density = calculate_density(roi)
        densities.append(density)

    # 自适应确定阈值
    threshold = np.mean(densities)

    # 找到密度最高的选项
    max_density = max(densities)

    if max_density >= threshold:
        answer_idx = densities.index(max_density)
        return chr(ord('A') + answer_idx)

    return None  # 未填涂
    \end{lstlisting}
\end{frame}

\section{完整实现}

\begin{frame}[fragile]{选择题识别器}
    \begin{lstlisting}[basicstyle=\ttfamily\scriptsize]
class ChoiceRecognizer:
    def __init__(self, threshold=0.3):
        self.threshold = threshold
        self.option_labels = ['A', 'B', 'C', 'D']

    def recognize_all(self, image, question_groups):
        """识别所有选择题"""
        results = []

        for i, option_positions in enumerate(question_groups):
            # 计算密度
            densities = [self.calculate_density(
                image[y:y+h, x:x+w])
                for x, y, w, h in option_positions
            ]

            # 找到填涂的选项
            if max(densities) >= self.threshold:
                idx = densities.index(max(densities))
                answer = self.option_labels[idx]
            else:
                answer = None

            results.append({'question': i+1, 'answer': answer})

        return results
    \end{lstlisting}
\end{frame}

\section{思考题}

\begin{frame}{课堂思考题}
    \begin{block}{问题1:OMR原理}
        \begin{itemize}
            \item 为什么通过像素密度就能判断填涂?
            \item 如果铅笔填涂很轻,密度会怎样变化?
        \end{itemize}
    \end{block}

    \vspace{0.3cm}

    \begin{block}{问题2:阈值选择}
        \begin{itemize}
            \item 固定阈值有什么缺点?
            \item 如何自动确定最佳阈值?
        \end{itemize}
    \end{block}
\end{frame}

\section{课后作业}

\begin{frame}{课后作业}
    \begin{block}{题目}
        实现选择题填涂识别模块
    \end{block}

    \textbf{要求:}
    \begin{enumerate}
        \item 实现选项区域定位
        \item 实现填涂密度计算
        \item 识别单选题答案
        \item 可视化标注识别结果
    \end{enumerate}

    \vspace{0.2cm}

    \textbf{评分标准:}
    \begin{itemize}
        \item 选项定位:25分
        \item 密度计算:25分
        \item 答案识别:30分
        \item 可视化:10分
        \item 代码质量:10分
    \end{itemize}
\end{frame}

\begin{frame}{下节预告}
    \begin{center}
        \Large \textbf{第6周:判断题识别(符号匹配)}

        \vspace{0.5cm}

        \normalsize
        故事问题:\textcolor{blue}{怎么看到是$\checkmark$还是$\times$?}

        \vspace{0.3cm}

        你将学会:
        \begin{itemize}
            \item 形状特征提取
            \item 轮廓形状分析
            \item 符号匹配识别
        \end{itemize}
    \end{center}
\end{frame}

\begin{frame}
    \begin{center}
        \Huge \textbf{谢谢!}
    \end{center}
\end{frame}

\end{document}
