%===========================================================
% summary.tex - 知识点总结与课后作业
%===========================================================
\section{总结与作业}

\begin{frame}{本周知识地图}
    \begin{center}
    \begin{tikzpicture}[
        node distance=0.5cm and 0.8cm,
        box/.style={draw, rectangle, rounded corners, minimum width=2cm, minimum height=0.8cm, align=center}
    ]
        \node[box, fill=blue!20] (omr) {OMR\\原理};
        \node[box, fill=yellow!20, right=of omr] (morph) {形态学\\处理};
        \node[box, fill=green!20, right=of morph] (conn) {连通域\\分析};
        \node[box, fill=orange!20, below=of morph] (loc) {选项\\定位};
        \node[box, fill=red!20, right=of loc] (detect) {填涂\\检测};
        \node[box, fill=purple!20, right=of detect] (impl) {完整\\实现};

        \draw[->] (omr) -- (morph);
        \draw[->] (morph) -- (conn);
        \draw[->] (morph) -- (loc);
        \draw[->] (conn) -- (loc);
        \draw[->] (loc) -- (detect);
        \draw[->] (detect) -- (impl);
    \end{tikzpicture}
    \end{center}

    \textbf{核心技能:}
    \begin{itemize}
        \item 掌握形态学操作(腐蚀、膨胀、开运算、闭运算)
        \item 理解连通域分析的应用
        \item 实现填涂检测算法
        \item 构建完整的识别系统
    \end{itemize}
\end{frame}

\begin{frame}{算法对比总结}
    \begin{table}
        \centering
        \small
        \begin{tabular}{l|p{3cm}|p{4cm}}
        \toprule
        技术 & 核心思想 & 适用场景 \\
        \midrule
        腐蚀 & 前景缩小 & 去除噪点 \\
        膨胀 & 前景扩大 & 填充孔洞 \\
        开运算 & 先腐后胀 & 去噪保形 \\
        闭运算 & 先胀后腐 & \textcolor{green}{填充孔洞(填涂检测常用)} \\
        连通域分析 & 区域标记 & 自动定位选项 \\
        密度统计 & 像素占比 & 判断是否填涂 \\
        \bottomrule
        \end{tabular}
    \end{table}
\end{frame}

\begin{frame}{智能阅卷系统进度}
    \begin{columns}
        \column{0.5\textwidth}
        \textbf{已完成模块:}
        \begin{enumerate}
            \item[\checkmark] 图像采集与预处理
            \item[\checkmark] 答题卡区域定位
            \item[\checkmark] 选择题填涂检测
        \end{enumerate}

        \column{0.5\textwidth}
        \textbf{待完成模块:}
        \begin{enumerate}
            \item[\_] 判断题符号识别
            \item[\_] 手写文字OCR
            \item[\_] 系统集成与部署
        \end{enumerate}
    \end{columns}

    \vspace{0.5cm}

    \begin{alertblock}{本周成果}
    完成选择题识别模块,为智能阅卷系统添加核心功能
    \end{alertblock}
\end{frame}

\begin{frame}{课后作业}
    \begin{block}{题目:实现选择题填涂识别模块}
    \end{block}

    \textbf{基础任务(60分):}
    \begin{enumerate}
        \item 实现选项区域定位(15分)
        \item 实现填涂密度计算(15分)
        \item 识别单选题答案(15分)
        \item 可视化标注识别结果(15分)
    \end{enumerate}

    \textbf{拓展任务(20分):}
    \begin{itemize}
        \item 实现形态学预处理,提高识别准确率
        \item 实现自适应阈值策略
    \end{itemize}

    \textbf{挑战任务(20分):}
    \begin{itemize}
        \item 处理多选题识别
        \item 实现几何校正,处理倾斜扫描
    \end{itemize}
\end{frame}

\begin{frame}{作业提交要求}
    \textbf{提交内容:}
    \begin{itemize}
        \item 代码文件(.py)
        \item 测试结果截图
        \item 简要说明文档(.md或.txt)
    \end{itemize}

    \vspace{0.3cm}

    \textbf{评分标准:}
    \begin{table}
        \centering
        \begin{tabular}{lc}
        \toprule
        评分项 & 分值 \\
        \midrule
        选项定位准确性 & 15分 \\
        密度计算正确性 & 15分 \\
        答案识别准确率 & 20分 \\
        可视化效果 & 10分 \\
        代码质量 & 10分 \\
        拓展功能 & 20分 \\
        文档完整性 & 10分 \\
        \bottomrule
        \end{tabular}
    \end{table}
\end{frame}

\begin{frame}{AI辅助作业提示}
    \begin{block}{可以用AI做什么?}
    \begin{itemize}
        \item \aihint{} "帮我设计ChoiceRecognizer类的结构"
        \item \aihint{} "如何用OpenCV实现形态学闭运算?"
        \item \aihint{} "帮我调试这个密度计算函数"
        \item \aihint{} "如何用Matplotlib并排显示多张图片?"
    \end{itemize}
    \end{block}

    \begin{alertblock}{学术诚信}
    \begin{itemize}
        \item 可以用AI生成代码框架
        \item 必须理解每一行代码的作用
        \item 在代码中标注AI辅助部分
    \end{itemize}
    \end{alertblock}
\end{frame}

\begin{frame}{延伸学习资源}
    \textbf{推荐阅读:}
    \begin{itemize}
        \item OpenCV官方文档:Image Processing
        \item 《数字图像处理》- 冈萨雷斯(形态学章节)
        \item OMR技术综述论文
    \end{itemize}

    \vspace{0.3cm}

    \textbf{在线资源:}
    \begin{itemize}
        \item OpenCV Python Tutorials
        \item Coursera: Digital Image Processing
    \end{itemize}

    \vspace{0.3cm}

    \textbf{开源项目参考:}
    \begin{itemize}
        \item GitHub: omr-python
        \item GitHub: bubble-sheet-scanner
    \end{itemize}
\end{frame}

\begin{frame}{下节预告}
    \begin{center}
        \Large \textbf{第6周:判断题识别(符号匹配)}

        \vspace{0.5cm}

        \normalsize
        故事问题:\textcolor{blue}{怎么看到是$\checkmark$还是$\times$?}

        \vspace{0.3cm}

        你将学会:
        \begin{itemize}
            \item 轮廓特征提取
            \item 模板匹配法
            \item $\checkmark$和$\times$符号识别
        \end{itemize}
    \end{center}
\end{frame}

\begin{frame}
    \begin{center}
        \Huge \textbf{谢谢!}

        \vspace{1cm}

        \large \textbf{提问与答疑}
    \end{center}
\end{frame}
