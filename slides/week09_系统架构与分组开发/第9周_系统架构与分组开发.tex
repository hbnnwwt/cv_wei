\documentclass[aspectratio=169, 12pt]{beamer}
\usepackage[UTF8]{ctex}
\usepackage{graphicx}
\usepackage{booktabs}
\usepackage{listings}
\usepackage{xcolor}
\usepackage{tikz}
\usepackage{hyperref}

\usetheme{Madrid}
\usecolortheme{whale}
\usefonttheme{professionalfonts}

\lstset{
    language=Python,
    basicstyle=\ttfamily\small,
    keywordstyle=\color{blue},
    commentstyle=\color{green!60!black},
    stringstyle=\color{orange},
    breaklines=true,
    frame=single,
    showstringspaces=false,
    backgroundcolor=\color{gray!10}
}

\title[系统架构与分组开发]{第9周:系统架构与分组开发}
\subtitle{把所有模块组合起来}
\author{北京石油化工学院\\人工智能研究院\\王文通}
\institute{通选课}
\date{2025-2026 学年}
\titlegraphic{
    \includegraphics[height=1.2cm]{../xiaohui.png}\hspace{2cm}
    \includegraphics[height=1.2cm]{../name.png}
}

\begin{document}

\begin{frame}
    \titlepage
\end{frame}

\begin{frame}{课程概览}
    \tableofcontents
\end{frame}

\section{项目回顾}

\begin{frame}{已学模块回顾}
    \begin{table}
        \centering
        \small
        \begin{tabular}{clp{5cm}}
            \toprule
            \textbf{周次} & \textbf{模块} & \textbf{功能} \\
            \midrule
            1-2 & 基础工具 & OpenCV基础、AI辅助编程 \\
            3 & 图像预处理 & 去噪、二值化、透视矫正 \\
            4 & 版面分析 & 边缘检测、轮廓检测、区域定位 \\
            5 & 选择题识别 & 填涂检测、OMR识别 \\
            6 & 判断题识别 & 符号匹配、形状识别 \\
            7-8 & 文字识别 & OCR印刷识别、手写识别 \\
            \bottomrule
        \end{tabular}
    \end{table}

    \textbf{本周目标:} 整合所有模块,构建完整系统!
\end{frame}

\section{系统架构设计}

\begin{frame}{系统架构}
    \begin{center}
        \begin{tikzpicture}[node distance=1.2cm]
            \node[draw, rectangle, fill=blue!10] (input) {输入:试卷图像};
            \node[draw, rectangle, fill=yellow!10, below of=input] (pre) {预处理模块};
            \node[draw, rectangle, fill=green!10, below of=pre] (layout) {版面分析模块};
            \node[draw, rectangle, fill=red!10, below of=layout] (rec) {识别模块};
            \node[draw, rectangle, fill=purple!10, below of=rec] (grade) {评分模块};
            \node[draw, rectangle, fill=orange!10, below of=grade] (output) {输出:评分报告};

            \draw[->, thick] (input) -- (pre);
            \draw[->, thick] (pre) -- (layout);
            \draw[->, thick] (layout) -- (rec);
            \draw[->, thick] (rec) -- (grade);
            \draw[->, thick] (grade) -- (output);
        \end{tikzpicture}
    \end{center}
\end{frame}

\begin{frame}{模块接口设计}
    \begin{table}
        \centering
        \small
        \begin{tabular}{lp{3cm}lp{4cm}}
            \toprule
            \textbf{模块} & \textbf{输入} & \textbf{输出} & \textbf{接口} \\
            \midrule
            预处理 & 原始图像 & 预处理图像 & preprocess(img) \\
            版面分析 & 预处理图像 & 区域坐标 & analyze(img) \\
            选择题 & 选项区域 & 答案(A/B/C/D) & recognize\_choice(roi) \\
            判断题 & 符号区域 & 答案(T/F) & recognize\_judge(roi) \\
            简答题 & 答题区域 & 文字内容 & recognize\_essay(roi) \\
            评分 & 所有答案 & 分数报告 & grade(answers) \\
            \bottomrule
        \end{tabular}
    \end{table}
\end{frame}

\section{项目框架}

\begin{frame}{代码结构}
    \textbf{项目目录结构:}
    \begin{verbatim}
auto_grading_system/
├── main.py                   # 主程序入口
├── modules/                  # 功能模块
│   ├── preprocess.py         # 预处理(已实现)
│   ├── layout.py             # 版面分析(已实现)
│   ├── choice_recognizer.py  # 选择题(待实现)
│   ├── judge_recognizer.py   # 判断题(待实现)
│   ├── essay_recognizer.py   # 简答题(待实现)
│   └── grading.py            # 评分(待实现)
├── utils/                    # 工具函数
├── data/                     # 数据目录
│   ├── input/                # 输入试卷
│   ├── output/               # 输出结果
│   └── templates/            # 标准答案
└── tests/                    # 单元测试
    \end{verbatim}
\end{frame}

\section{分组与任务}

\begin{frame}{分组原则}
    \textbf{分组建议:}
    \begin{itemize}
        \item 每组3-4人
        \item 包含不同专业背景
        \item 设立组长和技术负责人
    \end{itemize}

    \vspace{0.3cm}

    \textbf{角色分工:}
    \begin{table}
        \centering
        \small
        \begin{tabular}{lp{6cm}l}
            \toprule
            \textbf{角色} & \textbf{职责} & \textbf{适合} \\
            \midrule
            组长 & 统筹协调、进度管理 & 组织能力强的 \\
            技术负责人 & 架构设计、核心算法 & CS/EE专业 \\
            模块开发A & 选择题+判断题实现 & 有编程基础的 \\
            模块开发B & 简答题+评分实现 & 有编程基础的 \\
            \bottomrule
        \end{tabular}
    \end{table}
\end{frame}

\begin{frame}{开发任务清单}
    \textbf{任务优先级:}
    \begin{enumerate}
        \item \textbf{高优先级}
            \begin{itemize}
                \item 搭建开发环境
                \item 实现选择题识别
                \item 实现判断题识别
                \item 模块集成测试
            \end{itemize}
        \item \textbf{中优先级}
            \begin{itemize}
                \item 实现简答题识别
                \item 实现评分模块
            \end{itemize}
        \item \textbf{低优先级}
            \begin{itemize}
                \item 界面开发
                \item 文档编写
            \end{itemize}
    \end{enumerate}
\end{frame}

\section{开发指导}

\begin{frame}{开发环境搭建}
    \begin{lstlisting}
# 1. 创建项目
mkdir auto_grading_system
cd auto_grading_system

# 2. 创建虚拟环境
python -m venv venv
source venv/bin/activate  # Linux/Mac
venv\Scripts\activate     # Windows

# 3. 安装依赖
pip install opencv-python paddlepaddle paddleocr
pip install numpy pillow matplotlib

# 4. 创建项目结构
mkdir -p modules utils data/{input,output,templates,test}
    \end{lstlisting}
\end{frame}

\begin{frame}{调试工具}
    \textbf{调试工具函数:}
    \begin{lstlisting}
import cv2
from datetime import datetime

def save_debug_image(image, name):
    """保存调试图像"""
    timestamp = datetime.now().strftime('%Y%m%d_%H%M%S')
    filename = f"debug_{name}_{timestamp}.jpg"
    cv2.imwrite(filename, image)
    print(f"Debug saved: {filename}")

def draw_debug_boxes(image, boxes):
    """绘制调试框"""
    debug_img = image.copy()
    for (x, y, w, h) in boxes:
        cv2.rectangle(debug_img, (x, y), (x+w, y+h), (0, 255, 0), 2)
    return debug_img
    \end{lstlisting}
\end{frame}

\section{思考题}

\begin{frame}{课堂思考题}
    \begin{block}{问题1:系统设计}
        \begin{itemize}
            \item 为什么需要模块化设计?
            \item 模块之间如何传递数据?
        \end{itemize}
    \end{block}

    \vspace{0.3cm}

    \begin{block}{问题2:团队协作}
        \begin{itemize}
            \item 如何处理编程能力差异大的组员?
            \item 如何确保代码风格一致?
        \end{itemize}
    \end{block}
\end{frame}

\begin{frame}{本周任务}
    \begin{block}{分组任务}
        \begin{enumerate}
            \item 确定分组(3-4人/组)
            \item 确定组长和技术负责人
            \item 制定开发计划
            \item 搭建开发环境
            \item 开始模块开发(至少完成选择题框架)
        \end{enumerate}
    \end{block}

    \vspace{0.2cm}

    \textbf{提交内容:}
    \begin{itemize}
        \item 分组名单
        \item 开发计划
        \item 项目初始化代码
    \end{itemize}
\end{frame}

\begin{frame}{下节预告}
    \begin{center}
        \Large \textbf{第10周:核心开发与调试}

        \vspace{0.5cm}

        \normalsize
        目标:完成所有模块开发与集成

        \vspace{0.3cm}

        \textbf{这是最后一周完整开发时间!}
    \end{center}
\end{frame}

\begin{frame}
    \begin{center}
        \Huge \textbf{谢谢!}

        \vspace{1cm}

        \Large 有问题随时交流
    \end{center}
\end{frame}

\end{document}
