\section{识别引擎架构设计}

\begin{frame}{为什么需要模块化识别引擎?}
  \begin{columns}
    \column{0.45\textwidth}
      \begin{itemize}
        \item \textbf{问题1:代码耦合严重}
        \begin{itemize}
          \item 图像预处理与识别逻辑混在一起
          \item 新增题型需要修改核心代码
          \item 难以进行单元测试
        \end{itemize}
        \item \textbf{问题2:算法无法替换}
        \begin{itemize}
          \item 固定使用某种边缘检测算法
          \item 无法根据图像特点自适应选择
        \end{itemize}
      \end{itemize}
    \column{0.45\textwidth}
      \begin{itemize}
        \item \textbf{模块化解决思路}
        \begin{itemize}
          \item 定义统一接口规范
          \item 解耦预处理与识别逻辑
          \item 支持运行时动态加载算法
        \end{itemize}
        \item \textbf{设计目标}
        \begin{itemize}
          \item 开闭原则:对扩展开放,对修改关闭
          \item 依赖倒置:依赖于抽象而非具体
        \end{itemize}
      \end{itemize}
  \end{columns}
\end{frame}

\section{引擎接口设计}

\begin{frame}{识别器接口定义}
  所有识别器必须实现统一接口:

  \begin{block}{识别器接口规范}
    \begin{itemize}
      \item \texttt{detect\_question\_type(image) -> str}:检测图像中的题型
      \item \texttt{preprocess(image) -> image}:图像预处理
      \item \texttt{recognize(image) -> dict}:执行识别,返回结构化结果
      \item \texttt{get\_confidence(result) -> float}:计算识别置信度
    \end{itemize}
  \end{block}

  \begin{exampleblock}{接口设计原则}
    \begin{enumerate}
      \item 输入输出标准化:统一使用 NumPy 数组格式
      \item 错误处理统一:返回空字典表示识别失败
      \item 配置外部化:通过参数传入而非硬编码
    \end{enumerate}
  \end{exampleblock}
\end{frame}

\section{识别器实现}

\begin{frame}{选择题识别器实现}
  \begin{block}{选择题识别流程}
    \begin{enumerate}
      \item 题目区域定位:使用连通域分析找到题目块
      \item 选项分割:投影法分离各选项区域
      \item 填涂判断:计算选项内部填充比例
      \item 结果生成:选择填充比例最大的选项
    \end{enumerate}
  \end{block}

  \begin{alertblock}{关键代码逻辑}
    \texttt{如果 填充比例 > 0.6 且 方差 < 阈值:}\\
    \hspace{1cm} \texttt{判定为已填涂}
  \end{alertblock}

  \begin{itemize}
    \item 使用 Otsu 方法自适应确定二值化阈值
    \item 形态学闭操作填补小空洞
    \item 轮廓面积过滤去除噪点
  \end{itemize}
\end{frame}

\begin{frame}{判断题符号识别算法}
  \begin{columns}
    \column{0.5\textwidth}
      \begin{block}{判断依据}
        \begin{itemize}
          \item 正确($\checkmark$):斜向线条组合
          \item 错误($\times$):交叉十字形
        \end{itemize}
      \end{block}
    \column{0.45\textwidth}
      \begin{block}{识别步骤}
        \begin{enumerate}
          \item 霍夫变换检测直线
          \item 分析直线角度分布
          \item 计算交叉点位置
          \item 输出判断结果
        \end{enumerate}
      \end{block}
  \end{columns}

  \begin{algorithm}[H]
    \caption{判断题符号识别}
    \begin{algorithmic}
      \State 检测所有直线段
      \If {存在两条交叉直线且夹角 $\in [60^\circ, 120^\circ]$}
        \Return $\times$
      \ElseIf {存在两条夹角 $< 45^\circ$ 的直线}
        \Return $\checkmark$
      \EndIf
    \end{algorithmic}
  \end{algorithm}
\end{frame}

\section{高级特性}

\begin{frame}{算法选择策略}
  \begin{block}{自动算法选择机制}
    根据图像特征自动选择最优算法:

    \begin{itemize}
      \item \textbf{图像复杂度评估}:计算梯度幅值和边缘密度
      \item \textbf{噪声水平检测}:分析邻域像素差异
      \item \textbf{动态策略选择}
      \begin{itemize}
        \item 噪声高:使用中值滤波 + Canny 边缘检测
        \item 背景简单:直接使用 Otsu 二值化
        \item 目标模糊:使用自适应阈值
      \end{itemize}
    \end{itemize}
  \end{block}

  \begin{block}{策略模式实现}
    \texttt{根据 图像复杂度\_等级 选择 对应识别器}\\
    \texttt{case 低: 选用 简单二值化}\\
    \texttt{case 中: 选用 自适应阈值}\\
    \texttt{case 高: 选用 形态学增强}
  \end{block}
\end{frame}

\begin{frame}{结果融合与置信度评估}
  \begin{block}{多识别器融合}
    当多个识别器给出结果时,采用加权投票策略:

    \begin{itemize}
      \item 每个识别器输出结果和置信度
      \item 加权平均计算综合置信度
      \item 置信度超过阈值则输出结果
    \end{itemize}
  \end{block}

  \begin{block}{置信度计算公式}
    \texttt{综合置信度 = $\sum$ (识别器置信度 $\times$ 权重) / $\sum$ 权重}

    \begin{itemize}
      \item 选择题识别器权重:0.8
      \item 判断题识别器权重:0.7
      \item 位置校验器权重:0.5
    \end{itemize}
  \end{block}
\end{frame}

\begin{frame}{置信度阈值与降级策略}
  \begin{block}{三级置信度阈值}
    \begin{description}
      \item[高置信度 $>$ 0.9]:直接输出,无需复核
      \item[中置信度 0.7-0.9]:输出但标记为待复核
      \item[低置信度 $<$ 0.7]:触发降级策略或人工复核
    \end{description}
  \end{block}

  \begin{block}{降级策略}
    \begin{enumerate}
      \item 更换备用识别算法重新识别
      \item 尝试多角度/多阈值识别
      \item 返回结果附带警告信息
      \item 降级日志记录
    \end{enumerate}
  \end{block}

  \begin{alertblock}{关键阈值设置}
    \texttt{THRESHOLD\_HIGH = 0.9}\\
    \texttt{THRESHOLD\_MEDIUM = 0.7}\\
    \texttt{THRESHOLD\_LOW = 0.5}
  \end{alertblock}
\end{frame}

\begin{frame}{引擎使用完整示例}
  \begin{block}{引擎初始化与调用}
    \begin{enumerate}
      \item 创建引擎实例并注册识别器
      \item 加载配置参数
      \item 传入图像执行识别
      \item 获取并解析结果
    \end{enumerate}
  \end{block}

  \begin{exampleblock}{标准调用流程}
    \begin{enumerate}
      \item \texttt{engine = RecognitionEngine()}
      \item \texttt{engine.register("choice", ChoiceRecognizer())}
      \item \texttt{engine.register("judge", JudgeRecognizer())}
      \item \texttt{result = engine.recognize(image)}
      \item \texttt{if result.confidence > 0.7: print(result.value)}
    \end{enumerate}
  \end{exampleblock}

  \begin{itemize}
    \item 引擎自动完成:类型检测 $\rightarrow$ 算法选择 $\rightarrow$ 识别执行 $\rightarrow$ 结果融合
    \item 返回统一格式:\{type, value, confidence, details\}
  \end{itemize}
\end{frame}

\begin{frame}{性能监控与优化}
  \begin{block}{关键性能指标}
    \begin{columns}
      \column{0.45\textwidth}
        \begin{itemize}
          \item 识别耗时:单张图像处理时间
          \item 内存占用:运行时的内存峰值
          \item 准确率:正确识别样本/总样本
          \item 召回率:检出目标/实际目标
        \end{itemize}
      \column{0.45\textwidth}
        \begin{itemize}
          \item 识别成功率:成功返回结果/总调用
          \item 降级触发率:触发降级次数/总调用
          \item 置信度分布:各置信度区间的占比
          \item 算法切换频率:切换算法的次数
        \end{itemize}
    \end{columns}
  \end{block}

  \begin{block}{优化策略}
    \begin{itemize}
      \item \textbf{预处理优化}:使用查表法替代复杂计算
      \item \textbf{识别器懒加载}:首次使用时才初始化
      \item \textbf{批处理加速}:多图像合并处理减少开销
      \item \textbf{缓存机制}:相似图像复用识别结果
    \end{itemize}
  \end{block}
\end{frame}
