%===========================================================
% 08_pipeline.tex - 图像处理流水线实战
%===========================================================

\section{图像处理流水线架构设计}

\begin{frame}{流水线架构设计}
    \begin{definition}[图像处理流水线]
        将图像处理任务分解为一系列独立的处理步骤(过滤器),数据通过管道依次流经各过滤器完成处理。
    \end{definition}

    \vspace{0.3cm}

    \begin{center}
        \begin{tikzpicture}[scale=0.7, transform shape,
            box/.style={draw, rectangle, rounded corners, fill=blue!15, minimum width=2cm, minimum height=0.8cm, align=center},
            arrow/.style={->, thick}]
            % 输入
            \node[box, fill=green!20] (input) at (0,0) {原始图像};

            % 过滤器
            \node[box] (f1) at (2.5,0) {去噪\\过滤器};
            \node[box] (f2) at (5,0) {增强\\过滤器};
            \node[box] (f3) at (7.5,0) {二值化\\过滤器};
            \node[box] (f4) at (10,0) {矫正\\过滤器};

            % 输出
            \node[box, fill=red!20] (output) at (12.5,0) {处理后\\图像};

            % 连接
            \draw[arrow] (input) -- (f1);
            \draw[arrow] (f1) -- (f2);
            \draw[arrow] (f2) -- (f3);
            \draw[arrow] (f3) -- (f4);
            \draw[arrow] (f4) -- (output);
        \end{tikzpicture}
    \end{center}

    \vspace{0.3cm}

    \textbf{设计要点:}
    \begin{itemize}
        \item 每个过滤器职责单一
        \item 过滤器之间松耦合
        \item 支持动态组合
        \item 中间结果可追溯
    \end{itemize}
\end{frame}

\begin{frame}{过滤器接口设计}
    \begin{lstlisting}
from abc import ABC, abstractmethod
from dataclasses import dataclass
from typing import Any, Dict

@dataclass
class ImageData:
    """图像数据封装"""
    image: np.ndarray
    metadata: Dict[str, Any]

class FilterInterface(ABC):
    """过滤器接口"""

    @abstractmethod
    def process(self, data: ImageData) -> ImageData:
        """处理图像数据,返回处理后的数据"""
        pass

    @abstractmethod
    def get_name(self) -> str:
        """获取过滤器名称"""
        pass

    def configure(self, params: Dict[str, Any]) -> None:
        """配置参数(可选)"""
        pass
    \end{lstlisting}
\end{frame}

\begin{frame}{去噪过滤器实现}
    \begin{lstlisting}
class DenoiseFilter(FilterInterface):
    """去噪过滤器:高斯滤波"""

    def __init__(self, kernel_size=5):
        self.kernel_size = kernel_size
        self.name = "DenoiseFilter"

    def get_name(self) -> str:
        return self.name

    def process(self, data: ImageData) -> ImageData:
        denoised = cv2.GaussianBlur(
            data.image,
            (self.kernel_size, self.kernel_size),
            0
        )
        data.metadata['denoise_applied'] = True
        return ImageData(denoised, data.metadata)

    def configure(self, params: Dict[str, Any]) -> None:
        self.kernel_size = params.get('kernel_size', 5)
    \end{lstlisting}

    \vspace{0.3cm}
    \begin{itemize}
        \item 参数可配置:核大小可调
        \item 元数据记录:标记已应用去噪
        \item 纯函数设计:输入输出明确
    \end{itemize}
\end{frame}

\begin{frame}{二值化过滤器实现}
    \begin{lstlisting}
class BinarizeFilter(FilterInterface):
    """自适应二值化过滤器"""

    def __init__(self, method='adaptive', block_size=11):
        self.method = method
        self.block_size = block_size
        self.name = "BinarizeFilter"

    def get_name(self) -> str:
        return self.name

    def process(self, data: ImageData) -> ImageData:
        gray = cv2.cvtColor(data.image, cv2.COLOR_BGR2GRAY)

        if self.method == 'adaptive':
            binary = cv2.adaptiveThreshold(
                gray, 255, cv2.ADAPTIVE_THRESH_GAUSSIAN_C,
                cv2.THRESH_BINARY, self.block_size, 2
            )
        else:
            _, binary = cv2.threshold(gray, 127, 255, cv2.THRESH_BINARY)

        data.metadata['binarize_method'] = self.method
        return ImageData(binary, data.metadata)
    \end{lstlisting}
\end{frame}

\begin{frame}{透视矫正过滤器实现}
    \begin{lstlisting}
class PerspectiveCorrectionFilter(FilterInterface):
    """透视矫正过滤器"""

    def __init__(self, target_points=None):
        self.target_points = target_points
        self.name = "PerspectiveCorrectionFilter"

    def get_name(self) -> str:
        return self.name

    def process(self, data: ImageData) -> ImageData:
        # 从元数据获取角点
        corners = data.metadata.get('corners')
        if corners is None:
            return data  # 无需矫正

        # 计算透视变换矩阵
        h, w = data.image.shape[:2]
        target = np.float32([
            [0, 0], [w, 0], [w, h], [0, h]
        ])
        matrix = cv2.getPerspectiveTransform(corners, target)

        # 应用变换
        corrected = cv2.warpPerspective(
            data.image, matrix, (w, h)
        )
        return ImageData(corrected, data.metadata)
    \end{lstlisting}
\end{frame}

\begin{frame}{流水线执行引擎}
    \begin{lstlisting}
class ImagePipeline:
    """图像处理流水线"""

    def __init__(self, name: str = "default"):
        self.name = name
        self.filters: List[FilterInterface] = []
        self.intermediate_results: List[ImageData] = []

    def add_filter(self, filter_obj: FilterInterface) -> 'ImagePipeline':
        """添加过滤器(支持链式调用)"""
        self.filters.append(filter_obj)
        return self

    def execute(self, image: np.ndarray) -> ImageData:
        """执行流水线"""
        data = ImageData(image, {'pipeline': self.name})
        self.intermediate_results = [data]

        for filter_obj in self.filters:
            data = filter_obj.process(data)
            self.intermediate_results.append(data)

        return data
    \end{lstlisting}
\end{frame}

\begin{frame}{流水线执行与监控}
    \begin{lstlisting}
    def execute_with_monitoring(self, image: np.ndarray) -> Dict:
        """带监控的执行"""
        import time
        data = ImageData(image, {'pipeline': self.name})
        execution_log = []

        for filter_obj in self.filters:
            start_time = time.time()
            data = filter_obj.process(data)
            elapsed = time.time() - start_time

            execution_log.append({
                'filter': filter_obj.get_name(),
                'time_ms': elapsed * 1000,
                'image_shape': data.image.shape
            })

        return {
            'result': data,
            'execution_log': execution_log,
            'total_time_ms': sum(f['time_ms'] for f in execution_log)
        }
    \end{lstlisting}

    \vspace{0.3cm}
    \begin{itemize}
        \item 记录每个过滤器的执行时间
        \item 追踪图像尺寸变化
        \item 便于性能分析和优化
    \end{itemize}
\end{frame}

\begin{frame}{流水线使用示例}
    \begin{lstlisting}
# 构建预处理流水线
pipeline = ImagePipeline("exam_preprocess")

pipeline \
    .add_filter(DenoiseFilter(kernel_size=5)) \
    .add_filter(BinarizeFilter(method='adaptive')) \
    .add_filter(PerspectiveCorrectionFilter())

# 执行处理
result = pipeline.execute(image)
processed_image = result.image

# 带监控执行
monitored = pipeline.execute_with_monitoring(image)
print(f"总耗时: {monitored['total_time_ms']:.2f}ms")
for log in monitored['execution_log']:
    print(f"  {log['filter']}: {log['time_ms']:.2f}ms")
    \end{lstlisting}
\end{frame}

\begin{frame}{中间结果缓存}
    \begin{lstlisting}
class CachedPipeline(ImagePipeline):
    """带缓存的流水线"""

    def __init__(self, name: str = "cached", cache_dir: str = "./cache"):
        super().__init__(name)
        self.cache_dir = Path(cache_dir)
        self.cache_dir.mkdir(exist_ok=True)

    def _get_cache_key(self, image: np.ndarray, step: int) -> str:
        """生成缓存键"""
        import hashlib
        image_hash = hashlib.md5(image.tobytes()).hexdigest()
        return f"{self.name}_step{step}_{image_hash}.jpg"

    def execute(self, image: np.ndarray, use_cache: bool = True) -> ImageData:
        """支持缓存的执行"""
        # 检查缓存...
        # 命中则直接返回,否则执行并缓存
        \end{lstlisting}
\end{frame}

\begin{frame}{异常处理与恢复}
    \begin{lstlisting}
class RobustPipeline(ImagePipeline):
    """带异常处理的流水线"""

    def execute(self, image: np.ndarray) -> ImageData:
        data = ImageData(image, {'pipeline': self.name})

        for i, filter_obj in enumerate(self.filters):
            try:
                data = filter_obj.process(data)
            except Exception as e:
                print(f"过滤器 {filter_obj.get_name()} 失败: {e}")

                # 策略1: 跳过当前过滤器
                if self.skip_on_error:
                    continue

                # 策略2: 使用备用过滤器
                if i in self.fallback_filters:
                    fallback = self.fallback_filters[i]
                    data = fallback.process(data)

        return data
    \end{lstlisting}
\end{frame}

\begin{frame}{过滤器组合模式}
    \begin{lstlisting}
class CompositeFilter(FilterInterface):
    """组合过滤器:将多个过滤器视为一个"""

    def __init__(self, name: str):
        self.name = name
        self.filters: List[FilterInterface] = []

    def add(self, filter_obj: FilterInterface) -> 'CompositeFilter':
        self.filters.append(filter_obj)
        return self

    def process(self, data: ImageData) -> ImageData:
        for f in self.filters:
            data = f.process(data)
        return data

    def get_name(self) -> str:
        return self.name

# 使用:创建可复用的预处理组合
preprocess = CompositeFilter("preprocess") \
    .add(DenoiseFilter()) \
    .add(BinarizeFilter())
    \end{lstlisting}
\end{frame}

\begin{frame}{并行流水线}
    \begin{lstlisting}
class ParallelPipeline(ImagePipeline):
    """支持并行处理的流水线"""

    def __init__(self, name: str = "parallel", max_workers: int = 4):
        super().__init__(name)
        self.max_workers = max_workers

    def execute_parallel(self, images: List[np.ndarray]) -> List[ImageData]:
        """并行处理多张图像"""
        from concurrent.futures import ThreadPoolExecutor

        with ThreadPoolExecutor(max_workers=self.max_workers) as executor:
            results = list(executor.map(self.execute, images))

        return results

    def execute_filter_parallel(self, data: ImageData) -> ImageData:
        """过滤器内部并行(如分块处理)"""
        # 实现分块并行处理逻辑
        pass
    \end{lstlisting}
\end{frame}
