%===========================================================
% 07_patterns.tex - 架构设计模式与原则
%===========================================================

\section{分层架构模式}

\begin{frame}{分层架构的概念与优势}
    \begin{definition}[分层架构]
        将系统按职责划分为多个层次,每层只与相邻层交互,上层依赖下层,下层不依赖上层。
    \end{definition}

    \vspace{0.3cm}

    \begin{columns}
        \column{0.5\textwidth}
        \textbf{经典三层架构:}
        \begin{enumerate}
            \item \textbf{表现层}:UI、用户交互
            \item \textbf{业务层}:业务逻辑、规则
            \item \textbf{数据层}:数据访问、存储
        \end{enumerate}

        \column{0.5\textwidth}
        \begin{center}
            \begin{tikzpicture}[scale=0.8, transform shape,
                box/.style={draw, rectangle, minimum width=3cm, minimum height=0.8cm, align=center}]
                \node[box, fill=blue!20] (ui) at (0,0) {表现层\\(Presentation)};
                \node[box, fill=green!20] (biz) at (0,-1.2) {业务层\\(Business)};
                \node[box, fill=red!20] (data) at (0,-2.4) {数据层\\(Data)};

                \draw[->, thick] (ui) -- (biz);
                \draw[->, thick] (biz) -- (data);
            \end{tikzpicture}
        \end{center}
    \end{columns}

    \vspace{0.3cm}

    \textbf{分层架构的优势:}
    \begin{itemize}
        \item 关注点分离,职责清晰
        \item 易于测试(每层可独立测试)
        \item 便于维护(修改一层不影响其他层)
        \item 支持替换(如更换数据库)
    \end{itemize}
\end{frame}

\begin{frame}{层间通信与依赖管理}
    \textbf{依赖方向:}
    \begin{itemize}
        \item 依赖只能从上到下
        \item 上层调用下层接口
        \item 下层通过回调/事件通知上层
    \end{itemize}

    \vspace{0.3cm}

    \textbf{依赖倒置原则(DIP)应用:}
    \begin{lstlisting}
# 不好的做法:上层直接依赖下层实现
class GradeService:
    def __init__(self):
        self.db = MySQLDatabase()  # 直接依赖具体实现

# 好的做法:依赖抽象接口
class GradeService:
    def __init__(self, db: DatabaseInterface):
        self.db = db  # 依赖抽象接口
    \end{lstlisting}
\end{frame}

\begin{frame}{分层架构的适用场景}
    \begin{table}
        \centering
        \small
        \begin{tabular}{p{3cm}p{8cm}}
            \toprule
            \textbf{适用场景} \& \textbf{说明} \\
            \midrule
            企业应用 \& ERP、CRM等业务系统,业务逻辑复杂 \\
            Web应用 \& 前后端分离的Web系统 \\
            桌面应用 \& 具有复杂业务逻辑的单机软件 \\
            移动端应用 \& 具有本地业务处理的App \\
            \bottomrule
        \end{tabular}
    \end{table}

    \vspace{0.3cm}

    \begin{alertblock}{不适用场景}
        \begin{itemize}
            \item 简单脚本工具(过度设计)
            \item 高性能计算(层间开销)
            \item 事件驱动系统(更适合其他架构)
        \end{itemize}
    \end{alertblock}
\end{frame}

\section{微服务架构模式}

\begin{frame}{微服务的定义与特点}
    \begin{definition}[微服务架构]
        将应用拆分为一组小型、独立部署的服务,每个服务运行在自己的进程中,通过轻量级机制通信。
    \end{definition}

    \vspace{0.3cm}

    \textbf{微服务的核心特点:}
    \begin{columns}
        \column{0.5\textwidth}
        \begin{itemize}
            \item 单一职责
            \item 独立部署
            \item 技术异构
            \item 容错隔离
        \end{itemize}

        \column{0.5\textwidth}
        \begin{itemize}
            \item 去中心化治理
            \item 独立扩展
            \item 团队自治
            \item 自动化运维
        \end{itemize}
    \end{columns}

    \vspace{0.3cm}

    \begin{exampleblock}{阅卷系统微服务拆分}
        上传服务、预处理服务、识别服务、评分服务、报告服务
    \end{exampleblock}
\end{frame}

\begin{frame}{服务拆分策略}
    \textbf{拆分原则:}
    \begin{enumerate}
        \item \textbf{按业务能力拆分}:识别领域边界
        \item \textbf{按数据边界拆分}:数据独立管理
        \item \textbf{按团队边界拆分}:康威定律
    \end{enumerate}

    \vspace{0.3cm}

    \textbf{拆分粒度:}
    \begin{table}
        \centering
        \small
        \begin{tabular}{p{3cm}p{3cm}p{4.5cm}}
            \toprule
            \textbf{粒度} \& \textbf{服务数量} \& \textbf{适用场景} \\
            \midrule
            粗粒度 \& 5-10个 \& 中小型团队、业务稳定 \\
            中等粒度 \& 10-30个 \& 中大型团队、业务演进 \\
            细粒度 \& 30+个 \& 大型团队、平台化建设 \\
            \bottomrule
        \end{tabular}
    \end{table}
\end{frame}

\begin{frame}{服务间通信方式}
    \begin{columns}
        \column{0.5\textwidth}
        \textbf{同步通信:}
        \begin{itemize}
            \item \textbf{REST API}:HTTP + JSON,简单通用
            \item \textbf{gRPC}:高性能,基于HTTP/2
            \item \textbf{GraphQL}:灵活查询,减少往返
        \end{itemize}

        \vspace{0.3cm}

        \textbf{异步通信:}
        \begin{itemize}
            \item \textbf{消息队列}:RabbitMQ, Kafka
            \item \textbf{事件总线}:解耦服务
            \item \textbf{发布订阅}:一对多通知
        \end{itemize}

        \column{0.5\textwidth}
        \begin{block}{选择建议}
            \textbf{同步:}
            \begin{itemize}
                \item 实时性要求高
                \item 需要立即响应
                \item 调用链路短
            \end{itemize}

            \vspace{0.3cm}

            \textbf{异步:}
            \begin{itemize}
                \item 削峰填谷
                \item 最终一致性
                \item 解耦服务
            \end{itemize}
        \end{block}
    \end{columns}
\end{frame}

\section{管道-过滤器架构}

\begin{frame}{管道-过滤器模式}
    \begin{definition}[管道-过滤器架构]
        数据从数据源流入,经过一系列处理组件(过滤器)转换,最终输出到数据池。过滤器之间通过管道连接。
    \end{definition}

    \vspace{0.3cm}

    \begin{center}
        \begin{tikzpicture}[scale=0.8, transform shape,
            box/.style={draw, rectangle, rounded corners, fill=blue!15, minimum width=1.8cm, minimum height=1cm}]
            \node[box] (f1) at (0,0) {过滤器1};
            \node[box] (f2) at (3,0) {过滤器2};
            \node[box] (f3) at (6,0) {过滤器3};
            \node[box] (f4) at (9,0) {过滤器4};

            \draw[->, thick] (f1) -- (f2) node[midway, above] {管道};
            \draw[->, thick] (f2) -- (f3) node[midway, above] {管道};
            \draw[->, thick] (f3) -- (f4) node[midway, above] {管道};
        \end{tikzpicture}
    \end{center}

    \vspace{0.3cm}

    \textbf{特点:}
    \begin{itemize}
        \item 数据流驱动,单向流动
        \item 过滤器独立,可复用
        \item 易于扩展,新增过滤器
        \item 支持并行处理
    \end{itemize}
\end{frame}

\begin{frame}{过滤器组合与复用}
    \textbf{组合方式:}
    \begin{enumerate}
        \item \textbf{线性管道}:A → B → C
        \item \textbf{分支管道}:A → B/C → D
        \item \textbf{合并管道}:A/B → C
        \item \textbf{反馈管道}:输出反馈到输入
    \end{enumerate}

    \vspace{0.3cm}

    \begin{exampleblock}{图像处理流水线示例}
        \begin{center}
            \begin{tikzpicture}[scale=0.7, transform shape,
                box/.style={draw, rectangle, rounded corners, fill=green!15, minimum width=1.5cm, minimum height=0.8cm}]
                \node[box] (in) at (0,0) {输入图像};
                \node[box] (denoise) at (2.5,0) {去噪};
                \node[box] (binary) at (5,0) {二值化};
                \node[box] (correct) at (7.5,0) {矫正};
                \node[box] (out) at (10,0) {输出图像};

                \draw[->, thick] (in) -- (denoise);
                \draw[->, thick] (denoise) -- (binary);
                \draw[->, thick] (binary) -- (correct);
                \draw[->, thick] (correct) -- (out);
            \end{tikzpicture}
        \end{center}
    \end{exampleblock}
\end{frame}

\section{SOLID原则详解}

\begin{frame}{单一职责原则(SRP)}
    \begin{definition}[单一职责原则]
        一个类应该只有一个引起它变化的原因。换句话说,一个类应该只负责一件事。
    \end{definition}

    \vspace{0.3cm}

    \textbf{违反SRP的例子:}
    \begin{lstlisting}
class ExamProcessor:
    def load_image(self, path): ...  # 图像加载
    def preprocess(self, img): ...   # 预处理
    def recognize(self, img): ...    # 识别
    def save_result(self, result): ... # 保存结果
    \end{lstlisting}

    \vspace{0.3cm}

    \textbf{改进方案:}
    \begin{lstlisting}
class ImageLoader: ...      # 只负责加载
class Preprocessor: ...     # 只负责预处理
class Recognizer: ...       # 只负责识别
class ResultSaver: ...      # 只负责保存
    \end{lstlisting}
\end{frame}

\begin{frame}{开闭原则(OCP)}
    \begin{definition}[开闭原则]
        软件实体应该对扩展开放,对修改关闭。
    \end{definition}

    \vspace{0.3cm}

    \textbf{实现方式:}
    \begin{itemize}
        \item 抽象基类 + 具体实现
        \item 策略模式
        \item 依赖注入
    \end{itemize}

    \vspace{0.3cm}

    \begin{lstlisting}
class RecognizerInterface(ABC):
    @abstractmethod
    def recognize(self, image): pass

class ChoiceRecognizer(RecognizerInterface): ...
class JudgeRecognizer(RecognizerInterface): ...
class EssayRecognizer(RecognizerInterface): ...

# 新增识别器时,不需要修改原有代码
class NewRecognizer(RecognizerInterface): ...
    \end{lstlisting}
\end{frame}

\begin{frame}{里氏替换原则(LSP)}
    \begin{definition}[里氏替换原则]
        子类型必须能够替换其基类型,而程序的行为不会改变。
    \end{definition}

    \vspace{0.3cm}

    \textbf{关键要求:}
    \begin{itemize}
        \item 子类不改变父类的前置条件(输入)
        \item 子类不强化父类的后置条件(输出)
        \item 子类保持父类的不变性
    \end{itemize}

    \vspace{0.3cm}

    \begin{exampleblock}{正确示例}
        \begin{lstlisting}
class Recognizer:
    def recognize(self, img) -> Result:
        return Result(confidence=0.0, text="")

class OcrRecognizer(Recognizer):
    def recognize(self, img) -> Result:
        # 返回更精确的结果,但类型相同
        return Result(confidence=0.95, text="答案")
        \end{lstlisting}
    \end{exampleblock}
\end{frame}

\begin{frame}{接口隔离原则(ISP)}
    \begin{definition}[接口隔离原则]
        客户端不应该被迫依赖它不使用的方法。应该将大接口拆分为小接口。
    \end{definition}

    \vspace{0.3cm}

    \textbf{不好的设计:}
    \begin{lstlisting}
class BigInterface(ABC):  # 大接口
    @abstractmethod
    def preprocess(self, img): pass
    @abstractmethod
    def recognize(self, img): pass
    @abstractmethod
    def grade(self, answers): pass
    \end{lstlisting}

    \textbf{改进设计:}
    \begin{lstlisting}
class PreprocessorInterface(ABC): ...  # 小接口1
class RecognizerInterface(ABC): ...    # 小接口2
class GraderInterface(ABC): ...       # 小接口3

# 类按需实现接口
class ImageProcessor(PreprocessorInterface): ...
    \end{lstlisting}
\end{frame}

\begin{frame}{依赖倒置原则(DIP)}
    \begin{definition}[依赖倒置原则]
        高层模块不应该依赖低层模块,两者都应该依赖抽象。抽象不应该依赖细节,细节应该依赖抽象。
    \end{definition}

    \vspace{0.3cm}

    \textbf{传统方式(违反DIP):}
    \begin{lstlisting}
class GradeService:  # 高层
    def __init__(self):
        self.db = MySQLDatabase()  # 依赖具体低层
    \end{lstlisting}

    \textbf{改进方式(遵循DIP):}
    \begin{lstlisting}
class DatabaseInterface(ABC): ...  # 抽象

class MySQLDatabase(DatabaseInterface): ...
class MongoDB(DatabaseInterface): ...

class GradeService:  # 高层依赖抽象
    def __init__(self, db: DatabaseInterface):
        self.db = db
    \end{lstlisting}
\end{frame}

\section{设计模式}

\begin{frame}{创建型模式}
    \begin{table}
        \centering
        \small
        \begin{tabular}{p{2.5cm}p{4cm}p{4cm}}
            \toprule
            \textbf{模式} \& \textbf{用途} \& \textbf{应用场景} \\
            \midrule
            工厂模式 \& 创建对象,解耦创建与使用 \& 根据条件创建不同类型对象 \\
            单例模式 \& 确保只有一个实例 \& 数据库连接、配置管理 \\
            建造者模式 \& 分步骤构建复杂对象 \& 复杂配置对象构建 \\
            \bottomrule
        \end{tabular}
    \end{table}

    \vspace{0.3cm}

    \begin{lstlisting}
# 工厂模式示例
class RecognizerFactory:
    @staticmethod
    def create_recognizer(type_name):
        if type_name == "choice":
            return ChoiceRecognizer()
        elif type_name == "judge":
            return JudgeRecognizer()
        # ...

# 使用
recognizer = RecognizerFactory.create_recognizer("choice")
    \end{lstlisting}
\end{frame}

\begin{frame}{结构型模式}
    \begin{table}
        \centering
        \small
        \begin{tabular}{p{2.5cm}p{4cm}p{4cm}}
            \toprule
            \textbf{模式} \& \textbf{用途} \& \textbf{应用场景} \\
            \midrule
            适配器模式 \& 接口转换,兼容旧代码 \& 集成第三方库 \\
            装饰器模式 \& 动态添加功能 \& 日志、缓存、权限检查 \\
            代理模式 \& 控制对象访问 \& 延迟加载、访问控制 \\
            \bottomrule
        \end{tabular}
    \end{table}

    \vspace{0.3cm}

    \begin{lstlisting}
# 装饰器模式示例
class RecognizerDecorator:
    def __init__(self, recognizer):
        self._recognizer = recognizer

    def recognize(self, image):
        print(f"开始识别: {time.time()}")  # 添加日志
        result = self._recognizer.recognize(image)
        print(f"识别完成: {time.time()}")
        return result
    \end{lstlisting}
\end{frame}

\begin{frame}{行为型模式}
    \begin{table}
        \centering
        \small
        \begin{tabular}{p{2.5cm}p{4cm}p{4cm}}
            \toprule
            \textbf{模式} \& \textbf{用途} \& \textbf{应用场景} \\
            \midrule
            观察者模式 \& 状态变化通知 \& 事件监听、消息订阅 \\
            策略模式 \& 算法互换 \& 不同识别算法切换 \\
            模板方法 \& 定义算法骨架 \& 通用处理流程 \\
            \bottomrule
        \end{tabular}
    \end{table}

    \vspace{0.3cm}

    \begin{lstlisting}
# 策略模式示例
class RecognitionStrategy(ABC):
    @abstractmethod
    def recognize(self, image): pass

class TemplateMatchStrategy(RecognitionStrategy): ...
class DeepLearningStrategy(RecognitionStrategy): ...

# 运行时切换策略
recognizer.set_strategy(TemplateMatchStrategy())
    \end{lstlisting}
\end{frame}
