%===========================================================
% 11_cases.tex - 案例分析与互动
%===========================================================

\section{真实系统架构案例}

\begin{frame}{案例1:小型阅卷系统架构}
    \textbf{场景:}小型培训机构,单次阅卷<1000份

    \vspace{0.3cm}

    \begin{center}
        \begin{tikzpicture}[scale=0.7, transform shape,
            box/.style={draw, rectangle, rounded corners, fill=blue!15, minimum width=2.5cm, minimum height=0.8cm, align=center}]

            \node[box, fill=green!20] (ui) at (0,0) {Web界面\\Flask};
            \node[box, fill=yellow!20] (core) at (4,0) {处理核心\\OpenCV+OCR};
            \node[box, fill=red!20] (db) at (8,0) {SQLite\\本地存储};

            \draw[->, thick] (ui) -- (core) node[midway, above] {HTTP};
            \draw[->, thick] (core) -- (db) node[midway, above] {SQL};
        \end{tikzpicture}
    \end{center}

    \vspace{0.3cm}

    \textbf{架构特点:}
    \begin{itemize}
        \item 单体架构,单进程部署
        \item 使用Flask轻量Web框架
        \item SQLite零配置数据库
        \item 单机处理,无需分布式
    \end{itemize}

    \begin{exampleblock}{适用场景}
        团队小(2-3人)、预算有限、处理量小、快速上线
    \end{exampleblock}
\end{frame}

\begin{frame}{案例2:企业级阅卷系统架构}
    \textbf{场景:}大型教育机构,日均阅卷10万+

    \vspace{0.2cm}

    \begin{center}
        \begin{tikzpicture}[scale=0.6, transform shape,
            box/.style={draw, rectangle, rounded corners, fill=blue!15, minimum width=2cm, minimum height=0.6cm, align=center}]

            % 接入层
            \node[box, fill=cyan!20] (lb) at (0,0) {负载均衡\\Nginx};

            % 应用层
            \node[box, fill=green!20] (api1) at (-3,-1.5) {API服务1};
            \node[box, fill=green!20] (api2) at (-1,-1.5) {API服务2};
            \node[box, fill=green!20] (api3) at (1,-1.5) {API服务3};

            % 处理层
            \node[box, fill=yellow!20] (worker1) at (-3,-3) {处理节点1};
            \node[box, fill=yellow!20] (worker2) at (0,-3) {处理节点2};
            \node[box, fill=yellow!20] (worker3) at (3,-3) {处理节点3};

            % 存储层
            \node[box, fill=red!20] (db) at (-2,-4.5) {MySQL集群};
            \node[box, fill=orange!20] (cache) at (2,-4.5) {Redis缓存};
            \node[box, fill=purple!20] (oss) at (0,-5.5) {对象存储\\OSS};

            % 连接
            \draw[->] (lb) -- (api1);
            \draw[->] (lb) -- (api2);
            \draw[->] (lb) -- (api3);
            \draw[->] (api1) -- (worker1);
            \draw[->] (api2) -- (worker2);
            \draw[->] (api3) -- (worker3);
            \draw[->] (worker1) -- (db);
            \draw[->] (worker2) -- (cache);
            \draw[->] (worker3) -- (oss);
        \end{tikzpicture}
    \end{center}

    \textbf{架构特点:}微服务、水平扩展、读写分离、多级缓存
\end{frame}

\begin{frame}{案例3:云端SaaS阅卷服务}
    \textbf{场景:}多租户SaaS平台,学校按需使用

    \vspace{0.3cm}

    \begin{center}
        \begin{tikzpicture}[scale=0.6, transform shape,
            box/.style={draw, rectangle, rounded corners, fill=blue!15, minimum width=2.2cm, minimum height=0.6cm, align=center}]

            % 租户层
            \node[box, fill=green!15] (school1) at (-4,0) {学校A};
            \node[box, fill=green!15] (school2) at (-1,0) {学校B};
            \node[box, fill=green!15] (school3) at (2,0) {学校C};

            % 网关层
            \node[box, fill=cyan!20] (gateway) at (-1,-1.5) {API网关\\(鉴权/限流)};

            % 服务层
            \node[box, fill=yellow!20] (upload) at (-4,-3) {上传服务};
            \node[box, fill=yellow!20] (process) at (-1,-3) {处理服务};
            \node[box, fill=yellow!20] (report) at (2,-3) {报告服务};

            % 数据隔离
            \node[box, fill=red!20] (tenantdb) at (-1,-4.5) {租户数据隔离};

            \draw[->] (school1) -- (gateway);
            \draw[->] (gateway) -- (upload);
            \draw[->] (gateway) -- (process);
            \draw[->] (gateway) -- (report);
            \draw[->] (upload) -- (tenantdb);
        \end{tikzpicture}
    \end{center}

    \textbf{关键设计:}多租户隔离、按需计费、弹性伸缩、数据安全
\end{frame}

\begin{frame}{架构对比分析}
    \begin{table}
        \centering
        \scriptsize
        \begin{tabular}{p{1.8cm}p{2.8cm}p{2.8cm}p{2.8cm}}
            \toprule
            \textbf{维度} & \textbf{小型系统} & \textbf{企业级系统} & \textbf{SaaS平台} \\
            \midrule
            架构风格 & 单体 & 微服务 & 多租户微服务 \\
            部署方式 & 单机 & 私有云/机房 & 公有云 \\
            扩展能力 & 垂直扩展 & 水平扩展 & 弹性伸缩 \\
            数据存储 & SQLite & MySQL集群 & 分布式数据库 \\
            运维复杂度 & 低 & 高 & 很高 \\
            开发团队 & 2-3人 & 10-20人 & 50+人 \\
            成本 & 低 & 高 & 按需付费 \\
            适用场景 & 小机构 & 大机构自用 & 对外服务 \\
            \bottomrule
        \end{tabular}
    \end{table}

    \vspace{0.3cm}

    \begin{alertblock}{选型建议}
        根据团队规模、预算、业务量、技术能力选择合适架构。避免过度设计,也避免架构不足。
    \end{alertblock}
\end{frame}

\section{行业最佳实践}

\begin{frame}{微服务架构实践要点}
    \begin{enumerate}
        \item \textbf{服务拆分粒度}
        \begin{itemize}
            \item 按业务能力拆分,而非技术层次
            \item 服务数量适中(建议5-15个)
            \item 避免过细拆分导致运维噩梦
        \end{itemize}

        \item \textbf{数据管理}
        \begin{itemize}
            \item 每个服务独立数据库
            \item 避免直接访问其他服务数据库
            \item 通过API进行数据交互
        \end{itemize}

        \item \textbf{服务通信}
        \begin{itemize}
            \item 同步调用链路要短(<3层)
            \item 优先使用异步消息解耦
            \item 实现熔断降级机制
        \end{itemize}
    \end{enumerate}
\end{frame}

\begin{frame}{领域驱动设计(DDD)实践}
    \textbf{核心概念:}
    \begin{itemize}
        \item \textbf{限界上下文}:明确领域边界(如:识别域、评分域)
        \item \textbf{实体}:有唯一标识的对象(如:试卷、题目)
        \item \textbf{值对象}:无标识的描述属性(如:坐标、分数)
        \item \textbf{聚合}:一致性边界内的实体和值对象组合
        \item \textbf{领域服务}:跨实体的业务逻辑
    \end{itemize}

    \vspace{0.3cm}

    \begin{exampleblock}{阅卷系统DDD应用}
        \begin{itemize}
            \item 识别上下文:选择题识别、判断题识别、简答题识别
            \item 评分上下文:答案比对、分数计算、成绩汇总
            \item 考试上下文:考试创建、试卷分配、结果发布
        \end{itemize}
    \end{exampleblock}
\end{frame}

\begin{frame}{DevOps与CI/CD实践}
    \textbf{CI/CD流水线:}

    \begin{center}
        \begin{tikzpicture}[scale=0.7, transform shape,
            box/.style={draw, rectangle, rounded corners, fill=blue!15, minimum width=2cm, minimum height=0.7cm, align=center}]

            \node[box, fill=green!20] (code) at (0,0) {代码提交};
            \node[box] (build) at (2.5,0) {构建};
            \node[box] (test) at (5,0) {测试};
            \node[box] (deploy) at (7.5,0) {部署};
            \node[box, fill=red!20] (monitor) at (10,0) {监控};

            \draw[->, thick] (code) -- (build);
            \draw[->, thick] (build) -- (test);
            \draw[->, thick] (test) -- (deploy);
            \draw[->, thick] (deploy) -- (monitor);
        \end{tikzpicture}
    \end{center}

    \vspace{0.3cm}

    \textbf{实践要点:}
    \begin{itemize}
        \item 自动化测试:单元测试、集成测试、E2E测试
        \item 自动化构建:Docker镜像、依赖管理
        \item 自动化部署:蓝绿部署、金丝雀发布
        \item 监控告警:日志收集、性能监控、异常告警
    \end{itemize}
\end{frame}

\section{课堂互动与Quiz}

\begin{frame}{架构设计Quiz}
    \textbf{问题1:}以下哪种场景适合使用微服务架构?
    \begin{enumerate}
        \item[A] 个人博客系统
        \item[B] 团队2人、预算有限的学生项目
        \item[C] 50人团队、日均处理百万请求的教育平台
        \item[D] 内部使用的定时任务脚本
    \end{enumerate}

    \vspace{0.3cm}

    \textbf{问题2:}关于分层架构,以下说法正确的是?
    \begin{enumerate}
        \item[A] 层与层之间可以随意依赖
        \item[B] 上层依赖下层,下层不应依赖上层
        \item[C] 表现层应该直接访问数据库
        \item[D] 分层越多越好,至少分10层
    \end{enumerate}

    \vspace{0.3cm}

    \textbf{问题3:}下列哪个设计模式最适合实现多种识别算法的动态切换?
    \begin{enumerate}
        \item[A] 单例模式
        \item[B] 工厂模式
        \item[C] 策略模式
        \item[D] 观察者模式
    \end{enumerate}
\end{frame}

\begin{frame}{设计模式应用场景挑战}
    \textbf{场景分析:为以下场景选择合适的设计模式}

    \vspace{0.3cm}

    \begin{table}
        \centering
        \small
        \begin{tabular}{p{5cm}p{4cm}}
            \toprule
            \textbf{场景描述} & \textbf{建议模式} \\
            \midrule
            系统需要根据不同的题型创建对应的识别器 & 工厂模式 \\
            需要在不修改现有识别器的情况下添加日志功能 & 装饰器模式 \\
            多个识别器完成后需要通知评分模块 & 观察者模式 \\
            需要确保配置管理器只有一个实例 & 单例模式 \\
            需要为不同的阅卷规则定义统一的处理流程 & 模板方法模式 \\
            需要统一不同OCR库的接口 & 适配器模式 \\
            \bottomrule
        \end{tabular}
    \end{table}
\end{frame}

\begin{frame}{案例分析:架构问题诊断}
    \textbf{案例:}某团队开发的阅卷系统出现以下问题,请诊断:

    \vspace{0.3cm}

    \begin{block}{问题描述}
        \begin{enumerate}
            \item 修改选择题识别逻辑时,判断题识别也出bug
            \item 新增OCR引擎时,需要修改10多个文件
            \item 代码中充斥着复制粘贴的相似逻辑
            \item 无法单独测试某个识别模块,必须完整运行
        \end{enumerate}
    \end{block}

    \vspace{0.3cm}

    \textbf{诊断结果:}
    \begin{itemize}
        \item 问题1:违反单一职责,模块耦合严重
        \item 问题2:违反开闭原则,扩展困难
        \item 问题3:违反DRY原则,代码重复
        \item 问题4:缺乏接口抽象,无法单元测试
    \end{itemize}
\end{frame}
