%===========================================================
% 09_engine.tex - 模块化识别引擎实战
%===========================================================

\section{识别引擎架构设计}

\begin{frame}{识别引擎整体架构}
    \begin{center}
        \begin{tikzpicture}[scale=0.7, transform shape,
            box/.style={draw, rectangle, rounded corners, fill=blue!15, minimum width=2.5cm, minimum height=1cm, align=center},
            smallbox/.style={draw, rectangle, rounded corners, fill=green!15, minimum width=2cm, minimum height=0.7cm}]

            % 引擎核心
            \node[box, fill=yellow!30, minimum width=4cm] (engine) at (0,0) {识别引擎\\RecognitionEngine};

            % 算法插件
            \node[smallbox] (choice) at (-4,-2) {选择题\\识别器};
            \node[smallbox] (judge) at (-1.5,-2) {判断题\\识别器};
            \node[smallbox] (essay) at (1.5,-2) {简答题\\识别器};
            \node[smallbox] (new) at (4,-2) {新算法\\插件};

            % 输入输出
            \node[box, fill=green!20] (input) at (-4,2) {输入图像\\+ 区域信息};
            \node[box, fill=red!20] (output) at (4,2) {识别结果\\+ 置信度};

            % 连接
            \draw[->, thick] (input) -- (engine);
            \draw[->, thick] (engine) -- (output);
            \draw[<->, thick] (engine) -- (choice);
            \draw[<->, thick] (engine) -- (judge);
            \draw[<->, thick] (engine) -- (essay);
            \draw[<->, thick, dashed] (engine) -- (new);
        \end{tikzpicture}
    \end{center}

    \vspace{0.3cm}

    \textbf{核心特性:}
    \begin{itemize}
        \item 插件化架构:算法可动态注册
        \item 策略选择:根据题型自动选择算法
        \item 结果融合:多算法结果融合提高准确率
        \item 置信度评估:输出可信度指标
    \end{itemize}
\end{frame}

\begin{frame}{引擎接口设计}
    \begin{lstlisting}
from abc import ABC, abstractmethod
from dataclasses import dataclass
from typing import List, Dict, Optional
from enum import Enum

class QuestionType(Enum):
    CHOICE = "choice"
    JUDGE = "judge"
    ESSAY = "essay"

@dataclass
class RecognitionResult:
    """识别结果"""
    text: str
    confidence: float
    question_type: QuestionType
    raw_data: Dict = None

class RecognizerInterface(ABC):
    """识别器接口"""

    @abstractmethod
    def recognize(self, image: np.ndarray,
                  roi: Optional[tuple] = None) -> RecognitionResult:
        pass

    @abstractmethod
    def get_supported_types(self) -> List[QuestionType]:
        pass

    @abstractmethod
    def get_name(self) -> str:
        pass
    \end{lstlisting}
\end{frame}

\begin{frame}{识别引擎核心实现}
    \begin{lstlisting}
class RecognitionEngine:
    """模块化识别引擎"""

    def __init__(self):
        self._recognizers: Dict[QuestionType, List[RecognizerInterface]] = {
            qt: [] for qt in QuestionType
        }
        self._strategy = "best_confidence"

    def register(self, recognizer: RecognizerInterface) -> 'RecognitionEngine':
        """注册识别器"""
        for qtype in recognizer.get_supported_types():
            self._recognizers[qtype].append(recognizer)
        return self

    def recognize(self, image: np.ndarray,
                  question_type: QuestionType,
                  roi: Optional[tuple] = None) -> RecognitionResult:
        """执行识别"""
        recognizers = self._recognizers.get(question_type, [])

        if not recognizers:
            raise ValueError(f"没有支持 {question_type} 的识别器")

        # 收集所有识别器的结果
        results = [r.recognize(image, roi) for r in recognizers]

        # 根据策略选择最佳结果
        return self._select_best_result(results)
    \end{lstlisting}
\end{frame}

\begin{frame}{选择题识别器实现}
    \begin{lstlisting}
class ChoiceRecognizer(RecognizerInterface):
    """选择题填涂识别器"""

    def __init__(self, threshold: float = 0.5):
        self.threshold = threshold
        self.name = "ChoiceRecognizer"

    def get_name(self) -> str:
        return self.name

    def get_supported_types(self) -> List[QuestionType]:
        return [QuestionType.CHOICE]

    def recognize(self, image: np.ndarray,
                  roi: Optional[tuple] = None) -> RecognitionResult:
        """识别选择题答案"""
        if roi:
            x, y, w, h = roi
            image = image[y:y+h, x:x+w]

        # 分析选项区域
        options = self._detect_options(image)
        filled = self._find_filled_options(options)

        answer = ",".join(filled) if filled else "未作答"
        confidence = self._calculate_confidence(filled, options)

        return RecognitionResult(
            text=answer,
            confidence=confidence,
            question_type=QuestionType.CHOICE
        )
    \end{lstlisting}
\end{frame}

\begin{frame}{判断题识别器实现}
    \begin{lstlisting}
class JudgeRecognizer(RecognizerInterface):
    """判断题符号识别器"""

    def __init__(self):
        self.name = "JudgeRecognizer"
        # 预定义符号模板
        self.templates = {
            'T': self._load_template('true.png'),
            'F': self._load_template('false.png')
        }

    def recognize(self, image: np.ndarray,
                  roi: Optional[tuple] = None) -> RecognitionResult:
        """识别判断题符号"""
        if roi:
            x, y, w, h = roi
            image = image[y:y+h, x:x+w]

        # 形状分析
        symbol = self._analyze_symbol(image)

        # 匹配模板
        best_match, score = self._match_template(symbol)

        return RecognitionResult(
            text=best_match,
            confidence=score,
            question_type=QuestionType.JUDGE
        )

    def _analyze_symbol(self, image: np.ndarray) -> str:
        # 提取轮廓特征
        # 判断是√、×、T、F等
        pass
    \end{lstlisting}
\end{frame}

\begin{frame}{简答题识别器实现}
    \begin{lstlisting}
class EssayRecognizer(RecognizerInterface):
    """简答题OCR识别器"""

    def __init__(self, use_gpu: bool = False):
        self.name = "EssayRecognizer"
        self.use_gpu = use_gpu
        # 初始化OCR引擎(如PaddleOCR)
        self.ocr = self._init_ocr()

    def _init_ocr(self):
        from paddleocr import PaddleOCR
        return PaddleOCR(
            use_angle_cls=True,
            lang='ch',
            use_gpu=self.use_gpu
        )

    def recognize(self, image: np.ndarray,
                  roi: Optional[tuple] = None) -> RecognitionResult:
        """OCR识别手写文字"""
        if roi:
            x, y, w, h = roi
            image = image[y:y+h, x:x+w]

        # 执行OCR
        result = self.ocr.ocr(image, cls=True)

        # 提取文字和置信度
        text = self._extract_text(result)
        confidence = self._average_confidence(result)

        return RecognitionResult(
            text=text,
            confidence=confidence,
            question_type=QuestionType.ESSAY,
            raw_data={'ocr_result': result}
        )
    \end{lstlisting}
\end{frame}

\begin{frame}{算法注册与发现机制}
    \begin{lstlisting}
class RecognizerRegistry:
    """识别器注册表"""

    def __init__(self):
        self._recognizers: Dict[str, Type[RecognizerInterface]] = {}

    def register(self, name: str,
                 recognizer_class: Type[RecognizerInterface]) -> None:
        """注册识别器类"""
        self._recognizers[name] = recognizer_class

    def create(self, name: str, **kwargs) -> RecognizerInterface:
        """创建识别器实例"""
        if name not in self._recognizers:
            raise ValueError(f"未知的识别器: {name}")
        return self._recognizers[name](**kwargs)

    def list_recognizers(self) -> List[str]:
        """列出所有可用识别器"""
        return list(self._recognizers.keys())

# 全局注册表
registry = RecognizerRegistry()

# 注册内置识别器
registry.register("choice", ChoiceRecognizer)
registry.register("judge", JudgeRecognizer)
registry.register("essay", EssayRecognizer)
    \end{lstlisting}
\end{frame}

\begin{frame}{算法选择策略}
    \begin{lstlisting}
class SelectionStrategy(ABC):
    """算法选择策略接口"""

    @abstractmethod
    def select(self, results: List[RecognitionResult]) -> RecognitionResult:
        pass

class BestConfidenceStrategy(SelectionStrategy):
    """选择置信度最高的结果"""

    def select(self, results: List[RecognitionResult]) -> RecognitionResult:
        return max(results, key=lambda r: r.confidence)

class VotingStrategy(SelectionStrategy):
    """多数投票策略"""

    def select(self, results: List[RecognitionResult]) -> RecognitionResult:
        from collections import Counter
        texts = [r.text for r in results]
        winner = Counter(texts).most_common(1)[0][0]

        # 取所有相同结果的平均置信度
        winners = [r for r in results if r.text == winner]
        avg_conf = sum(r.confidence for r in winners) / len(winners)

        return RecognitionResult(
            text=winner,
            confidence=avg_conf,
            question_type=winners[0].question_type
        )
    \end{lstlisting}
\end{frame}

\begin{frame}{结果融合与置信度评估}
    \begin{lstlisting}
class ResultFusion:
    """结果融合器"""

    @staticmethod
    def weighted_fusion(results: List[RecognitionResult],
                        weights: List[float]) -> RecognitionResult:
        """加权融合多个结果"""
        # 加权平均置信度
        total_weight = sum(weights)
        weighted_conf = sum(
            r.confidence * w for r, w in zip(results, weights)
        ) / total_weight

        # 选择加权后置信度最高的文本
        best = max(results, key=lambda r: r.confidence)

        return RecognitionResult(
            text=best.text,
            confidence=weighted_conf,
            question_type=best.question_type
        )

    @staticmethod
    def calculate_overall_confidence(results: List[RecognitionResult]) -> float:
        """计算整体置信度"""
        if not results:
            return 0.0
        # 使用几何平均
        import math
        product = math.prod(r.confidence for r in results)
        return product ** (1 / len(results))
    \end{lstlisting}
\end{frame}

\begin{frame}{置信度阈值与降级策略}
    \begin{lstlisting}
class ConfidenceBasedFallback:
    """基于置信度的降级策略"""

    def __init__(self, engine: RecognitionEngine,
                 threshold: float = 0.8):
        self.engine = engine
        self.threshold = threshold

    def recognize_with_fallback(self, image: np.ndarray,
                                 question_type: QuestionType) -> Dict:
        """带降级的识别"""
        result = self.engine.recognize(image, question_type)

        response = {
            'result': result,
            'confidence': result.confidence,
            'needs_review': False,
            'fallback_used': False
        }

        if result.confidence < self.threshold:
            response['needs_review'] = True

            # 尝试备用算法
            if question_type == QuestionType.ESSAY:
                # 切换到更精确的模型
                fallback_result = self._use_precise_model(image)
                if fallback_result.confidence > result.confidence:
                    response['result'] = fallback_result
                    response['fallback_used'] = True

        return response
    \end{lstlisting}
\end{frame}

\begin{frame}{引擎使用完整示例}
    \begin{lstlisting}
# 1. 创建引擎
engine = RecognitionEngine()

# 2. 注册识别器
engine \
    .register(ChoiceRecognizer(threshold=0.5)) \
    .register(JudgeRecognizer()) \
    .register(EssayRecognizer(use_gpu=True))

# 3. 设置选择策略
engine.set_strategy("best_confidence")

# 4. 执行识别
result = engine.recognize(
    image=exam_image,
    question_type=QuestionType.CHOICE,
    roi=(100, 200, 300, 400)  # 指定区域
)

print(f"识别结果: {result.text}")
print(f"置信度: {result.confidence:.2%}")
    \end{lstlisting}
\end{frame}

\begin{frame}{性能监控与优化}
    \begin{lstlisting}
class MonitoredEngine(RecognitionEngine):
    """带性能监控的识别引擎"""

    def __init__(self):
        super().__init__()
        self.metrics = {
            'total_calls': 0,
            'total_time': 0.0,
            'by_type': {}
        }

    def recognize(self, image: np.ndarray,
                  question_type: QuestionType,
                  roi: Optional[tuple] = None) -> RecognitionResult:
        import time

        start = time.time()
        result = super().recognize(image, question_type, roi)
        elapsed = time.time() - start

        # 记录指标
        self.metrics['total_calls'] += 1
        self.metrics['total_time'] += elapsed

        if question_type not in self.metrics['by_type']:
            self.metrics['by_type'][question_type] = {
                'count': 0, 'time': 0.0
            }
        self.metrics['by_type'][question_type]['count'] += 1
        self.metrics['by_type'][question_type]['time'] += elapsed

        return result
    \end{lstlisting}
\end{frame}
