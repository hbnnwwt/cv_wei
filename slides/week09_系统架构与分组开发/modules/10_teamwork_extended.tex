%===========================================================
% 10_teamwork_extended.tex - 团队协作开发实战(扩充版)
%===========================================================

\section{AI辅助编程实战}

\begin{frame}{AI编程工具推荐}
    \textbf{课程推荐工具(按优先级):}
    \begin{columns}
        \column{0.33\textwidth}
        \begin{block}{Cursor - 首选}
            \begin{itemize}
                \item AI原生IDE
                \item Ctrl+K 原地编辑
                \item Ctrl+L 上下文对话
                \item 最适合初学者
            \end{itemize}
        \end{block}

        \column{0.33\textwidth}
        \begin{block}{Claude Code}
            \begin{itemize}
                \item 命令行AI助手
                \item 适合高级用户
                \item 强大的代码理解
            \end{itemize}
        \end{block}

        \column{0.33\textwidth}
        \begin{block}{通义灵码}
            \begin{itemize}
                \item JetBrains插件
                \item 国内访问友好
                \item 免费使用
            \end{itemize}
        \end{block}
    \end{columns}
\end{frame}

\begin{frame}{Cursor实战演示}
    \textbf{Cursor核心快捷键:}
    \begin{center}
        \begin{tikzpicture}[scale=0.9, transform shape,
            key/.style={draw, rectangle, rounded corners, fill=blue!15, minimum width=3cm, minimum height=0.6cm, align=center},
            arrow/.style={->, thick}]

            \node[key] (ctrlk) at (0,0) {\texttt{Ctrl+K} - 原地编辑代码};
            \node[key] (ctrlkdesc) at (4,0) {选中代码 → 按Ctrl+K → 输入指令};

            \node[key] (ctrlk2) at (0,-1) {\texttt{Ctrl+L} - 上下文对话};
            \node[key] (ctrldesc) at (4,-1) {询问代码逻辑、解释错误原因};

            \node[key] (ctrlk3) at (0,-2) {\texttt{Ctrl+I} - 生成新代码};
            \node[key] (ctrldesc3) at (4,-2) {描述需求 → 自动生成代码框架};
        \end{tikzpicture}
    \end{center}
\end{frame}

\begin{frame}{AI Prompt模板:RTF框架}
    \textbf{RTF框架示例 - 生成接口代码:}
    \begin{block}{Role-Task-Format}
        \begin{itemize}
            \item \textbf{Role}: 你是一个Python架构师
            \item \textbf{Task}: 设计图像预处理流水线的Filter接口
            \item \textbf{Format}: 包含类定义、方法签名、类型注解、注释
        \end{itemize}
    \end{block}

    \textbf{思维链引导示例:}
    \texttt{请一步步思考:首先确定输入输出数据结构,然后定义过滤器接口,最后实现流水线引擎}

    \textbf{少样本提示示例:}
    \texttt{请参考以下接口设计,为二值化过滤器创建类似的接口:\\
    \indent class DenoiseFilter:\\
    \indent \indent def process(self, data: ImageData) -> ImageData:}
\end{frame}

\begin{frame}{AI辅助调试三部曲}
    \textbf{遇到错误时,这样问AI:}

    \begin{enumerate}
        \item \textbf{Traceback复制}:粘贴完整的错误信息
        \item \textbf{环境说明}:Python版本、使用的库
        \item \textbf{数据贴出}:关键变量值或输入数据
    \end{enumerate}

    \begin{block}{示例Prompt}
        我遇到了这个错误:\texttt{TypeError: 'NoneType' object...}\\
        环境:Python 3.10, OpenCV 4.8\\
        代码在第56行,请帮我分析原因并修复
    \end{block}
\end{frame}

\section{调试工具函数}

\begin{frame}{调试工具函数:save\_debug\_image}
    \begin{block}{utils/debug.py}
        \lstinputlisting[
            language=Python,
            basicstyle=\ttfamily\scriptsize,
            numbers=left,
            frame=single
        ]{modules/05_development.tex}
    \end{block}
\end{frame}

\begin{frame}{调试工具函数:draw\_debug\_boxes}
    \begin{block}{绘制调试框}
        \lstinputlisting[
            language=Python,
            firstline=30,
            lastline=46,
            basicstyle=\ttfamily\scriptsize,
            numbers=left,
            frame=single
        ]{modules/05_development.tex}
    \end{block}

    \textbf{使用示例:}
    \begin{itemize}
        \item 检测到答题框后,用 \texttt{draw\_debug\_boxes} 可视化
        \item 调试时保存中间结果到 \texttt{debug/} 目录
        \item 结合 \texttt{save\_debug\_image} 记录处理步骤
    \end{itemize}
\end{frame}

\section{Git工作流实践详解}

\begin{frame}{分支策略:Git Flow详解}
    \begin{center}
        \begin{tikzpicture}[scale=0.65, transform shape,
            box/.style={draw, rectangle, rounded corners, fill=blue!15, minimum width=2.2cm, minimum height=0.6cm, align=center},
            main/.style={draw, rectangle, rounded corners, fill=red!15, minimum width=2.2cm, minimum height=0.6cm, align=center},
            develop/.style={draw, rectangle, rounded corners, fill=green!15, minimum width=2.2cm, minimum height=0.6cm, align=center}]

            % 主分支
            \node[main] (master) at (0,0) {master\\生产分支};
            \node[develop] (dev) at (0,-1.2) {develop\\开发分支};

            % 功能分支
            \node[box] (f1) at (4,-1.8) {feature/choice};
            \node[box] (f2) at (4,-2.5) {feature/judge};
            \node[box] (f3) at (4,-3.2) {feature/essay};

            % 发布分支
            \node[box, fill=yellow!15] (rel) at (8,-0.6) {release/v1.0};

            % 热修复分支
            \node[box, fill=orange!15] (hotfix) at (8,-2) {hotfix/fix};

            % 连接
            \draw[->, thick] (master) -- (dev);
            \draw[->, thick] (dev) -- (f1);
            \draw[->, thick] (dev) -- (f2);
            \draw[->, thick] (dev) -- (f3);
            \draw[->, thick] (dev) -- (rel);
            \draw[->, thick] (rel) -- (master);
            \draw[->, thick] (hotfix) -- (master);
            \draw[->, thick] (hotfix) -- (dev);
        \end{tikzpicture}
    \end{center}

    \textbf{分支说明:}
    \begin{itemize}
        \item \textbf{master}:生产分支,只接受合并,永远保持可发布状态
        \item \textbf{develop}:开发分支,功能集成分支
        \item \textbf{feature/*}:功能分支,从develop创建
        \item \textbf{release/*}:发布分支,准备上线
        \item \textbf{hotfix/*}:紧急修复,从master创建
    \end{itemize}
\end{frame}

\begin{frame}{分支命名规范}
    \begin{table}
        \centering
        \small
        \begin{tabular}{p{3cm}p{6cm}}
            \toprule
            \textbf{分支类型} & \textbf{命名规范} \\
            \midrule
            功能分支 & feature/模块-功能描述 \\
            & 如:feature/choice-recognizer \\
            \hline
            发布分支 & release/v版本号 \\
            & 如:release/v1.0.0 \\
            \hline
            修复分支 & hotfix/问题描述 \\
            & 如:hotfix/correct-bug-001 \\
            \hline
            实验分支 & dev/实验描述 \\
            & 如:dev/ocr-experiment \\
            \bottomrule
        \end{tabular}
    \end{table}

    \begin{alertblock}{命名原则}
        \begin{itemize}
            \item 使用小写字母
            \item 使用连字符(-)而非下划线
            \item 描述清晰,避免缩写
            \item 包含任务编号(如有)
        \end{itemize}
    \end{alertblock}
\end{frame}

\section{分组任务清单}

\begin{frame}{开发任务分解}
    \textbf{高优先级(P0)- 本周必须完成:}
    \begin{columns}
        \column{0.5\textwidth}
        \begin{block}{选择题识别器}
            \begin{itemize}
                \item ChoiceRecognizer类框架
                \item 填涂区域检测
                \item 像素密度统计
                \item 置信度计算
                \item 单元测试
            \end{itemize}
        \end{block}

        \column{0.5\textwidth}
        \begin{block}{判断题识别器}
            \begin{itemize}
                \item JudgeRecognizer类框架
                \item 符号轮廓提取
                \item 形状特征分析
                \item √/× 分类
                \item 单元测试
            \end{itemize}
        \end{block}
    \end{columns}

    \textbf{中优先级(P1)- 下周完成:}
    \begin{itemize}
        \item 简答题识别器(集成PaddleOCR)
        \item 评分模块
        \item 模块集成测试
    \end{itemize}

    \textbf{低优先级(P2)- 选做:}
    \begin{itemize}
        \item 性能优化
        \item 图形界面
        \item 高级功能
    \end{itemize}
\end{frame}

\begin{frame}{角色任务清单}
    \begin{table}
        \centering
        \small
        \begin{tabular}{p{2cm}p{7cm}p{3cm}}
            \toprule
            \textbf{角色} & \textbf{本周任务} & \textbf{交付物} \\
            \midrule
            组长 & 协调分工、进度跟踪、代码审查、主程序整合 & 项目周报 \\
            技术负责人 & 架构设计、技术难点攻关、代码审查 & 架构设计文档 \\
            模块开发A & 选择题识别器实现、单元测试 & choice\_recognizer.py \\
            模块开发B & 判断题识别器实现、单元测试 & judge\_recognizer.py \\
            \bottomrule
        \end{tabular}
    \end{table}

    \begin{alertblock}{协作要求}
        每日站会(15分钟):各自汇报进度、遇到的问题
    \end{alertblock}
\end{frame}

\begin{frame}{GitHub Flow简化工作流}
    \textbf{适合小团队的简化流程:}

    \begin{enumerate}
        \item \textbf{主分支保护}:master分支受保护,禁止直接推送
        \item \textbf{功能分支开发}:从master创建feature分支
        \item \textbf{提交Pull Request}:完成功能后提交PR
        \item \textbf{代码审查}:至少1人review通过
        \item \textbf{合并到主分支}:CI通过后合并
        \item \textbf{删除功能分支}:保持仓库整洁
    \end{enumerate}

    \begin{center}
        \begin{tikzpicture}[scale=0.7, transform shape,
            box/.style={draw, rectangle, rounded corners, fill=blue!15, minimum width=2.5cm, minimum height=0.7cm}]
            \node[box, fill=green!20] (master) at (0,0) {master};
            \node[box] (feature) at (3,0) {feature/xxx};
            \node[box, fill=purple!15] (pr) at (6,0) {Pull Request};
            \node[box, fill=green!20] (master2) at (9,0) {master (合并后)};

            \draw[->, thick] (master) -- (feature);
            \draw[->, thick] (feature) -- (pr);
            \draw[->, thick] (pr) -- (master2);
        \end{tikzpicture}
    \end{center}
\end{frame}

\begin{frame}{Git命令实战}
    \begin{block}{创建功能分支}
        \begin{itemize}
            \item \texttt{git checkout -b feature/choice-recognizer} - 创建并切换
            \item \texttt{git add .} - 添加修改
            \item \texttt{git commit -m "feat: 实现选择题识别器"} - 提交
        \end{itemize}
    \end{block}

    \begin{block}{推送与同步}
        \begin{itemize}
            \item \texttt{git push origin feature/choice-recognizer} - 推送到远程
            \item \texttt{git fetch origin} - 拉取远程
            \item \texttt{git rebase origin/develop} - 变基到最新
        \end{itemize}
    \end{block}
\end{frame}

\begin{frame}{代码审查流程}
    \begin{block}{代码审查(Code Review)原则}
        \begin{itemize}
            \item 所有代码必须经过review才能合并
            \item 审查关注:功能、性能、安全、风格
            \item 提出问题要具体,给出改进建议
            \item 保持尊重,对代码不对人
            \item 及时响应审查意见
        \end{itemize}
    \end{block}

    \textbf{审查清单:}
    \begin{table}
        \centering
        \small
        \begin{tabular}{p{2cm}p{8cm}}
            \toprule
            \textbf{检查项} & \textbf{内容} \\
            \midrule
            功能性 & 代码是否实现了需求?边界条件处理了吗? \\
            可读性 & 命名是否清晰?注释是否充分? \\
            可维护性 & 是否遵循设计原则?是否高内聚低耦合? \\
            性能 & 是否有明显性能问题?算法复杂度如何? \\
            安全性 & 是否有注入风险?敏感信息处理? \\
            测试 & 是否有单元测试?覆盖率高吗? \\
            \bottomrule
        \end{tabular}
    \end{table}
\end{frame}

\begin{frame}{审查评论规范}
    \begin{columns}
        \column{0.5\textwidth}
        \textbf{好的评论:}
        \begin{itemize}
            \item "这个变量名建议改为 total\_score,更清晰表达含义"
            \item "这里建议添加边界检查,防止数组越界"
            \item "这个逻辑可以用列表推导式简化,更Pythonic"
        \end{itemize}

        \column{0.5\textwidth}
        \textbf{避免的评论:}
        \begin{itemize}
            \item "这写得不对"(太笼统)
            \item "重写这段代码"(没有说明原因)
            \item "这么简单都写错"(人身攻击)
        \end{itemize}
    \end{columns}

    \begin{block}{审查评论标记}
        \begin{itemize}
            \item \textbf{[suggestion]}:建议性改进
            \item \textbf{[nitpick]}:小的格式问题
            \item \textbf{[required]}:必须修改的问题
            \item \textbf{[blocking]}:阻塞性问题
        \end{itemize}
    \end{block}
\end{frame}

\begin{frame}{冲突解决与合并}
    \begin{block}{避免冲突的最佳实践}
        \begin{itemize}
            \item 频繁提交,小步迭代
            \item 开发前先从主分支拉取最新代码
            \item 不要长时间在分支上开发
            \item 与团队成员沟通,避免同时修改同一文件
            \item 使用相同的代码格式化工具
        \end{itemize}
    \end{block}

    \textbf{解决冲突步骤:}
    \begin{center}
        \begin{tikzpicture}[scale=0.7, transform shape,
            box/.style={draw, rectangle, fill=blue!15, minimum width=5cm, minimum height=0.8cm}]
            \node[box] (step1) at (0,0) {1. git pull origin develop};
            \node[box] (step2) at (0,-1) {2. 识别冲突文件};
            \node[box] (step3) at (0,-2) {3. 编辑解决冲突};
            \node[box] (step4) at (0,-3) {4. git add <冲突文件>};
            \node[box] (step5) at (0,-4) {5. git commit};

            \draw[->, thick] (step1) -- (step2);
            \draw[->, thick] (step2) -- (step3);
            \draw[->, thick] (step3) -- (step4);
            \draw[->, thick] (step4) -- (step5);
        \end{tikzpicture}
    \end{center}
\end{frame}

\begin{frame}{版本发布管理}
    \textbf{语义化版本(Semantic Versioning):}

    \begin{center}
        \Large \textbf{MAJOR.MINOR.PATCH}
    \end{center}

    \begin{columns}
        \column{0.4\textwidth}
        \begin{itemize}
            \item \textbf{MAJOR}:不兼容的API修改
            \item \textbf{MINOR}:向下兼容的功能新增
            \item \textbf{PATCH}:向下兼容的问题修复
        \end{itemize}

        \column{0.6\textwidth}
        \begin{block}{版本示例}
            \begin{itemize}
                \item v0.1.0(内测版)
                \item v0.9.0(公测版)
                \item v1.0.0(正式版)
                \item v1.1.0(功能更新)
                \item v1.1.1(Bug修复)
            \end{itemize}
        \end{block}
    \end{columns}

    \textbf{标签操作:}
    \begin{block}{版本标签命令}
        \begin{itemize}
            \item \texttt{git tag -a v1.0.0 -m "第一个正式版本"} - 创建标签
            \item \texttt{git push origin v1.0.0} - 推送标签
            \item \texttt{git tag} - 查看标签
        \end{itemize}
    \end{block}
\end{frame}

\section{代码规范与文档}

\begin{frame}{PEP 8代码规范详解}
    \textbf{命名规范:}
    \begin{table}
        \centering
        \small
        \begin{tabular}{p{2.5cm}p{3cm}p{4.5cm}}
            \toprule
            \textbf{类型} & \textbf{规范} & \textbf{示例} \\
            \midrule
            模块/包 & 小写,短名称 & \texttt{recognition}, \texttt{utils} \\
            类 & 驼峰命名 & \texttt{ChoiceRecognizer} \\
            函数/方法 & 小写下划线 & \texttt{recognize\_choice()} \\
            变量 & 小写下划线 & \texttt{image\_data} \\
            常量 & 大写下划线 & \texttt{MAX\_IMAGE\_SIZE} \\
            私有变量 & 前导下划线 & \texttt{\_internal\_data} \\
            保护成员 & 双前导下划线 & \texttt{\_\_init\_\_} \\
            \bottomrule
        \end{tabular}
    \end{table}

    \textbf{代码布局:}
    \begin{itemize}
        \item 缩进:4个空格(禁用Tab)
        \item 行宽:最大79字符
        \item 空行:顶级函数/类间2行,方法间1行
        \item 导入:标准库→第三方→本地,每组空一行
    \end{itemize}
\end{frame}

\begin{frame}{Python代码风格检查}
    \textbf{使用工具保证代码质量:}

    \begin{columns}
        \column{0.5\textwidth}
        \textbf{black - 代码格式化}
        \begin{itemize}
            \item 强制统一代码风格
            \item 减少代码审查中的格式争论
            \item 配置简单
        \end{itemize}
        \begin{block}{安装使用}
            \begin{itemize}
                \item \texttt{pip install black}
                \item \texttt{black src/}
            \end{itemize}
        \end{block}

        \column{0.5\textwidth}
        \textbf{flake8 - 代码检查}
        \begin{itemize}
            \item 检查代码风格
            \item 发现潜在问题
        \end{itemize}
        \begin{block}{安装使用}
            \begin{itemize}
                \item \texttt{pip install flake8}
                \item \texttt{flake8 src/}
            \end{itemize}
        \end{block}
    \end{columns}
\end{frame}

\begin{frame}{类型注解与文档字符串}
    \begin{block}{类型注解示例}
        \begin{itemize}
            \item \texttt{from typing import List, Optional, Dict}
            \item \texttt{import numpy as np}
            \item 参数:\texttt{image: np.ndarray}
            \item 返回:\texttt{-> Dict[str, str]}
        \end{itemize}
    \end{block}

    \begin{block}{文档字符串规范}
        \begin{itemize}
            \item \textbf{Args}: 参数说明
            \item \textbf{Returns}: 返回值说明
            \item \textbf{Raises}: 异常说明
            \item \textbf{Example}: 使用示例
        \end{itemize}
    \end{block}
\end{frame}

\begin{frame}{接口文档生成}
    \textbf{使用工具自动生成文档:}

    \begin{columns}
        \column{0.5\textwidth}
        \textbf{Sphinx:}
        \begin{itemize}
            \item Python标准文档工具
            \item 支持reStructuredText
            \item 生成HTML/PDF/ePub
            \item 支持自动生成API文档
        \end{itemize}

        \column{0.5\textwidth}
        \textbf{MkDocs:}
        \begin{itemize}
            \item 基于Markdown
            \item 配置简单
            \item 美观的主题
            \item 适合项目文档
        \end{itemize}
    \end{columns}

        \begin{block}{pdoc文档生成}
            \begin{itemize}
                \item \texttt{pip install pdoc}
                \item \texttt{pdoc --http : mymodule} - 启动文档服务器
                \item \texttt{pdoc --html mymodule} - 生成HTML
            \end{itemize}
        \end{block}
\end{frame}

\begin{frame}{README文档规范}
    \textbf{项目README应包含:}
    \begin{table}
        \centering
        \small
        \begin{tabular}{p{3cm}p{6cm}}
            \toprule
            \textbf{章节} & \textbf{内容} \\
            \midrule
            项目名称 & 清晰的项目标题 \\
            项目简介 & 一句话描述项目目的 \\
            功能特性 & 主要功能列表 \\
            技术栈 & 使用的技术、库、版本 \\
            安装部署 & 环境要求、安装步骤 \\
            使用说明 & 快速上手示例 \\
            目录结构 & 项目文件组织 \\
            贡献指南 & 如何参与开发 \\
            许可证 & 开源协议 \\
            \bottomrule
        \end{tabular}
    \end{table}
\end{frame}

\section{团队协作实践}

\begin{frame}{任务分配与跟踪}
    \textbf{任务分解原则(WBS):}
    \begin{itemize}
        \item 每个任务可独立完成(1-3天)
        \item 任务有明确的验收标准
        \item 任务间依赖关系清晰
        \item 预留缓冲时间应对风险
    \end{itemize}

    \textbf{任务模板:}
    \begin{table}
        \centering
        \small
        \begin{tabular}{|l|l|}
            \hline
            \textbf{字段} & \textbf{示例} \\
            \hline
            任务名称 & 实现选择题识别器 \\
            任务描述 & 识别答题卡上选择题的填涂答案 \\
            验收标准 & 准确率>95\%,有单元测试 \\
            负责人 & 张三 \\
            计划工时 & 3天 \\
            截止时间 & 第10周周三 \\
            优先级 & 高 \\
            依赖任务 & 图像预处理模块 \\
            \hline
        \end{tabular}
    \end{table}
\end{frame}

\begin{frame}{每日站会与进度同步}
    \begin{block}{每日站会(Daily Stand-up)}
        \begin{itemize}
            \item 时间:固定时间,15分钟以内
            \item 形式:站立进行,保持高效
            \item 每人回答三个问题:
                \begin{enumerate}
                    \item 昨天完成了什么?
                    \item 今天计划做什么?
                    \item 有什么阻碍?
                \end{enumerate}
        \end{itemize}
    \end{block}

    \begin{columns}
        \column{0.5\textwidth}
        \textbf{进度同步工具:}
        \begin{itemize}
            \item 看板(Kanban):直观展示任务状态
            \item 燃尽图(Burndown):跟踪进度趋势
            \item 甘特图(Gantt):展示任务时间安排
        \end{itemize}

        \column{0.5\textwidth}
        \textbf{会议记录模板:}
        \begin{itemize}
            \item 日期和参会人员
            \item 各成员进度汇报
            \item 遇到的问题
            \item 解决方案
            \item 下次会议时间
        \end{itemize}
    \end{columns}
\end{frame}

\begin{frame}{问题沟通与解决}
    \textbf{问题升级路径:}

    \begin{center}
        \begin{tikzpicture}[scale=0.7, transform shape,
            box/.style={draw, rectangle, rounded corners, fill=blue!15, minimum width=4cm, minimum height=1cm, text width=3.5cm, align=center}]

            \node[box] (self) at (0,0) {1. 自主解决\\查阅文档、Google搜索};
            \node[box] (team) at (0,-1.5) {2. 团队讨论\\请教组员、技术负责人};
            \node[box] (lead) at (0,-3) {3. 组长介入\\调整方案、协调资源};
            \node[box] (teacher) at (0,-4.5) {4. 求助老师\\技术难点、方向决策};

            \draw[->, thick] (self) -- (team) node[midway, right] {30分钟};
            \draw[->, thick] (team) -- (lead) node[midway, right] {2小时};
            \draw[->, thick] (lead) -- (teacher) node[midway, right] {半天};
        \end{tikzpicture}
    \end{center}

    \begin{alertblock}{沟通原则}
        遇到问题不要憋太久,及时求助。但同时也要先自己尝试解决,带着思考去提问。
    \end{alertblock}
\end{frame}

\begin{frame}{团队协作工具链}
    \begin{center}
        \begin{tikzpicture}[scale=0.65, transform shape,
            box/.style={draw, rectangle, rounded corners, fill=blue!15, minimum width=2.5cm, minimum height=0.8cm, text width=2.3cm, align=center}]

            % 代码协作
            \node[box, fill=green!20] (git) at (0,0) {Git\\版本控制};
            \node[box, fill=green!20] (github) at (3,0) {GitHub\\代码托管};

            % 项目管理
            \node[box, fill=yellow!20] (jira) at (0,-1.2) {Jira\\任务管理};
            \node[box, fill=yellow!20] (trello) at (3,-1.2) {Trello\\看板};

            % 沟通协作
            \node[box, fill=purple!20] (slack) at (0,-2.4) {Slack\\即时通讯};
            \node[box, fill=purple!20] (confluence) at (3,-2.4) {Confluence\\文档};

            % CI/CD
            \node[box, fill=red!20] (actions) at (0,-3.6) {GitHub Actions\\CI/CD};
            \node[box, fill=red!20] (docker) at (3,-3.6) {Docker\\容器化};

            \draw[<->, thick] (git) -- (github);
            \draw[<->, thick] (jira) -- (trello);
            \draw[<->, thick] (slack) -- (confluence);
        \end{tikzpicture}
    \end{center}
\end{frame}

\begin{frame}{代码评审最佳实践}
    \textbf{评审流程:}
    \begin{center}
        \begin{tikzpicture}[scale=0.7, transform shape,
            box/.style={draw, rectangle, rounded corners, fill=blue!15, minimum width=2.5cm, minimum height=0.8cm}]
            \node[box] (create) at (0,0) {创建PR};
            \node[box] (notify) at (2.5,0) {通知Reviewer};
            \node[box] (review) at (5,0) {代码审查};
            \node[box] (revise) at (7.5,0) {修改反馈};
            \node[box, fill=green!20] (merge) at (10,0) {合并};

            \draw[->, thick] (create) -- (notify);
            \draw[->, thick] (notify) -- (review);
            \draw[->, thick] (review) -- (revise);
            \draw[->, thick] (revise) -- (merge);
        \end{tikzpicture}
    \end{center}

    \textbf{PR描述模板:}
    \begin{table}
        \centering
        \small
        \begin{tabular}{|l|p{6cm}|}
            \hline
            \textbf{字段} & \textbf{内容} \\
            \hline
            描述 & 实现选择题识别功能 \\
            关联Issue & \#123 \\
            改动内容 & 新增ChoiceRecognizer类 \\
            测试用例 & test\_choice\_recognition.py \\
            \hline
        \end{tabular}
    \end{table}
\end{frame}
