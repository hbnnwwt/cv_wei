%===========================================================
% summary.tex - 总结与作业
%===========================================================

\section{知识点总结}

\begin{frame}{系统架构设计完整流程}
    \begin{center}
        \begin{tikzpicture}[scale=0.65, transform shape,
            box/.style={draw, rectangle, rounded corners, fill=blue!15, minimum width=2.5cm, minimum height=0.8cm, align=center},
            arrow/.style={->, thick}]

            \node[box, fill=green!20] (req) at (0,0) {需求分析};
            \node[box] (arch) at (3,0) {架构设计};
            \node[box] (module) at (6,0) {模块划分};
            \node[box] (interface) at (9,0) {接口定义};
            \node[box] (impl) at (12,0) {模块实现};
            \node[box] (test) at (12,-2) {集成测试};
            \node[box] (deploy) at (9,-2) {部署运维};

            \draw[arrow] (req) -- (arch);
            \draw[arrow] (arch) -- (module);
            \draw[arrow] (module) -- (interface);
            \draw[arrow] (interface) -- (impl);
            \draw[arrow] (impl) -- (test);
            \draw[arrow] (test) -- (deploy);
            \draw[arrow, dashed] (test) -- (interface) node[midway, left] {问题反馈};
        \end{tikzpicture}
    \end{center}

    \vspace{0.3cm}

    \textbf{核心要点:}
    \begin{itemize}
        \item 架构设计要面向需求,而非面向技术
        \item 模块划分遵循高内聚低耦合原则
        \item 接口设计要稳定,实现可以变化
        \item 持续重构,保持架构活力
    \end{itemize}
\end{frame}

\begin{frame}{架构风格速查表}
    \begin{table}
        \centering
        \scriptsize
        \begin{tabular}{p{2cm}p{3cm}p{3cm}p{3.5cm}}
            \toprule
            \textbf{架构风格} & \textbf{核心思想} & \textbf{适用场景} & \textbf{实现技术} \\
            \midrule
            分层架构 & 按职责分层 & 企业应用 & Spring, Django \\
            微服务 & 服务拆分独立部署 & 大型分布式系统 & Docker, K8s \\
            管道-过滤器 & 数据流处理 & 图像/数据处理 & OpenCV, FFmpeg \\
            事件驱动 & 异步消息通信 & 实时系统 & Kafka, RabbitMQ \\
            \bottomrule
        \end{tabular}
    \end{table}

    \vspace{0.3cm}

    \textbf{选择原则:}
    \begin{itemize}
        \item 小团队、简单业务:单体/分层架构
        \item 大团队、复杂业务:微服务架构
        \item 数据处理流水线:管道-过滤器
        \item 高并发、实时性:事件驱动
    \end{itemize}
\end{frame}

\begin{frame}{设计模式速查表}
    \begin{table}
        \centering
        \scriptsize
        \begin{tabular}{p{1.8cm}p{2.5cm}p{4cm}p{3.5cm}}
            \toprule
            \textbf{类型} & \textbf{模式} & \textbf{解决问题} & \textbf{应用场景} \\
            \midrule
            \multirow{3}{*}{创建型}
            & 工厂模式 & 对象创建与使用解耦 & 根据类型创建不同对象 \\
            & 单例模式 & 全局唯一实例 & 配置管理、数据库连接 \\
            & 建造者模式 & 复杂对象构建 & 多参数对象构造 \\
            \midrule
            \multirow{3}{*}{结构型}
            & 适配器模式 & 接口不兼容转换 & 集成第三方库 \\
            & 装饰器模式 & 动态添加功能 & 日志、缓存、权限 \\
            & 代理模式 & 控制对象访问 & 延迟加载、远程代理 \\
            \midrule
            \multirow{3}{*}{行为型}
            & 策略模式 & 算法 interchangeable & 多种识别算法切换 \\
            & 观察者模式 & 状态变化通知 & 事件监听、消息订阅 \\
            & 模板方法 & 算法骨架复用 & 通用处理流程 \\
            \bottomrule
        \end{tabular}
    \end{table}
\end{frame}

\begin{frame}{SOLID原则速查}
    \begin{table}
        \centering
        \small
        \begin{tabular}{clp{6cm}}
            \toprule
            \textbf{原则} & \textbf{核心} & \textbf{实践要点} \\
            \midrule
            SRP & 单一职责 & 一个类只做一件事,修改原因只有一个 \\
            OCP & 开闭原则 & 扩展新功能不修改已有代码 \\
            LSP & 里氏替换 & 子类可以无缝替换父类 \\
            ISP & 接口隔离 & 客户端不依赖不需要的接口 \\
            DIP & 依赖倒置 & 依赖抽象接口,而非具体实现 \\
            \bottomrule
        \end{tabular}
    \end{table}

    \vspace{0.3cm}

    \begin{block}{记忆口诀}
        "职责单一要封闭,替换子类没问题,接口隔离倒依赖"
    \end{block}
\end{frame}

\begin{frame}{常见问题与解决方案}
    \begin{table}
        \centering
        \scriptsize
        \begin{tabular}{p{4cm}p{7cm}}
            \toprule
            \textbf{问题} & \textbf{解决方案} \\
            \midrule
            模块耦合严重,修改一处影响全局 & 应用DIP,引入接口抽象;使用依赖注入 \\
            代码重复,复制粘贴多 & 提取公共逻辑,应用DRY原则 \\
            扩展困难,新增功能需要修改多处 & 应用OCP,使用策略/工厂模式 \\
            类职责不清,功能臃肿 & 应用SRP,拆分小类 \\
            团队协作冲突多 & 规范Git工作流,明确代码审查流程 \\
            接口频繁变更 & 接口版本化,向下兼容 \\
            性能瓶颈 & 引入缓存、异步处理、并行计算 \\
            \bottomrule
        \end{tabular}
    \end{table}
\end{frame}

\section{课后作业}

\begin{frame}{作业要求详解}
    \begin{block}{作业1:架构设计文档(必做)}
        为你的小组项目设计系统架构:
        \begin{enumerate}
            \item 绘制系统架构图(模块划分、接口关系)
            \item 说明选择的架构风格及理由
            \item 定义各模块的接口(输入、输出、职责)
            \item 说明应用了哪些设计模式
        \end{enumerate}
        \textbf{提交格式:}Markdown文档 + 架构图(PNG/PDF)
    \end{block}

    \vspace{0.3cm}

    \begin{block}{作业2:流水线实现(必做)}
        实现图像预处理流水线:
        \begin{enumerate}
            \item 定义FilterInterface接口
            \item 实现至少3个过滤器(去噪、二值化、矫正)
            \item 实现Pipeline执行引擎
            \item 编写测试用例验证
        \end{enumerate}
        \textbf{提交格式:}Python代码 + 测试报告
    \end{block}
\end{frame}

\begin{frame}{作业评分标准}
    \begin{table}
        \centering
        \small
        \begin{tabular}{p{3cm}p{3cm}p{3cm}p{2cm}}
            \toprule
            \textbf{评分项} & \textbf{优秀(90-100)} & \textbf{良好(75-89)} & \textbf{及格(60-74)} \\
            \midrule
            架构完整性 & 设计合理,扩展性强 & 设计合理 & 基本可用 \\
            代码规范 & 规范+类型注解+文档 & 规范 & 基本规范 \\
            设计模式应用 & 恰当使用3+种 & 使用2种 & 使用1种 \\
            测试覆盖 & 单元测试覆盖主要逻辑 & 有测试 & 无测试 \\
            团队协作 & Git提交规范,PR清晰 & 提交较规范 & 提交混乱 \\
            \bottomrule
        \end{tabular}
    \end{table}

    \vspace{0.3cm}

    \begin{alertblock}{提交要求}
        \begin{itemize}
            \item 截止时间:第10周上课前
            \item 提交方式:GitHub仓库链接
            \item 命名规范:cv2025-groupX-week9
            \item 提交内容:代码 + 文档 + README说明
        \end{itemize}
    \end{alertblock}
\end{frame}

\section{延伸学习资源}

\begin{frame}{推荐书籍与论文}
    \textbf{架构设计:}
    \begin{itemize}
        \item 《软件架构设计:大型网站技术架构与业务架构融合之道》— 余洪春
        \item 《实现领域驱动设计》— Vaughn Vernon
        \item 《微服务设计》— Sam Newman
        \item 《软件架构模式》— Mark Richards(免费电子书)
    \end{itemize}

    \vspace{0.3cm}

    \textbf{设计模式:}
    \begin{itemize}
        \item 《设计模式:可复用面向对象软件的基础》— GoF
        \item 《Head First 设计模式》— Eric Freeman
        \item 《Python设计模式》— Brandon Rhodes
    \end{itemize}

    \vspace{0.3cm}

    \textbf{代码规范:}
    \begin{itemize}
        \item 《代码整洁之道》— Robert C. Martin
        \item 《重构:改善既有代码的设计》— Martin Fowler
    \end{itemize}
\end{frame}

\begin{frame}{在线课程与教程}
    \textbf{中文资源:}
    \begin{itemize}
        \item 极客时间《从0开始学架构》— 李运华
        \item 极客时间《设计模式之美》— 王争
        \item B站《软件架构设计》系列课程
        \item 慕课网《Python设计模式》
    \end{itemize}

    \vspace{0.3cm}

    \textbf{英文资源:}
    \begin{itemize}
        \item Coursera: Software Architecture for the Internet of Things
        \item Udemy: Software Architecture & Technology of Large-Scale Systems
        \item YouTube: System Design Primer系列
        \item Refactoring.Guru: 设计模式教程(有中文)
    \end{itemize}

    \vspace{0.3cm}

    \textbf{官方文档:}
    \begin{itemize}
        \item PEP 8 -- Python代码风格指南
        \item Git官方文档与Pro Git电子书
    \end{itemize}
\end{frame}

\begin{frame}{开源项目参考}
    \textbf{推荐学习项目:}
    \begin{table}
        \centering
        \small
        \begin{tabular}{p{3.5cm}p{3cm}p{4cm}}
            \toprule
            \textbf{项目} & \textbf{语言} & \textbf{学习点} \\
            \midrule
            OCRmyPDF & Python & 图像处理流水线、插件架构 \\
            OpenCV-Python & Python/C++ & 计算机视觉算法实现 \\
            FastAPI & Python & 现代Web框架设计 \\
            Celery & Python & 分布式任务队列 \\
            Pytest & Python & 测试框架设计 \\
            \bottomrule
        \end{tabular}
    \end{table}

    \vspace{0.3cm}

    \textbf{参与开源:}
    \begin{itemize}
        \item 从阅读源码开始,理解项目架构
        \item 修复good-first-issue标签的问题
        \item 参与代码审查,学习最佳实践
    \end{itemize}
\end{frame}

\begin{frame}
    \begin{center}
        \Huge \textbf{谢谢!}

        \vspace{1cm}

        \Large 有问题随时交流

        \vspace{1cm}

        \normalsize
        \textbf{下节预告:}第10周 核心开发与调试

        \vspace{0.3cm}

        \textit{目标:完成所有模块开发与集成}
    \end{center}
\end{frame}
