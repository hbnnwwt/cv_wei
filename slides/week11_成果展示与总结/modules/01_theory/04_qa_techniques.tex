%===========================================================
% 04_qa_techniques.tex - 问答环节技巧
%===========================================================

\begin{frame}{Q&A环节:回答问题的技巧}
    \textbf{问答环节展示应变能力}

    \vspace{0.3cm}

    \textbf{回答问题的标准流程:}
    \begin{enumerate}
        \item \textbf{倾听}:
            \begin{itemize}
                \item 认真听完问题
                \item 必要时重复确认
                \item 不要急于打断
            \end{itemize}

        \item \textbf{思考}:
            \begin{itemize}
                \item 花几秒组织答案
                \item 确保理解问题核心
                \item 决定回答的深度
            \end{itemize}

        \item \textbf{回答}:
            \begin{itemize}
                \item 先给出直接答案
                \item 再展开解释说明
                \item 必要时举例说明
            \end{itemize}

        \item \textbf{确认}:
            \begin{itemize}
                \item 确认回答了问题
                \item 询问是否需要补充
            \end{itemize}
    \end{enumerate}
\end{frame}

\begin{frame}{Q&A环节:处理不同类型的问题}
    \textbf{针对不同问题采用不同策略}

    \vspace{0.3cm}

    \begin{table}
        \centering
        \small
        \begin{tabular}{lp{6cm}l}
            \toprule
            \textbf{问题类型} & \textbf{应对策略} & \textbf{示例} \\
            \midrule
            技术细节 & 解释思路,不必展开代码 & "我们采用X方法,核心是..." \\
            质疑效果 & 用数据说话,诚实对待 & "准确率95\%,基于100张测试..." \\
            超出范围 & 说明限制,聚焦核心 & "这超出我们范围,但..." \\
            多个问题 & 分别回答,控制时间 & "关于A...关于B..." \\
            不太确定 & 诚实承认,提供思路 & "我不太确定,但可能是..." \\
            \bottomrule
        \end{tabular}
    \end{table}

    \vspace{0.3cm}

    \textbf{万能回答框架:}
    \begin{itemize}
        \item "这是个好问题,我们确实考虑过..."
        \item "这个问题的核心在于..."
        \item "我们从XX角度来解决..."
    \end{itemize}
\end{frame}

\begin{frame}{Q&A环节:处理刁钻问题}
    \textbf{面对挑战性问题保持专业}

    \vspace{0.3cm}

    \textbf{常见刁钻问题与应对:}
    \begin{columns}
        \column{0.5\textwidth}
        \begin{block}{"你们的项目有什么创新?"}
            \begin{itemize}
                \item 避免夸大
                \item 说明改进点
                \item 强调应用场景
            \end{itemize}
        \end{block}

        \begin{block}{"这个技术很简单啊"}
            \begin{itemize}
                \item 承认技术基础
                \item 强调工程实践
                \item 说明集成价值
            \end{itemize}
        \end{block}

        \column{0.5\textwidth}
        \begin{block}{"准确率这么低?"}
            \begin{itemize}
                \item 说明测试条件
                \item 解释误差来源
                \item 说明改进方向
            \end{itemize}
        \end{block}

        \begin{block}{"这个不是你们写的吧"}
            \begin{itemize}
                \item 诚实说明引用
                \item 强调集成工作
                \item 说明改进部分
            \end{itemize}
        \end{block}
    \end{columns}

    \vspace{0.3cm}

    \begin{alertblock}{应对原则}
    保持冷静、诚实专业、聚焦价值
    \end{alertblock}
\end{frame}

\begin{frame}{Q&A环节:不知道答案怎么办}
    \textbf{诚实是最专业的选择}

    \vspace{0.3cm}

    \textbf{不知道答案时的应对策略:}
    \begin{enumerate}
        \item \textbf{诚实承认}:
            \begin{itemize}
                \item "这个问题很有意思,但我目前没有相关数据"
                \item "我不太确定,需要进一步研究"
            \end{itemize}

        \item \textbf{提供思路}:
            \begin{itemize}
                \item "从XX角度看,可能是..."
                \item "类似情况下,通常..."
            \end{itemize}

        \item \textbf{转移焦点}:
            \begin{itemize}
                \item "我们主要关注的是..."
                \item "在我们的应用场景下..."
            \end{itemize}

        \item \textbf{承诺跟进}:
            \begin{itemize}
                \item "我可以会后研究一下,再和您交流"
                \item "这是个很好的方向,我们可以后续探讨"
            \end{itemize}
    \end{enumerate}

    \vspace{0.3cm}

    \begin{block}{记住}
    \textbf{没有人知道一切}。诚实承认比胡乱回答更专业
    \end{block}
\end{frame}

\begin{frame}{Q&A环节:时间管理}
    \textbf{在有限时间内高效回答}

    \vspace{0.3cm}

    \textbf{时间控制技巧:}
    \begin{itemize}
        \item \textbf{快速判断}:评估问题重要性,决定回答深度
        \item \textbf{简洁回答}:先给核心答案,再根据时间展开
        \item \textbf{合并回答}:多个相似问题合并回答
        \item \textbf{延后讨论}:复杂问题建议会后讨论
    \end{itemize}

    \vspace{0.3cm}

    \textbf{结束Q&A的时机:}
    \begin{itemize}
        \item 预定时间快到时
        \item 回答了3-5个主要问题后
        \item 问题开始重复或跑题时
    \end{itemize}

    \vspace{0.3cm}

    \textbf{优雅结束Q&A:}
    \begin{itemize}
        \item "由于时间关系,我们再回答最后一个问题"
        \item "感谢大家的提问,感兴趣的同学可以会后继续交流"
        \item "这个问题很有价值,我们可以会后深入讨论"
    \end{itemize}
\end{frame}
