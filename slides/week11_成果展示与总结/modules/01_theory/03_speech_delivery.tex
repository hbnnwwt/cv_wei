%===========================================================
% 03_speech_delivery.tex - 演讲表达技巧
%===========================================================

\begin{frame}{演讲表达:语言与声音}
    \textbf{让你的表达更有感染力}

    \vspace{0.3cm}

    \textbf{语言表达技巧:}
    \begin{columns}
        \column{0.5\textwidth}
        \begin{block}{语速控制}
            \begin{itemize}
                \item 正常:120-150字/分钟
                \item 重点处:放慢
                \item 过渡处:加快
                \item 关键数字:重复强调
            \end{itemize}
        \end{block}

        \column{0.5\textwidth}
        \begin{block}{停顿艺术}
            \begin{itemize}
                \item 章节之间停顿
                \item 重点之前停顿
                \item 提问之后停顿
                \item 让听众思考
            \end{itemize}
        \end{block}
    \end{columns}

    \vspace{0.3cm}

    \textbf{声音控制要点:}
    \begin{itemize}
        \item \textbf{音量}:足够大,后排能听清
        \item \textbf{语调}:有起伏,避免单调
        \item \textbf{清晰}:发音清楚,避免吞字
        \item \textbf{热情}:传递自信和能量
    \end{itemize}
\end{frame}

\begin{frame}{演讲表达:肢体语言}
    \textbf{身体语言强化表达效果}

    \vspace{0.3cm}

    \textbf{站姿与移动:}
    \begin{itemize}
        \item \textbf{站姿}:双脚与肩同宽,挺胸抬头
        \item \textbf{移动}:适度移动,不要晃动
        \item \textbf{面向}:面向观众,不要背对观众
    \end{itemize}

    \vspace{0.3cm}

    \textbf{手势运用:}
    \begin{table}
        \centering
        \begin{tabular}{ll}
            \toprule
            \textbf{手势类型} & \textbf{适用场景} \\
            \midrule
            开放式手势 & 欢迎观众、展示开放态度 \\
            指示性手势 & 引导观众看屏幕、强调重点 \\
            切分式手势 & 列举要点、区分不同内容 \\
            强调式手势 & 配合重点、加强语气 \\
            \bottomrule
        \end{tabular}
    \end{table}

    \vspace{0.3cm}

    \begin{alertblock}{避免}
    双手交叉抱胸、玩弄笔/翻页器、频繁触摸面部
    \end{alertblock}
\end{frame}

\begin{frame}{演讲表达:眼神交流}
    \textbf{建立与观众的联系}

    \vspace{0.3cm}

    \textbf{眼神交流技巧:}
    \begin{enumerate}
        \item \textbf{扫视全场}:
            \begin{itemize}
                \item 不要盯着屏幕或地面
                \item 定期扫视全场观众
                \item 确保各个区域都被覆盖
            \end{itemize}

        \item \textbf{短暂对视}:
            \begin{itemize}
                \item 与单个观众短暂眼神接触(3-5秒)
                \item 传递信心和真诚
                \item 获得观众反馈
            \end{itemize}

        \item \textbf{关注反馈}:
            \begin{itemize}
                \item 观察观众的反应
                \item 看到困惑时放慢节奏
                \item 看到兴趣时展开讲解
            \end{itemize}
    \end{enumerate}

    \vspace{0.3cm}

    \textbf{特殊情况处理:}
    \begin{itemize}
        \item 紧张时看观众友善的面孔
        \item 忘词时看屏幕或笔记
        \item 回答问题时看提问者
    \end{itemize}
\end{frame}

\begin{frame}{演讲表达:应对紧张}
    \textbf{紧张是正常的,学会管理它}

    \vspace{0.3cm}

    \textbf{演讲前的准备:}
    \begin{itemize}
        \item \textbf{充分准备}:准备越充分,越有信心
        \item \textbf{提前到场}:熟悉环境,测试设备
        \item \textbf{深呼吸}:上场前做几次深呼吸
        \item \textbf{积极想象}:想象成功演讲的场景
    \end{itemize}

    \vspace{0.3cm}

    \textbf{演讲中的应对:}
    \begin{itemize}
        \item \textbf{承认紧张}:说出来反而会放松
        \item \textbf{放慢语速}:给自己思考的时间
        \item \textbf{记住内容}:关注内容而非自己
        \item \textbf{寻找支持}:从友善的观众获得鼓励
    \end{itemize}

    \vspace{0.3cm}

    \begin{block}{记住}
    \textbf{紧张 = 重视}。适度紧张有助于发挥最佳水平
    \end{block}
\end{frame}
