%===========================================================
% 02_speech_design.tex - 幻灯片设计原则
%===========================================================

\begin{frame}{幻灯片设计:视觉设计原则}
    \textbf{优秀的幻灯片设计让演讲更有效}

    \vspace{0.3cm}

    \textbf{核心设计原则:}
    \begin{enumerate}
        \item \textbf{简洁原则}:
            \begin{itemize}
                \item 一页一个核心观点
                \item 去除不必要的装饰
                \item 留白让内容呼吸
            \end{itemize}

        \item \textbf{视觉原则}:
            \begin{itemize}
                \item 用图表代替文字
                \item 用截图代替描述
                \item 用动画展示流程
            \end{itemize}

        \item \textbf{一致性原则}:
            \begin{itemize}
                \item 统一的字体和字号
                \item 统一的配色方案
                \item 统一的排版风格
            \end{itemize}

        \item \textbf{层次原则}:
            \begin{itemize}
                \item 标题、正文、注释层次分明
                \item 重点内容突出显示
                \item 信息分组清晰
            \end{itemize}
    \end{enumerate}
\end{frame}

\begin{frame}{幻灯片设计:信息层次与排版}
    \textbf{如何组织信息层次?}

    \vspace{0.3cm}

    \textbf{层次化的信息组织:}
    \begin{table}
        \centering
        \begin{tabular}{llp{6cm}}
            \toprule
            \textbf{层级} & \textbf{字号} & \textbf{用途} \\
            \midrule
            一级标题 & 36-44pt & 幻灯片标题,最醒目 \\
            二级标题 & 28-32pt & 章节标题,次醒目 \\
            正文内容 & 18-24pt & 核心内容,易读 \\
            辅助说明 & 14-16pt & 注释、来源 \\
            \bottomrule
        \end{tabular}
    \end{table}

    \vspace{0.3cm}

    \textbf{排版技巧:}
    \begin{itemize}
        \item \textbf{对齐}:左对齐为主,保持视觉流线
        \item \textbf{间距}:行距1.2-1.5倍,段落间距明显
        \item \textbf{分组}:相关内容靠近,用框架定边界
        \item \textbf{对比}:用大小、颜色突出重点
    \end{itemize}
\end{frame}

\begin{frame}{幻灯片设计:图表与可视化}
    \textbf{让数据说话,让图表表达}

    \vspace{0.3cm}

    \textbf{常用图表类型:}
    \begin{columns}
        \column{0.5\textwidth}
        \textbf{展示比较:}
        \begin{itemize}
            \item 柱状图:数量对比
            \item 条形图:排名对比
            \item 对比图:前后对比
        \end{itemize}

        \textbf{展示趋势:}
        \begin{itemize}
            \item 折线图:时间变化
            \item 面积图:累积变化
        \end{itemize}

        \column{0.5\textwidth}
        \textbf{展示占比:}
        \begin{itemize}
            \item 饼图:部分与整体
            \item 环形图:现代感的饼图
        \end{itemize}

        \textbf{展示关系:}
        \begin{itemize}
            \item 流程图:步骤关系
            \item 架构图:系统结构
            \item 思维导图:概念关系
        \end{itemize}
    \end{columns}

    \vspace{0.3cm}

    \textbf{技术演讲必备的可视化:}
    \begin{itemize}
        \item 系统架构图
        \item 数据流程图
        \item 代码截图
        \item 运行效果截图/GIF
    \end{itemize}
\end{frame}

\begin{frame}{幻灯片设计:动画与过渡}
    \textbf{恰当使用动画增强表达}

    \vspace{0.3cm}

    \textbf{动画使用原则:}
    \begin{enumerate}
        \item \textbf{有目的性}:
            \begin{itemize}
                \item 逐步展示内容,避免信息过载
                \item 强调重点,吸引注意力
                \item 展示流程,说明顺序
            \end{itemize}

        \item \textbf{克制使用}:
            \begin{itemize}
                \item 不要为了动画而动画
                \item 避免花哨的转场效果
                \item 保持一致的动画风格
            \end{itemize}

        \item \textbf{适度简洁}:
            \begin{itemize}
                \item 淡入淡出最安全
                \item 从左/右飞入适合列表
                \item 缩放适合强调重点
            \end{itemize}
    \end{enumerate}

    \vspace{0.3cm}

    \textbf{适合技术演讲的动画场景:}
    \begin{itemize}
        \item 逐条显示列表项
        \item 分步展示系统架构
        \item 逐步显示代码执行过程
    \end{itemize}
\end{frame}
