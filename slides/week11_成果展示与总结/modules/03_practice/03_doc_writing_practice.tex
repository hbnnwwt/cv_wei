%===========================================================
% 03_doc_writing_practice.tex - 项目文档撰写实战
%===========================================================

\subsection{项目文档撰写实战}

\begin{frame}{文档实战:README编写}
    \textbf{README是项目的门面}

    \vspace{0.3cm}

    \begin{block}{标准README结构}
    \begin{enumerate}
        \item \textbf{项目标题}:简洁有力
        \item \textbf{项目简介}:1-2句话说明功能
        \item \textbf{功能特性}:列出核心功能
        \item \textbf{安装指南}:如何运行项目
        \item \textbf{使用说明}:基本使用方法
        \item \textbf{项目结构}:代码组织说明
        \item \textbf{技术栈}:使用的技术和工具
        \item \textbf{贡献指南}:如何参与开发
        \item \textbf{许可证}:开源协议
    \end{enumerate}
    \end{block}

    \vspace{0.3cm}

    \begin{exampleblock}{本周项目README模板}
    \small
    \# AI阅卷助手\\
    \textbf{基于OpenCV和PaddleOCR的自动阅卷系统}\\
    本项目能自动识别选择题和判断题,支持多种答题卡格式...
    \end{exampleblock}
\end{frame}

\begin{frame}[fragile]{文档实战:技术文档撰写}
    \textbf{技术文档的类型与要点}

    \vspace{0.3cm}

    \begin{columns}
        \column{0.5\textwidth}
        \textbf{1. 架构设计文档}
        \begin{itemize}
            \item 系统整体架构
            \item 模块划分
            \item 数据流向
            \item 接口定义
        \end{itemize}

        \vspace{0.2cm}

        \textbf{2. API接口文档}
        \begin{itemize}
            \item 函数签名
            \item 参数说明
            \item 返回值
            \item 使用示例
        \end{itemize}

        \column{0.5\textwidth}
        \textbf{3. 部署文档}
        \begin{itemize}
            \item 环境要求
            \item 依赖安装
            \item 配置说明
            \item 部署步骤
        \end{itemize}

        \vspace{0.2cm}

        \textbf{4. 用户手册}
        \begin{itemize}
            \item 功能说明
            \item 操作步骤
            \item 常见问题
            \item 故障排查
        \end{itemize}
    \end{columns}

    \vspace{0.3cm}

    \begin{block}{文档编写原则}
    \begin{itemize}
        \item 面向读者:清楚文档写给谁看
        \item 由浅入深:从概览到细节
        \item 图文并茂:用图表辅助说明
        \item 保持更新:代码改了文档也要改
    \end{itemize}
    \end{block}
\end{frame}

\begin{frame}[fragile]{文档实战:API文档示例}
    \textbf{清晰的API文档示例}

    \vspace{0.3cm}

    \begin{exampleblock}{示例:填涂检测函数文档}
\begin{lstlisting}[language=Python]
def detect_fill(image: np.ndarray,
                region: tuple,
                threshold: float = 0.4) -> tuple[bool, float]:
    """
    检测答题卡指定区域的填涂情况

    Args:
        image: 输入图像(灰度图)
        region: 检测区域 (x, y, w, h)
        threshold: 判断阈值,默认0.4

    Returns:
        (is_filled, ratio): 是否填涂及填涂比例

    Example:
        >>> is_filled, ratio = detect_fill(img, (10, 10, 20, 20))
        >>> print(f"填涂比例: {ratio:.2%}")
    """
\end{lstlisting}
    \end{exampleblock}
\end{frame}

\begin{frame}{文档实战:代码注释规范}
    \textbf{好的代码自带文档}

    \vspace{0.3cm}

    \begin{block}{注释三要素}
    \begin{enumerate}
        \item \textbf{为什么}:解释为什么这样写,而非写了什么
        \item \textbf{注意事项}:标明边界条件、特殊情况
        \item \textbf{示例}:复杂逻辑附带使用示例
    \end{enumerate}
    \end{block}

    \vspace{0.3cm}

    \textbf{注释示例对比:}
    \begin{columns}
        \column{0.48\textwidth}
        \begin{alertblock}{\small 不好的注释}
\begin{verbatim}
# 二值化
_, binary = cv2.threshold(
    img, 127, 255,
    cv2.THRESH_BINARY
)

# 判断填涂
if ratio > 0.4:
    result = True
\end{verbatim}
        \end{alertblock}

        \column{0.48\textwidth}
        \begin{exampleblock}{\small 好的注释}
\begin{verbatim}
# 自适应阈值:处理光照不均
# OTSU算法自动选择最佳阈值
_, binary = cv2.threshold(
    img, 0, 255,
    cv2.THRESH_BINARY | cv2.THRESH_OTSU
)

# 阈值0.4:基于实验确定
# 可根据实际填涂深度调整
if ratio > 0.4:
    result = True
\end{verbatim}
        \end{exampleblock}
    \end{columns}
\end{frame}

\begin{frame}{文档实战:知识沉淀}
    \textbf{将经验转化为团队财富}

    \vspace{0.3cm}

    \textbf{本周项目的知识沉淀:}
    \begin{enumerate}
        \item \textbf{技术博客}
            \begin{itemize}
                \item 记录开发过程
                \item 分享踩坑经验
                \item 总结最佳实践
            \end{itemize}

        \item \textbf{团队Wiki}
            \begin{itemize}
                \item 开发环境配置
                \item 常见问题FAQ
                \item 代码规范说明
            \end{itemize}

        \item \textbf{代码注释与文档}
            \begin{itemize}
                \item 完善的README
                \item 详细的API文档
                \item 清晰的代码注释
            \end{itemize}

        \item \textbf{分享材料}
            \begin{itemize}
                \item 技术分享PPT
                \item 演示视频
                \item 使用教程
            \end{itemize}
    \end{enumerate}
\end{frame}
