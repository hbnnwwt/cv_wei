%===========================================================
% 04_review_practice.tex - 项目总结与复盘实战
%===========================================================

\subsection{项目总结与复盘实战}

\begin{frame}{复盘实战:项目复盘框架}
    \textbf{系统化的项目复盘方法}

    \vspace{0.3cm}

    \begin{block}{复盘四步法}
    \begin{enumerate}
        \item \textbf{目标回顾}:
            \begin{itemize}
                \item 最初的目标是什么?
                \item 期望达成什么结果?
            \end{itemize}

        \item \textbf{过程分析}:
            \begin{itemize}
                \item 实际发生了什么?
                \item 哪些做得好?哪些不到位?
                \item 遇到了什么意外?
            \end{itemize}

        \item \textbf{结果评估}:
            \begin{itemize}
                \item 目标完成度如何?
                \item 与预期差距在哪?
                \item 有哪些意外收获?
            \end{itemize}

        \item \textbf{经验总结}:
            \begin{itemize}
                \item 学到了什么?
                \item 有哪些可以复用?
                \item 下次如何改进?
            \end{itemize}
    \end{enumerate}
    \end{block}
\end{frame}

\begin{frame}{复盘实战:本周项目复盘}
    \textbf{AI阅卷助手项目复盘}

    \vspace{0.3cm}

    \begin{columns}
        \column{0.5\textwidth}
        \textbf{目标回顾:}
        \begin{itemize}
            \item[$\checkmark$] 实现答题卡识别
            \item[$\checkmark$] 支持选择题判断题
            \item[$\checkmark$] 达到90\%+准确率
            \item[$\approx$] 完整系统集成
        \end{itemize}

        \vspace{0.2cm}

        \textbf{过程亮点:}
        \begin{itemize}
            \item 团队协作顺畅
            \item 模块化设计成功
            \item AI工具有效使用
        \end{itemize}

        \column{0.5\textwidth}
        \textbf{遇到挑战:}
        \begin{itemize}
            \item 答题卡定位困难
            \item 光照不均影响识别
            \item 手写字体多样
            \item 时间紧张
        \end{itemize}

        \vspace{0.2cm}

        \textbf{解决方法:}
        \begin{itemize}
            \item 使用定位锚点
            \item 自适应二值化
            \item TrOCR手写识别
            \item 分工并行推进
        \end{itemize}
    \end{columns}
\end{frame}

\begin{frame}{复盘实战:STAR法则应用}
    \textbf{用STAR法则总结项目经验}

    \vspace{0.3cm}

    \begin{block}{STAR法则}
    \begin{itemize}
        \item \textbf{S - Situation(情境)}:项目背景是什么?
        \item \textbf{T - Task(任务)}:需要解决什么问题?
        \item \textbf{A - Action(行动)}:采取了什么措施?
        \item \textbf{R - Result(结果)}:取得了什么成果?
    \end{itemize}
    \end{block}

    \vspace{0.3cm}

    \begin{exampleblock}{示例:答题卡定位问题}
    \small
    \begin{itemize}
        \item \textbf{S}:答题卡位置不固定,扫描时有偏移
        \item \textbf{T}:需要自动定位答题卡位置
        \item \textbf{A}:设计定位锚点,使用霍夫变换检测
        \item \textbf{R}:定位成功率从70\%提升到95\%
    \end{itemize}
    \end{exampleblock}
\end{frame}

\begin{frame}{复盘实战:技术总结撰写}
    \textbf{如何撰写有价值的技术总结}

    \vspace{0.3cm}

    \textbf{技术总结的核心内容:}
    \begin{enumerate}
        \item \textbf{技术亮点}:
            \begin{itemize}
                \item 创新的解决方案
                \item 巧妙的实现技巧
                \item 性能优化成果
            \end{itemize}

        \item \textbf{问题与解决}:
            \begin{itemize}
                \item 遇到的技术难题
                \item 尝试过的方案
                \item 最终的解决方案
                \item 为什么这个方案有效
            \end{itemize}

        \item \textbf{最佳实践}:
            \begin{itemize}
                \item 可复用的代码片段
                \item 可推广的经验
                \item 可避免的坑
            \end{itemize}

        \item \textbf{技术债务}:
            \begin{itemize}
                \item 已知的问题
                \item 待优化的部分
                \item 后续改进方向
            \end{itemize}
    \end{enumerate}
\end{frame}

\begin{frame}{复盘实战:团队协作总结}
    \textbf{项目中的协作经验}

    \vspace{0.3cm}

    \textbf{协作亮点:}
    \begin{itemize}
        \item \textbf{角色分工}:谁负责什么?
        \item \textbf{沟通方式}:如何协调进度?
        \item \textbf{冲突解决}:如何处理分歧?
        \item \textbf{知识共享}:如何互相学习?
    \end{itemize}

    \vspace{0.3cm}

    \textbf{可以改进的地方:}
    \begin{itemize}
        \item[$\square$] 早期需求沟通
        \item[$\square$] 定期进度同步
        \item[$\square$] 代码规范统一
        \item[$\square$] 文档及时更新
        \item[$\square$] 测试充分覆盖
    \end{itemize}

    \vspace{0.3cm}

    \begin{block}{团队协作黄金法则}
    \textbf{沟通} $+$ \textbf{信任} $+$ \textbf{责任} $=$ \textbf{高效团队}
    \end{block}
\end{frame}
