%===========================================================
% 05_ai_programming_review.tex - AI辅助编程实践回顾
%===========================================================

\begin{frame}{AI辅助编程能力成长}
    \textbf{从第2周到第10周,AI工具使用能力的提升}

    \vspace{0.3cm}

    \begin{center}
        \begin{tikzpicture}[
            node distance=0.5cm,
            phase/.style={draw, fill=blue!10, minimum width=2.2cm, minimum height=0.8cm, align=center, font=\small},
            arrow/.style={-Stealth, thick}
        ]
            \node[phase] (w2) {第2周\\学习Prompt};
            \node[phase, right=of w2] (w3) {第3-5周\\AI辅助开发};
            \node[phase, right=of w3] (w6) {第6-8周\\AI调试优化};
            \node[phase, right=of w6] (w9) {第9周\\AI代码审查};
            \node[phase, right=of w9] (w10) {第10周\\AI重构测试};

            \draw[arrow] (w2) -- (w3);
            \draw[arrow] (w3) -- (w6);
            \draw[arrow] (w6) -- (w9);
            \draw[arrow] (w9) -- (w10);
        \end{tikzpicture}
    \end{center}

    \vspace{0.3cm}

    \textbf{能力提升路径:}
    \begin{enumerate}
        \item \textbf{第2周}:学习RTF框架、思维链、少样本提示
        \item \textbf{第3-5周}:用AI生成代码、解释OpenCV函数、调试图像预处理
        \item \textbf{第6-8周}:用AI优化识别算法、分析OCR结果、解决手写识别问题
        \item \textbf{第9周}:用AI进行代码审查、生成测试用例、优化架构
        \item \textbf{第10周}:用AI重构代码、性能分析、生成文档
    \end{enumerate}
\end{frame}

\begin{frame}{AI辅助编程最佳实践}
    \textbf{项目开发中成功使用AI的经验}

    \vspace{0.3cm}

    \textbf{Prompt工程技巧:}
    \begin{itemize}
        \item \textbf{RTF框架}:Role(角色)- Task(任务)- Format(格式)
        \item \textbf{思维链}:引导AI逐步思考,展示推理过程
        \item \textbf{少样本}:提供示例,让AI模仿学习
        \item \textbf{多轮迭代}:根据AI回答不断优化Prompt
    \end{itemize}

    \vspace{0.3cm}

    \textbf{成功案例:}
    \begin{itemize}
        \item 用AI快速实现轮廓检测算法,节省2小时
        \item 用AI解释复杂的OpenCV参数,快速上手
        \item 用AI生成测试用例,提高测试覆盖率
        \item 用AI重构长函数,代码可读性提升
    \end{itemize}
\end{frame}

\begin{frame}{AI辅助编程避坑指南}
    \textbf{避免这些常见问题}

    \vspace{0.3cm}

    \textbf{常见陷阱:}
    \begin{itemize}
        \item[$\times$] \textbf{过度依赖}:完全复制AI生成的代码而不理解
        \item[$\times$] \textbf{幻觉问题}:AI生成不存在的函数或API
        \item[$\times$] \textbf{上下文不足}:Prompt缺乏关键信息,AI答非所问
        \item[$\times$] \textbf{安全风险}:将敏感信息发送给AI
    \end{itemize}

    \vspace{0.3cm}

    \textbf{最佳实践:}
    \begin{itemize}
        \item[$\checkmark$] 理解代码逻辑再使用
        \item[$\checkmark$] 验证AI生成的API是否存在
        \item[$\checkmark$] 提供足够的上下文信息
        \item[$\checkmark$] 不发送密码、Key等敏感信息
        \item[$\checkmark$] 记录与AI的对话作为学习资料
    \end{itemize}
\end{frame}

\begin{frame}{AI协作能力评估}
    \textbf{你掌握了哪些AI协作技能?}

    \vspace{0.3cm}

    \begin{block}{自我评估清单}
    \begin{itemize}
        \item[$\square$] 能设计高质量的Prompt(RTF框架)
        \item[$\square$] 能用思维链引导AI思考复杂问题
        \item[$\square$] 能用少样本让AI学习特定风格
        \item[$\square$] 能识别AI的幻觉并验证结果
        \item[$\square$] 能进行多轮迭代优化AI的回答
        \item[$\square$] 能用AI辅助调试和定位问题
        \item[$\square$] 能用AI进行代码审查和重构
        \item[$\square$] 能用AI生成文档和测试用例
    \end{itemize}
    \end{block}

    \vspace{0.3cm}

    \textbf{持续学习:}
    \begin{itemize}
        \item 关注AI工具的新功能和最佳实践
        \item 参与AI编程社区,分享经验
        \item 将AI协作作为一项核心竞争力
    \end{itemize}
\end{frame}
