%===========================================================
% 04_ai_presentation_tools.tex - AI辅助演讲准备工具
%===========================================================

\begin{frame}{AI辅助演讲准备}
    \textbf{用AI工具提升演讲质量}

    \vspace{0.3cm}

    \textbf{AI工具在演讲准备中的应用:}
    \begin{itemize}
        \item \textbf{内容生成}:演讲大纲、开场白、总结语
        \item \textbf{内容优化}:语言润色、逻辑梳理、精简表达
        \item \textbf{素材生成}:架构图、流程图、数据可视化
        \item \textbf{问答准备}:预测问题、准备答案、模拟答辩
        \item \textbf{排练辅助}:语音分析、语速评估、计时提醒
    \end{itemize}

    \vspace{0.3cm}

    \begin{block}{核心理念}
    AI不是代替你准备演讲,而是帮助你更好地准备演讲
    \end{block}
\end{frame}

\begin{frame}[fragile]{用AI生成演讲大纲}
    \textbf{场景:快速生成5分钟演讲大纲}

    \begin{columns}
        \column{0.48\textwidth}
        \textbf{Prompt(RTF框架):}
\begin{lstlisting}[basicstyle=\ttfamily\tiny, frame=single]
角色:技术演讲专家
任务:为AI阅卷助手项目生成演讲大纲
格式:按时间分点,标注时间分配

要求:
1. 总时长5分钟
2. 开场30秒:Hook听众
3. 演示3分钟:展示核心功能
4. 讲解1分钟:技术亮点
5. 总结30秒:价值与展望
\end{lstlisting}

        \column{0.48\textwidth}
        \textbf{AI会生成:}
\begin{lstlisting}[basicstyle=\ttfamily\tiny, frame=single]
0:00-0:30 开场引入
- 问题:期末阅卷耗时
- 解决:AI阅卷助手
- Hook:3秒阅卷演示

0:30-3:30 核心演示
- 图像加载与预处理
- 答题卡区域定位
- 填涂识别与评分
- 结果输出与导出

3:30-4:30 技术亮点
- 自动阈值调整算法
- 模块化架构设计
- 准确率95%+

4:30-5:00 总结展望
- 核心价值:80%时间节省
- 未来:支持主观题识别
\end{lstlisting}
    \end{columns}
\end{frame}

\begin{frame}[fragile]{用AI优化PPT内容}
    \textbf{场景:让PPT更简洁有效}

    \begin{columns}
        \column{0.48\textwidth}
        \textbf{优化前的PPT:}
\begin{lstlisting}[basicstyle=\ttfamily\tiny, frame=single]
我们的系统采用了OpenCV进行图像预处理,
包括高斯滤波、中值滤波、双边滤波等去噪方法,
然后使用Otsu自适应阈值进行二值化处理,
接着通过轮廓检测定位答题区域...
\end{lstlisting}

        \column{0.48\textwidth}
        \textbf{Prompt:}
\begin{lstlisting}[basicstyle=\ttfamily\tiny, frame=single]
将这段PPT内容优化为:

要求:
1. 每行不超过20字
2. 使用要点列表
3. 关键词突出
4. 适合3秒阅读
\end{lstlisting}
    \end{columns}

    \vspace{0.3cm}

    \textbf{AI优化后:}
    \begin{itemize}
        \item \textbf{图像预处理}:去噪(高斯/中值/双边)
        \item \textbf{二值化}:Otsu自适应阈值
        \item \textbf{区域定位}:轮廓检测算法
        \item \textbf{识别引擎}:OMR + OCR
    \end{itemize}
\end{frame}

\begin{frame}[fragile]{用AI生成展示素材}
    \textbf{场景:生成架构图和流程图}

    \begin{columns}
        \column{0.48\textwidth}
        \textbf{Prompt:}
\begin{lstlisting}[basicstyle=\ttfamily\tiny, frame=single]
用Mermaid语法绘制系统架构图:

系统名称:AI阅卷助手

模块:
1. 图像预处理模块
2. 区域检测模块
3. 识别模块
4. 评分模块
5. 结果导出模块

要求:
- 从上到下流程
- 清晰的模块划分
- 简洁的命名
\end{lstlisting}

        \column{0.48\textwidth}
        \textbf{AI生成Mermaid代码:}
\begin{lstlisting}[basicstyle=\ttfamily\tiny, frame=single]
graph TD
    A[输入图像] --> B[预处理]
    B --> C[区域检测]
    C --> D{识别}
    D --> E[选择题]
    D --> F[判断题]
    E --> G[评分]
    F --> G
    G --> H[结果导出]
\end{lstlisting}
    \end{columns}

    \vspace{0.3cm}

    \textbf{其他AI生成素材:}
    \begin{itemize}
        \item 用AI生成流程图(Mermaid/PlantUML)
        \item 用AI生成数据可视化建议
        \item 用AI生成图标和配色方案
    \end{itemize}
\end{frame}

\begin{frame}[fragile]{用AI预测问答}
    \textbf{场景:准备答辩问题}

    \begin{columns}
        \column{0.48\textwidth}
        \textbf{Prompt:}
\begin{lstlisting}[basicstyle=\ttfamily\tiny, frame=single]
作为技术专家,请为这个项目生成
可能的答辩问题:

项目:AI阅卷助手
功能:选择题和判断题自动识别
技术栈:OpenCV + PaddleOCR
准确率:95%

要求:
1. 生成5-10个问题
2. 按难度分类
3. 提供简要答案要点
\end{lstlisting}

        \column{0.48\textwidth}
        \textbf{AI会预测:}
\begin{lstlisting}[basicstyle=\ttfamily\tiny, frame=single]
基础问题:
Q: 系统的核心功能是什么?
A: 自动识别答题卡,输出评分结果

进阶问题:
Q: 如何提高填涂识别准确率?
A: 自适应阈值算法+机器学习辅助

深入问题:
Q: 系统能处理哪些异常情况?
A: 空白试卷、部分填涂、多选等
\end{lstlisting}
    \end{columns}

    \vspace{0.3cm}

    \textbf{答辩准备策略:}
    \begin{itemize}
        \item 用AI预测评委可能问的问题
        \item 准备数据和实验结果支撑
        \item 诚实面对不足,展示改进方向
    \end{itemize}
\end{frame}
