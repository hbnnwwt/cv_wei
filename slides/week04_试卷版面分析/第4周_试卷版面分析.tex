\documentclass[aspectratio=169, 12pt]{beamer}
\usepackage[UTF8]{ctex}
\usepackage{graphicx}
\usepackage{booktabs}
\usepackage{listings}
\usepackage{xcolor}
\usepackage{tikz}
\usepackage{hyperref}

\usetheme{Madrid}
\usecolortheme{whale}
\usefonttheme{professionalfonts}

% 页脚logo(缩小显示)
\logo{\includegraphics[height=0.8cm]{../xiaohui.png}}

\lstset{
    language=Python,
    basicstyle=\ttfamily\small,
    keywordstyle=\color{blue},
    commentstyle=\color{green!60!black},
    stringstyle=\color{orange},
    breaklines=true,
    frame=single,
    showstringspaces=false,
    backgroundcolor=\color{gray!10}
}

\title[试卷版面分析]{第4周:试卷版面分析}
\subtitle{怎么知道选择题、简答题在哪里?}
\author{北京石油化工学院\\人工智能研究院\\王文通}
\institute{通选课}
\date{2025-2026 学年}
\titlegraphic{
    \includegraphics[height=1.2cm]{../xiaohui.png}\hspace{2cm}
    \includegraphics[height=1.2cm]{../name.png}
}

\begin{document}

\begin{frame}
    \titlepage
\end{frame}

\begin{frame}{课程概览}
    \tableofcontents
\end{frame}

\section{版面分析概述}

\begin{frame}{什么是版面分析?}
    \begin{block}{定义}
        从文档图像中识别和定位不同区域(标题、正文、表格等)
    \end{block}

    \vspace{0.3cm}

    \textbf{在阅卷系统中的作用:}
    \begin{enumerate}
        \item 找到试卷边界
        \item 定位选择题区域
        \item 定位判断题区域
        \item 定位简答题区域
    \end{enumerate}
\end{frame}

\begin{frame}{版面分析流程}
    \begin{center}
        \begin{tikzpicture}[node distance=1.8cm]
            \node[draw, rectangle, rounded corners, fill=blue!10] (input) {输入图像};
            \node[draw, rectangle, rounded corners, fill=yellow!10, right of=input] (pre) {预处理};
            \node[draw, rectangle, rounded corners, fill=green!10, right of=pre] (edge) {边缘检测};
            \node[draw, rectangle, rounded corners, fill=red!10, right of=edge] (contour) {轮廓检测};
            \node[draw, rectangle, rounded corners, fill=purple!10, right of=contour] (region) {区域筛选};

            \draw[->, thick] (input) -- (pre);
            \draw[->, thick] (pre) -- (edge);
            \draw[->, thick] (edge) -- (contour);
            \draw[->, thick] (contour) -- (region);
        \end{tikzpicture}
    \end{center}
\end{frame}

\section{边缘检测}

\begin{frame}{Canny边缘检测}
    \textbf{为什么选择Canny?}
    \begin{itemize}
        \item 最优边缘检测算法
        \item 检测准确、定位精确
        \item 单边缘响应
    \end{itemize}

    \vspace{0.3cm}

    \textbf{参数说明:}
    \begin{itemize}
        \item threshold1:低阈值(50-150)
        \item threshold2:高阈值(100-300)
        \item 高阈值通常是低阈值的2-3倍
    \end{itemize}
\end{frame}

\begin{frame}[fragile]{Canny边缘检测代码}
    \begin{lstlisting}
import cv2
import numpy as np

# 预处理
gray = cv2.cvtColor(img, cv2.COLOR_BGR2GRAY)
blur = cv2.GaussianBlur(gray, (5, 5), 0)

# Canny边缘检测
edges = cv2.Canny(blur, 50, 150)

# 可视化
cv2.imshow('Edges', edges)
cv2.waitKey(0)
    \end{lstlisting}

    \textbf{调参技巧:}
    \begin{itemize}
        \item 噪声多 $\to$ 提高低阈值
        \item 边缘弱 $\to$ 降低高阈值
    \end{itemize}
\end{frame}

\section{轮廓检测}

\begin{frame}{轮廓vs边缘}
    \begin{columns}
        \column{0.5\textwidth}
        \begin{block}{边缘(Edges)}
            \begin{itemize}
                \item 不连续的像素集合
                \item 梯度变化的位置
                \item 原始数据
            \end{itemize}
        \end{block}

        \column{0.5\textwidth}
        \begin{block}{轮廓(Contours)}
            \begin{itemize}
                \item 连续的边界曲线
                \item 封闭的、有序的
                \item 可用于形状分析
            \end{itemize}
        \end{block}
    \end{columns}

    \vspace{0.5cm}

    \textbf{关系:} 边缘 $\to$ 连接 $\to$ 轮廓
\end{frame}

\begin{frame}[fragile]{轮廓检测API}
    \begin{lstlisting}
# 查找轮廓
contours, hierarchy = cv2.findContours(
    binary_image,           # 输入(二值图)
    cv2.RETR_EXTERNAL,      # 外层轮廓
    cv2.CHAIN_APPROX_SIMPLE # 压缩像素
)

# 绘制轮廓
output = cv2.drawContours(
    image.copy(), contours, -1, (0, 255, 0), 2
)
    \end{lstlisting}

    \textbf{轮廓模式选择:}
    \begin{itemize}
        \item RETR\_EXTERNAL:只检测最外层
        \item RETR\_TREE:检测所有层级
    \end{itemize}
\end{frame}

\section{形状特征}

\begin{frame}{轮廓特征}
    \textbf{常用特征:}
    \begin{table}
        \centering
        \small
        \begin{tabular}{lp{5cm}l}
            \toprule
            \textbf{特征} & \textbf{说明} & \textbf{OpenCV函数} \\
            \midrule
            面积 & 轮廓所围区域大小 & cv2.contourArea() \\
            周长 & 轮廓长度 & cv2.arcLength() \\
            边界矩形 & 外接矩形 & cv2.boundingRect() \\
            凸包 & 最小凸多边形 & cv2.convexHull() \\
            \bottomrule
        \end{tabular}
    \end{table}

    \vspace{0.2cm}

    \textbf{应用:} 根据面积筛选目标轮廓
\end{frame}

\begin{frame}[fragile]{找到试卷轮廓}
    \begin{lstlisting}[basicstyle=\ttfamily\scriptsize]
def find_paper_contour(contours, image_area):
    """从轮廓中找到试卷"""
    for contour in contours:
        area = cv2.contourArea(contour)

        # 面积筛选:应该是图像的一定比例
        if area > image_area * 0.5:
            # 周长
            peri = cv2.arcLength(contour, True)

            # 多边形逼近
            approx = cv2.approxPolyDP(contour, 0.02 * peri, True)

            # 如果是四边形
            if len(approx) == 4:
                return approx

    return None
    \end{lstlisting}
\end{frame}

\section{区域定位}

\begin{frame}{投影法}
    \textbf{水平投影:} 统计每行的白色像素数量

    \begin{columns}
        \column{0.6\textwidth}
        \begin{lstlisting}
def horizontal_projection(binary):
    """水平投影"""
    proj = np.sum(binary, axis=1)
    return proj

def find_divider_lines(proj, threshold=10):
    """找到分隔线"""
    dividers = []
    for i, value in enumerate(proj):
        if value < threshold:
            dividers.append(i)
    return dividers
        \end{lstlisting}

        \column{0.4\textwidth}
        \begin{center}
            \begin{tikzpicture}
                \draw[step=0.5, gray!20] (0,0) grid (3,4);
                \foreach \y in {1,3} \draw[red, thick] (0,\y) -- (3,\y);
                \node at (1.5,2) {分隔线};
            \end{tikzpicture}
        \end{center}
    \end{columns}
\end{frame}

\begin{frame}{连通域分析}
    \textbf{什么是连通域?}

    相邻的相同像素值构成的区域

    \vspace{0.3cm}

    \textbf{应用:识别填涂区域}
    \begin{itemize}
        \item 找到所有连通区域
        \item 根据面积、形状筛选
        \item 定位选项气泡
    \end{itemize}

    \vspace{0.3cm}

    \begin{alertblock}{下周预告}
        下周将用连通域分析来识别选择题填涂!
    \end{alertblock}
\end{frame}

\section{思考题}

\begin{frame}{课堂思考题}
    \begin{block}{问题1:轮廓检测}
        \begin{itemize}
            \item 如何从多个轮廓中找到试卷轮廓?
            \item 如果试卷边缘有破损,怎么办?
        \end{itemize}
    \end{block}

    \vspace{0.3cm}

    \begin{block}{问题2:区域定位}
        \begin{itemize}
            \item 投影法检测分隔线的原理是什么?
            \item 如何处理不同版式的试卷?
        \end{itemize}
    \end{block}
\end{frame}

\section{课后作业}

\begin{frame}{课后作业}
    \begin{block}{题目}
        实现试卷版面分析,标注三种题型区域
    \end{block}

    \textbf{要求:}
    \begin{enumerate}
        \item 使用轮廓检测找到试卷边界
        \item 使用投影法分析试卷结构
        \item 标注出选择题、判断题、简答题区域
        \item 提交标注结果的可视化图像
    \end{enumerate}

    \vspace{0.2cm}

    \textbf{评分标准:}
    \begin{itemize}
        \item 轮廓检测:25分
        \item 区域定位:35分
        \item 可视化:25分
        \item 代码质量:15分
    \end{itemize}
\end{frame}

\begin{frame}{下节预告}
    \begin{center}
        \Large \textbf{第5周:选择题识别(填涂检测)}

        \vspace{0.5cm}

        \normalsize
        故事问题:\textcolor{blue}{怎么知道选了A还是B?}

        \vspace{0.3cm}

        你将学会:
        \begin{itemize}
            \item OMR光学标记识别原理
            \item 填涂密度计算
            \item 选择题自动识别
        \end{itemize}
    \end{center}
\end{frame}

\begin{frame}
    \begin{center}
        \Huge \textbf{谢谢!}

        \vspace{1cm}

        \Large 有问题随时交流
    \end{center}
\end{frame}

\end{document}
