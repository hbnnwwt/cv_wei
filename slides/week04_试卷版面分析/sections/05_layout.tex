%===========================================================
% 05_layout.tex - 投影法与区域定位
%===========================================================

\section{区域定位}

\begin{frame}{投影法原理}
    \textbf{什么是投影?}
    \begin{itemize}
        \item 统计图像在某个方向上的像素分布
        \item \textbf{水平投影}:统计每行的白色像素数
        \item \textbf{垂直投影}:统计每列的白色像素数
    \end{itemize}

    \vspace{0.3cm}

    \begin{columns}
        \column{0.5\textwidth}
        \textbf{水平投影应用:}
        \begin{itemize}
            \item 检测文本行分隔
            \item 找到题目之间的空白
            \item 切割试卷区域
        \end{itemize}

        \column{0.5\textwidth}
        \textbf{垂直投影应用:}
        \begin{itemize}
            \item 检测字符分隔
            \item 找到列边界
            \item 对齐文本
        \end{itemize}
    \end{columns}
\end{frame}

\begin{frame}[fragile]{水平投影}
    \begin{lstlisting}[basicstyle=\ttfamily\scriptsize]
def horizontal_projection(binary):
    """
    水平投影:统计每行的白色像素数
    binary: 二值图像(0=黑, 255=白)
    """
    # 将图像归一化到 0-1
    binary = binary // 255

    # 沿水平方向求和(每行的白色像素数)
    proj = np.sum(binary, axis=1)

    return proj

# 使用
binary = cv2.adaptiveThreshold(gray, 255, ...)
h_proj = horizontal_projection(binary)
    \end{lstlisting}
\end{frame}

\begin{frame}[fragile]{垂直投影}
    \begin{lstlisting}[basicstyle=\ttfamily\scriptsize]
def vertical_projection(binary):
    """
    垂直投影:统计每列的白色像素数
    """
    binary = binary // 255
    proj = np.sum(binary, axis=0)
    return proj

# 使用
v_proj = vertical_projection(binary)

# 可视化
import matplotlib.pyplot as plt
plt.figure(figsize=(10, 4))
plt.plot(v_proj)
plt.title('Vertical Projection')
plt.show()
    \end{lstlisting}
\end{frame}

\begin{frame}{投影波形分析}
    \begin{columns}
        \column{0.6\textwidth}
        \begin{center}
            \begin{tikzpicture}[scale=0.8]
                \draw[->] (0,0) -- (5,0) node[right] {行/列};
                \draw[->] (0,0) -- (0,3) node[above] {像素数};
                \draw[blue, thick] (0,0.5) -- (1,2) -- (2,2.5) -- (3,1) -- (4,2.5) -- (5,1);
                \draw[red, dashed] (0,0.5) -- (5,0.5);
                \node[red, font=\tiny] at (2.5,0.2) {阈值};
            \end{tikzpicture}
        \end{center}

        \column{0.4\textwidth}
        \textbf{波峰(Peak):}
        \begin{itemize}
            \item 像素数多的位置
            \item 对应文本/图形区域
        \end{itemize}

        \textbf{波谷(Valley):}
        \begin{itemize}
            \item 像素数少的位置
            \item 对应空白/分隔
        \end{itemize}
    \end{columns}

    \vspace{0.3cm}

    \textbf{分隔线检测:} 找投影值低于阈值的波谷位置
\end{frame}

\begin{frame}[fragile]{找到分隔线}
    \begin{lstlisting}[basicstyle=\ttfamily\scriptsize]
def find_divider_lines(proj, threshold=10, min_length=5):
    """
    找到分隔线(波谷)
    proj: 投影数组
    threshold: 波谷阈值
    min_length: 最小连续长度(过滤噪声)
    """
    dividers = []
    current_start = None

    for i, value in enumerate(proj):
        if value < threshold:
            if current_start is None:
                current_start = i
        else:
            if current_start is not None:
                length = i - current_start
                if length >= min_length:
                    dividers.append((current_start, i))
                current_start = None

    return dividers

# 使用
dividers = find_divider_lines(h_proj, threshold=10)
    \end{lstlisting}
\end{frame}

\begin{frame}[fragile]{投影法实战}
    \begin{lstlisting}[basicstyle=\ttfamily\tiny]
def layout_analysis(binary):
    """
    版面分析:找到题目分隔线
    """
    # 水平投影
    h_proj = np.sum(binary // 255, axis=1)

    # 找分隔线
    threshold = np.mean(h_proj) * 0.1  # 自适应阈值
    dividers = find_divider_lines(h_proj, threshold)

    # 计算题目区域
    regions = []
    prev_end = 0
    for start, end in dividers:
        if start > prev_end:
            regions.append((prev_end, start))
        prev_end = end

    # 最后一个区域
    if prev_end < binary.shape[0]:
        regions.append((prev_end, binary.shape[0]))

    return regions

# 使用
regions = layout_analysis(binary)
print(f"检测到 {len(regions)} 个区域")
    \end{lstlisting}
\end{frame}

\begin{frame}{连通域分析}
    \textbf{什么是连通域?}
    \begin{itemize}
        \item 相邻的相同像素值构成的区域
        \item 4连通:上下左右
        \item 8连通:包括对角线
    \end{itemize}

    \vspace{0.3cm}

    \textbf{应用场景:}
    \begin{itemize}
        \item 识别填涂区域(选择题气泡)
        \item 分离独立字符
        \item 去除噪声斑点
    \end{itemize}

    \begin{alertblock}{下周预告}
        下周将用连通域分析来识别选择题填涂!
    \end{alertblock}
\end{frame}

\begin{frame}[fragile]{连通域检测}
    \begin{lstlisting}[basicstyle=\ttfamily\scriptsize]
# 连通域检测
num_labels, labels, stats, centroids = cv2.connectedComponentsWithStats(
    binary,      # 输入(二值图,0=背景,>0=前景)
    connectivity=8  # 8连通
)

# stats 格式:[x, y, width, height, area]
# centroids 格式:[cx, cy]

# 遍历所有连通域(跳过背景 label=0)
for i in range(1, num_labels):
    x, y, w, h, area = stats[i]
    cx, cy = centroids[i]

    # 筛选:面积在一定范围内
    if 100 < area < 10000:
        cv2.rectangle(img, (x, y), (x+w, y+h), (0, 255, 0), 2)
    \end{lstlisting}
\end{frame}

\begin{frame}{区域定位完整流程}
    \begin{enumerate}
        \item \textbf{输入}:预处理后的二值图像
        \item \textbf{轮廓检测}:找到主要区域边界
        \item \textbf{投影分析}:定位水平/垂直分隔
        \item \textbf{连通域分析}:识别填涂区域
        \item \textbf{区域标注}:绘制识别结果
    \end{enumerate}

    \vspace{0.3cm}

    \begin{block}{组合策略}
        轮廓检测 + 投影法 + 连通域分析 = 完整版面分析
    \end{block}
\end{frame}
