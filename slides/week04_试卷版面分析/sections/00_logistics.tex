%===========================================================
% 00_logistics.tex - 教学设计模块(时间分配、分组策略、预备知识、学习路径)
%===========================================================

%===========================================================
% 页面1:本周时间分配
%===========================================================
\begin{frame}{本周时间分配(160分钟 = 3学时)}
    \begin{columns}
        \column{0.5\textwidth}
        \textbf{第1学时(50分钟):}
        \begin{itemize}
            \item[00:00-00:10] 概览(10min):版面分析概述
            \item[00:10-00:35] 边缘检测(25min):梯度、Sobel、Canny
            \item[00:35-00:50] 讨论(15min):边缘检测问题讨论
        \end{itemize}

        \textbf{第2学时(50分钟):}
        \begin{itemize}
            \item[00:50-01:15] 轮廓检测(25min):findContours API
            \item[01:15-01:40] 形状特征(25min):面积、周长、多边形逼近
        \end{itemize}

        \column{0.5\textwidth}
        \textbf{第3学时(60分钟):}
        \begin{itemize}
            \item[01:40-02:05] 区域定位(25min):投影法、连通域
            \item[02:05-02:30] 实战任务(25min):标注三种题型区域
            \item[02:30-02:50] 总结与作业(20min)
            \item[02:50-03:00] 课间休息(10min)
        \end{itemize}
    \end{columns}

    \vspace{0.5cm}

    \begin{alertblock}{时间控制提示}
    如果进度落后,建议跳过"Hu矩"和"凸缺陷"部分
    \end{alertblock}
\end{frame}

%===========================================================
% 页面2:分组策略
%===========================================================
\begin{frame}{本周分组策略}
    \textbf{分组原则:}
    \begin{itemize}
        \item 每4人为一组
        \item 确保不同专业背景混合
        \item 建议包含:理工科、文科、无编程基础、有编程基础
    \end{itemize}

    \vspace{0.3cm}

    \textbf{角色分工:}
    \begin{table}
        \centering
        \small
        \begin{tabular}{lp{6cm}l}
            \toprule
            \textbf{角色} & \textbf{职责} & \textbf{适合} \\
            \midrule
            组长 & 统筹协调、进度管理 & 组织能力强的 \\
            算法实现者 & 实现边缘检测、轮廓检测 & 有编程基础的 \\
            参数调优者 & 调整Canny阈值、轮廓筛选参数 & 细心负责的 \\
            测试者 & 收集测试用例、报告问题 & 细心负责的 \\
            \bottomrule
        \end{tabular}
    \end{table}

    \vspace{0.3cm}

    \begin{block}{本周协作任务}
        用Canny边缘检测和轮廓检测实现试卷版面分析,标注选择题、判断题、简答题区域
    \end{block}
\end{frame}

%===========================================================
% 页面3:预备知识视频说明
%===========================================================
\begin{frame}{预备知识(课前5分钟视频)}
    \begin{columns}
        \column{0.5\textwidth}
        \textbf{梯度与边缘的数学基础:}
        \begin{itemize}
            \item 导数的概念
            \item 偏导数(对x方向、对y方向)
            \item 梯度向量
            \item 梯度幅值和方向
        \end{itemize}

        \column{0.5\textwidth}
        \textbf{坐标系与几何变换基础:}
        \begin{itemize}
            \item 图像坐标系(原点在左上角)
            \item 坐标变换(平移、旋转、缩放)
            \item 仿射变换
            \item 透视变换
        \end{itemize}
    \end{columns}

    \vspace{0.5cm}

    \begin{alertblock}{观看要求}
    请在课前观看预备知识视频,为本周学习做好准备
    \end{alertblock}
\end{frame}

%===========================================================
% 页面4:并行学习路径
%===========================================================
\begin{frame}{并行学习路径}
    \textbf{观察者路径:}
    \begin{itemize}
        \item 理解版面分析原理
        \item 看老师演示边缘检测、轮廓检测
        \item 完成基础任务:运行示例代码
    \end{itemize}

    \vspace{0.3cm}

    \textbf{使用者路径:}
    \begin{itemize}
        \item 使用示例代码处理自己的试卷图像
        \item 调整Canny阈值、轮廓筛选参数
        \item 完成核心任务:标注三种题型区域
    \end{itemize}

    \vspace{0.3cm}

    \textbf{创造者路径:}
    \begin{itemize}
        \item 设计自己的版面分析算法
        \item 处理复杂版式的试卷
        \item 完成挑战任务:实现自适应版面分析
    \end{itemize}
\end{frame}

%===========================================================
% 页面5:AI工具回顾(Week 2内容)
%===========================================================
\begin{frame}{AI编程辅助工具回顾}
    \textbf{Week 2 学到的AI工具:}
    \begin{columns}
        \column{0.5\textwidth}
        \begin{block}{Cursor}
            \begin{itemize}
                \item AI代码编辑器
                \item 支持代码补全和生成
                \item 可以直接询问编程问题
            \end{itemize}
        \end{block}

        \column{0.5\textwidth}
        \begin{block}{ChatGPT / Claude}
            \begin{itemize}
                \item 通用AI助手
                \item 可以解释算法原理
                \item 可以调试代码错误
            \end{itemize}
        \end{block}
    \end{columns}

    \vspace{0.5cm}

    \begin{exampleblock}{本周AI辅助重点}
        \begin{itemize}
            \item 理解Canny算法的4个步骤
            \item 调试边缘检测和轮廓检测代码
            \item 优化Canny阈值参数
            \item 筛选合适的轮廓
        \end{itemize}
    \end{exampleblock}
\end{frame}

%===========================================================
% 页面6:课堂互动环节设计
%===========================================================
\begin{frame}{课堂互动环节设计}
    \textbf{互动1:实时投票(问卷星)}
    \begin{itemize}
        \item \textbf{时机}:边缘检测基础讲解后
        \item \textbf{问题}:边缘是图像中像素值发生什么变化的位置?
        \item \textbf{选项}:A. 剧烈变化  B. 缓慢变化  C. 无变化
        \item \textbf{目的}:检查学生对边缘概念的理解
    \end{itemize}

    \vspace{0.3cm}

    \textbf{互动2:代码拼图(小组竞赛)}
    \begin{itemize}
        \item \textbf{时机}:Canny边缘检测代码讲解后
        \item \textbf{内容}:将Canny边缘检测代码打乱顺序
        \item \textbf{任务}:小组竞赛复原代码
        \item \textbf{目的}:巩固代码逻辑,培养团队协作
    \end{itemize}

    \vspace{0.3cm}

    \textbf{互动3:错误找茬(集体debug)}
    \begin{itemize}
        \item \textbf{时机}:轮廓检测代码讲解后
        \item \textbf{内容}:展示有bug的轮廓检测代码
        \item \textbf{问题}:findContours输入图像格式错误
        \item \textbf{目的}:培养调试能力,预防常见错误
    \end{itemize}
\end{frame}
