%===========================================================
% 01_overview.tex - 版面分析概述与流程
%===========================================================

\section{版面分析概述}

\begin{frame}{什么是版面分析?}
    \begin{block}{定义}
        从文档图像中识别和定位不同区域(标题、正文、表格、图片等)
    \end{block}

    \vspace{0.3cm}

    \textbf{在阅卷系统中的作用:}
    \begin{enumerate}
        \item 找到试卷边界(定位试卷)
        \item 定位选择题区域(OMR识别)
        \item 定位判断题区域(符号匹配)
        \item 定位简答题区域(手写识别)
        \item 定位填空题区域(内容提取)
    \end{enumerate}
\end{frame}

\begin{frame}{为什么需要版面分析?}
    \begin{columns}
        \column{0.5\textwidth}
        \textbf{没有版面分析:}
        \begin{itemize}
            \item 不知道答题卡在哪
            \item 无法区分题型
            \item OCR识别范围过大
            \item 处理效率低下
        \end{itemize}

        \column{0.5\textwidth}
        \textbf{有了版面分析:}
        \begin{itemize}
            \item 精确定位各区域
            \item 分类处理不同题型
            \item 提高识别准确率
            \item 加快处理速度
        \end{itemize}
    \end{columns}

    \vspace{0.5cm}

    \begin{alertblock}{核心价值}
        版面分析是自动阅卷的"地图导航"!
    \end{alertblock}
\end{frame}

\begin{frame}{版面分析完整流程}
    \begin{center}
        \begin{tikzpicture}[node distance=1.5cm, auto,
            block/.style={draw, rectangle, rounded corners, fill=blue!10, minimum height=0.8cm, minimum width=2cm},
            arrow/.style={->, thick}]

            \node[block] (input) {输入图像};
            \node[block, right of=input, node distance=2.2cm, fill=yellow!10] (pre) {预处理};
            \node[block, right of=pre, node distance=2.2cm, fill=green!10] (edge) {边缘检测};
            \node[block, right of=edge, node distance=2.2cm, fill=red!10] (contour) {轮廓检测};
            \node[block, below of=contour, node distance=1.5cm, fill=purple!10] (filter) {轮廓筛选};
            \node[block, left of=filter, node distance=2.2cm, fill=orange!10] (layout) {版面分析};
            \node[block, left of=layout, node distance=2.2cm, fill=cyan!10] (output) {区域输出};

            \draw[arrow] (input) -- (pre);
            \draw[arrow] (pre) -- (edge);
            \draw[arrow] (edge) -- (contour);
            \draw[arrow] (contour) -- (filter);
            \draw[arrow] (filter) -- (layout);
            \draw[arrow] (layout) -- (output);
        \end{tikzpicture}
    \end{center}
\end{frame}

\begin{frame}{版面分析方法分类}
    \textbf{传统方法:}
    \begin{itemize}
        \item \textbf{边缘检测}:Canny、Sobel
        \item \textbf{轮廓检测}:findContours
        \item \textbf{投影法}:水平/垂直投影
        \item \textbf{连通域分析}:blob检测
    \end{itemize}

    \vspace{0.3cm}

    \textbf{深度学习方法:}
    \begin{itemize}
        \item \textbf{目标检测}:YOLO、Faster R-CNN
        \item \textbf{语义分割}:U-Net、DeepLab
        \item \textbf{版面分析模型}:LayoutLM、DocFormer
    \end{itemize}

    \vspace{0.3cm}

    \begin{block}{本课程重点}
        传统方法(简单、高效、可控)
    \end{block}
\end{frame}

\begin{frame}{应用场景}
    \begin{columns}
        \column{0.33\textwidth}
        \begin{block}{教育阅卷}
            \begin{itemize}
                \item 答题卡识别
                \item 试卷自动批改
                \item 成绩统计
            \end{itemize}
        \end{block}

        \column{0.33\textwidth}
        \begin{block}{文档处理}
            \begin{itemize}
                \item 发票识别
                \item 表单提取
                \item 合同分析
            \end{itemize}
        \end{block}

        \column{0.33\textwidth}
        \begin{block}{档案数字化}
            \begin{itemize}
                \item 版面重建
                \item 区域提取
                \item 内容索引
            \end{itemize}
        \end{block}
    \end{columns}
\end{frame}
