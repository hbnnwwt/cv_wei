%===========================================================
% 06_conclusion.tex - 思考题、作业与预告
%===========================================================

\section{思考题}

\begin{frame}{课堂思考题}
    \begin{block}{问题1:边缘检测}
        \begin{itemize}
            \item Canny 的双阈值如何选择?
            \item 如果阈值设置不当会怎样?
            \item 如何自动确定最优阈值?
        \end{itemize}
    \end{block}

    \vspace{0.3cm}

    \begin{block}{问题2:轮廓检测}
        \begin{itemize}
            \item 如何从多个轮廓中找到试卷轮廓?
            \item 如果试卷边缘有破损,怎么办?
            \item RETR\_EXTERNAL 和 RETR\_TREE 有什么区别?
        \end{itemize}
    \end{block}
\end{frame}

\begin{frame}{课堂思考题(续)}
    \begin{block}{问题3:区域定位}
        \begin{itemize}
            \item 投影法检测分隔线的原理是什么?
            \item 如何处理不同版式的试卷?
            \item 连通域分析在选择题识别中的作用?
        \end{itemize}
    \end{block}

    \vspace{0.3cm}

    \begin{block}{问题4:综合应用}
        \begin{itemize}
            \item 如何设计一个通用的试卷版面分析系统?
            \item 如何处理倾斜的试卷?
            \item 如何处理有多栏的试卷?
        \end{itemize}
    \end{block}
\end{frame}

\section{课后作业}

\begin{frame}{课后作业:三级任务设计}
    \begin{block}{题目}
        实现试卷版面分析,标注三种题型区域
    \end{block}

    \vspace{0.3cm}

    \textbf{基础任务(60分,必做):}
    \begin{itemize}
        \item[25分] 使用 Canny 边缘检测找到试卷边界
        \item[25分] 使用 findContours 检测轮廓并筛选试卷轮廓
        \item[10分] 标注出选择题、判断题、简答题的大致区域
    \end{itemize}

    \vspace{0.2cm}

    \textbf{拓展任务(20分,选做):}
    \begin{itemize}
        \item[10分] 实现自适应Canny阈值选择(auto\_canny函数)
        \item[10分] 使用多边形逼近判断试卷是否为四边形
    \end{itemize}

    \vspace{0.2cm}

    \textbf{挑战任务(20分,选做):}
    \begin{itemize}
        \item[10分] 设计通用版面分析算法,处理多种试卷版式
        \item[10分] 使用投影法分析试卷结构,自动划分题目区域
    \end{itemize}
\end{frame}

\begin{frame}{作业提交要求}
    \textbf{提交内容:}
    \begin{enumerate}
        \item Python 源代码(.py 或 .ipynb)
        \item 测试图像(原图 + 标注结果)
        \item 实验报告(PDF)
    \end{enumerate}

    \vspace{0.3cm}

    \textbf{实验报告包含:}
    \begin{itemize}
        \item 算法流程说明
        \item 参数选择理由
        \item 遇到的问题与解决方案
        \item 处理结果展示
    \end{itemize}

    \vspace{0.3cm}

    \textbf{截止时间:} 下次上课前

    \textbf{提交方式:} 教学平台上传
\end{frame}

\section{下节预告}

\begin{frame}{下节预告}
    \begin{center}
        \Large \textbf{第5周:选择题识别(填涂检测)}

        \vspace{0.5cm}

        \normalsize
        故事问题:\textcolor{blue}{怎么知道选了A还是B?}

        \vspace{0.3cm}

        你将学会:
        \begin{itemize}
            \item OMR 光学标记识别原理
            \item 填涂密度计算方法
            \item 选择题自动识别流程
            \item 多选项处理逻辑
        \end{itemize}
    \end{center}
\end{frame}

\begin{frame}{延伸学习资源}
    \textbf{推荐阅读:}
    \begin{itemize}
        \item OpenCV 官方文档 - Contours 章节
        \item 《数字图像处理》冈萨雷斯 - 边缘检测章节
        \item 论文:Canny, J. (1986). "A Computational Approach to Edge Detection"
    \end{itemize}

    \vspace{0.3cm}

    \textbf{实践项目:}
    \begin{itemize}
        \item 答题卡识别系统
        \item 发票表格提取
        \item 文档版面重建
    \end{itemize}

    \vspace{0.3cm}

    \textbf{在线资源:}
    \begin{itemize}
        \item OpenCV Python Tutorials
        \item LearnOpenCV 网站教程
    \end{itemize}
\end{frame}

\begin{frame}{知识点总结}
    \textbf{核心方法回顾:}
    \begin{table}
        \centering
        \small
        \begin{tabular}{llp{5cm}}
            \toprule
            \textbf{步骤} & \textbf{方法} & \textbf{关键函数} \\
            \midrule
            边缘检测 & Canny & cv2.Canny() \\
            轮廓检测 & findContours & cv2.findContours() \\
            形状分析 & 多边形逼近 & cv2.approxPolyDP() \\
            区域定位 & 投影法 & np.sum(axis=1/0) \\
            细节识别 & 连通域 & cv2.connectedComponents() \\
            \bottomrule
        \end{tabular}
    \end{table}

    \vspace{0.3cm}

    \textbf{关键参数:}
    \begin{itemize}
        \item Canny 阈值:(50, 150)
        \item 逼近精度:0.02 × 周长
        \item 面积筛选:图像面积的 10%-80%
    \end{itemize}
\end{frame}

\begin{frame}
    \begin{center}
        \Huge \textbf{谢谢!}

        \vspace{1cm}

        \Large 有问题随时交流

        \vspace{0.5cm}

        \normalsize
        \textcolor{blue}{邮箱:} wwtong@bipt.edu.cn
    \end{center}
\end{frame}
