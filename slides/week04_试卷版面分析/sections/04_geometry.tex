%===========================================================
% 04_geometry.tex - 形状特征与试卷定位
%===========================================================

\section{形状特征}

\begin{frame}{轮廓特征}
    \textbf{常用几何特征:}
    \begin{table}
        \centering
        \small
        \begin{tabular}{lp{5cm}l}
            \toprule
            \textbf{特征} & \textbf{说明} & \textbf{OpenCV函数} \\
            \midrule
            面积 & 轮廓所围区域大小 & cv2.contourArea() \\
            周长 & 轮廓长度 & cv2.arcLength() \\
            边界矩形 & 外接矩形 & cv2.boundingRect() \\
            最小外接矩形 & 旋转矩形 & cv2.minAreaRect() \\
            最小外接圆 & 外接圆 & cv2.minEnclosingCircle() \\
            凸包 & 最小凸多边形 & cv2.convexHull() \\
            凸缺陷 & 凹陷部分 & cv2.convexityDefects() \\
            \bottomrule
        \end{tabular}
    \end{table}
\end{frame}

\begin{frame}[fragile]{面积与周长}
    \begin{lstlisting}[basicstyle=\ttfamily\scriptsize]
# 计算面积
area = cv2.contourArea(contour)

# 计算周长
perimeter = cv2.arcLength(contour, True)  # True=封闭

# 面积比(用于筛选)
image_area = img.shape[0] * img.shape[1]
area_ratio = area / image_area

# 筛选:保留面积占图像 10%-80% 的轮廓
if 0.1 < area_ratio < 0.8:
    # 保留该轮廓
    pass
    \end{lstlisting}

    \vspace{0.2cm}

    \textbf{应用:} 根据面积筛选目标轮廓
\end{frame}

\begin{frame}[fragile]{边界矩形}
    \textbf{直立边界矩形(不考虑旋转):}
    \begin{lstlisting}[basicstyle=\ttfamily\scriptsize]
x, y, w, h = cv2.boundingRect(contour)

# 绘制矩形
cv2.rectangle(img, (x, y), (x+w, y+h), (0, 255, 0), 2)
    \end{lstlisting}

    \vspace{0.2cm}

    \textbf{最小外接矩形(考虑旋转):}
    \begin{lstlisting}[basicstyle=\ttfamily\scriptsize]
rect = cv2.minAreaRect(contour)
box = cv2.boxPoints(rect)
box = np.int0(box)

# 绘制旋转矩形
cv2.drawContours(img, [box], 0, (0, 0, 255), 2)
    \end{lstlisting}
\end{frame}

\begin{frame}{多边形逼近}
    \textbf{原理:} 用较少的点逼近轮廓形状

    \vspace{0.2cm}

    \textbf{approxPolyDP 函数:}
    $$\epsilon = \text{precision} \times \text{perimeter}$$

    \begin{itemize}
        \item \textbf{precision}:逼近精度(0.01-0.05)
        \item 值越小,逼近越精确(点越多)
        \item 值越大,逼近越粗糙(点越少)
    \end{itemize}

    \vspace{0.3cm}

    \textbf{应用:} 判断轮廓形状
    \begin{itemize}
        \item 4个点 $\rightarrow$ 四边形(试卷)
        \item 3个点 $\rightarrow$ 三角形
        \item 点很多 $\rightarrow$ 圆形
    \end{itemize}
\end{frame}

\begin{frame}[fragile]{多边形逼近代码}
    \begin{lstlisting}[basicstyle=\ttfamily\scriptsize]
# 计算周长
peri = cv2.arcLength(contour, True)

# 多边形逼近
approx = cv2.approxPolyDP(
    contour,      # 输入轮廓
    0.02 * peri,  # 精度参数
    True          # 轮廓是否封闭
)

# 获取顶点数
num_vertices = len(approx)

print(f"顶点数: {num_vertices}")

# 绘制逼近结果
cv2.drawContours(img, [approx], 0, (0, 255, 0), 2)
    \end{lstlisting}
\end{frame}

\begin{frame}{凸包与凸缺陷}
    \textbf{凸包(Convex Hull):}
    \begin{itemize}
        \item 包含轮廓的最小凸多边形
        \item 类似于"橡皮筋"包住轮廓
    \end{itemize}

    \vspace{0.2cm}

    \textbf{凸缺陷(Convexity Defects):}
    \begin{itemize}
        \item 轮廓与凸包之间的凹陷
        \item 用于手势识别、形状分析
    \end{itemize}

    \begin{lstlisting}[basicstyle=\ttfamily\scriptsize]
# 计算凸包
hull = cv2.convexHull(contour)

# 计算凸缺陷
hull_idx = cv2.convexHull(contour, returnPoints=False)
defects = cv2.convexityDefects(contour, hull_idx)
    \end{lstlisting}
\end{frame}

\begin{frame}[fragile]{找到试卷轮廓}
    \begin{lstlisting}[basicstyle=\ttfamily\tiny]
def find_paper_contour(contours, image_area):
    """
    从轮廓中找到试卷
    """
    for contour in contours:
        area = cv2.contourArea(contour)

        # 面积筛选:应该是图像的一定比例
        if area > image_area * 0.5:
            # 计算周长
            peri = cv2.arcLength(contour, True)

            # 多边形逼近
            approx = cv2.approxPolyDP(contour, 0.02 * peri, True)

            # 如果是四边形
            if len(approx) == 4:
                return approx

    return None

# 使用
image_area = gray.shape[0] * gray.shape[1]
paper_contour = find_paper_contour(contours, image_area)
    \end{lstlisting}
\end{frame}

\begin{frame}{形状匹配}
    \textbf{Hu矩:}
    \begin{itemize}
        \item 具有旋转、缩放、平移不变性
        \item 用于形状相似度比较
    \end{itemize}

    \begin{lstlisting}[basicstyle=\ttfamily\scriptsize]
# 计算 Hu 矩
moments = cv2.moments(contour)
hu_moments = cv2.HuMoments(moments)

# 形状匹配
match = cv2.matchShapes(
    contour1,
    contour2,
    cv2.CONTOURS_MATCH_I1,  # 匹配方法
    0.0
)
    \end{lstlisting}

    \textbf{应用:} 识别标准形状(圆形、方形等)
\end{frame}
