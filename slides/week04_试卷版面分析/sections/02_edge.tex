%===========================================================
% 02_edge.tex - 边缘检测
%===========================================================

\section{边缘检测}

\begin{frame}{边缘检测基础}
    \textbf{什么是边缘?}
    \begin{itemize}
        \item 图像中像素值发生剧烈变化的位置
        \item 反映了物体边界、纹理变化等信息
        \item 是图像的重要特征
    \end{itemize}

    \vspace{0.3cm}

    \textbf{边缘类型:}
    \begin{columns}
        \column{0.5\textwidth}
        \begin{center}
            \begin{tikzpicture}
                \draw[gray!20] (0,0) rectangle (4,1);
                \fill[black] (0,0) rectangle (2,1);
                \draw[red, thick] (2,0) -- (2,1);
                \node[below] at (2,0) {阶跃边缘};
            \end{tikzpicture}
        \end{center}

        \column{0.5\textwidth}
        \begin{center}
            \begin{tikzpicture}
                \shade[left color=black, right color=white] (0,0) rectangle (4,1);
                \node at (2,0.5) {斜坡边缘};
            \end{tikzpicture}
        \end{center}
    \end{columns}
\end{frame}

\begin{frame}{梯度与边缘}
    \textbf{梯度原理:}
    $$\nabla f = \begin{bmatrix} \frac{\partial f}{\partial x} \\ \frac{\partial f}{\partial y} \end{bmatrix}$$

    \vspace{0.2cm}

    \textbf{梯度幅值:}
    $$|\nabla f| = \sqrt{(\frac{\partial f}{\partial x})^2 + (\frac{\partial f}{\partial y})^2}$$

    \vspace{0.2cm}

    \textbf{梯度方向:}
    $$\theta = \arctan(\frac{\partial f}{\partial y} / \frac{\partial f}{\partial x})$$

    \vspace{0.3cm}

    \begin{block}{核心思想}
        边缘处梯度幅值最大!
    \end{block}
\end{frame}

\begin{frame}[fragile]{Sobel 算子}
    \textbf{水平方向卷积核(检测垂直边缘):}
    $$
    G_x = \begin{bmatrix}
    -1 & 0 & 1 \\
    -2 & 0 & 2 \\
    -1 & 0 & 1
    \end{bmatrix}
    $$

    \textbf{垂直方向卷积核(检测水平边缘):}
    $$
    G_y = \begin{bmatrix}
    -1 & -2 & -1 \\
     0 &  0 &  0 \\
     1 &  2 &  1
    \end{bmatrix}
    $$

    \begin{lstlisting}[basicstyle=\ttfamily\scriptsize]
# Sobel 边缘检测
sobel_x = cv2.Sobel(gray, cv2.CV_64F, 1, 0, ksize=3)
sobel_y = cv2.Sobel(gray, cv2.CV_64F, 0, 1, ksize=3)
sobel = np.sqrt(sobel_x**2 + sobel_y**2)
    \end{lstlisting}
\end{frame}

\begin{frame}{Canny 边缘检测}
    \textbf{为什么选择 Canny?}
    \begin{itemize}
        \item 最优边缘检测算法(1986年提出)
        \item 检测准确率高(低误检率)
        \item 定位精确(边缘位置准确)
        \item 单边缘响应(每个边缘只有一条响应线)
    \end{itemize}

    \vspace{0.3cm}

    \textbf{Canny 算法步骤:}
    \begin{enumerate}
        \item 高斯滤波降噪
        \item 计算梯度幅值和方向
        \item 非极大值抑制
        \item 双阈值检测和边缘连接
    \end{enumerate}
\end{frame}

\begin{frame}{Canny 算法详解(1/4)}
    \textbf{步骤1:高斯滤波}
    \begin{itemize}
        \item 去除图像噪声
        \item 防止噪声被误认为边缘
        \item 通常使用 5×5 高斯核
    \end{itemize}

    \vspace{0.5cm}

    \textbf{步骤2:计算梯度}
    \begin{itemize}
        \item 使用 Sobel 算子计算梯度
        \item 得到梯度幅值和方向
        \item 梯度方向垂直于边缘方向
    \end{itemize}
\end{frame}

\begin{frame}{Canny 算法详解(2/4)}
    \textbf{步骤3:非极大值抑制}

    \begin{columns}
        \column{0.6\textwidth}
        \textbf{目的:} 将粗边缘变成细边缘(单像素宽)

        \vspace{0.2cm}

        \textbf{方法:}
        \begin{enumerate}
            \item 在梯度方向上比较当前像素
            \item 如果不是局部最大值,则抑制(置0)
            \item 保留梯度方向上的最大值点
        \end{enumerate}

        \column{0.4\textwidth}
        \centering
        \begin{tikzpicture}
            \draw[step=0.5, gray!30] (0,0) grid (3,3);
            \fill[red] (1.5,1.5) circle (0.15);
            \draw[->, blue] (1.5,1.5) -- (2.5,2.5);
            \node[below, font=\tiny] at (1.5,0) {梯度方向};
        \end{tikzpicture}
    \end{columns}
\end{frame}

\begin{frame}{Canny 算法详解(3/4)}
    \textbf{步骤4:双阈值检测}

    \begin{columns}
        \column{0.5\textwidth}
        \textbf{两个阈值:}
        \begin{itemize}
            \item \textbf{高阈值}(threshold2):强边缘
            \item \textbf{低阈值}(threshold1):弱边缘
        \end{itemize}

        \textbf{分类:}
        \begin{itemize}
            \item 梯度 > 高阈值 $\rightarrow$ 强边缘
            \item 低阈值 < 梯度 < 高阈值 $\rightarrow$ 弱边缘
            \item 梯度 < 低阈值 $\rightarrow$ 非边缘
        \end{itemize}

        \column{0.5\textwidth}
        \textbf{边缘连接:}
        \begin{itemize}
            \item 强边缘直接保留
            \item 弱边缘如果与强边缘相连则保留
            \item 孤立的弱边缘被抑制
        \end{itemize}
    \end{columns}

    \vspace{0.3cm}

    \begin{block}{推荐比例}
        高阈值 = 低阈值 × 2~3
    \end{block}
\end{frame}

\begin{frame}[fragile]{Canny 代码实现}
    \begin{lstlisting}[basicstyle=\ttfamily\scriptsize]
import cv2
import numpy as np

# 预处理
gray = cv2.cvtColor(img, cv2.COLOR_BGR2GRAY)
blur = cv2.GaussianBlur(gray, (5, 5), 0)

# Canny 边缘检测
edges = cv2.Canny(
    blur,           # 输入图像
    50,             # 低阈值 threshold1
    150             # 高阈值 threshold2
)

# 可视化
cv2.imshow('Edges', edges)
cv2.waitKey(0)
    \end{lstlisting}
\end{frame}

\begin{frame}[fragile]{Canny 参数调优}
    \textbf{参数影响:}
    \begin{table}
        \centering
        \small
        \begin{tabular}{llp{5cm}}
            \toprule
            \textbf{参数} & \textbf{调整方向} & \textbf{效果} \\
            \midrule
            低阈值 & 提高 & 减少噪声,但可能丢失弱边缘 \\
            低阈值 & 降低 & 检测更多边缘,但噪声增多 \\
            高阈值 & 提高 & 只保留强边缘,边缘更少 \\
            高阈值 & 降低 & 保留更多边缘,可能引入噪声 \\
            \bottomrule
        \end{tabular}
    \end{table}

    \textbf{试卷处理推荐值:}
    \begin{itemize}
        \item 低阈值:50-100
        \item 高阈值:150-200
        \item 可根据图像质量调整
    \end{itemize}
\end{frame}

\begin{frame}[fragile]{自动阈值选择}
    \textbf{问题:} 手动调参太繁琐

    \vspace{0.2cm}

    \textbf{解决方案:} 基于图像中值自动计算

    \begin{lstlisting}[basicstyle=\ttfamily\scriptsize]
def auto_canny(gray, sigma=0.33):
    """
    自动计算 Canny 阈值
    sigma: 控制阈值范围,通常 0.33
    """
    # 计算图像中值
    v = np.median(gray)

    # 计算阈值
    lower = int(max(0, (1.0 - sigma) * v))
    upper = int(min(255, (1.0 + sigma) * v))

    # 应用 Canny
    return cv2.Canny(gray, lower, upper)

# 使用
edges = auto_canny(blur)
    \end{lstlisting}
\end{frame}
