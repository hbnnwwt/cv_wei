%===========================================================
% 03_contour.tex - 轮廓检测原理与API
%===========================================================

\section{轮廓检测}

\begin{frame}{轮廓 vs 边缘}
    \begin{columns}
        \column{0.5\textwidth}
        \begin{block}{边缘(Edges)}
            \begin{itemize}
                \item 不连续的像素集合
                \item 梯度变化的位置
                \item 原始数据
                \item 来自 Canny/Sobel
            \end{itemize}
        \end{block}

        \column{0.5\textwidth}
        \begin{block}{轮廓(Contours)}
            \begin{itemize}
                \item 连续的边界曲线
                \item 封闭的、有序的
                \item 可用于形状分析
                \item 来自 findContours
            \end{itemize}
        \end{block}
    \end{columns}

    \vspace{0.5cm}

    \textbf{关系:} 边缘 $\xrightarrow{\text{连接}}$ 轮廓

    \vspace{0.3cm}

    \begin{center}
        边缘图像 $\rightarrow$ 轮廓检测 $\rightarrow$ 轮廓列表
    \end{center}
\end{frame}

\begin{frame}{轮廓检测流程}
    \begin{enumerate}
        \item \textbf{输入}:二值图像(通常来自 Canny 或 threshold)
        \item \textbf{扫描}:遍历图像,寻找边界
        \item \textbf{连接}:将相邻边缘点连接成轮廓
        \item \textbf{层级}:建立轮廓之间的层级关系
        \item \textbf{输出}:轮廓点集 + 层级结构
    \end{enumerate}

    \vspace{0.3cm}

    \begin{alertblock}{注意}
        findContours 会修改输入图像!如果需要保留,请先复制。
    \end{alertblock}
\end{frame}

\begin{frame}[fragile]{findContours API}
    \textbf{函数原型:}
    \begin{lstlisting}[basicstyle=\ttfamily\scriptsize]
contours, hierarchy = cv2.findContours(
    image,              # 输入(二值图,最好是白底黑字)
    mode,               # 轮廓检索模式
    method              # 轮廓逼近方法
)
    \end{lstlisting}

    \textbf{轮廓检索模式(mode):}
    \begin{table}
        \centering
        \small
        \begin{tabular}{ll}
            \toprule
            \textbf{模式} & \textbf{说明} \\
            \midrule
            cv2.RETR\_EXTERNAL & 只检测最外层轮廓 \\
            cv2.RETR\_LIST & 检测所有轮廓,不建立层级 \\
            cv2.RETR\_CCOMP & 检测所有轮廓,建立两级层级 \\
            cv2.RETR\_TREE & 检测所有轮廓,建立完整层级树 \\
            \bottomrule
        \end{tabular}
    \end{table}
\end{frame}

\begin{frame}[fragile]{轮廓逼近方法(method)}
    \textbf{存储方式:}
    \begin{table}
        \centering
        \begin{tabular}{ll}
            \toprule
            \textbf{方法} & \textbf{说明} \\
            \midrule
            cv2.CHAIN\_APPROX\_NONE & 存储所有边界点 \\
            cv2.CHAIN\_APPROX\_SIMPLE & 压缩水平、垂直、对角线段 \\
            cv2.CHAIN\_APPROX\_TC89\_L1 & Teh-Chin 链逼近算法 \\
            \bottomrule
        \end{tabular}
    \end{table}

    \begin{lstlisting}[basicstyle=\ttfamily\scriptsize]
# 常用配置
contours, hierarchy = cv2.findContours(
    binary,
    cv2.RETR_EXTERNAL,      # 只要外层轮廓
    cv2.CHAIN_APPROX_SIMPLE # 压缩存储,节省内存
)
    \end{lstlisting}

    \textbf{推荐:} RETR\_EXTERNAL + CHAIN\_APPROX\_SIMPLE
\end{frame}

\begin{frame}{层级结构(hierarchy)}
    \textbf{什么是层级?}
    \begin{itemize}
        \item 轮廓之间的包含关系
        \item 外轮廓 vs 内轮廓(孔洞)
        \item 用树形结构表示
    \end{itemize}

    \vspace{0.3cm}

    \textbf{hierarchy 数组格式:}
    $$
    [next, previous, first\_child, parent]
    $$

    \begin{itemize}
        \item \textbf{next}:同级下一个轮廓
        \item \textbf{previous}:同级前一个轮廓
        \item \textbf{first\_child}:第一个子轮廓
        \item \textbf{parent}:父轮廓
    \end{itemize}
\end{frame}

\begin{frame}{层级结构示例}
    \begin{columns}
        \column{0.4\textwidth}
        \centering
        \begin{tikzpicture}
            \draw[thick] (0,0) rectangle (3,3);
            \draw[thick] (0.5,0.5) rectangle (1.5,1.5);
            \draw[thick] (2,1) rectangle (2.8,2);
            \node at (1.5,3.3) {外轮廓 0};
            \node at (1,0.2) {内轮廓 1};
            \node at (2.4,0.7) {内轮廓 2};
        \end{tikzpicture}

        \column{0.6\textwidth}
        \textbf{hierarchy 示例:}
        \begin{table}
            \centering
            \tiny
            \begin{tabular}{cccc}
                \toprule
                next & prev & child & parent \\
                \midrule
                1 & -1 & 2 & -1 \\
                2 & 0 & -1 & 0 \\
                -1 & 1 & -1 & 0 \\
                \bottomrule
            \end{tabular}
        \end{table}

        \small
        \begin{itemize}
            \item 轮廓0:外轮廓(无父级)
            \item 轮廓1:内轮廓(父级=0)
            \item 轮廓2:内轮廓(父级=0)
        \end{itemize}
    \end{columns}
\end{frame}

\begin{frame}[fragile]{绘制轮廓}
    \textbf{drawContours 函数:}
    \begin{lstlisting}[basicstyle=\ttfamily\scriptsize]
# 绘制所有轮廓
output = cv2.drawContours(
    image.copy(),    # 目标图像(会修改原图)
    contours,        # 轮廓列表
    -1,              # -1=绘制所有,0=绘制第一个
    (0, 255, 0),     # 颜色 BGR
    2                # 线宽
)

# 只绘制特定轮廓
output = cv2.drawContours(
    image.copy(), contours, 0, (0, 0, 255), 3
)
    \end{lstlisting}

    \vspace{0.2cm}

    \textbf{填充轮廓:}
    \begin{lstlisting}[basicstyle=\ttfamily\scriptsize]
# 线宽设为 -1 表示填充
output = cv2.drawContours(
    image.copy(), contours, -1, (0, 255, 0), -1
)
    \end{lstlisting}
\end{frame}

\begin{frame}[fragile]{轮廓检测完整示例}
    \begin{lstlisting}[basicstyle=\ttfamily\tiny]
import cv2
import numpy as np

# 读取图像
img = cv2.imread('exam.jpg')
gray = cv2.cvtColor(img, cv2.COLOR_BGR2GRAY)

# 预处理
blur = cv2.GaussianBlur(gray, (5, 5), 0)
edges = cv2.Canny(blur, 50, 150)

# 查找轮廓
contours, hierarchy = cv2.findContours(
    edges,
    cv2.RETR_EXTERNAL,
    cv2.CHAIN_APPROX_SIMPLE
)

# 绘制轮廓
output = img.copy()
cv2.drawContours(output, contours, -1, (0, 255, 0), 2)

print(f"检测到 {len(contours)} 个轮廓")
cv2.imshow('Contours', output)
cv2.waitKey(0)
    \end{lstlisting}
\end{frame}

\begin{frame}{轮廓筛选技巧}
    \textbf{按面积筛选:}
    \begin{itemize}
        \item 排除过小的轮廓(噪声)
        \item 排除过大的轮廓(背景)
    \end{itemize}

    \textbf{按位置筛选:}
    \begin{itemize}
        \item 只保留图像中心区域
        \item 排除边缘区域的轮廓
    \end{itemize}

    \textbf{按形状筛选:}
    \begin{itemize}
        \item 按长宽比筛选
        \item 按纵横比筛选
        \item 按凸包面积比筛选
    \end{itemize}
\end{frame}
