%===========================================================
% 05_binary_coding.tex - 二值化实战代码(TODO框架)
%===========================================================

\begin{frame}[fragile]{二值化代码框架}
    \begin{alertblock}{AI辅助提示}
        使用Cursor/Claude Code可以快速生成以下代码:
        \begin{itemize}
            \item "用OpenCV实现自适应阈值二值化"
            \item "对比全局阈值、Otsu和自适应阈值的效果"
        \end{itemize}
    \end{alertblock}

    \begin{lstlisting}[basicstyle=\ttfamily\tiny]
import cv2
import numpy as np

def binarize_exam(img_path, method='adaptive'):
    """
    试卷二值化处理
    method: 'global', 'otsu', 'adaptive'
    """
    # TODO 1: 读取图像(灰度模式)
    # 提示:使用 cv2.imread(img_path, 0)
    img = ____

    if method == 'global':
        # TODO 2: 全局阈值二值化
        # 提示:cv2.threshold(img, 127, 255, cv2.THRESH_BINARY_INV)
        _, binary = cv2.threshold(____, ____, ____, ____)

    elif method == 'otsu':
        # TODO 3: Otsu自动阈值
        # 提示:添加 cv2.THRESH_OTSU 标志
        _, binary = cv2.threshold(____, ____, ____, ____)

    elif method == 'adaptive':
        # TODO 4: 自适应阈值(推荐用于试卷)
        # 提示:cv2.adaptiveThreshold(img, 255, cv2.ADAPTIVE_THRESH_GAUSSIAN_C,
        #       cv2.THRESH_BINARY_INV, blockSize, C)
        binary = cv2.adaptiveThreshold(____, ____, ____, ____, ____, ____)

    return binary

# 使用
result = binarize_exam('exam.jpg', method='adaptive')
    \end{lstlisting}
\end{frame}

\begin{frame}[fragile]{完整参考代码(课后查看)}
    \begin{lstlisting}[basicstyle=\ttfamily\tiny]
def binarize_exam(img_path, method='adaptive'):
    """完整版二值化函数"""
    # 读取图像(灰度模式)
    img = cv2.imread(img_path, 0)

    if method == 'global':
        # 全局阈值
        _, binary = cv2.threshold(img, 127, 255, cv2.THRESH_BINARY_INV)

    elif method == 'otsu':
        # Otsu自动阈值
        _, binary = cv2.threshold(img, 0, 255,
            cv2.THRESH_BINARY_INV + cv2.THRESH_OTSU)

    elif method == 'adaptive':
        # 自适应阈值(推荐用于手机拍照的试卷)
        binary = cv2.adaptiveThreshold(
            img, 255,
            cv2.ADAPTIVE_THRESH_GAUSSIAN_C,
            cv2.THRESH_BINARY_INV,
            15, 2
        )

    return binary

# 三种方法对比
img = cv2.imread('exam.jpg', 0)
global_bin = binarize_exam('exam.jpg', 'global')
otsu_bin = binarize_exam('exam.jpg', 'otsu')
adaptive_bin = binarize_exam('exam.jpg', 'adaptive')
    \end{lstlisting}
\end{frame}

\begin{frame}[fragile]{参数调优实验}
    \textbf{自适应阈值参数调优:}

    \begin{lstlisting}[basicstyle=\ttfamily\tiny]
# TODO: 实验不同的blockSize和C值
# blockSize必须是奇数,值越大考虑的邻域越大
# C值是从均值减去的常数,值越大阈值越高

# 参数组1:默认参数
binary1 = cv2.adaptiveThreshold(img, 255,
    cv2.ADAPTIVE_THRESH_GAUSSIAN_C,
    cv2.THRESH_BINARY_INV, 15, 2)

# TODO 2: 尝试调整参数
# 如果文字太细:增大blockSize或减小C
# 如果噪声太多:减小blockSize或增大C
binary2 = cv2.adaptiveThreshold(img, 255,
    cv2.ADAPTIVE_THRESH_GAUSSIAN_C,
    cv2.THRESH_BINARY_INV, ____, ____)
    \end{lstlisting}

    \vspace{0.2cm}

    \textbf{参数调优Prompt示例:}
    \begin{lstlisting}[basicstyle=\ttfamily\tiny]
"自适应阈值二值化后,试卷暗区域文字消失了,应该如何调整参数?
当前:blockSize=15, C=2"
    \end{lstlisting}
\end{frame}

\begin{frame}{二值化方法总结}
    \begin{table}
        \centering
        \begin{tabular}{lccc}
            \toprule
            \textbf{方法} & \textbf{光照均匀} & \textbf{光照不均} & \textbf{速度} \\
            \midrule
            全局阈值 & 良好 & 差 & 快 \\
            Otsu & \textbf{优秀} & 差 & 快 \\
            自适应阈值 & 良好 & \textbf{优秀} & 较慢 \\
            \bottomrule
        \end{tabular}
    \end{table}

    \vspace{0.5cm}

    \begin{exampleblock}{试卷二值化建议}
        \begin{itemize}
            \item \textbf{扫描件}:Otsu(速度快,效果好)
            \item \textbf{手机拍照}:自适应阈值(应对光照不均)
            \item \textbf{参数推荐}:blockSize=15, C=2-5
        \end{itemize}
    \end{exampleblock}
\end{frame}

\begin{frame}[fragile]{代码找茬环节}
    \textbf{找出以下代码的错误:}

    \begin{lstlisting}[basicstyle=\ttfamily\small]
# 有错误的代码
def binarize_exam(img_path):
    img = cv2.imread(img_path)  # 错误1
    _, binary = cv2.threshold(img, 127, 255, cv2.THRESH_BINARY)
    return binary
    \end{lstlisting}

    \vspace{0.3cm}

    \textbf{错误清单:}
    \begin{enumerate}
        \item \textbf{错误1}:\texttt{imread}默认读取彩色图,应加参数\texttt{0}读取灰度图
        \item \textbf{错误2}:\texttt{threshold}需要灰度图,彩色图会报错
        \item \textbf{错误3}:试卷识别应该用\texttt{THRESH\_BINARY\_INV}(反转黑白)
    \end{enumerate}

    \vspace{0.3cm}

    \textbf{修正后的代码:}
    \begin{lstlisting}[basicstyle=\ttfamily\small]
def binarize_exam(img_path):
    img = cv2.imread(img_path, 0)  # 读取灰度图
    _, binary = cv2.threshold(img, 127, 255, cv2.THRESH_BINARY_INV)
    return binary
    \end{lstlisting}
\end{frame}
