%===========================================================
% 05_binary_coding.tex - 二值化实战代码
%===========================================================

\begin{frame}[fragile]{二值化完整流程}
    \begin{lstlisting}[basicstyle=\ttfamily\tiny]
import cv2
import numpy as np

def binarize_exam(img_path, method='adaptive'):
    """
    试卷二值化处理
    method: 'global', 'otsu', 'adaptive'
    """
    # 读取图像
    img = cv2.imread(img_path, 0)

    if method == 'global':
        _, binary = cv2.threshold(img, 127, 255, cv2.THRESH_BINARY_INV)

    elif method == 'otsu':
        _, binary = cv2.threshold(img, 0, 255,
            cv2.THRESH_BINARY_INV + cv2.THRESH_OTSU)

    elif method == 'adaptive':
        # 推荐用于试卷
        binary = cv2.adaptiveThreshold(
            img, 255,
            cv2.ADAPTIVE_THRESH_GAUSSIAN_C,
            cv2.THRESH_BINARY_INV,
            15, 2
        )

    return binary

# 使用
result = binarize_exam('exam.jpg', method='adaptive')
cv2.imwrite('exam_binary.jpg', result)
    \end{lstlisting}
\end{frame}

\begin{frame}[fragile]{三种方法对比展示}
    \begin{lstlisting}[basicstyle=\ttfamily\tiny]
import matplotlib.pyplot as plt

# 读取图像
img = cv2.imread('exam.jpg', 0)

# 三种二值化方法
_, global_bin = cv2.threshold(img, 127, 255, cv2.THRESH_BINARY_INV)
_, otsu_bin = cv2.threshold(img, 0, 255,
    cv2.THRESH_BINARY_INV + cv2.THRESH_OTSU)
adaptive_bin = cv2.adaptiveThreshold(img, 255,
    cv2.ADAPTIVE_THRESH_GAUSSIAN_C,
    cv2.THRESH_BINARY_INV, 15, 2)

# 对比显示
fig, axes = plt.subplots(2, 2, figsize=(12, 10))
axes[0,0].imshow(img, cmap='gray')
axes[0,0].set_title('原图')
axes[0,1].imshow(global_bin, cmap='gray')
axes[0,1].set_title('全局阈值')
axes[1,0].imshow(otsu_bin, cmap='gray')
axes[1,0].set_title('Otsu')
axes[1,1].imshow(adaptive_bin, cmap='gray')
axes[1,1].set_title('自适应阈值(推荐)')
plt.show()
    \end{lstlisting}
\end{frame}

\begin{frame}{二值化方法总结}
    \begin{table}
        \centering
        \begin{tabular}{lccc}
            \toprule
            \textbf{方法} & \textbf{光照均匀} & \textbf{光照不均} & \textbf{速度} \\
            \midrule
            全局阈值 & 良好 & 差 & 快 \\
            Otsu & \textbf{优秀} & 差 & 快 \\
            自适应阈值 & 良好 & \textbf{优秀} & 较慢 \\
            \bottomrule
        \end{tabular}
    \end{table}

    \vspace{0.5cm}

    \begin{exampleblock}{试卷二值化建议}
        \begin{itemize}
            \item \textbf{扫描件}:Otsu(速度快,效果好)
            \item \textbf{手机拍照}:自适应阈值(应对光照不均)
            \item \textbf{参数推荐}:blockSize=15, C=2-5
        \end{itemize}
    \end{exampleblock}
\end{frame}
