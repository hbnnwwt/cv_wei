%===========================================================
% 00_binary_intro.tex - 二值化概述
%===========================================================

\section{图像二值化}

\begin{frame}{什么是二值化?}
    \begin{block}{定义}
        将灰度图像转换为只有黑白两种颜色的图像(0或255)
    \end{block}

    \vspace{0.3cm}

    \begin{columns}
        \column{0.5\textwidth}
        \textbf{转换规则:}
        $$f(x) = \begin{cases} 255 & \text{if } I(x,y) > T \\ 0 & \text{if } I(x,y) \le T \end{cases}$$

        \textbf{为什么要二值化?}
        \begin{itemize}
            \item 简化数据(减少信息量)
            \item 突出目标(文字边缘清晰)
            \item 便于后续处理(OCR输入)
        \end{itemize}

        \column{0.5\textwidth}
        \centering
        \begin{tikzpicture}
            \shade[ball color=gray] (0,0) circle (0.6cm);
            \node at (0,-1) {灰度图};
            \node at (1.5,0) {\Huge $\rightarrow$};
            \begin{scope}[shift={(3,0)}]
                \draw[fill=white] (-0.6,-0.6) rectangle (0.6,0.6);
                \fill[black] (-0.3,-0.3) rectangle (0,0.3);
                \fill[black] (0.1,-0.2) rectangle (0.6,0.2);
                \node at (0,-1) {二值图};
            \end{scope}
        \end{tikzpicture}
    \end{columns}
\end{frame}

\begin{frame}{二值化的应用场景}
    \begin{columns}
        \column{0.5\textwidth}
        \begin{block}{OCR 文字识别}
            \begin{itemize}
                \item 试卷答题卡识别
                \item 身份证识别
                \item 发票处理
                \item 文档数字化
            \end{itemize}
        \end{block}

        \column{0.5\textwidth}
        \begin{block}{其他应用}
            \begin{itemize}
                \item 边缘检测预处理
                \item 形态学操作输入
                \item 条形码识别
                \item 图像压缩
            \end{itemize}
        \end{block}
    \end{columns}

    \vspace{0.5cm}

    \begin{alertblock}{关键点}
        二值化质量直接影响 OCR 识别率!
    \end{alertblock}
\end{frame}
