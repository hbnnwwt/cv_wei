%===========================================================
% 02_otsu.tex - Otsu 自动阈值算法
%===========================================================

\begin{frame}{Otsu 算法原理}
    \textbf{核心思想:} 最大化类间方差

    \vspace{0.2cm}

    \textbf{算法步骤:}
    \begin{enumerate}
        \item 遍历所有可能的阈值 $T$
        \item 计算每个阈值下的类间方差 $\sigma^2$
        \item 选择使 $\sigma^2$ 最大的阈值作为最优阈值
    \end{enumerate}

    \vspace{0.2cm}

    $$
    \sigma^2 = \omega_0 \omega_1 (\mu_0 - \mu_1)^2
    $$

    其中:
    \begin{itemize}
        \item $\omega_0, \omega_1$:两类像素的比例
        \item $\mu_0, \mu_1$:两类像素的平均灰度
    \end{itemize}

    \vspace{0.2cm}

    \begin{block}{适用条件}
        直方图呈\textbf{双峰分布}(前景与背景明显分离)
    \end{block}
\end{frame}

\begin{frame}[fragile]{Otsu 算法代码实现}
    \begin{lstlisting}[basicstyle=\ttfamily\scriptsize]
# Otsu 自动阈值
ret, otsu = cv2.threshold(
    gray,
    0,                           # 自动计算
    255,
    cv2.THRESH_BINARY + cv2.THRESH_OTSU
)

print(f"Otsu 最优阈值: {ret}")

# 反色 Otsu(试卷常用)
ret, otsu_inv = cv2.threshold(
    gray,
    0,
    255,
    cv2.THRESH_BINARY_INV + cv2.THRESH_OTSU
)
    \end{lstlisting}

    \vspace{0.2cm}

    \begin{exampleblock}{Otsu 优势}
        \begin{itemize}
            \item 自动寻找最优阈值
            \item 适合光照均匀的图像
            \item 计算效率高
        \end{itemize}
    \end{exampleblock}
\end{frame}

\begin{frame}{Otsu 算法的局限}
    \begin{alertblock}{问题场景}
        \begin{itemize}
            \item 光照不均(单峰直方图)
            \item 噪声较多
            \item 前景背景比例悬殊
        \end{itemize}
    \end{alertblock}

    \vspace{0.3cm}

    \begin{columns}
        \column{0.5\textwidth}
        \textbf{适用:}
        \begin{itemize}
            \item 扫描仪扫描
            \item 光照均匀拍摄
            \item 双峰直方图
        \end{itemize}

        \column{0.5\textwidth}
        \textbf{不适用:}
        \begin{itemize}
            \item 手机拍照(光照不均)
            \item 阴影遮挡
            \item 复杂背景
        \end{itemize}
    \end{columns}

    \vspace{0.3cm}

    \textbf{解决方案:} 使用自适应阈值
\end{frame}
