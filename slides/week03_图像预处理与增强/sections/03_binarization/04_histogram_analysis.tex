%===========================================================
% 04_histogram_analysis.tex - 直方图分析与二值化质量评估
%===========================================================

\begin{frame}[fragile]{直方图指导阈值选择}
    \begin{lstlisting}[basicstyle=\ttfamily\scriptsize]
import cv2
import matplotlib.pyplot as plt

# 计算直方图
hist = cv2.calcHist([gray], [0], None, [256], [0, 256])

# 绘制直方图,标记 Otsu 阈值
plt.figure(figsize=(10, 4))
plt.plot(hist, color='black')
plt.axvline(x=ret, color='red', linestyle='--',
            label=f'Otsu阈值={ret:.1f}')
plt.xlabel('像素值')
plt.ylabel('像素数量')
plt.legend()
plt.title('灰度直方图与最优阈值')
plt.grid(alpha=0.3)
plt.show()
    \end{lstlisting}

    \vspace{0.2cm}

    \textbf{直方图分析:}
    \begin{itemize}
        \item \textbf{双峰明显}:适合 Otsu
        \item \textbf{单峰分布}:考虑自适应阈值
        \item \textbf{多峰分布}:可能需要分段处理
    \end{itemize}
\end{frame}

\begin{frame}[fragile]{二值化质量评估}
    \begin{lstlisting}[basicstyle=\ttfamily\scriptsize]
def evaluate_binary(binary_img):
    """
    评估二值化质量
    """
    # 黑白像素比例
    black_ratio = np.sum(binary_img == 0) / binary_img.size
    white_ratio = np.sum(binary_img == 255) / binary_img.size

    # 噪声评估(孤立白点)
    contours, _ = cv2.findContours(
        255 - binary_img,
        cv2.RETR_EXTERNAL,
        cv2.CHAIN_APPROX_SIMPLE
    )
    small_noise = sum(1 for c in contours if cv2.contourArea(c) < 10)

    return {
        'black_ratio': black_ratio,
        'white_ratio': white_ratio,
        'noise_count': small_noise
    }
    \end{lstlisting}

    \vspace{0.2cm}

    \textbf{质量指标:}
    \begin{itemize}
        \item 黑白比例合理(文档约 3:7 或 4:6)
        \item 噪声点数量少
        \item 字符边缘清晰
    \end{itemize}
\end{frame}
