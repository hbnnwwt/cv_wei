%===========================================================
% 01_global_thresh.tex - 全局阈值方法
%===========================================================

\begin{frame}[fragile]{全局阈值法}
    \textbf{原理:} 整个图像使用同一个阈值

    \begin{lstlisting}[basicstyle=\ttfamily\scriptsize]
# 固定阈值二值化
ret, binary = cv2.threshold(
    gray,                # 输入图像
    127,                 # 阈值
    255,                 # 最大值
    cv2.THRESH_BINARY    # 类型
)

# 反色二值化(试卷常用)
ret, binary_inv = cv2.threshold(
    gray,
    127,
    255,
    cv2.THRESH_BINARY_INV  # 黑白反转
)
    \end{lstlisting}

    \vspace{0.2cm}

    \textbf{阈值类型:}
    \begin{table}
        \centering
        \small
        \begin{tabular}{ll}
            \toprule
            \textbf{类型} & \textbf{说明} \\
            \midrule
            THRESH\_BINARY & $>T \rightarrow 255$, $\le T \rightarrow 0$ \\
            THRESH\_BINARY\_INV & $>T \rightarrow 0$, $\le T \rightarrow 255$ \\
            THRESH\_TRUNC & $>T \rightarrow T$, $\le T \rightarrow$ 不变 \\
            THRESH\_TOZERO & $>T \rightarrow$ 不变, $\le T \rightarrow 0$ \\
            \bottomrule
        \end{tabular}
    \end{table}
\end{frame}

\begin{frame}{全局阈值的局限}
    \begin{alertblock}{问题}
        光照不均时,单一阈值无法适应所有区域!
    \end{alertblock}

    \vspace{0.3cm}

    \begin{columns}
        \column{0.5\textwidth}
        \textbf{阈值过高:}
        \begin{itemize}
            \item 亮部细节丢失
            \item 背景变为黑色
        \end{itemize}

        \column{0.5\textwidth}
        \textbf{阈值过低:}
        \begin{itemize}
            \item 暗部噪声保留
            \item 背景变为白色
        \end{itemize}
    \end{columns}

    \vspace{0.5cm}

    \begin{center}
        \textbf{解决方案:自适应阈值 / Otsu算法}
    \end{center}
\end{frame}
