%===========================================================
% 03_adaptive.tex - 自适应阈值
%===========================================================

\begin{frame}{自适应阈值原理}
    \textbf{问题:} 光照不均时,全局阈值无法适应

    \vspace{0.2cm}

    \textbf{解决方案:} 为每个像素点计算局部阈值

    \vspace{0.2cm}

    \begin{columns}
        \column{0.5\textwidth}
        \textbf{均值自适应:}
        $$T(x,y) = \text{mean}(\text{邻域}) - C$$

        \textbf{特点:}
        \begin{itemize}
            \item 计算简单
            \item 对噪声敏感
        \end{itemize}

        \column{0.5\textwidth}
        \textbf{高斯自适应:}
        $$T(x,y) = \text{加权均值}(\text{邻域}) - C$$

        \textbf{特点:}
        \begin{itemize}
            \item 加权平均
            \item 效果更平滑
            \item \textbf{推荐使用}
        \end{itemize}
    \end{columns}
\end{frame}

\begin{frame}[fragile]{自适应阈值代码实现}
    \begin{lstlisting}[basicstyle=\ttfamily\scriptsize]
# 自适应阈值 - 均值法
adaptive_mean = cv2.adaptiveThreshold(
    gray,
    255,
    cv2.ADAPTIVE_THRESH_MEAN_C,
    cv2.THRESH_BINARY,
    11,     # 邻域大小(奇数)
    2       # 常数 C
)

# 自适应阈值 - 高斯法(推荐)
adaptive_gaussian = cv2.adaptiveThreshold(
    gray,
    255,
    cv2.ADAPTIVE_THRESH_GAUSSIAN_C,
    cv2.THRESH_BINARY_INV,  # 反色,试卷常用
    11,     # 邻域大小(奇数)
    2       # 常数 C
)
    \end{lstlisting}

    \vspace{0.2cm}

    \textbf{参数说明:}
    \begin{itemize}
        \item \textbf{blockSize}:邻域大小,通常 11、15、21(必须是奇数)
        \item \textbf{C}:从均值中减去的常数,用于微调
    \end{itemize}
\end{frame}

\begin{frame}{自适应阈值参数调优}
    \textbf{blockSize 选择:}
    \begin{itemize}
        \item 太小(如 5):对噪声敏感,产生细碎斑点
        \item \textbf{适中(11-21):推荐范围}
        \item 太大(如 31):接近全局阈值,失去自适应效果
    \end{itemize}

    \vspace{0.3cm}

    \textbf{C 值选择:}
    \begin{itemize}
        \item 正值:使阈值降低,更多像素变为黑色
        \item 负值:使阈值升高,更多像素变为白色
        \item \textbf{通常取 2-10}
    \end{itemize}

    \vspace{0.3cm}

    \begin{exampleblock}{试卷二值化推荐配置}
        \texttt{blockSize=15, C=5, THRESH\_BINARY\_INV}
    \end{exampleblock}
\end{frame}
