%===========================================================
% 03_geometry_coding.tex - 几何变换实战代码
%===========================================================

\begin{frame}[fragile]{完整的文档矫正代码}
    \begin{lstlisting}[basicstyle=\ttfamily\tiny]
def four_point_transform(img, pts):
    """
    四点透视变换
    """
    # 排序四个角点
    rect = order_points(pts)
    (tl, tr, br, bl) = rect

    # 计算输出尺寸
    widthA = np.linalg.norm(br - bl)
    widthB = np.linalg.norm(tr - tl)
    maxWidth = max(int(widthA), int(widthB))

    heightA = np.linalg.norm(tr - br)
    heightB = np.linalg.norm(tl - bl)
    maxHeight = max(int(heightA), int(heightB))

    # 目标点
    dst = np.array([
        [0, 0],
        [maxWidth - 1, 0],
        [maxWidth - 1, maxHeight - 1],
        [0, maxHeight - 1]
    ], dtype="float32")

    # 计算变换矩阵并应用
    M = cv2.getPerspectiveTransform(rect, dst)
    warped = cv2.warpPerspective(img, M, (maxWidth, maxHeight))

    return warped
    \end{lstlisting}
\end{frame}

\begin{frame}[fragile]{几何变换总结}
    \begin{table}
        \centering
        \begin{tabular}{lcc}
            \toprule
            \textbf{变换类型} & \textbf{所需点数} & \textbf{应用场景} \\
            \midrule
            平移 & 1个位移向量 & 图像移动 \\
            旋转 & 1个中心点+角度 & 旋转修正 \\
            仿射 & 3个点对 & 倾斜矫正 \\
            透视 & 4个点对 & \textbf{文档矫正} \\
            \bottomrule
        \end{tabular}
    \end{table}

    \vspace{0.3cm}

    \begin{exampleblock}{试卷矫正推荐方案}
        \begin{enumerate}
            \item 边缘检测(Canny)
            \item 找最大轮廓并近似为四边形
            \item 排序四个角点
            \item 透视变换矫正
        \end{enumerate}
    \end{exampleblock}

    \begin{alertblock}{注意}
        下周将深入讲解边缘检测与轮廓查找!
    \end{alertblock}
\end{frame}
