%===========================================================
% 00_affine.tex - 仿射变换
%===========================================================

\section{几何变换}

\begin{frame}{仿射变换基础}
    \textbf{仿射变换:} 保持"平直线"和"平行性"的变换

    \vspace{0.3cm}

    \textbf{包含的操作:}
    \begin{itemize}
        \item 平移(Translation)
        \item 旋转(Rotation)
        \item 缩放(Scaling)
        \item 倾斜/剪切(Shear)
    \end{itemize}

    \vspace{0.3cm}

    \textbf{变换矩阵(2×3):}
    $$
    M = \begin{bmatrix}
    a & b & c \\
    d & e & f
    \end{bmatrix}
    $$

    新坐标:$\begin{bmatrix} x' \\ y' \end{bmatrix} = \begin{bmatrix} a & b \\ d & e \end{bmatrix} \begin{bmatrix} x \\ y \end{bmatrix} + \begin{bmatrix} c \\ f \end{bmatrix}$
\end{frame}

\begin{frame}[fragile]{仿射变换代码实现}
    \begin{lstlisting}[basicstyle=\ttfamily\scriptsize]
import cv2
import numpy as np

# 图像平移
def translate(img, x, y):
    M = np.float32([[1, 0, x], [0, 1, y]])
    return cv2.warpAffine(img, M, (img.shape[1], img.shape[0]))

# 图像旋转
def rotate(img, angle, center=None):
    h, w = img.shape[:2]
    if center is None:
        center = (w // 2, h // 2)

    M = cv2.getRotationMatrix2D(center, angle, 1.0)
    return cv2.warpAffine(img, M, (w, h))

# 使用
translated = translate(img, 50, 30)   # 向右50,向下30
rotated = rotate(img, 15)             # 旋转15度
    \end{lstlisting}

    \vspace{0.2cm}

    \textbf{注意:} 仿射变换需要3个点对来确定变换矩阵
\end{frame}

\begin{frame}{仿射变换应用场景}
    \begin{columns}
        \column{0.5\textwidth}
        \begin{block}{文档矫正}
            \begin{itemize}
                \item 轻微倾斜修正
                \item 水平对齐
            \end{itemize}
        \end{block}

        \column{0.5\textwidth}
        \begin{block}{图像增强}
            \begin{itemize}
                \item 数据增强
                \item 随机变换
            \end{itemize}
        \end{block}
    \end{columns}

    \vspace{0.3cm}

    \begin{alertblock}{局限}
        仿射变换不能处理透视变形(近大远小)
    \end{alertblock}
\end{frame}
