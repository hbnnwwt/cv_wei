%===========================================================
% 02_auto_correct.tex - 自动文档矫正
%===========================================================

\begin{frame}{自动文档矫正流程}
    \textbf{核心步骤:}
    \begin{enumerate}
        \item 边缘检测(Canny)
        \item 轮廓查找(findContours)
        \item 四边形近似(approxPolyDP)
        \item 透视变换
    \end{enumerate}

    \vspace{0.3cm}

    \begin{alertblock}{关键点}
        如何从图像中找到四个角点?
    \end{alertblock}

    \begin{itemize}
        \item 假设最大轮廓是文档边缘
        \item 将轮廓近似为四边形
        \item 按顺序排列四个角点
    \end{itemize}
\end{frame}

\begin{frame}[fragile]{自动矫正代码框架}
    \begin{lstlisting}[basicstyle=\ttfamily\tiny]
def auto_correct_document(img):
    # 1. 预处理
    gray = cv2.cvtColor(img, cv2.COLOR_BGR2GRAY)
    blur = cv2.GaussianBlur(gray, (5, 5), 0)
    edged = cv2.Canny(blur, 50, 150)

    # 2. 查找轮廓
    contours, _ = cv2.findContours(
        edged, cv2.RETR_EXTERNAL, cv2.CHAIN_APPROX_SIMPLE
    )

    # 3. 找到最大轮廓(假设是文档)
    doc_contour = max(contours, key=cv2.contourArea)

    # 4. 近似为四边形
    peri = cv2.arcLength(doc_contour, True)
    approx = cv2.approxPolyDP(doc_contour, 0.02 * peri, True)

    # 5. 透视变换
    if len(approx) == 4:
        # 获取四个角点并排序
        pts = order_points(approx.reshape(4, 2))
        # ... 执行透视变换

    return result
    \end{lstlisting}
\end{frame}

\begin{frame}{角点排序}
    \textbf{问题:} 轮廓检测返回的四个点顺序是随机的

    \vspace{0.2cm}

    \textbf{解决方案:} 统一排序为 [左上, 右上, 右下, 左下]

    \begin{lstlisting}[basicstyle=\ttfamily\scriptsize]
def order_points(pts):
    """
    将四个点按左上、右上、右下、左下排序
    """
    # 按x坐标排序
    x_sorted = pts[np.argsort(pts[:, 0]), :]

    # 分为左右两组
    left = x_sorted[:2, :]
    right = x_sorted[2:, :]

    # 左边按y排序(上→下)
    left = left[np.argsort(left[:, 1]), :]
    # 右边按y排序(上→下)
    right = right[np.argsort(right[:, 1]), :]

    # 返回:左上、右上、右下、左下
    return np.array([
        left[0],   # 左上
        right[0],  # 右上
        right[1],  # 右下
        left[1]    # 左下
    ])
    \end{lstlisting}
\end{frame}
