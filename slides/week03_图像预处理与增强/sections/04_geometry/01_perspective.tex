%===========================================================
% 01_perspective.tex - 透视变换
%===========================================================

\begin{frame}{透视变换原理}
    \textbf{问题:} 拍照时相机与试卷不平行,产生透视变形

    \vspace{0.3cm}

    \begin{columns}
        \column{0.5\textwidth}
        \centering
        \begin{tikzpicture}
            \draw[fill=gray!20] (0,0) -- (2.5,0.5) -- (2.8,3.5) -- (0.3,3) -- cycle;
            \node at (1.5,1.8) {透视变形};
        \end{tikzpicture}

        \column{0.5\textwidth}
        \centering
        \begin{tikzpicture}
            \draw[fill=white] (0,0) rectangle (2.5,3.5);
            \node at (1.25,1.75) {矫正结果};
        \end{tikzpicture}
    \end{columns}

    \vspace{0.5cm}

    \textbf{透视变换矩阵(3×3):}
    $$
    H = \begin{bmatrix}
    h_{00} & h_{01} & h_{02} \\
    h_{10} & h_{11} & h_{12} \\
    h_{20} & h_{21} & h_{22}
    \end{bmatrix}
    $$

    需要\textbf{4个点对}来确定变换矩阵
\end{frame}

\begin{frame}[fragile]{透视变换实现}
    \begin{lstlisting}[basicstyle=\ttfamily\scriptsize]
import numpy as np
import cv2

# 定义四个角点(原图像)
pts1 = np.float32([
    [100, 150],   # 左上
    [450, 120],   # 右上
    [480, 380],   # 右下
    [80, 400]     # 左下
])

# 定义目标矩形
width, height = 400, 300
pts2 = np.float32([
    [0, 0],                    # 左上
    [width - 1, 0],            # 右上
    [width - 1, height - 1],   # 右下
    [0, height - 1]            # 左下
])

# 计算透视变换矩阵
M = cv2.getPerspectiveTransform(pts1, pts2)

# 应用透视变换
result = cv2.warpPerspective(img, M, (width, height))
    \end{lstlisting}
\end{frame}

\begin{frame}[fragile]{输出尺寸计算}
    \textbf{如何确定输出尺寸?}

    \begin{lstlisting}[basicstyle=\ttfamily\scriptsize]
import numpy as np

def calculate_output_size(pts):
    """
    根据四个角点计算输出尺寸
    """
    # 计算宽度(取上下边长的最大值)
    w1 = np.linalg.norm(pts[1] - pts[0])
    w2 = np.linalg.norm(pts[2] - pts[3])
    width = max(int(w1), int(w2))

    # 计算高度(取左右边长的最大值)
    h1 = np.linalg.norm(pts[3] - pts[0])
    h2 = np.linalg.norm(pts[2] - pts[1])
    height = max(int(h1), int(h2))

    return width, height

# 使用
width, height = calculate_output_size(pts1)
    \end{lstlisting}
\end{frame}
