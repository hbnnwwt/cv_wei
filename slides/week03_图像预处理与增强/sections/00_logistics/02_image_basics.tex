%===========================================================
% 02_image_basics.tex - 数字图像基础回顾
%===========================================================

\section{数字图像基础}

\begin{frame}{图像的基本概念}
    \begin{columns}
        \column{0.6\textwidth}
        \textbf{核心概念:}
        \begin{itemize}
            \item \textbf{像素}:图像的最小单位
            \item \textbf{分辨率}:宽度 × 高度(如 1920×1080)
            \item \textbf{位深}:每个像素的比特数(8bit = 256级灰度)
        \end{itemize}

        \vspace{0.3cm}
        \textbf{OpenCV 中的图像表示:}
        \begin{lstlisting}[language=Python, basicstyle=\ttfamily\scriptsize]
import cv2
import numpy as np

# 读取图像
img = cv2.imread('exam.jpg')

# 图像是 numpy 数组
print(type(img))  # <class 'numpy.ndarray'>
print(img.shape)  # (height, width, channels)
        \end{lstlisting}

        \column{0.4\textwidth}
        \centering
        \begin{tikzpicture}
            \draw[step=0.5, gray!30] (0,0) grid (4,4);
            \fill[red] (1,3) rectangle (1.5,3.5);
            \node at (2,2) {图像 = 像素矩阵};
            \node[font=\tiny] at (1.25,3.25) {像素};
        \end{tikzpicture}
    \end{columns}
\end{frame}

\begin{frame}[fragile]{色彩空间与图像格式}
    \textbf{常见色彩空间:}
    \begin{table}
        \centering
        \small
        \begin{tabular}{lp{5cm}l}
            \toprule
            \textbf{色彩空间} & \textbf{特点} & \textbf{应用} \\
            \midrule
            RGB & 红、绿、蓝三通道 & 显示、存储 \\
            GRAY & 单通道灰度 & 图像处理、OCR \\
            HSV & 色调、饱和度、亮度 & 颜色分割 \\
            YCrCb & 亮度与色度分离 & 视频压缩 \\
            \bottomrule
        \end{tabular}
    \end{table}

    \vspace{0.3cm}

    \textbf{图像格式对比:}
    \begin{itemize}
        \item \textbf{PNG}:无损压缩,适合处理中间结果
        \item \textbf{JPEG}:有损压缩,不适合后续处理
        \item \textbf{BMP}:无压缩,文件较大
    \end{itemize}

    \begin{lstlisting}[language=Python]
# 色彩空间转换
gray = cv2.cvtColor(img, cv2.COLOR_BGR2GRAY)
hsv = cv2.cvtColor(img, cv2.COLOR_BGR2HSV)
    \end{lstlisting}
\end{frame}
