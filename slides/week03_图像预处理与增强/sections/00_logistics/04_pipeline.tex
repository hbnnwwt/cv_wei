%===========================================================
% 04_pipeline.tex - 预处理流水线详解
%===========================================================

\section{预处理流水线}

\begin{frame}{标准预处理流水线}
    \begin{center}
        \begin{tikzpicture}[node distance=1.5cm, auto,
            block/.style={draw, rectangle, rounded corners, fill=blue!10, minimum height=1cm, minimum width=2cm},
            arrow/.style={->, thick}]

            \node[block] (input) {原始图像};
            \node[block, right of=input, node distance=2.5cm, fill=yellow!10] (denoise) {去噪};
            \node[block, right of=denoise, node distance=2.5cm, fill=green!10] (enhance) {增强};
            \node[block, right of=enhance, node distance=2.5cm, fill=orange!10] (binary) {二值化};
            \node[block, right of=binary, node distance=2.5cm, fill=red!10] (correct) {几何矫正};
            \node[block, right of=correct, node distance=2.5cm, fill=purple!10] (output) {输出};

            \draw[arrow] (input) -- (denoise);
            \draw[arrow] (denoise) -- (enhance);
            \draw[arrow] (enhance) -- (binary);
            \draw[arrow] (binary) -- (correct);
            \draw[arrow] (correct) -- (output);
        \end{tikzpicture}
    \end{center}

    \vspace{0.5cm}

    \textbf{各步骤作用:}
    \begin{table}
        \centering
        \small
        \begin{tabular}{lp{8cm}}
            \toprule
            \textbf{步骤} & \textbf{作用} \\
            \midrule
            去噪 & 消除拍摄噪声、传感器噪声 \\
            增强 & 提升对比度、恢复细节 \\
            二值化 & 简化数据、突出目标 \\
            几何矫正 & 消除透视变形、统一尺寸 \\
            \bottomrule
        \end{tabular}
    \end{table}
\end{frame}

\begin{frame}{流水线顺序的重要性}
    \begin{alertblock}{错误顺序示例}
        先二值化 $\rightarrow$ 再去噪 $\times$

        \textbf{问题:} 二值化后的图像只有0和255,滤波效果极差!
    \end{alertblock}

    \vspace{0.3cm}

    \begin{exampleblock}{正确顺序}
        去噪 $\rightarrow$ 增强 $\rightarrow$ 二值化 $\rightarrow$ 矫正 \checkmark
    \end{exampleblock}

    \vspace{0.3cm}

    \textbf{本课程学习顺序:}
    \begin{enumerate}
        \item 图像去噪(消除噪声)
        \item 图像增强(提升质量)
        \item 图像二值化(简化数据)
        \item 几何变换(矫正变形)
    \end{enumerate}
\end{frame}

\begin{frame}{课程概览}
    \tableofcontents
\end{frame}
