%===========================================================
% 00_prerequisites.tex - 预备知识
%===========================================================

\section{预备知识}

\begin{frame}{课前预备知识}
    \begin{alertblock}{课前学习(5分钟视频)}
        如果你是非计算机专业背景,请先观看以下预备知识视频:
    \end{alertblock}

    \vspace{0.3cm}

    \begin{columns}
        \column{0.5\textwidth}
        \begin{block}{1. 图像的数字表示}
            \begin{itemize}
                \item 像素:图像的最小单位
                \item 灰度图:0-255的数字矩阵
                \item RGB:三个颜色通道叠加
            \end{itemize}
        \end{block}

        \column{0.5\textwidth}
        \begin{block}{2. 卷积运算基础}
            \begin{itemize}
                \item 卷积核:一个小矩阵
                \item 滑动窗口:逐像素计算
                \item 加权求和:得到新像素值
            \end{itemize}
        \end{block}
    \end{columns}

    \vspace{0.3cm}

    \begin{exampleblock}{3. NumPy数组操作}
        \begin{itemize}
            \item 图像 = NumPy数组
            \item \texttt{img.shape} 获取尺寸
            \item \texttt{img[y, x]} 访问像素
        \end{itemize}
    \end{exampleblock}
\end{frame}

\begin{frame}[fragile]{快速回顾:图像是数字矩阵}
    \begin{columns}
        \column{0.6\textwidth}
        \textbf{灰度图示例:}
        \begin{lstlisting}[language=Python, basicstyle=\ttfamily\scriptsize]
import cv2
import numpy as np

# 读取图像
img = cv2.imread('exam.jpg', 0)

# 图像是一个数字矩阵
print(img.shape)  # (高度, 宽度)
print(img.dtype)  # uint8 (0-255)

# 访问单个像素
pixel_value = img[100, 200]
print(f"像素值: {pixel_value}")
        \end{lstlisting}

        \column{0.4\textwidth}
        \centering
        \begin{tikzpicture}
            \draw[step=0.5, gray!30] (0,0) grid (3,3);
            \foreach \x in {0.25,0.75,...,2.75}
                \foreach \y in {0.25,0.75,...,2.75}
                    \node[font=\tiny] at (\x,\y) {\pgfmathsetmacro{\val}{int(random(0,255))}\val};
            \node at (1.5,-0.5) {图像 = 数字矩阵};
        \end{tikzpicture}
    \end{columns}

    \vspace{0.3cm}

    \begin{itemize}
        \item 每个像素是一个数字(0=黑,255=白)
        \item 滤波 = 对这些数字进行数学运算
    \end{itemize}
\end{frame}

\begin{frame}{快速回顾:卷积运算}
    \textbf{什么是卷积?}
    \begin{itemize}
        \item 用一个小矩阵(卷积核)在图像上滑动
        \item 对每个位置,计算加权求和
        \item 得到新的像素值
    \end{itemize}

    \vspace{0.3cm}

    \begin{center}
        \begin{tikzpicture}[scale=0.7]
            % 原图像区域
            \draw[step=1, gray] (0,0) grid (3,3);
            \node at (1.5,-0.5) {原图像};

            % 卷积核
            \draw[step=1, blue, thick] (4,0) grid (6,2);
            \node at (5,-0.5) {卷积核};

            % 结果
            \draw[step=1, green!50!black] (7,0) grid (9,3);
            \node at (8,-0.5) {结果};

            % 箭头
            \draw[->, thick] (3.2,1) -- (3.8,1);
            \draw[->, thick] (6.2,1) -- (6.8,1);
        \end{tikzpicture}
    \end{center}

    \vspace{0.3cm}

    \begin{exampleblock}{本节课的核心}
        所有的图像处理(去噪、增强、边缘检测)都是不同卷积核的应用!
    \end{exampleblock}
\end{frame}
