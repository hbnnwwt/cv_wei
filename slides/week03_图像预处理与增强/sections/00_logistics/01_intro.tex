%===========================================================
% 01_intro.tex - 预处理的重要性
%===========================================================

\section{为什么需要预处理?}

\begin{frame}{预处理是计算机视觉的"门卫"}
    \begin{columns}
        \column{0.5\textwidth}
        \begin{alertblock}{现实问题}
            \begin{itemize}
                \item 拍摄角度不正
                \item 光照不均匀
                \item 纸张有折痕
                \item 背景有杂物
                \item \textbf{手机拍照 vs 扫描仪}
            \end{itemize}
        \end{alertblock}

        \column{0.5\textwidth}
        \begin{exampleblock}{预处理目标}
            \begin{enumerate}
                \item 去除噪声干扰
                \item 增强目标特征
                \item 规范图像格式
                \item \textbf{提升识别准确率}
            \end{enumerate}
        \end{exampleblock}
    \end{columns}

    \vspace{0.5cm}
    \centering
    \textbf{\textcolor{red}{预处理质量直接决定后续识别效果!}}
\end{frame}

\begin{frame}{真实场景案例分析}
    \begin{columns}
        \column{0.33\textwidth}
        \centering
        \textbf{低光照拍摄}\\
        \vspace{0.2cm}
        \begin{tikzpicture}
            \draw[fill=gray!30] (0,0) rectangle (2.5,3);
            \node at (1.25,1.5) {暗淡有噪点};
        \end{tikzpicture}

        \column{0.33\textwidth}
        \centering
        \textbf{光照不均}\\
        \vspace{0.2cm}
        \begin{tikzpicture}
            \shade[left color=white, right color=gray!50] (0,0) rectangle (2.5,3);
            \node at (1.25,1.5) {阴影干扰};
        \end{tikzpicture}

        \column{0.33\textwidth}
        \centering
        \textbf{拍摄角度}\\
        \vspace{0.2cm}
        \begin{tikzpicture}
            \draw[fill=blue!10, rotate=10] (0,0) rectangle (2.5,3);
            \node at (1.25,1.5) {透视变形};
        \end{tikzpicture}
    \end{columns}

    \vspace{0.5cm}

    \begin{block}{核心结论}
        没有好的预处理,再先进的算法也无法发挥威力!
    \end{block}
\end{frame}
