%===========================================================
% 05_learning_modes.tex - 学习模式说明
%===========================================================

\section{学习模式}

\begin{frame}{三种学习路径}
    根据你的基础和时间,选择适合的学习模式:

    \vspace{0.5cm}

    \begin{columns}
        \column{0.32\textwidth}
        \begin{block}{观察者模式}
            \textbf{适合:}
            \begin{itemize}
                \item 编程基础较弱
                \item 时间有限
                \item 侧重理解原理
            \end{itemize}

            \vspace{0.2cm}
            \textbf{学习方式:}
            \begin{itemize}
                \item 理解滤波器原理
                \item 观看演示效果
                \item 运行完整代码
                \item 修改参数对比
            \end{itemize}

            \vspace{0.2cm}
            \textbf{目标:} 能解释预处理的作用
        \end{block}

        \column{0.32\textwidth}
        \begin{block}{使用者模式}
            \textbf{适合:}
            \begin{itemize}
                \item 有一定编程基础
                \item 能独立调试
                \item 希望动手实践
            \end{itemize}

            \vspace{0.2cm}
            \textbf{学习方式:}
            \begin{itemize}
                \item 运行代码模板
                \item 调整滤波参数
                \item 对比不同方法
                \item 调试代码错误
            \end{itemize}

            \vspace{0.2cm}
            \textbf{目标:} 能完成基本预处理任务
        \end{block}

        \column{0.32\textwidth}
        \begin{block}{创造者模式}
            \textbf{适合:}
            \begin{itemize}
                \item 编程基础扎实
                \item 喜欢挑战
                \item 追求深入理解
            \end{itemize}

            \vspace{0.2cm}
            \textbf{学习方式:}
            \begin{itemize}
                \item 从零编写代码
                \item 优化算法性能
                \item 探索新方法
                \item 创新应用场景
            \end{itemize}

            \vspace{0.2cm}
            \textbf{目标:} 能设计预处理方案
        \end{block}
    \end{columns}
\end{frame}

\begin{frame}{本周任务分层}
    \begin{table}
        \centering
        \begin{tabular}{lp{8cm}l}
            \toprule
            \textbf{模式} & \textbf{任务要求} & \textbf{评分} \\
            \midrule
            观察者 & 理解原理,运行代码,完成实验报告 & 最高60分 \\
            使用者 & 修改参数,对比效果,调试代码 & 最高85分 \\
            创造者 & 编写代码,优化算法,创新应用 & 100分+15分 \\
            \bottomrule
        \end{tabular}
    \end{table}

    \vspace{0.5cm}

    \begin{alertblock}{重要提示}
        \begin{itemize}
            \item 所有模式都需要提交实验报告
            \item 可以在学期中随时切换到更高级模式
            \item 鼓励跨越模式学习,挑战自我
        \end{itemize}
    \end{alertblock}
\end{frame}

\begin{frame}[fragile]{AI辅助:三种模式的Prompt示例}
    \textbf{观察者模式Prompt:}
    \begin{lstlisting}
"请解释中值滤波和高斯滤波的区别,用通俗的语言"
    \end{lstlisting}

    \vspace{0.3cm}

    \textbf{使用者模式Prompt:}
    \begin{lstlisting}
"帮我用OpenCV实现中值滤波去噪,并解释每个参数的作用"
    \end{lstlisting}

    \vspace{0.3cm}

    \textbf{创造者模式Prompt:}
    \begin{lstlisting}
"如何设计一个自适应滤波器,能根据噪声类型自动选择
最合适的滤波方法?请给出完整方案和代码实现"
    \end{lstlisting}

    \vspace{0.3cm}

    \begin{exampleblock}{提示}
        选择适合你当前水平的Prompt,逐步提升难度!
    \end{exampleblock}
\end{frame}
