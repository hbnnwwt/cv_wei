%===========================================================
% 06_grouping.tex - 分组策略说明
%===========================================================

\section{分组策略}

\begin{frame}{分组原则}
    \begin{columns}
        \column{0.5\textwidth}
        \textbf{分组规则:}
        \begin{itemize}
            \item 每组 \textbf{4人}
            \item 不同专业背景混合
            \item 强弱搭配,鼓励互助
        \end{itemize}

        \vspace{0.3cm}

        \textbf{为什么需要分组?}
        \begin{itemize}
            \item 计算机视觉是跨学科领域
            \item 团队协作是真实工作场景
            \item 互相学习,共同进步
        \end{itemize}

        \column{0.5\textwidth}
        \begin{alertblock}{分组示例}
            \textbf{理想组合:}
            \begin{itemize}
                \item 1人:计算机/软件专业
                \item 1人:数学/统计专业
                \item 1人:其他理工科专业
                \item 1人:文科/商科专业
            \end{itemize}
        \end{alertblock}
    \end{columns}
\end{frame}

\begin{frame}{角色分工}
    \begin{columns}
        \column{0.5\textwidth}
        \begin{block}{组长}
            \begin{itemize}
                \item 协调小组进度
                \item 分配任务
                \item 组织讨论
                \item 与教师沟通
            \end{itemize}
        \end{block}

        \begin{block}{技术负责人}
            \begin{itemize}
                \item 把关代码质量
                \item 解决技术难题
                \item 审查代码规范
                \item 指导其他成员
            \end{itemize}
        \end{block}

        \column{0.5\textwidth}
        \begin{block}{开发者A}
            \begin{itemize}
                \item 负责去噪模块
                \item 编写相关代码
                \item 准备模块演示
                \item 编写模块文档
            \end{itemize}
        \end{block}

        \begin{block}{开发者B}
            \begin{itemize}
                \item 负责增强模块
                \item 编写相关代码
                \item 准备模块演示
                \item 编写模块文档
            \end{itemize}
        \end{block}
    \end{columns}
\end{frame}

\begin{frame}{本周协作任务}
    \textbf{小组任务:构建完整的试卷预处理系统}

    \vspace{0.3cm}

    \begin{table}
        \centering
        \small
        \begin{tabular}{lp{6cm}l}
            \toprule
            \textbf{角色} & \textbf{本周任务} & \textbf{交付物} \\
            \midrule
            组长 & 协调进度,整合代码,准备汇报 & 完整系统代码 \\
            技术负责人 & 审查代码,解决bug,优化性能 & 代码审查报告 \\
            开发者A & 实现去噪功能,对比不同方法 & 去噪模块+测试 \\
            开发者B & 实现增强功能,调试参数 & 增强模块+测试 \\
            \bottomrule
        \end{tabular}
    \end{table}

    \vspace{0.3cm}

    \begin{exampleblock}{协作流程}
        1. 讨论设计方案 $\rightarrow$ 2. 分模块开发 $\rightarrow$ 3. 代码集成 $\rightarrow$ 4. 互相测试 $\rightarrow$ 5. 准备汇报
    \end{exampleblock}
\end{frame}

\begin{frame}{分组时间安排}
    \textbf{第5分钟:快速分组}

    \vspace{0.2cm}

    \begin{enumerate}
        \item 自由组队(找队友)
        \item 确定角色(谁做什么)
        \item 建立群组(微信/QQ)
        \item 共享代码(GitHub/Gitee)
    \end{enumerate}

    \vspace{0.3cm}

    \textbf{协作工具推荐:}
    \begin{itemize}
        \item \textbf{代码共享}:GitHub、Gitee
        \item \textbf{实时协作}:腾讯文档、石墨文档
        \item \textbf{沟通}:微信群、Discord
        \item \textbf{AI辅助}:共享Prompt模板
    \end{itemize}
\end{frame}

\begin{frame}{互助机制}
    \begin{columns}
        \column{0.5\textwidth}
        \begin{block}{强帮弱}
            \begin{itemize}
                \item 编程好的帮助基础弱的
                \item 数学好的帮助理论弱的
                \item 互相讲解,共同进步
            \end{itemize}
        \end{block}

        \column{0.5\textwidth}
        \begin{block}{同伴学习}
            \begin{itemize}
                \item 代码互审
                \item 问题共解
                \item 成果共享
            \end{itemize}
        \end{block}
    \end{columns}

    \vspace{0.5cm}

    \begin{alertblock}{评价机制}
        \begin{itemize}
            \item 个人贡献:60分(完成自己的任务)
            \item 小组协作:20分(帮助他人、代码质量)
            \item 团队成果:20分(系统完整性、演示效果)
        \end{itemize}
    \end{alertblock}
\end{frame}
