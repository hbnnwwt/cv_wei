%===========================================================
% 07_ai_assistant.tex - AI辅助编程说明
%===========================================================

\section{AI辅助编程}

\begin{frame}{AI编程工具回顾}
    \textbf{Week 2核心工具:}

    \vspace{0.3cm}

    \begin{columns}
        \column{0.5\textwidth}
        \begin{block}{Cursor}
            \begin{itemize}
                \item VS Code增强版
                \item AI自动补全
                \item 自然语言生成代码
            \end{itemize}
        \end{block}

        \column{0.5\textwidth}
        \begin{block}{Claude Code}
            \begin{itemize}
                \item 命令行AI助手
                \item 文件操作专家
                \item 问题诊断能手
            \end{itemize}
        \end{block}
    \end{columns}

    \vspace{0.5cm}

    \textbf{本节课AI应用场景:}
    \begin{itemize}
        \item 理解卷积运算原理
        \item 生成对比实验代码
        \item 调试滤波参数
        \item 优化预处理流程
    \end{itemize}
\end{frame}

\begin{frame}[fragile]{AI辅助:理解概念}
    \textbf{示例1:理解卷积运算}

    \vspace{0.2cm}

    \textbf{Prompt:}
    \begin{lstlisting}
"请用通俗的语言解释什么是卷积运算,为什么它能用于图像去噪?
用3x3矩阵的例子说明"
    \end{lstlisting}

    \vspace{0.3cm}

    \textbf{示例2:理解滤波器区别}
    \begin{lstlisting}
"高斯滤波和中值滤波有什么区别?分别适合什么类型的噪声?
用表格对比说明"
    \end{lstlisting}

    \vspace{0.3cm}

    \begin{exampleblock}{提示}
        理解概念时,可以让AI"用通俗语言解释"、"用例子说明"。
    \end{exampleblock}
\end{frame}

\begin{frame}[fragile]{AI辅助:生成代码}
    \textbf{示例1:生成去噪代码}

    \vspace{0.2cm}

    \textbf{Prompt(RTF框架):}
    \begin{lstlisting}
# Role
你是一个OpenCV图像处理专家

# Task
编写一个函数,对试卷图像进行去噪处理:
1. 使用中值滤波(kernel=5)
2. 计算并返回PSNR值
3. 包含完整的注释

# Format
返回Python代码,使用cv2和numpy
    \end{lstlisting}

    \vspace{0.3cm}

    \textbf{示例2:生成对比代码}
    \begin{lstlisting}
"生成Python代码,对比高斯滤波和中值滤波对椒盐噪声的效果,
使用matplotlib并排显示结果"
    \end{lstlisting}
\end{frame}

\begin{frame}[fragile]{AI辅助:调试代码}
    \textbf{示例1:调试错误}
    \begin{lstlisting}
# 错误代码
img = cv2.imread('exam.jpg')
_, binary = cv2.threshold(img, 127, 255, cv2.THRESH_BINARY)
# 报错:error: (-215:Assertion failed)

# Prompt
"这段代码报错error: (-215:Assertion failed),请帮我找出问题并修正"
    \end{lstlisting}

    \vspace{0.3cm}

    \textbf{示例2:优化参数}
    \begin{lstlisting}
"CLAHE处理后对比度过高,文字细节丢失,应该如何调整参数?
当前参数:clipLimit=2.0, tileGridSize=(8,8)"
    \end{lstlisting}

    \vspace{0.3cm}

    \begin{alertblock}{AI调试三部曲}
        1. 描述问题 + 粘贴代码 $\rightarrow$ AI分析
        2. 理解AI的建议 $\rightarrow$ 判断是否正确
        3. 测试修复方案 $\rightarrow$ 验证效果
    \end{alertblock}
\end{frame}

\begin{frame}[fragile]{AI辅助:参数调优}
    \textbf{场景:} 自适应阈值效果不理想

    \vspace{0.2cm}

    \textbf{Prompt:}
    \begin{lstlisting}
# 背景
试卷光照不均,使用自适应阈值二值化后,部分区域文字丢失

# 当前参数
blockSize=15, C=2

# 问题
1. 暗区域文字消失
2. 亮区域噪声过多

# 请求
请分析问题原因,并给出参数调优建议,解释每个参数的作用
    \end{lstlisting}

    \vspace{0.3cm}

    \textbf{AI可能的分析:}
    \begin{itemize}
        \item blockSize太小导致局部过度敏感
        \item C值太大导致阈值偏高
        \item 建议增大blockSize到21-31
        \item 建议减小C值到1-2
    \end{itemize}
\end{frame}

\begin{frame}[fragile]{本周Prompt模板汇总}
    \textbf{模板1:理解概念}
    \begin{lstlisting}
"请用通俗的语言解释[概念],[具体要求]"
    \end{lstlisting}

    \textbf{模板2:生成代码}
    \begin{lstlisting}
"用OpenCV实现[功能],参数设置为[值],包含详细注释"
    \end{lstlisting}

    \textbf{模板3:调试问题}
    \begin{lstlisting}
"这段代码[描述问题]:[粘贴代码],请帮我[具体要求]"
    \end{lstlisting}

    \textbf{模板4:参数调优}
    \begin{lstlisting}
"[功能]效果不理想,当前参数[列出],请分析原因并建议调优方案"
    \end{lstlisting}

    \vspace{0.3cm}

    \begin{exampleblock}{最佳实践}
        \begin{itemize}
            \item 提供足够的背景信息
            \item 明确你想要的结果
            \item 要求AI解释原因
            \item 验证AI给出的答案
        \end{itemize}
    \end{exampleblock}
\end{frame}
