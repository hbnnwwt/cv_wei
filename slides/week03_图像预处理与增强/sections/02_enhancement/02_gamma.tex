%===========================================================
% 02_gamma.tex - 伽马校正
%===========================================================

\begin{frame}{伽马校正原理}
    \textbf{原理:} 幂律变换 $O = (I/255)^\gamma \times 255$

    \begin{columns}
        \column{0.5\textwidth}
        \textbf{伽马值的影响:}
        \begin{itemize}
            \item $\gamma < 1$:提亮暗部(曲线向上)
            \item $\gamma = 1$:无变化
            \item $\gamma > 1$:压暗亮部(曲线向下)
        \end{itemize}

        \column{0.5\textwidth}
        \begin{center}
            \begin{tikzpicture}[scale=0.8]
                \draw[->] (0,0) -- (4,0) node[right] {输入};
                \draw[->] (0,0) -- (0,4) node[above] {输出};
                \draw[dashed] (0,0) -- (4,4);
                \draw[blue, thick] (0,0) .. controls (1,0.5) and (2,2) .. (4,4);
                \node[blue] at (3,0.5) {$\gamma<1$};
                \draw[red, thick] (0,0) .. controls (2,0.5) and (3,2) .. (4,4);
                \node[red] at (0.5,3) {$\gamma>1$};
            \end{tikzpicture}
        \end{center}
    \end{columns}

    \vspace{0.3cm}

    \textbf{应用场景:}
    \begin{itemize}
        \item 暗部细节恢复($\gamma < 1$)
        \item 过曝图像修正($\gamma > 1$)
        \item 显示设备校正
    \end{itemize}
\end{frame}

\begin{frame}[fragile]{伽马校正代码实现}
    \begin{lstlisting}[basicstyle=\ttfamily\scriptsize]
import numpy as np

def gamma_correction(img, gamma=1.0):
    """
    伽马校正
    gamma < 1: 提亮暗部
    gamma > 1: 压暗亮部
    """
    # 构建查找表
    inv_gamma = 1.0 / gamma
    table = np.array([
        ((i / 255.0) ** inv_gamma) * 255
        for i in np.arange(0, 256)
    ]).astype("uint8")

    # 应用查找表
    return cv2.LUT(img, table)

# 使用示例
# 提亮暗部(适合暗光拍摄)
gamma_bright = gamma_correction(gray, gamma=0.6)

# 压暗亮部(适合过曝图像)
gamma_dark = gamma_correction(gray, gamma=1.5)
    \end{lstlisting}

    \vspace{0.2cm}

    \begin{alertblock}{试卷处理建议}
        通常 $\gamma = 0.7 \sim 0.9$,提升暗部细节
    \end{alertblock}
\end{frame}
