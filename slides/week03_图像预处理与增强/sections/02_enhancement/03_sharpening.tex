%===========================================================
% 03_sharpening.tex - 锐化技术
%===========================================================

\begin{frame}{图像锐化原理}
    \textbf{目的:} 增强边缘,使图像更清晰

    \vspace{0.3cm}

    \textbf{常用锐化方法:}
    \begin{enumerate}
        \item \textbf{拉普拉斯锐化}:基于二阶导数
        \item \textbf{USM锐化}:Unsharp Mask,反锐化掩模
        \item \textbf{高反差保留}:保留高频细节
    \end{enumerate}

    \vspace{0.3cm}

    \begin{alertblock}{注意}
        锐化会放大噪声!建议在去噪后进行。
    \end{alertblock}
\end{frame}

\begin{frame}[fragile]{拉普拉斯锐化}
    \textbf{拉普拉斯核:}
    $$
    \begin{bmatrix}
    0 & -1 & 0 \\
    -1 & 5 & -1 \\
    0 & -1 & 0
    \end{bmatrix}
    $$

    \begin{lstlisting}[basicstyle=\ttfamily\scriptsize]
# 拉普拉斯锐化核
kernel = np.array([
    [0, -1, 0],
    [-1, 5, -1],
    [0, -1, 0]
])

# 应用锐化
sharpened = cv2.filter2D(img, -1, kernel)

# 更强的锐化核
kernel_strong = np.array([
    [-1, -1, -1],
    [-1, 9, -1],
    [-1, -1, -1]
])
    \end{lstlisting}
\end{frame}

\begin{frame}[fragile]{USM 锐化(推荐)}
    \textbf{原理:} 原图 + (原图 - 模糊图) × 强度

    \begin{lstlisting}[basicstyle=\ttfamily\scriptsize]
def usm_sharpen(img, sigma=1.0, strength=1.5):
    """
    Unsharp Mask 锐化
    sigma: 高斯模糊半径
    strength: 锐化强度
    """
    # 高斯模糊
    blurred = cv2.GaussianBlur(img, (0, 0), sigma)

    # 计算差值(高频成分)
    high_freq = cv2.subtract(img, blurred)

    # 叠加高频成分
    sharpened = cv2.addWeighted(
        img, 1,
        high_freq, strength,
        0
    )

    return sharpened

# 使用
sharp = usm_sharpen(gray, sigma=1.0, strength=1.5)
    \end{lstlisting}

    \begin{block}{参数建议}
        \begin{itemize}
            \item sigma = 0.5 ~ 2.0
            \item strength = 1.0 ~ 2.0
        \end{itemize}
    \end{block}
\end{frame}

\begin{frame}{图像增强总结}
    \begin{table}
        \centering
        \begin{tabular}{lp{6cm}l}
            \toprule
            \textbf{方法} & \textbf{作用} & \textbf{适用场景} \\
            \midrule
            直方图均衡化 & 提升整体对比度 & 低对比度图像 \\
            CLAHE & 局部自适应增强 & \textbf{试卷增强} \\
            伽马校正 & 调整亮度分布 & 暗部/亮部修正 \\
            锐化 & 增强边缘细节 & 模糊图像 \\
            \bottomrule
        \end{tabular}
    \end{table}

    \vspace{0.3cm}

    \begin{exampleblock}{试卷增强推荐流程}
        去噪 $\rightarrow$ CLAHE $\rightarrow$ 轻度锐化
    \end{exampleblock}
\end{frame}
