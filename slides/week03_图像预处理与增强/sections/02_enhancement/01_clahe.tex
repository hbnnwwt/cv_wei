%===========================================================
% 01_clahe.tex - CLAHE自适应直方图均衡化
%===========================================================

\begin{frame}[fragile]{CLAHE - 自适应直方图均衡化}
    \textbf{问题:} 全局均衡化可能导致局部过度增强

    \vspace{0.2cm}

    \textbf{CLAHE(Contrast Limited Adaptive Histogram Equalization):}
    \begin{itemize}
        \item 分块处理,保留局部对比度
        \item 限制对比度,避免噪声放大
    \end{itemize}

    \begin{lstlisting}[basicstyle=\ttfamily\scriptsize]
# 创建CLAHE对象
clahe = cv2.createCLAHE(
    clipLimit=2.0,      # 对比度限制
    tileGridSize=(8, 8) # 网格大小
)

# 应用CLAHE
enhanced = clahe.apply(gray)
    \end{lstlisting}

    \vspace{0.2cm}

    \textbf{参数说明:}
    \begin{itemize}
        \item \textbf{clipLimit}:对比度限制(1-3),值越大对比度越高
        \item \textbf{tileGridSize}:网格大小(4×4 到 16×16),越小越局部
    \end{itemize}
\end{frame}

\begin{frame}[fragile]{CLAHE 在试卷增强中的应用}
    \begin{lstlisting}[basicstyle=\ttfamily\tiny]
# 试卷增强最佳实践
def enhance_exam(img_path):
    # 读取图像
    img = cv2.imread(img_path, 0)

    # 去噪
    denoised = cv2.medianBlur(img, 3)

    # CLAHE增强
    clahe = cv2.createCLAHE(clipLimit=2.0, tileGridSize=(8, 8))
    enhanced = clahe.apply(denoised)

    return enhanced

# 使用
result = enhance_exam('exam.jpg')

# 保存结果
cv2.imwrite('exam_enhanced.jpg', result)
    \end{lstlisting}

    \vspace{0.3cm}

    \begin{exampleblock}{试卷增强推荐参数}
        \begin{itemize}
            \item clipLimit = 2.0(平衡对比度与噪声)
            \item tileGridSize = (8, 8)(适合文字区域)
        \end{itemize}
    \end{exampleblock}
\end{frame}
