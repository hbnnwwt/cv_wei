%===========================================================
% 00_noise_models.tex - 噪声模型与分类
%===========================================================

\section{图像去噪}

\begin{frame}{噪声类型与特征}
    \begin{table}
        \centering
        \begin{tabular}{lp{6cm}l}
            \toprule
            \textbf{噪声类型} & \textbf{特征} & \textbf{典型场景} \\
            \midrule
            高斯噪声 & 随机分布的亮度变化 & 传感器噪声、低光拍摄 \\
            椒盐噪声 & 随机的黑点或白点 & 传输错误、老化传感器 \\
            泊松噪声 & 与信号强度相关 & 低光摄影、X射线 \\
            周期噪声 & 规则的干扰条纹 & 电气干扰、扫描缺陷 \\
            \bottomrule
        \end{tabular}
    \end{table}

    \vspace{0.3cm}

    \begin{columns}
        \column{0.5\textwidth}
        \begin{block}{高斯噪声}
            服从正态分布 $N(\mu, \sigma^2)$\\
            \textit{最常见,容易处理}
        \end{block}

        \column{0.5\textwidth}
        \begin{block}{椒盐噪声}
            随机像素变为0或255\\
            \textit{需要非线性滤波}
        \end{block}
    \end{columns}
\end{frame}

\begin{frame}{试卷扫描常见噪声}
    \begin{alertblock}{实际问题}
        \begin{itemize}
            \item 手机拍照:ISO噪声(高斯)
            \item 压缩传输:块效应、JPEG伪影
            \item 扫描仪:灰尘颗粒(椒盐)
            \item 纸张:纹理干扰
        \end{itemize}
    \end{alertblock}

    \vspace{0.5cm}

    \textbf{噪声对识别的影响:}
    \begin{itemize}
        \item 字符边缘模糊 $\rightarrow$ OCR识别率下降
        \item 背景噪声干扰 $\rightarrow$ 边缘检测失败
        \item 压缩伪影 $\rightarrow$ 细节丢失
    \end{itemize}

    \begin{center}
        \textbf{结论:试卷预处理首选中值滤波!}
    \end{center}
\end{frame}
