%===========================================================
% 01_convolution.tex - 卷积原理
%===========================================================

\begin{frame}{空间域滤波原理}
    \textbf{卷积操作:}
    \begin{itemize}
        \item 滤波核(Kernel)在图像上滑动
        \item 每个位置计算加权求和
        \item 输出新的像素值
    \end{itemize}

    \vspace{0.3cm}

    \begin{columns}
        \column{0.5\textwidth}
        \begin{block}{3×3 均值滤波核}
            $$
            K = \frac{1}{9}
            \begin{bmatrix}
            1 & 1 & 1 \\
            1 & 1 & 1 \\
            1 & 1 & 1
            \end{bmatrix}
            $$
        \end{block}

        \column{0.5\textwidth}
        \begin{block}{3×3 高斯滤波核}
            $$
            K \approx
            \begin{bmatrix}
            1 & 2 & 1 \\
            2 & 4 & 2 \\
            1 & 2 & 1
            \end{bmatrix}
            $$
        \end{block}
    \end{columns}

    \vspace{0.3cm}

    \begin{block}{边界处理}
        \begin{itemize}
            \item \textbf{零填充}:边界外补0
            \item \textbf{复制边界}:复制边缘像素
            \item \textbf{镜像填充}:镜像边界(推荐)
        \end{itemize}
    \end{block}
\end{frame}

\begin{frame}[fragile]{OpenCV 卷积实现}
    \textbf{自定义滤波核:}
    \begin{lstlisting}
import cv2
import numpy as np

# 自定义锐化核
kernel = np.array([
    [-1, -1, -1],
    [-1,  9, -1],
    [-1, -1, -1]
])

# 应用卷积
result = cv2.filter2D(img, -1, kernel)
    \end{lstlisting}

    \vspace{0.2cm}

    \textbf{常用滤波函数:}
    \begin{lstlisting}
# 均值滤波
blur = cv2.blur(img, (5, 5))

# 高斯滤波
gaussian = cv2.GaussianBlur(img, (5, 5), 0)

# 中值滤波
median = cv2.medianBlur(img, 5)
    \end{lstlisting}
\end{frame}
