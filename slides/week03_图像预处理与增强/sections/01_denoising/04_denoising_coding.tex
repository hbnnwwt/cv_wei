%===========================================================
% 04_denoising_coding.tex - 去噪实战代码(TODO框架)
%===========================================================

\begin{frame}[fragile]{去噪实战:代码框架}
    \begin{alertblock}{AI辅助提示}
        使用Cursor/Claude Code可以快速生成以下代码:
        \begin{itemize}
            \item "用OpenCV实现中值滤波去噪,kernel=5"
            \item "对比高斯滤波和中值滤波的效果"
        \end{itemize}
    \end{alertblock}

    \begin{lstlisting}[basicstyle=\ttfamily\tiny]
import cv2
import numpy as np
import matplotlib.pyplot as plt

def denoise_exam(img_path):
    """
    试卷图像去噪处理
    """
    # TODO 1: 读取图像
    # 提示:使用 cv2.imread()
    img = ____

    # TODO 2: 转为灰度图
    # 提示:使用 cv2.cvtColor(img, cv2.COLOR_BGR2GRAY)
    gray = ____

    # TODO 3: 应用中值滤波(适合试卷的椒盐噪声)
    # 提示:使用 cv2.medianBlur(gray, 5)
    median = ____

    # TODO 4 (可选): 尝试高斯滤波,对比效果
    gaussian = ____

    return gray, median, gaussian

# 使用
original, median_denoised, gaussian_denoised = denoise_exam('exam_noisy.jpg')
    \end{lstlisting}
\end{frame}

\begin{frame}[fragile]{完整参考代码(课后查看)}
    \begin{lstlisting}[basicstyle=\ttfamily\tiny]
def denoise_exam(img_path):
    """完整版去噪函数"""
    img = cv2.imread(img_path)
    gray = cv2.cvtColor(img, cv2.COLOR_BGR2GRAY)

    # 中值滤波(推荐用于试卷)
    median = cv2.medianBlur(gray, 5)

    # 高斯滤波(适合高斯噪声)
    gaussian = cv2.GaussianBlur(gray, (5, 5), 0)

    # 双边滤波(保边去噪)
    bilateral = cv2.bilateralFilter(gray, 9, 75, 75)

    return gray, median, gaussian, bilateral

def psnr(img1, img2):
    """计算PSNR值"""
    mse = np.mean((img1 - img2) ** 2)
    if mse == 0:
        return float('inf')
    return 20 * np.log10(255.0 / np.sqrt(mse))

# 使用示例
gray, median, gaussian, bilateral = denoise_exam('exam_noisy.jpg')
print(f"中值滤波PSNR: {psnr(gray, median):.2f} dB")
    \end{lstlisting}
\end{frame}

\begin{frame}[fragile]{PSNR 评估去噪质量}
    \textbf{PSNR(峰值信噪比):} 评估图像质量的指标

    \begin{lstlisting}
def psnr(img1, img2):
    mse = np.mean((img1 - img2) ** 2)
    if mse == 0:
        return float('inf')
    return 20 * np.log10(255.0 / np.sqrt(mse))

# TODO: 对比各方法的PSNR值
# 提示:调用psnr函数对比原图和去噪后的图像
print(f"中值滤波: {psnr(gray, __):.2f} dB")
print(f"高斯滤波: {psnr(gray, __):.2f} dB")
    \end{lstlisting}

    \vspace{0.2cm}

    \textbf{PSNR 参考值:}
    \begin{itemize}
        \item $> 40$ dB:优秀
        \item $30-40$ dB:良好
        \item $< 30$ dB:较差
    \end{itemize}
\end{frame}

\begin{frame}[fragile]{三种学习模式的代码任务}
    \begin{columns}
        \column{0.33\textwidth}
        \textbf{观察者模式:}
        \begin{lstlisting}[basicstyle=\ttfamily\tiny]
# 只运行代码
denoised = cv2.medianBlur(gray, 5)
# 修改参数对比
denoised = cv2.medianBlur(gray, 7)
        \end{lstlisting}

        \column{0.33\textwidth}
        \textbf{使用者模式:}
        \begin{lstlisting}[basicstyle=\ttfamily\tiny]
# 完成TODO
median = cv2.medianBlur(gray, ____)
# 对比不同方法
gaussian = cv2.GaussianBlur(gray, ____, ____)
        \end{lstlisting}

        \column{0.33\textwidth}
        \textbf{创造者模式:}
        \begin{lstlisting}[basicstyle=\ttfamily\tiny]
# 从零实现
def my_median_filter(img, k):
    # 自己实现中值滤波
    pass
        \end{lstlisting}
    \end{columns}
\end{frame}
