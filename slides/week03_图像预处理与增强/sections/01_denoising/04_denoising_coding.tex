%===========================================================
% 04_denoising_coding.tex - 去噪实战代码
%===========================================================

\begin{frame}[fragile]{去噪实战:完整代码}
    \begin{lstlisting}[basicstyle=\ttfamily\tiny]
import cv2
import numpy as np
import matplotlib.pyplot as plt

# 读取图像
img = cv2.imread('exam_noisy.jpg')
gray = cv2.cvtColor(img, cv2.COLOR_BGR2GRAY)

# 方法1:高斯滤波(适合高斯噪声)
gaussian = cv2.GaussianBlur(gray, (5, 5), 0)

# 方法2:中值滤波(适合椒盐噪声)- 推荐用于试卷
median = cv2.medianBlur(gray, 5)

# 方法3:双边滤波(保边去噪)
bilateral = cv2.bilateralFilter(gray, 9, 75, 75)

# 方法4:非局部均值(效果最好,速度较慢)
nlm = cv2.fastNlMeansDenoising(gray, None, 10, 7, 21)

# 显示对比
titles = ['原图', '高斯', '中值', '双边', 'NLM']
images = [gray, gaussian, median, bilateral, nlm]

plt.figure(figsize=(15, 3))
for i in range(5):
    plt.subplot(1, 5, i+1)
    plt.imshow(images[i], cmap='gray')
    plt.title(titles[i])
    plt.axis('off')
plt.show()
    \end{lstlisting}
\end{frame}

\begin{frame}[fragile]{PSNR 评估去噪质量}
    \textbf{PSNR(峰值信噪比):} 评估图像质量的指标

    \begin{lstlisting}
def psnr(img1, img2):
    mse = np.mean((img1 - img2) ** 2)
    if mse == 0:
        return float('inf')
    return 20 * np.log10(255.0 / np.sqrt(mse))

# 对比各方法PSNR
print(f"高斯滤波: {psnr(gray, gaussian):.2f} dB")
print(f"中值滤波: {psnr(gray, median):.2f} dB")
print(f"双边滤波: {psnr(gray, bilateral):.2f} dB")
print(f"NLM去噪: {psnr(gray, nlm):.2f} dB")
    \end{lstlisting}

    \vspace{0.2cm}

    \textbf{PSNR 参考值:}
    \begin{itemize}
        \item $> 40$ dB:优秀
        \item $30-40$ dB:良好
        \item $< 30$ dB:较差
    \end{itemize}
\end{frame}
