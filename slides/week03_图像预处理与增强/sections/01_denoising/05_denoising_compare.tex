%===========================================================
% 05_denoising_compare.tex - 效果对比总结
%===========================================================

\begin{frame}{去噪方法总结对比}
    \begin{table}
        \centering
        \small
        \begin{tabular}{lcccc}
            \toprule
            \textbf{方法} & \textbf{速度} & \textbf{去噪效果} & \textbf{边缘保留} & \textbf{适用场景} \\
            \midrule
            均值滤波 & ★★★★★ & ★★ & ★ & 快速预览 \\
            高斯滤波 & ★★★★☆ & ★★★ & ★★ & 高斯噪声 \\
            中值滤波 & ★★★☆☆ & ★★★★ & ★★★ & \textbf{椒盐噪声} \\
            双边滤波 & ★★☆☆☆ & ★★★★ & ★★★★ & 保边去噪 \\
            NLM去噪 & ★☆☆☆☆ & ★★★★★ & ★★★★★ & \textbf{高质量去噪} \\
            \bottomrule
        \end{tabular}
    \end{table}

    \vspace{0.3cm}

    \begin{exampleblock}{试卷预处理推荐方案}
        \textbf{首选:} 中值滤波(快速、有效、保边)

        \textbf{备选:} NLM去噪(质量要求高时)

        \textbf{不推荐:} 均值滤波(边缘模糊严重)
    \end{exampleblock}
\end{frame}

\begin{frame}{去噪参数调优建议}
    \textbf{核大小选择:}
    \begin{itemize}
        \item 3×3:轻微去噪,保留细节
        \item 5×5:\textbf{平衡选择}(推荐)
        \item 7×7:强去噪,可能模糊细节
    \end{itemize}

    \vspace{0.3cm}

    \textbf{组合策略:}
    \begin{enumerate}
        \item 先用中值滤波去除椒盐噪声
        \item 再用高斯滤波平滑处理
        \item 最后用NLM精细化处理(可选)
    \end{enumerate}

    \vspace{0.3cm}

    \begin{alertblock}{注意}
        过度去噪会导致字符边缘模糊,反而降低OCR识别率!
    \end{alertblock}
\end{frame}
