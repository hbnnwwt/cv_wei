%===========================================================
% 03_advanced_filters.tex - 高级滤波器
%===========================================================

\begin{frame}[fragile]{双边滤波 - 保边去噪}
    \textbf{原理:} 同时考虑空间距离和像素值差异

    \vspace{0.2cm}

    \begin{columns}
        \column{0.6\textwidth}
        \begin{lstlisting}
# 双边滤波参数说明
bilateral = cv2.bilateralFilter(
    img,          # 输入图像
    9,            # 邻域直径
    75,           # 颜色空间标准差
    75            # 坐标空间标准差
)
        \end{lstlisting}

        \column{0.4\textwidth}
        \textbf{特点:}
        \begin{itemize}
            \item 保持边缘清晰
            \item 去除平滑区域噪声
            \item 计算量较大
        \end{itemize}
    \end{columns}

    \vspace{0.3cm}

    \begin{alertblock}{应用场景}
        \begin{itemize}
            \item 人像磨皮(卡通效果)
            \item 边缘检测前的预处理
            \item 需要保留边缘的场景
        \end{itemize}
    \end{alertblock}
\end{frame}

\begin{frame}[fragile]{非局部均值去噪(NLM)}
    \textbf{原理:} 利用图像的自相似性

    \vspace{0.2cm}

    \begin{lstlisting}[basicstyle=\ttfamily\scriptsize]
# 快速非局部均值去噪
denoised = cv2.fastNlMeansDenoisingColored(
    img,               # 输入图像
    None,              # 输出
    10,                # 滤波强度 (h)
    10,                # 滤波强度 (hColor)
    7,                 # 模板窗口大小
    21                 # 搜索窗口大小
)
    \end{lstlisting}

    \vspace{0.2cm}

    \textbf{参数说明:}
    \begin{itemize}
        \item \textbf{h}:滤波强度,值越大去噪越强(但会丢失细节)
        \item \textbf{templateWindowSize}:通常取7(奇数)
        \item \textbf{searchWindowSize}:通常取21(奇数,大于template)
    \end{itemize}

    \begin{block}{优势}
        去噪效果最好,能保留纹理细节!
    \end{block}
\end{frame}
