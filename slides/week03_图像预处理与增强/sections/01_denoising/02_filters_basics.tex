%===========================================================
% 02_filters_basics.tex - 基础滤波器
%===========================================================

\begin{frame}[fragile]{基础滤波器对比}
    \begin{columns}
        \column{0.33\textwidth}
        \textbf{均值滤波}
        \begin{lstlisting}[basicstyle=\ttfamily\tiny]
blur = cv2.blur(
    img, (5, 5)
)
        \end{lstlisting}
        \begin{itemize}
            \item 简单平均
            \item 速度快
            \item 边缘模糊
        \end{itemize}

        \column{0.33\textwidth}
        \textbf{高斯滤波}
        \begin{lstlisting}[basicstyle=\ttfamily\tiny]
blur = cv2.GaussianBlur(
    img, (5, 5), 0
)
        \end{lstlisting}
        \begin{itemize}
            \item 加权平均
            \item 自然模糊
            \item 适合高斯噪声
        \end{itemize}

        \column{0.33\textwidth}
        \textbf{中值滤波}
        \begin{lstlisting}[basicstyle=\ttfamily\tiny]
blur = cv2.medianBlur(
    img, 5
)
        \end{lstlisting}
        \begin{itemize}
            \item 中值替代
            \item 保边缘
            \item 适合椒盐噪声
        \end{itemize}
    \end{columns}
\end{frame}

\begin{frame}{滤波效果对比}
    \begin{table}
        \centering
        \begin{tabular}{lccc}
            \toprule
            \textbf{滤波器} & \textbf{高斯噪声} & \textbf{椒盐噪声} & \textbf{边缘保留} \\
            \midrule
            均值滤波 & 良好 & 一般 & 差 \\
            高斯滤波 & \textbf{优秀} & 差 & 一般 \\
            中值滤波 & 一般 & \textbf{优秀} & 良好 \\
            \bottomrule
        \end{tabular}
    \end{table}

    \vspace{0.5cm}

    \begin{block}{选择建议}
        \begin{itemize}
            \item \textbf{试卷扫描}:中值滤波(去除点状噪声)
            \item \textbf{照片美化}:高斯滤波(自然模糊)
            \item \textbf{边缘检测前}:双边滤波(保边去噪)
        \end{itemize}
    \end{block}
\end{frame}
