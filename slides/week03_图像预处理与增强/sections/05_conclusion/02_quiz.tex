%===========================================================
% 02_quiz.tex - 课堂互动
%===========================================================

\section{课堂互动}

\begin{frame}{Quiz 1:噪声识别}
    \textbf{问题:} 下面图像中的噪声属于哪种类型?

    \begin{columns}
        \column{0.5\textwidth}
        \begin{block}{图像 A}
            有随机分布的黑点和白点
        \end{block}
        \visible<2->{
            \textcolor{green}{\textbf{答案:椒盐噪声}}
        }

        \column{0.5\textwidth}
        \begin{block}{图像 B}
            整体有朦胧感,细节模糊
        \end{block}
        \visible<2->{
            \textcolor{green}{\textbf{答案:高斯噪声}}
        }
    \end{columns}

    \vspace{0.5cm}

    \visible<3->{
        \textbf{思考:} 应该分别用什么滤波器处理?
    }
\end{frame}

\begin{frame}{Quiz 2:场景选择}
    \textbf{问题:} 以下场景应该使用哪种二值化方法?

    \begin{enumerate}
        \item<1-> 扫描仪扫描的试卷
            \visible<2->{\textcolor{blue}{$\rightarrow$ Otsu 算法}}

        \item<1-> 手机拍照的试卷(有阴影)
            \visible<3->{\textcolor{blue}{$\rightarrow$ 自适应阈值}}

        \item<1-> 光照均匀的发票
            \visible<4->{\textcolor{blue}{$\rightarrow$ 全局阈值 或 Otsu}}
    \end{enumerate}
\end{frame}

\begin{frame}{Quiz 3:参数调优}
    \textbf{问题:} 自适应阈值参数如何选择?

    \begin{columns}
        \column{0.5\textwidth}
        \begin{block}{场景 A:大号文字}
            选择 blockSize:
            \begin{itemize}
                \item<2-> A. 5
                \item<2-> B. 11
                \item<2-> C. 21 \checkmark
            \end{itemize}
        \end{block}

        \column{0.5\textwidth}
        \begin{block}{场景 B:背景噪声多}
            选择 C 值:
            \begin{itemize}
                \item<3-> A. 2
                \item<3-> B. 10 \checkmark
                \item<3-> C. -5
            \end{itemize}
        \end{block}
    \end{columns}
\end{frame}
