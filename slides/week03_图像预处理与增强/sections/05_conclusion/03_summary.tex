%===========================================================
% 03_summary.tex - 知识点总结
%===========================================================

\section{知识点总结}

\begin{frame}{预处理完整流程回顾}
    \begin{center}
        \begin{tikzpicture}[node distance=1.2cm, auto,
            block/.style={draw, rectangle, rounded corners, fill=blue!10, minimum height=0.8cm, minimum width=1.8cm, font=\small},
            arrow/.style={->, thick}]

            \node[block] (input) {原始图像};
            \node[block, right of=input, node distance=2cm, fill=yellow!10] (denoise) {去噪};
            \node[block, right of=denoise, node distance=2cm, fill=green!10] (enhance) {增强};
            \node[block, right of=enhance, node distance=2cm, fill=orange!10] (binary) {二值化};
            \node[block, right of=binary, node distance=2cm, fill=red!10] (correct) {几何矫正};
            \node[block, right of=correct, node distance=2cm, fill=purple!10] (output) {输出};

            \draw[arrow] (input) -- (denoise);
            \draw[arrow] (denoise) -- (enhance);
            \draw[arrow] (enhance) -- (binary);
            \draw[arrow] (binary) -- (correct);
            \draw[arrow] (correct) -- (output);
        \end{tikzpicture}
    \end{center}

    \vspace{0.3cm}

    \textbf{核心方法对比:}
    \begin{table}
        \centering
        \tiny
        \begin{tabular}{llll}
            \toprule
            \textbf{类别} & \textbf{基础方法} & \textbf{进阶方法} & \textbf{推荐场景} \\
            \midrule
            去噪 & 高斯/均值滤波 & 中值/NLM & 试卷用中值 \\
            增强 & 直方图均衡化 & CLAHE/伽马 & 试卷用CLAHE \\
            二值化 & 全局阈值 & Otsu/自适应 & 拍照用自适应 \\
            几何 & 仿射变换 & 透视变换 & 文档用透视 \\
            \bottomrule
        \end{tabular}
    \end{table}
\end{frame}

\begin{frame}{参数速查表}
    \begin{table}
        \centering
        \small
        \begin{tabular}{llp{5cm}}
            \toprule
            \textbf{操作} & \textbf{参数} & \textbf{推荐值} \\
            \midrule
            中值滤波 & kernel & 5 \\
            CLAHE & clipLimit & 2.0 \\
            CLAHE & tileGridSize & (8, 8) \\
            自适应阈值 & blockSize & 15 \\
            自适应阈值 & C & 2-5 \\
            伽马校正 & gamma & 0.7-0.9 \\
            USM锐化 & sigma & 1.0 \\
            USM锐化 & strength & 1.5 \\
            \bottomrule
        \end{tabular}
    \end{table}
\end{frame}
