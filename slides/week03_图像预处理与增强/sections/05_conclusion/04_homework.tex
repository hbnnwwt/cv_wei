%===========================================================
% 04_homework.tex - 课后作业
%===========================================================

\section{课后作业}

\begin{frame}{作业:试卷图像预处理完整实现}
    \textbf{任务描述:}
    实现一个完整的试卷图像预处理系统

    \vspace{0.3cm}

    \textbf{基本要求(60分):}
    \begin{enumerate}
        \item 实现去噪功能(至少2种方法对比)
        \item 实现二值化功能(至少2种方法对比)
        \item 生成处理前后对比图
        \item 代码规范,有注释
    \end{enumerate}

    \vspace{0.3cm}

    \textbf{进阶要求(40分):}
    \begin{enumerate}
        \item 实现 CLAHE 增强功能
        \item 实现透视矫正功能
        \item 编写批量处理脚本
        \item 实现 PSNR 等质量评估指标
    \end{enumerate}
\end{frame}

\begin{frame}{提交要求}
    \textbf{提交内容:}
    \begin{enumerate}
        \item Python 代码(.py 文件或 Jupyter Notebook)
        \item 测试图像(处理前/后对比)
        \item 实验报告(PDF)
    \end{enumerate}

    \vspace{0.3cm}

    \textbf{实验报告包含:}
    \begin{itemize}
        \item 不同方法的对比分析
        \item 参数调优过程
        \item 遇到的问题与解决方案
        \item 处理效果评估
    \end{itemize}

    \vspace{0.3cm}

    \textbf{截止时间:} 下次上课前

    \textbf{提交方式:} 教学平台上传
\end{frame}

\begin{frame}{评分标准}
    \begin{table}
        \centering
        \begin{tabular}{lc}
            \toprule
            \textbf{项目} & \textbf{分值} \\
            \midrule
            去噪功能实现 & 15分 \\
            二值化功能实现 & 15分 \\
            增强功能实现 & 10分 \\
            几何矫正实现 & 10分 \\
            代码规范 & 15分 \\
            实验报告质量 & 20分 \\
            创新点(可选) & +15分 \\
            \midrule
            \textbf{总分} & \textbf{100分} \\
            \bottomrule
        \end{tabular}
    \end{table}
\end{frame}
