%===========================================================
% 00_real_cases.tex - 真实案例
%===========================================================

\section{案例分析}

\begin{frame}{案例1:手机拍摄的模糊试卷}
    \textbf{问题描述:}
    \begin{itemize}
        \item 拍摄角度不正
        \item 光照不均匀
        \item 手持模糊
    \end{itemize}

    \vspace{0.3cm}

    \textbf{处理方案:}
    \begin{enumerate}
        \item \textbf{去噪}:中值滤波去除噪点
        \item \textbf{增强}:CLAHE 提升对比度
        \item \textbf{锐化}:轻度 USM 锐化
        \item \textbf{二值化}:自适应阈值
        \item \textbf{矫正}:透视变换展平
    \end{enumerate}

    \vspace{0.3cm}

    \textbf{效果:} OCR 识别率从 60\% 提升到 95\%
\end{frame}

\begin{frame}{案例2:光照不均的扫描件}
    \textbf{问题描述:}
    \begin{itemize}
        \item 左侧有阴影
        \item 中心过曝
        \item 对比度低
    \end{itemize}

    \vspace{0.3cm}

    \textbf{处理方案:}
    \begin{enumerate}
        \item \textbf{直方图分析}:识别光照分布
        \item \textbf{CLAHE}:局部自适应增强
        \item \textbf{伽马校正}:$\gamma=0.8$ 提亮暗部
        \item \textbf{自适应二值化}:blockSize=15
    \end{enumerate}

    \vspace{0.3cm}

    \textbf{关键:} 避免使用全局阈值和全局均衡化
\end{frame}

\begin{frame}{案例3:有折痕的老试卷}
    \textbf{问题描述:}
    \begin{itemize}
        \item 纸张有明显折痕
        \item 折痕处有阴影
        \item 字迹被折痕遮挡
    \end{itemize}

    \vspace{0.3cm}

    \textbf{处理方案:}
    \begin{enumerate}
        \item \textbf{去噪}:NLM 去除折痕噪声
        \item \textbf{方向滤波}:沿文字方向平滑
        \item \textbf{Inpainting}:修复折痕区域(高级)
        \item \textbf{增强}:对比度拉伸
    \end{enumerate}

    \vspace{0.3cm}

    \begin{alertblock}{注意}
        严重折痕可能需要手动干预或深度学习修复
    \end{alertblock}
\end{frame}
