\documentclass[aspectratio=169, 12pt]{beamer}
\usepackage[UTF8]{ctex}
\usepackage{graphicx}
\usepackage{booktabs}
\usepackage{listings}
\usepackage{xcolor}
\usepackage{tikz}
\usepackage{hyperref}

\usetheme{Madrid}
\usecolortheme{whale}
\usefonttheme{professionalfonts}

% 页脚logo(缩小显示)
\logo{\includegraphics[height=0.8cm]{../xiaohui.png}}

\lstset{
    language=Python,
    basicstyle=\ttfamily\small,
    keywordstyle=\color{blue},
    commentstyle=\color{green!60!black},
    stringstyle=\color{orange},
    breaklines=true,
    frame=single,
    showstringspaces=false,
    backgroundcolor=\color{gray!10}
}

\title[图像预处理与增强]{第3周:图像预处理与增强}
\subtitle{试卷拍照模糊怎么办?}
\author{北京石油化工学院\\人工智能研究院\\王文通}
\institute{通选课}
\date{2025-2026 学年}
\titlegraphic{
    \includegraphics[height=1.2cm]{../xiaohui.png}\hspace{2cm}
    \includegraphics[height=1.2cm]{../name.png}
}

\begin{document}

\begin{frame}
    \titlepage
\end{frame}

\begin{frame}{课程概览}
    \tableofcontents
\end{frame}

\section{预处理概述}

\begin{frame}{为什么需要预处理?}
    \textbf{现实问题:}
    \begin{itemize}
        \item 拍摄角度不正
        \item 光照不均匀
        \item 纸张有折痕
        \item 背景有杂物
    \end{itemize}

    \vspace{0.5cm}

    \textbf{预处理目标:}
    \begin{enumerate}
        \item 去除噪声干扰
        \item 增强目标特征
        \item 规范图像格式
        \item \textbf{提升识别准确率}
    \end{enumerate}
\end{frame}

\begin{frame}{预处理流水线}
    \begin{center}
        \begin{tikzpicture}[node distance=2cm]
            \node[draw, rectangle, rounded corners, fill=blue!10] (input) {原始图像};
            \node[draw, rectangle, rounded corners, fill=yellow!10, right of=input] (denoise) {去噪};
            \node[draw, rectangle, rounded corners, fill=green!10, right of=denoise] (binary) {二值化};
            \node[draw, rectangle, rounded corners, fill=red!10, right of=binary] (correct) {矫正};
            \node[draw, rectangle, rounded corners, fill=purple!10, right of=correct] (output) {输出};

            \draw[->, thick] (input) -- (denoise);
            \draw[->, thick] (denoise) -- (binary);
            \draw[->, thick] (binary) -- (correct);
            \draw[->, thick] (correct) -- (output);
        \end{tikzpicture}
    \end{center}

    \vspace{0.5cm}

    \textbf{关键:} 预处理质量直接决定后续识别效果!
\end{frame}

\section{图像去噪}

\begin{frame}{常见噪声类型}
    \begin{table}
        \centering
        \begin{tabular}{lp{6cm}l}
            \toprule
            \textbf{噪声类型} & \textbf{特征} & \textbf{典型场景} \\
            \midrule
            高斯噪声 & 随机分布的亮度变化 & 传感器噪声、低光拍摄 \\
            椒盐噪声 & 随机的黑点或白点 & 传输错误、老化传感器 \\
            周期噪声 & 规则的干扰条纹 & 电气干扰 \\
            \bottomrule
        \end{tabular}
    \end{table}

    \vspace{0.3cm}

    \begin{alertblock}{试卷扫描常见问题}
        \begin{itemize}
            \item 拍照时有噪点
            \item 扫描时有灰尘
            \item 压缩失真
        \end{itemize}
    \end{alertblock}
\end{frame}

\begin{frame}[fragile]{滤波方法对比}
    \textbf{1. 高斯滤波} - 适合高斯噪声
    \begin{lstlisting}
blur = cv2.GaussianBlur(img, (5, 5), 0)
    \end{lstlisting}

    \textbf{2. 中值滤波} - 适合椒盐噪声,保边
    \begin{lstlisting}
median = cv2.medianBlur(img, 5)
    \end{lstlisting}

    \textbf{3. 双边滤波} - 保边去噪
    \begin{lstlisting}
bilateral = cv2.bilateralFilter(img, 9, 75, 75)
    \end{lstlisting}

    \vspace{0.2cm}

    \textbf{建议:} 试卷预处理使用\textcolor{blue}{中值滤波}
\end{frame}

\section{图像二值化}

\begin{frame}{什么是二值化?}
    \begin{block}{定义}
        将灰度图像转换为只有黑白两种颜色的图像
    \end{block}

    \vspace{0.3cm}

    \begin{columns}
        \column{0.5\textwidth}
        \textbf{转换规则:}
        \begin{itemize}
            \item 像素值 $>$ 阈值 $\to$ 白色(255)
            \item 像素值 $\le$ 阈值 $\to$ 黑色(0)
        \end{itemize}

        \column{0.5\textwidth}
        \begin{center}
            \begin{tikzpicture}
                \shade[ball color=gray] (0,0) circle (0.8cm);
                \node at (0,-1.2) {灰度};
                \node at (2,0) {\Huge $\to$};
                \shade[ball color=white, draw=black] (3.5,0) circle (0.8cm);
                \node at (3.5,-1.2) {二值};
            \end{tikzpicture}
        \end{center}
    \end{columns}
\end{frame}

\begin{frame}[fragile]{二值化方法}
    \textbf{1. 全局阈值}
    \begin{lstlisting}
ret, binary = cv2.threshold(gray, 127, 255, cv2.THRESH_BINARY)
    \end{lstlisting}

    \textbf{2. Otsu自动阈值}
    \begin{lstlisting}
ret, otsu = cv2.threshold(gray, 0, 255,
    cv2.THRESH_BINARY + cv2.THRESH_OTSU)
    \end{lstlisting}

    \textbf{3. 自适应阈值} - 推荐用于光照不均
    \begin{lstlisting}
adaptive = cv2.adaptiveThreshold(gray, 255,
    cv2.ADAPTIVE_THRESH_GAUSSIAN_C,
    cv2.THRESH_BINARY, 11, 2)
    \end{lstlisting}
\end{frame}

\section{透视矫正}

\begin{frame}{透视矫正}
    \textbf{问题:} 拍照时试卷角度不正

    \vspace{0.3cm}

    \begin{columns}
        \column{0.5\textwidth}
        \begin{center}
            \begin{tikzpicture}
                \draw[fill=gray!20] (0,0) -- (2,0.3) -- (2.1,2.8) -- (0.1,2.5) -- cycle;
                \node at (1,1.5) {倾斜};
            \end{tikzpicture}
        \end{center}

        \column{0.5\textwidth}
        \begin{center}
            \begin{tikzpicture}
                \draw[fill=white] (0,0) rectangle (2,2.5);
                \node at (1,1.25) {矫正};
            \end{tikzpicture}
        \end{center}
    \end{columns}

    \vspace{0.5cm}

    \textbf{解决:} 透视变换(Perspective Transform)
\end{frame}

\begin{frame}[fragile]{透视变换实现}
    \begin{lstlisting}[basicstyle=\ttfamily\scriptsize]
import numpy as np

# 四个角点坐标(实际需要检测)
pts = np.float32([[100, 150], [450, 120], [480, 380], [80, 400]])

# 目标矩形
width, height = 400, 300
dst = np.float32([[0, 0], [width-1, 0],
                  [width-1, height-1], [0, height-1]])

# 计算变换矩阵
M = cv2.getPerspectiveTransform(pts, dst)

# 应用变换
warped = cv2.warpPerspective(img, M, (width, height))
    \end{lstlisting}

    \textbf{关键:} 需要先检测试卷的四个角点(下周内容)
\end{frame}

\section{思考题}

\begin{frame}{课堂思考题}
    \begin{block}{问题1:图像预处理}
        \begin{itemize}
            \item 什么情况下使用高斯滤波?什么情况下使用中值滤波?
            \item 为什么二值化时要用反色(THRESH\_BINARY\_INV)?
        \end{itemize}
    \end{block}

    \vspace{0.3cm}

    \begin{block}{问题2:透视变换}
        \begin{itemize}
            \item 如何自动检测试卷的四个角点?
            \item 如果四个角点检测不准确,会影响什么?
        \end{itemize}
    \end{block}
\end{frame}

\section{课后作业}

\begin{frame}{课后作业}
    \begin{block}{题目}
        实现试卷图像预处理完整流程
    \end{block}

    \textbf{要求:}
    \begin{enumerate}
        \item 对试卷图像进行去噪处理
        \item 实现二值化(至少两种方法对比)
        \item 提交处理前后对比图
        \item 分析不同参数对结果的影响
    \end{enumerate}

    \vspace{0.2cm}

    \textbf{评分标准:}
    \begin{itemize}
        \item 代码实现:40分
        \item 效果对比:30分
        \item 参数分析:20分
        \item 代码规范:10分
    \end{itemize}
\end{frame}

\begin{frame}{下节预告}
    \begin{center}
        \Large \textbf{第4周:试卷版面分析}

        \vspace{0.5cm}

        \normalsize
        故事问题:\textcolor{blue}{怎么知道选择题、简答题在哪里?}

        \vspace{0.3cm}

        你将学会:
        \begin{itemize}
            \item 边缘检测与轮廓查找
            \item 区域定位与分割
            \item 版面结构分析
        \end{itemize}
    \end{center}
\end{frame}

\begin{frame}
    \begin{center}
        \Huge \textbf{谢谢!}

        \vspace{1cm}

        \Large 有问题随时交流
    \end{center}
\end{frame}

\end{document}
