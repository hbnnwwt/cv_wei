%===========================================================
% 教师版讲义 - 时间分配设计
%===========================================================

\section{时间分配设计}

\begin{frame}{3课时时间分配(160分钟)}
    \begin{center}
        \begin{tikzpicture}[scale=0.8, every node/.style={transform shape}]
            % 时间轴
            \draw[->, thick] (0,0) -- (12,0);

            % 时间段
            \foreach \x/\label/\minutes/\color in {
                0/09:00-09:10/10/blue,
                1.5/09:10-09:25/15/green,
                3/09:25-09:35/10/yellow,
                4.5/09:35-09:50/15/orange,
                6/09:50-10:00/10/gray,
                7.5/10:00-10:20/20/red,
                9.5/10:20-10:30/10/gray,
                10.5/10:30-10:50/20/green,
                12/10:50-11:05/15/yellow,
                13.5/11:05-11:25/20/orange,
                15/11:25-11:40/15/red
            } {
                \draw[fill=\color!30, draw=\color] (\x,0.2) rectangle ++(1.5,1);
                \node[font=\tiny] at (\x+0.75,0.7) {\label};
                \node[font=\tiny] at (\x+0.75,1.3) {\minutes min};
            }

            % 标签
            \node[below] at (0.75,-0.3) {分组};
            \node[below] at (2.25,-0.3) {理论1};
            \node[below] at (3.75,-0.3) {演示1};
            \node[below] at (5.25,-0.3) {实践1};
            \node[below] at (6.75,-0.3) {投票};
            \node[below] at (8.25,-0.3) {休息};
            \node[below] at (10.5,-0.3) {理论2};
            \node[below] at (12.75,-0.3) {演示2};
            \node[below] at (14.25,-0.3) {实践2};
            \node[below] at (15.75,-0.3) {分享};
        \end{tikzpicture}
    \end{center}

    \vspace{0.3cm}

    \begin{table}
        \centering
        \tiny
        \begin{tabular}{llp{5cm}l}
            \toprule
            \textbf{时间} & \textbf{环节} & \textbf{内容} & \textbf{形式} \\
            \midrule
            09:00-09:10 & 课程导入 & 预处理重要性、分组说明 & 讲授 \\
            09:10-09:25 & 理论讲解1 & 噪声模型、滤波器原理 & 主屏讲授 \\
            09:25-09:35 & 现场演示1 & 去噪效果对比、参数调优 & 侧屏演示 \\
            09:35-09:50 & 实践环节1 & 运行去噪代码、对比方法 & 学生编码 \\
            09:50-10:00 & 互动投票 & 选择最适合的去噪方法 & 问卷星 \\
            10:00-10:20 & 综合实践 & 完整去噪流程、小组协作 & 教师巡视 \\
            \midrule
            10:20-10:30 & \textbf{课间休息} & & \\
            \midrule
            10:30-10:50 & 理论讲解2 & 增强、二值化原理 & 主屏讲授 \\
            10:50-11:05 & 现场演示2 & CLAHE、Otsu效果 & 侧屏演示 \\
            11:05-11:25 & 实践环节2 & 完整预处理流程 & 学生编码 \\
            11:25-11:40 & 成果分享 & 展示处理效果、评选最佳 & 小组汇报 \\
            \bottomrule
        \end{tabular}
    \end{table}
\end{frame}

\begin{frame}{第1课时详细安排(60分钟)}
    \textbf{主题:图像去噪}

    \vspace{0.3cm}

    \begin{table}
        \centering
        \small
        \begin{tabular}{lp{8cm}l}
            \toprule
            \textbf{时间} & \textbf{教师活动} & \textbf{学生活动} \\
            \midrule
            09:00-09:10 & 讲解预处理重要性,说明分组规则 & 听讲、记录、分组 \\
            09:10-09:25 & 讲授噪声模型、滤波器原理(主屏) & 理解原理、记笔记 \\
            09:25-09:35 & 演示去噪效果,调优参数(侧屏) & 观察效果、提问 \\
            09:35-09:50 & 巡视指导,解答问题 & 运行代码、对比方法 \\
            09:50-10:00 & 发起投票,讲解答案 & 投票选择、分享理由 \\
            10:00-10:20 & 深入指导,帮助困难组 & 完善代码、小组讨论 \\
            \bottomrule
        \end{tabular}
    \end{table}

    \vspace{0.3cm}

    \begin{alertblock}{时间管理提示}
        \begin{itemize}
            \item 严格控制讲授时间,留足实践时间
            \item 投票环节不超过10分钟
            \item 巡视时重点关注困难小组
        \end{itemize}
    \end{alertblock}
\end{frame}

\begin{frame}{第2课时详细安排(60分钟)}
    \textbf{主题:图像增强与二值化}

    \vspace{0.3cm}

    \begin{table}
        \centering
        \small
        \begin{tabular}{lp{8cm}l}
            \toprule
            \textbf{时间} & \textbf{教师活动} & \textbf{学生活动} \\
            \midrule
            10:20-10:30 & \multicolumn{2}{c}{\textbf{课间休息}} \\
            \midrule
            10:30-10:50 & 讲授CLAHE、二值化原理(主屏) & 理解原理、记笔记 \\
            10:50-11:05 & 演示CLAHE效果、Otsu对比(侧屏) & 观察效果、提问 \\
            11:05-11:25 & 代码找茬环节,集体debug & 找错误、修复代码 \\
            11:25-11:40 & 组织分享、点评成果 & 展示成果、互评 \\
            11:40-11:50 & 总结知识点、布置作业 & 记录作业、提问 \\
            \bottomrule
        \end{tabular}
    \end{table}

    \vspace{0.3cm}

    \begin{exampleblock}{代码找茬示例}
        展示有错误的二值化代码:
        \begin{itemize}
            \item 没有转为灰度图
            \item threshold返回两个值
            \item 参数设置不当
        \end{itemize}
    \end{exampleblock}
\end{frame}

\begin{frame}{第3课时详细安排(40分钟)}
    \textbf{主题:几何变换与综合实践}

    \vspace{0.3cm}

    \begin{table}
        \centering
        \small
        \begin{tabular}{lp{8cm}l}
            \toprule
            \textbf{时间} & \textbf{教师活动} & \textbf{学生活动} \\
            \midrule
            11:50-12:00 & 讲授透视变换原理(主屏) & 理解原理、记笔记 \\
            12:00-12:10 & 演示自动文档矫正(侧屏) & 观察效果、提问 \\
            12:10-12:25 & 综合实践指导,协助集成 & 完整流程实现 \\
            12:25-12:35 & TODO挑战环节(AI辅助) & 用AI补全代码 \\
            12:35-12:50 & 成果分享、评选最佳 & 展示效果、投票 \\
            12:50-13:00 & 课程总结、作业答疑 & 记录要点、提问 \\
            \bottomrule
        \end{tabular}
    \end{table}

    \vspace{0.3cm}

    \begin{alertblock}{综合实践要点}
        \begin{itemize}
            \item 鼓励使用AI工具辅助编程
            \item 重点查看完整预处理流程
            \item 评选最佳处理效果
        \end{itemize}
    \end{alertblock}
\end{frame}

\begin{frame}{动静结合原则实施}
    \textbf{每45分钟的标准节奏:}

    \vspace{0.3cm}

    \begin{columns}
        \column{0.33\textwidth}
        \begin{block}{讲授20min}
            \begin{itemize}
                \item 传递新知识
                \item 主屏展示理论
                \item 配合示意图解
            \end{itemize}
        \end{block}

        \column{0.33\textwidth}
        \begin{block}{实践20min}
            \begin{itemize}
                \item 动手操作
                \item 侧屏演示
                \item 学生编码
            \end{itemize}
        \end{block}

        \column{0.33\textwidth}
        \begin{block}{互动5min}
            \begin{itemize}
                \item 投票/问答
                \item 分享/讨论
                \item 巩固知识
            \end{itemize}
        \end{block}
    \end{columns}

    \vspace{0.5cm}

    \textbf{本课程调整:}
    \begin{itemize}
        \item 理论重的部分(卷积原理)可延长讲授时间
        \item 实践重的部分(代码编写)可延长实践时间
        \item 每个模块结束后都有5分钟分享/投票
    \end{itemize}
\end{frame}
