%===========================================================
% 教师版讲义 - 互动环节设计
%===========================================================

\section{互动环节设计}

\begin{frame}{互动环节总览}
    \textbf{本课程设计的互动环节:}

    \vspace{0.3cm}

    \begin{table}
        \centering
        \begin{tabular}{llp{6cm}l}
            \toprule
            \textbf{类型} & \textbf{环节名称} & \textbf{内容} & \textbf{时间} \\
            \midrule
            实时投票 & 去噪方法选择 & 根据噪声类型选择滤波器 & 5分钟 \\
            代码找茬 & 二值化代码 & 找出threshold函数的错误 & 10分钟 \\
            TODO挑战 & 自动矫正 & 用AI补全透视变换代码 & 10分钟 \\
            成果分享 & 最佳效果 & 展示处理结果、投票评选 & 10分钟 \\
            快速问答 & 知识回顾 & 核心概念快速确认 & 5分钟 \\
            \bottomrule
        \end{tabular}
    \end{table}

    \vspace{0.3cm}

    \begin{exampleblock}{互动原则}
        \begin{itemize}
            \item 每个互动环节不超过10分钟
            \item 紧密结合当前讲授内容
            \item 鼓励全员参与
            \item 及时反馈点评
        \end{itemize}
    \end{exampleblock}
\end{frame}

\begin{frame}[fragile]{互动1:去噪方法选择投票}
    \textbf{时间:} 09:50-10:00(第1课时末)

    \vspace{0.3cm}

    \textbf{场景描述:}
    \begin{itemize}
        \item 试卷有明显的黑白噪点(椒盐噪声)
        \item 需要选择最合适的去噪方法
    \end{itemize}

    \vspace{0.3cm}

    \textbf{投票选项:}
    \begin{enumerate}
        \item A. 高斯滤波
        \item B. 中值滤波
        \item C. 双边滤波
        \item D. 非局部均值
    \end{enumerate}

    \vspace{0.3cm}

    \textbf{实施步骤:}
    \begin{enumerate}
        \item 展示噪声图像效果
        \item 发起问卷星投票
        \item 学生扫码投票
        \item 公布结果、讲解答案
    \end{enumerate}

    \vspace{0.3cm}

    \begin{alertblock}{正确答案:B. 中值滤波}
        椒盐噪声的特点是随机黑白点,中值滤波能有效去除。
    \end{alertblock}
\end{frame}

\begin{frame}[fragile]{互动2:代码找茬}
    \textbf{时间:} 11:05-11:15(第2课时中)

    \vspace{0.3cm}

    \textbf{任务:找出代码中的错误}

    \begin{lstlisting}[basicstyle=\ttfamily\small]
# 有错误的代码
def binarize_exam(img_path):
    img = cv2.imread(img_path)  # 错误1
    _, binary = cv2.threshold(img, 127, 255,
                              cv2.THRESH_BINARY)
    return binary
    \end{lstlisting}

    \vspace{0.3cm}

    \textbf{错误清单:}
    \begin{enumerate}
        \item \textbf{错误1}:\texttt{imread}默认读取彩色图,应加参数\texttt{0}
        \item \textbf{错误2}:\texttt{threshold}需要灰度图
        \item \textbf{错误3}:试卷识别应该用\texttt{THRESH\_BINARY\_INV}
    \end{enumerate}

    \vspace{0.3cm}

    \textbf{实施方式:}
    \begin{itemize}
        \item 小组讨论2分钟
        \item 抢答得分
        \item 集体修正代码
    \end{itemize}
\end{frame}

\begin{frame}[fragile]{互动3:TODO挑战(AI辅助)}
    \textbf{时间:} 12:10-12:20(第3课时中)

    \vspace{0.3cm}

    \textbf{任务:用AI补全代码}

    \begin{lstlisting}[basicstyle=\ttfamily\small]
def order_points(pts):
    """
    将四个点按左上、右上、右下、左下排序
    """
    # TODO 1: 按x坐标排序,分为左右两组
    x_sorted = pts[np.argsort(_____), :]

    # TODO 2: 左边按y排序(上→下)
    left = x_sorted[:2, :]
    left = left[np.argsort(left[_____, :]), :]

    # TODO 3: 右边按y排序(上→下)
    right = x_sorted[2:, :]
    right = right[np.argsort(right[_____, :]), :]

    # TODO 4: 返回正确顺序
    return np.array([left[0], right[0], right[1], left[1]])
    \end{lstlisting}

    \vspace{0.3cm}

    \textbf{实施方式:}
    \begin{itemize}
        \item 学生用Cursor/Claude Code补全TODO
        \item 对比不同AI的答案
        \item 讲评最佳方案
    \end{itemize}
\end{frame}

\begin{frame}[fragile]{互动4:成果分享}
    \textbf{时间:} 11:25-11:35(第2课时末)

    \vspace{0.3cm}

    \textbf{任务:展示处理效果,评选最佳}

    \vspace{0.3cm}

    \textbf{评选标准:}
    \begin{enumerate}
        \item 去噪效果(30分):噪声去除程度
        \item 细节保留(30分):文字清晰度
        \item 整体美观(20分):视觉效果
        \item 创新性(20分):方法或参数创新
    \end{enumerate}

    \vspace{0.3cm}

    \textbf{实施流程:}
    \begin{enumerate}
        \item 每组展示处理结果(1分钟)
        \item 全班投票评选(问卷星)
        \item 教师点评亮点
        \item 颁发"最佳效果奖"
    \end{enumerate}

    \vspace{0.3cm}

    \begin{exampleblock}{投票提示}
        扫码查看各组效果,为你认为最佳的处理结果投票!
    \end{exampleblock}
\end{frame}

\begin{frame}[fragile]{互动5:快速问答}
    \textbf{时间:} 插入在任意理论讲解后

    \vspace{0.3cm}

    \textbf{问题示例:}
    \begin{enumerate}
        \item 高斯滤波适合哪种噪声?
        \item CLAHE的全称是什么?
        \item Otsu方法的全局阈值有什么特点?
        \item 自适应阈值最适合什么场景?
        \item 透视变换需要几个点?
    \end{enumerate}

    \vspace{0.3cm}

    \textbf{实施方式:}
    \begin{itemize}
        \item 快速抢答
        \item 或使用问卷星快速收集
        \item 答对者加分
    \end{itemize}

    \vspace{0.3cm}

    \begin{alertblock}{设计原则}
        \begin{itemize}
            \item 问题要简单直接
            \item 答案可在课件中找到
            \item 目的是巩固概念理解
        \end{itemize}
    \end{alertblock}
\end{frame}

\begin{frame}{问卷星题目模板}
    \textbf{投票1:去噪方法选择}

    \vspace{0.2cm}

    试卷有明显的黑白噪点,应该选择哪种去噪方法?

    \begin{itemize}
        \item[$\square$] A. 高斯滤波
        \item[$\square$] B. 中值滤波
        \item[$\square$] C. 双边滤波
        \item[$\square$] D. 非局部均值
    \end{itemize}

    \vspace{0.3cm}

    \textbf{投票2:最佳效果评选}

    \vspace{0.2cm}

    哪一组的预处理效果最好?(扫码查看各组效果图)

    \begin{itemize}
        \item[$\square$] 第1组
        \item[$\square$] 第2组
        \item[$\square$] 第3组
        \item[$\square$] 第4组
        \item[$\square$] 第5组
        \item[$\square$] 第6组
    \end{itemize}
\end{frame}
