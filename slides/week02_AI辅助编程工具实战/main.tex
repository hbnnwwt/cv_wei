%=============================================================================
% 主控文件:第2周 AI辅助编程工具实战
% 根据 week1_advice.md 扩展策略重构
%=============================================================================

\documentclass[aspectratio=169, 12pt]{beamer}

% 1. 导入宏包与样式
%===========================================================
% preamble.tex - Beamer 配置文件
%===========================================================

% 中文支持
\usepackage[UTF8]{ctex}

% 图形与表格
\usepackage{graphicx}
\usepackage{booktabs}

% 颜色与图形(必须在listings之前加载)
\usepackage{xcolor}
\usepackage{tikz}
\usetikzlibrary{shapes, arrows.meta, positioning}

% 数学公式
\usepackage{amsmath}
\usepackage{amssymb}

% 代码高亮
\usepackage{listings}

\lstset{
    language=Python,
    basicstyle=\small\ttfamily,
    keywordstyle=\color{blue},
    commentstyle=\color{green!60!black},
    stringstyle=\color{orange},
    breaklines=true,
    showstringspaces=false,
    keepspaces=true
}

% 超链接
\usepackage{hyperref}

%===========================================================
% 主题设置
%===========================================================
\usetheme{Madrid}
\usecolortheme{whale}
\usefonttheme{professionalfonts}

%===========================================================
% 课程信息
%===========================================================
\title[核心开发与调试]{第10周:核心开发与调试}
\subtitle{让系统真正跑起来}
\author{北京石油化工学院\textbackslash 人工智能研究院\textbackslash 王文通}
\institute{通选课}
\date{2025-2026 学年}

%===========================================================
% 自定义命令
%===========================================================
% 高亮命令
\newcommand{\highlight}[1]{\textcolor{red}{\textbf{#1}}}


% 2. 学校信息
\institute{%
    \raisebox{-0.5cm}{\includegraphics[height=1.2cm]{../name.png}}\hspace{0.5cm}%
    \raisebox{-0.5cm}{\includegraphics[height=1.2cm]{../xiaohui.png}}\hspace{0.3cm}%
    \begin{minipage}{6cm}
        \centering
        \textbf{北京石油化工学院}\\
        \textit{人工智能研究院}
    \end{minipage}
}

\title[AI辅助编程工具实战]{第2周:AI辅助编程工具实战}
\subtitle{怎么让AI帮我写代码?}
\author{王文通}
\date{2025-2026 学年}

\begin{document}

% -----------------------------------------------------------------------------
% 标题页与目录
% -----------------------------------------------------------------------------
\begin{frame}
	\titlepage
\end{frame}

\section*{课程概览}
\begin{frame}{课程概览}
	\begin{columns}
		\column{0.5\textwidth}
		\textbf{本周内容:}
		\begin{enumerate}
			\item LLM 时代的编程新范式
			\item AI 编程工具全景
			\item Prompt 工程精讲
			\item CV 领域深度实战
			\item 调试、重构与安全
		\end{enumerate}

		\column{0.5\textwidth}
		\textbf{核心技能:}
		\begin{itemize}
			\item 结构化 Prompt 设计(RTF)
			\item 思维链(Chain-of-Thought)
			\item 少样本提示(Few-shot)
			\item 多轮迭代优化
			\item 代码审查与验证
		\end{itemize}
	\end{columns}

	\vspace{0.3cm}

	\begin{block}{本周作业}
		用 AI 辅助实现 \textbf{答题卡边界检测},提交完整的 AI 对话记录、代码和测试报告
	\end{block}
\end{frame}

% -----------------------------------------------------------------------------
% 教学安排(新增)
% -----------------------------------------------------------------------------
\begin{frame}{教学安排}
	\begin{block}{3课时时间分配}
		\begin{table}
			\centering
			\small
			\begin{tabular}{cll}
				\toprule
				\textbf{时间} & \textbf{环节} & \textbf{内容} \\
				\midrule
				09:00-09:20 & 理论讲解 & AI编程范式与工具发展史 \\
				09:20-09:40 & 现场演示 & Cursor IDE实战演示 \\
				09:40-10:20 & 实践环节 & Prompt工程练习(小组协作) \\
				10:20-10:30 & \textbf{课间休息} & 休息10分钟 \\
				10:30-11:00 & 小组讨论 & 案例分析与代码优化 \\
				11:00-11:15 & 难点突破 & 调试与重构技巧 \\
				11:15-11:30 & 成果分享 & 小组展示与总结 \\
				\bottomrule
			\end{tabular}
		\end{table}
	\end{block}
\end{frame}

% -----------------------------------------------------------------------------
% 学习模式(新增)
% -----------------------------------------------------------------------------
\begin{frame}{三种参与模式}
	\begin{columns}
		\column{0.33\textwidth}
		\begin{block}{观察者模式}
			\textbf{适合:}编程基础较弱的同学
			\begin{itemize}
				\item 理解AI编程概念
				\item 观看教师演示
				\item 理解Prompt设计原理
			\end{itemize}
			\textbf{目标:}能用自然语言描述需求
		\end{block}

		\column{0.33\textwidth}
		\begin{block}{使用者模式}
			\textbf{适合:}有一定编程基础的同学
			\begin{itemize}
				\item 使用AI工具生成代码
				\item 运行并测试代码
				\item 调整参数优化结果
			\end{itemize}
			\textbf{目标:}能独立完成基础任务
		\end{block}

		\column{0.33\textwidth}
		\begin{block}{创造者模式}
			\textbf{适合:}编程基础较好的同学
			\begin{itemize}
				\item 设计高质量的Prompt
				\item 修改和优化AI生成的代码
				\item 探索创新解决方案
			\end{itemize}
			\textbf{目标:}能设计和优化系统
		\end{block}
	\end{columns}

	\vspace{0.5cm}

	\begin{center}
		\textit{提示:本周作业提供基础版/进阶版/挑战版,可根据自身水平选择}
	\end{center}
\end{frame}

% -----------------------------------------------------------------------------
% 小组协作(新增)
% -----------------------------------------------------------------------------
\begin{frame}{小组协作安排}
	\begin{block}{分组原则}
		\begin{itemize}
			\item 每\textbf{4人}一组,不同专业背景混合
			\item 确保每组至少有一名有编程基础的同学
			\item 鼓励强弱搭配,互相帮助
		\end{itemize}
	\end{block}

	\vspace{0.3cm}

	\begin{columns}
		\column{0.5\textwidth}
		\textbf{角色分工:}
		\begin{enumerate}
			\item \textbf{组长}:协调进度,分配任务
			\item \textbf{技术负责人}:把关代码质量,审核AI生成的代码
			\item \textbf{开发者A}:负责Prompt设计
			\item \textbf{开发者B}:负责代码测试与优化
		\end{enumerate}

		\column{0.5\textwidth}
		\textbf{协作建议:}
		\begin{itemize}
			\item 观察者模式同学可担任组长
			\item 使用者模式同学担任开发者
			\item 创造者模式同学担任技术负责人
			\item 鼓励角色轮换,体验不同角色
		\end{itemize}
	\end{columns}
\end{frame}

% -----------------------------------------------------------------------------
% 教学模块
% -----------------------------------------------------------------------------
%=============================================================================
% 模块零:AI编程工具发展史与背景知识 (约 8-12 页)
% 根据 week2_reconstruct.md 新增
%=============================================================================

\section{AI编程工具发展史}

% -----------------------------------------------------------------------------
% 1. AI编程工具发展史 (3-4页)
% -----------------------------------------------------------------------------

\begin{frame}{AI编程工具发展史:从传统IDE到AI助手}
	\begin{center}
		\begin{tikzpicture}[scale=0.9]
			% 时间轴
			\draw[->, thick] (0,0) -- (12,0);

			% 时间节点
			\node[below] at (1,-0.3) {\small 2018};
			\node[below] at (4,-0.3) {\small 2020};
			\node[below] at (7,-0.3) {\small 2022};
			\node[below] at (10,-0.3) {\small 2024-26};

			% 事件
			\node[draw, fill=blue!20, align=center] at (1,1.5) {\small TabNine\\\small 代码补全};
			\node[draw, fill=green!20, align=center] at (4,1.5) {\small GPT-3发布\\\small 代码生成};
			\node[draw, fill=orange!20, align=center] at (7,1.5) {\small ChatGPT\\\small Copilot爆发};
			\node[draw, fill=red!20, align=center] at (10,1.5) {\small Claude/Cursor\\\small 原生AI IDE};
		\end{tikzpicture}
	\end{center}

	\begin{block}{里程碑事件}
		\begin{itemize}
			\item \textbf{2018}:TabNine首次将深度学习用于代码补全
			\item \textbf{2020}:GPT-3展示强大的代码生成能力
			\item \textbf{2022}:ChatGPT和GitHub Copilot引爆AI编程浪潮
			\item \textbf{2024-26}:Claude、Cursor等原生AI IDE重新定义编程体验
		\end{itemize}
	\end{block}
\end{frame}

\begin{frame}{行业影响:编程效率提升10倍+}
	\begin{columns}
		\column{0.5\textwidth}
		\textbf{GitHub Copilot 统计数据(2024):}
		\begin{itemize}
			\item 全球开发者超过 \textbf{1300万}
			\item 代码接受率约 \textbf{30-40\%}
			\item 编码速度提升 \textbf{55\%}
			\item 开发者满意度 \textbf{75\%+}
		\end{itemize}

		\vspace{0.3cm}

		\textbf{Cursor 用户反馈:}
		\begin{itemize}
			\item 代码重构效率提升 \textbf{10倍}
			\item 新功能开发周期缩短 \textbf{50\%}
			\item 代码质量显著改善
		\end{itemize}

		\column{0.5\textwidth}
		\textbf{对开发者的影响:}
		\begin{enumerate}
			\item \textbf{从"记忆"到"理解"}
			\begin{itemize}
				\item 不再需要记忆大量API
				\item 更需要理解原理和设计
			\end{itemize}

			\item \textbf{从"编写"到"审查"}
			\begin{itemize}
				\item AI生成初稿
				\item 人审查、修改、优化
			\end{itemize}

			\item \textbf{从"实现"到"设计"}
			\begin{itemize}
				\item 更多时间用于架构设计
				\item 更少时间用于编码细节
			\end{itemize}
		\end{enumerate}
	\end{columns}
\end{frame}

\begin{frame}{AI编程辅助的优势}
	\begin{columns}
		\column{0.5\textwidth}
		\begin{block}{1. 快速生成代码框架}
			\begin{itemize}
				\item 根据描述生成完整函数
				\item 自动生成类结构
				\item 提供多种实现方案
			\end{itemize}
		\end{block}

		\vspace{0.3cm}

		\begin{block}{2. 解释错误原因}
			\begin{itemize}
				\item 分析报错信息
				\item 定位问题根源
				\item 提供修复建议
			\end{itemize}
		\end{block}

		\column{0.5\textwidth}
		\begin{block}{3. 提供优化建议}
			\begin{itemize}
				\item 性能优化方案
				\item 代码重构建议
				\item 最佳实践推荐
			\end{itemize}
		\end{block}

		\vspace{0.3cm}

		\begin{block}{4. 降低学习门槛}
			\begin{itemize}
				\item 即时答疑解惑
				\item 提供学习路径
				\item 降低技术门槛
			\end{itemize}
		\end{block}
	\end{columns}
\end{frame}

%=============================================================================
% 模块一:认知升级——LLM 时代的编程新范式 (约 12 页)
%=============================================================================

\section{LLM时代的编程新范式}

% -----------------------------------------------------------------------------
% 1. 回顾与痛点
% -----------------------------------------------------------------------------

\begin{frame}[fragile]{回顾:第1周的挑战}
	\begin{columns}
		\column{0.5\textwidth}
		\textbf{还记得这些痛点吗?}
		\begin{itemize}
			\item \texttt{cv2.imread()} 返回 \texttt{None},找不到原因
			\item 图像显示颜色异常
			\item 数组越界、维度不匹配
			\item API 参数太多,记不住
		\end{itemize}

		\vspace{0.3cm}

		\begin{alertblock}{典型错误代码}
			\begin{lstlisting}[language=Python, basicstyle=\ttfamily\tiny]
# 死循环!
while img is not None:
    cv2.imshow('image', img)
    if cv2.waitKey(1) == 'q':
        break
\end{lstlisting}
		\end{alertblock}

		\column{0.5\textwidth}
		\textbf{传统解决方式:}
		\begin{enumerate}
			\item 翻文档(\texttt{docs.opencv.org})
			\item 搜索 StackOverflow
			\item 问同学/老师
			\item 试错(耗费大量时间)
		\end{enumerate}

		\vspace{0.3cm}

		\begin{exampleblock}{AI 辅助的新方式}
			\begin{itemize}
				\item 直接问 AI
				\item 解释错误原因
				\item 给出修改建议
				\item \textbf{降低学习门槛!}
			\end{itemize}
		\end{exampleblock}
	\end{columns}
\end{frame}

% -----------------------------------------------------------------------------
% 2. LLM 的本质
% -----------------------------------------------------------------------------

\begin{frame}{大模型写代码的本质}
	\begin{center}
		\Huge \textbf{Token 预测} \\[0.5cm]
		\Large + \\[0.5cm]
		\Huge \textbf{模式匹配}
	\end{center}

	\vspace{0.5cm}

	\begin{block}{为什么 AI 擅长语法但不一定懂逻辑?}
		\begin{itemize}
			\item \textbf{擅长:} 基于海量代码库的模式复刻
			\item \textbf{不擅长:} 理解你的具体业务逻辑
		\end{itemize}
	\end{block}
\end{frame}

% -----------------------------------------------------------------------------
% 3. 幻觉专题
% -----------------------------------------------------------------------------

\begin{frame}[fragile]{幻觉专题}
	\begin{alertblock}{什么是 AI 幻觉?}
		AI 自信满满地生成看似合理但实际上错误或不存在的信息。
	\end{alertblock}

	\vspace{0.3cm}

	\textbf{案例:AI 发明了一个不存在的 OpenCV 函数}

	\begin{exampleblock}{AI 的"创意"代码}
		\begin{lstlisting}[language=Python, basicstyle=\ttfamily\tiny]
import cv2

img = cv2.imread('exam.jpg')

# 不存在的函数!
fixed = cv2.auto_fix_exposure(img)
\end{lstlisting}
	\end{exampleblock}

	\begin{block}{如何识别幻觉?}
		\begin{itemize}
			\item 查官方文档 (docs.opencv.org)
			\item 在 Python 中 \texttt{dir(cv2)} 查看所有可用函数
			\item 多问一句:"这个函数真的存在吗?"
		\end{itemize}
	\end{block}
\end{frame}

% -----------------------------------------------------------------------------
% 4. 编程范式演进
% -----------------------------------------------------------------------------

\begin{frame}{编程范式的演进}
	\begin{columns}
		\column{0.33\textwidth}
		\textbf{1.0 阶段:查阅文档/StackOverflow}
		\begin{itemize}
			\item 手动搜索
			\item 阅读文档
			\item 试错调试
		\end{itemize}
		\textbf{效率:}两星

		\column{0.33\textwidth}
		\textbf{2.0 阶段:AI 辅助}
		\begin{itemize}
			\item 从"写作者"转变为"审查者"
			\item AI 生成初稿,人审查修改
			\item 快速原型验证
		\end{itemize}
		\textbf{效率:}四星

		\column{0.33\textwidth}
		\textbf{3.0 阶段:AI 原生?}
		\begin{itemize}
			\item AI 直接理解需求生成完整系统
			\item 人主要负责需求定义和架构设计
			\item 可能的未来方向
		\end{itemize}
		\textbf{效率:}五星?
	\end{columns}

	\vspace{0.5cm}

	\begin{block}{核心结论:程序员的核心竞争力转变}
		\begin{center}
			\Large 从 \textbf{``记忆语法''} 转向 \textbf{``问题定义''} 与 \textbf{``代码审查''}
		\end{center}
	\end{block}
\end{frame}

% =============================================================================
% 编程范式对跨专业学习者的启示(新增)
% =============================================================================

\begin{frame}{不同专业背景的学习路径}
	\begin{columns}
		\column{0.33\textwidth}
		\begin{block}{理工科背景}
			\textbf{优势:}编程基础较好
			\begin{itemize}
				\item 直接进入使用者/创造者模式
				\item 重点学习Prompt工程技巧
				\item 探索复杂场景的AI协作
			\end{itemize}
		\end{block}

		\column{0.33\textwidth}
		\begin{block}{文科/经管背景}
			\textbf{优势:}表达与逻辑能力
			\begin{itemize}
				\item 从观察者模式开始
				\item 发挥自然语言优势设计Prompt
				\item 逐步掌握基础代码理解
			\end{itemize}
		\end{block}

		\column{0.33\textwidth}
		\begin{block}{艺术/设计背景}
			\textbf{优势:}视觉敏感度高
			\begin{itemize}
				\item 适合图像处理任务
				\item 关注可视化结果
				\item AI辅助实现创意想法
			\end{itemize}
		\end{block}
	\end{columns}

	\vspace{0.5cm}

	\begin{center}
		\textit{AI辅助编程降低了技术门槛,不同专业都能找到适合自己的切入点}
	\end{center}
\end{frame}

%===========================================================
% 模块02:工具与环境介绍 (5-8页)
%===========================================================

\section{工具与环境}

%----------------------------------------------------------
\subsection{Python开发工具}
%----------------------------------------------------------

\begin{frame}{IDE选择与对比}
    \begin{columns}
        \begin{column}{0.32\textwidth}
            \textbf{PyCharm}
            \begin{itemize}
                \item 专业Python IDE
                \item 智能代码补全
                \item 强大的调试器
                \item 集成测试工具
                \item 专业版收费
            \end{itemize}
        \end{column}
        \begin{column}{0.32\textwidth}
            \textbf{VS Code}
            \begin{itemize}
                \item 轻量编辑器
                \item 插件生态丰富
                \item Python扩展强大
                \item 免费开源
                \item 多语言支持
            \end{itemize}
        \end{column}
        \begin{column}{0.32\textwidth}
            \textbf{Jupyter}
            \begin{itemize}
                \item 交互式环境
                \item 适合数据探索
                \item 可视化集成
                \item 文档混排
                \item 教学演示友好
            \end{itemize}
        \end{column}
    \end{columns}

    \vspace{0.5cm}

    \begin{block}{推荐选择}
        \begin{itemize}
            \item 大型项目开发:PyCharm(专业版)
            \item 日常/多语言开发:VS Code
            \item 数据分析/教学:Jupyter Notebook
        \end{itemize}
    \end{block}
\end{frame}

\begin{frame}[fragile]{调试工具详解}
    \textbf{Python内置调试器pdb:}

    \begin{lstlisting}[basicstyle=\ttfamily\scriptsize]
import pdb

def process_image(image_path):
    image = load_image(image_path)
    pdb.set_trace()  \#设置断点
    result = analyze(image)
    return result

\#常用pdb命令
\#n - 下一行(Step Over)
\#s - 进入函数(Step Into)
\#c - 继续执行
\#p var - 打印变量
\#q - 退出调试
    \end{lstlisting}

    \textbf{ipdb增强版:}
    \begin{itemize}
        \item 语法高亮
        \item Tab自动补全
        \item 更好的交互体验
    \end{itemize}
\end{frame}

\begin{frame}[fragile]{性能分析工具}
    \textbf{cProfile - 标准库性能分析:}

    \begin{lstlisting}[basicstyle=\ttfamily\scriptsize]
import cProfile
import pstats

\#分析代码
profiler = cProfile.Profile()
profiler.enable()

\#要分析的代码
process_batch(images)

profiler.disable()
stats = pstats.Stats(profiler)
stats.sort_stats('cumulative')
stats.print_stats(20)  \#打印前20个热点
    \end{lstlisting}

    \vspace{0.3cm}

    \textbf{line\_profiler - 逐行分析:}
    \begin{itemize}
        \item 精确到每行代码的耗时
        \item 使用\texttt{@profile}装饰器标记
        \item 适合定位性能瓶颈
    \end{itemize}
\end{frame}

\begin{frame}[fragile]{代码质量工具}
    \begin{columns}
        \begin{column}{0.48\textwidth}
            \textbf{静态检查工具}
            \begin{itemize}
                \item \textbf{pylint}:全面代码检查
                \item \textbf{flake8}:PEP 8 + 复杂度检查
                \item \textbf{mypy}:静态类型检查
                \item \textbf{bandit}:安全漏洞扫描
            \end{itemize}
        \end{column}
        \begin{column}{0.48\textwidth}
            \textbf{代码格式化工具}
            \begin{itemize}
                \item \textbf{black}:严格格式化,无配置
                \item \textbf{autopep8}:PEP 8自动修复
                \item \textbf{isort}:import排序
                \item \textbf{yapf}:Google出品,可配置
            \end{itemize}
        \end{column}
    \end{columns}

    \vspace{0.5cm}

    \begin{lstlisting}[basicstyle=\ttfamily\scriptsize]
\#使用示例
$ flake8 myproject/          \#检查代码风格
$ black myproject/           \#自动格式化
$ mypy myproject/            \#类型检查
$ bandit -r myproject/       \#安全扫描
    \end{lstlisting}
\end{frame}

%----------------------------------------------------------
\subsection{测试框架与工具}
%----------------------------------------------------------

\begin{frame}{pytest测试框架}
    \textbf{为什么选择pytest?}

    \begin{itemize}
        \item 简洁的断言语法(无需\texttt{assertEqual})
        \item 强大的Fixture机制
        \item 丰富的插件生态
        \item 详细的失败报告
    \end{itemize}

    \vspace{0.3cm}

    \textbf{unittest风格:}
    \begin{itemize}
        \item[] \texttt{class TestRecognizer(unittest.TestCase):}
        \item[] \hspace{1em} \texttt{def test(self):}
        \item[] \hspace{2em} \texttt{assert result == 'A'}
    \end{itemize}

    \textbf{pytest风格:}
    \begin{itemize}
        \item[] \texttt{def test\_recognize():}
        \item[] \hspace{1em} \texttt{assert result == 'A'}
    \end{itemize}
\end{frame}

\begin{frame}[fragile]{Mock与测试隔离}
    \textbf{为什么要Mock?}

    \begin{itemize}
        \item 隔离被测代码,避免外部依赖
        \item 控制测试环境,确保可重复
        \item 模拟异常情况,提高覆盖率
        \item 加速测试执行(避免真实IO)
    \end{itemize}

    \vspace{0.3cm}

    \textbf{Mock对象示例:}
    \begin{itemize}
        \item[] \texttt{from unittest.mock import Mock}
        \item[] \texttt{mock = Mock()}
        \item[] \texttt{mock.method.return\_value = 42}
        \item[] \texttt{assert mock.method() == 42}
    \end{itemize}

    \textbf{Patch装饰器:}
    \begin{itemize}
        \item[] \texttt{@patch('module.function')}
        \item[] \texttt{def test(mocked\_func):}
        \item[] \hspace{1em} \texttt{mocked\_func.return\_value = 42}
    \end{itemize}
\end{frame}

\begin{frame}{测试覆盖率}
    \textbf{coverage.py使用:}

    \begin{itemize}
        \item[] \texttt{\$ pip install coverage pytest-cov}
        \item[] \texttt{\$ coverage run -m pytest}
        \item[] \texttt{\$ coverage report}
        \item[] \texttt{\$ coverage html}
        \item[] \texttt{\$ open htmlcov/index.html}
    \end{itemize}

    \vspace{0.3cm}

    \textbf{覆盖率目标:}
    \begin{itemize}
        \item 核心逻辑:80\%以上
        \item 工具类:90\%以上
        \item 但高覆盖率不等于高质量,关键是有效测试
    \end{itemize}
\end{frame}

%----------------------------------------------------------
\subsection{版本控制与协作}
%----------------------------------------------------------

\begin{frame}{Git高级操作}
    \textbf{常用高级命令:}

    \begin{itemize}
        \item \textbf{交互式暂存:} \texttt{git add -p}
        \item \textbf{修改提交:} \texttt{git commit --amend}
        \item \textbf{储存修改:} \texttt{git stash} / \texttt{git stash pop}
        \item \textbf{变基整理:} \texttt{git rebase -i HEAD\textasciitilde{}3}
        \item \textbf{二分调试:} \texttt{git bisect start}
    \end{itemize}
\end{frame}

\begin{frame}{Git工作流}
    \begin{columns}
        \begin{column}{0.48\textwidth}
            \textbf{Feature Branch工作流}
            \begin{itemize}
                \item 主分支保持稳定
                \item 每个功能新建分支
                \item 完成后Code Review合并
                \item 适合大多数团队
            \end{itemize}
        \end{column}
        \begin{column}{0.48\textwidth}
            \textbf{Forking工作流}
            \begin{itemize}
                \item 开发者fork仓库
                \item 在自己的仓库开发
                \item 通过Pull Request贡献
                \item 适合开源项目
            \end{itemize}
        \end{column}
    \end{columns}

    \vspace{0.5cm}

    \textbf{Commit Message规范:}
    \begin{itemize}
        \item \texttt{feat:} 新功能
        \item \texttt{fix:} 修复bug
        \item \texttt{docs:} 文档更新
        \item \texttt{refactor:} 重构
        \item \texttt{test:} 测试相关
        \item \texttt{chore:} 构建/工具
    \end{itemize}
\end{frame}

%----------------------------------------------------------
\subsection{AI编程工具}
%----------------------------------------------------------

\begin{frame}{AI编程工具概览}
    \textbf{AI改变编程方式:}

    \vspace{0.3cm}

    \begin{columns}
        \begin{column}{0.48\textwidth}
            \textbf{传统编程流程:}
            \begin{enumerate}
                \item 查阅文档
                \item 编写代码
                \item 调试测试
                \item 优化重构
            \end{enumerate}
        \end{column}
        \begin{column}{0.48\textwidth}
            \textbf{AI辅助编程流程:}
            \begin{enumerate}
                \item 描述需求
                \item AI生成代码
                \item 验证测试
                \item AI辅助优化
            \end{enumerate}
        \end{column}
    \end{columns}

    \vspace{0.5cm}

    \textbf{主流AI编程工具:}
    \begin{itemize}
        \item \textbf{Cursor}:AI原生IDE,内置GPT-4
        \item \textbf{Claude Code}:Anthropic官方CLI工具
        \item \textbf{GitHub Copilot}:代码自动补全
    \end{itemize}
\end{frame}

\begin{frame}[fragile]{Cursor:AI原生IDE}
    \textbf{核心特性:}
    \begin{itemize}
        \item \textbf{AI Chat}:内置对话界面,无需切换窗口
        \item \textbf{代码生成}:根据描述生成完整函数
        \item \textbf{代码解释}:选中代码即可获得详细解释
        \item \textbf{重构建议}:自动识别Code Smell并优化
    \end{itemize}

    \vspace{0.3cm}

    \begin{columns}
        \begin{column}{0.48\textwidth}
            \textbf{快捷键:}
            \begin{itemize}
                \item \texttt{Cmd+L}:打开AI Chat
                \item \texttt{Cmd+K}:生成/编辑代码
                \item \texttt{Cmd+I}:询问当前文件
            \end{itemize}
        \end{column}
        \begin{column}{0.48\textwidth}
            \textbf{适用场景:}
            \begin{itemize}
                \item 快速原型开发
                \item 代码理解学习
                \item 重构优化代码
            \end{itemize}
        \end{column}
    \end{columns}
\end{frame}

\begin{frame}[fragile]{Claude Code:命令行AI助手}
    \textbf{Claude Code特点:}

    \vspace{0.3cm}

    \begin{columns}
        \begin{column}{0.48\textwidth}
            \textbf{优势:}
            \begin{itemize}
                \item 长上下文窗口(200K tokens)
                \item 精准代码理解
                \item 多文件编辑能力
                \item 命令行无缝集成
            \end{itemize}
        \end{column}
        \begin{column}{0.48\textwidth}
            \textbf{使用方式:}
\begin{verbatim}
\#安装
npm install -g @anthropic-ai/claude-code

\#使用
claude  \#启动交互会话
\end{verbatim}
        \end{column}
    \end{columns}

    \vspace{0.3cm}

    \textbf{实战示例:}
\begin{verbatim}
\#用户:解释这个检测填涂的函数
[选中代码]

\#Claude:这个函数通过灰度化和阈值判断...
\#- 先转为灰度图
\#- 计算像素密度
\#- 返回是否填涂
\end{verbatim}
\end{frame}

\begin{frame}[fragile]{GitHub Copilot:智能代码补全}
    \textbf{Copilot工作原理:}
    \begin{itemize}
        \item 基于OpenAI Codex模型
        \item 从GitHub公开代码学习
        \item 根据上下文自动补全
    \end{itemize}

    \vspace{0.3cm}

    \textbf{使用场景:}
    \begin{columns}
        \begin{column}{0.48\textwidth}
            \textbf{最佳场景:}
            \begin{itemize}
                \item 编写样板代码
                \item 生成测试用例
                \item API调用示例
                \item 正则表达式
            \end{itemize}
        \end{column}
        \begin{column}{0.48\textwidth}
            \textbf{使用技巧:}
            \begin{itemize}
                \item 写注释描述意图
                \item 函数命名要清晰
                \item Tab接受建议
                \item Esc忽略建议
            \end{itemize}
        \end{column}
    \end{columns}

    \vspace{0.3cm}

    \begin{exampleblock}{Copilot示例}
\begin{verbatim}
\#注释:计算答题卡填涂区域密度
def calculate_density(image):
    \#Copilot自动补全以下代码
    gray = cv2.cvtColor(image, cv2.COLOR_BGR2GRAY)
    return np.sum(gray > 127) / gray.size
\end{verbatim}
    \end{exampleblock}
\end{frame}

\begin{frame}[fragile]{AI辅助代码审查}
    \textbf{用AI进行Code Review:}

    \vspace{0.3cm}

    \begin{columns}
        \begin{column}{0.48\textwidth}
            \textbf{Prompt模板:}
\begin{verbatim}
请审查这段代码的质量:

1. 代码规范问题
2. 潜在bug
3. 性能优化建议
4. 重构建议

[粘贴代码]
\end{verbatim}
        \end{column}
        \begin{column}{0.48\textwidth}
            \textbf{AI会检查:}
            \begin{itemize}
                \item 命名是否规范
                \item 是否有重复代码
                \item 错误处理是否完善
                \item 边界条件是否考虑
                \item 性能是否可优化
            \end{itemize}
        \end{column}
    \end{columns}

    \vspace{0.3cm}

    \textbf{实战案例:}
\begin{verbatim}
\#原始代码
def proc(img,p):
    return cv2.threshold(img,p,255,1)[1]

\#AI建议:
\#1. 函数名太短,应改为 process_image
\#2. 参数p应改为 threshold_value
\#3. 应添加参数验证
\#4. 应添加文档字符串
\end{verbatim}
\end{frame}

\begin{frame}[fragile]{AI辅助调试实战}
    \textbf{场景:代码运行报错,看不懂错误信息}

    \vspace{0.2cm}

    \textbf{错误信息:}
\begin{verbatim}
cv2.error: OpenCV(4.8.0) :-1: error: (-5:Bad argument)
in function 'threshold'
> Overwhelming requirement: (img.depth() == CV_8U ||
 img.depth() == CV_32F)
\end{verbatim}

    \vspace{0.2cm}

    \textbf{向AI提问:}
\begin{verbatim}
这段OpenCV代码报错,是什么原因?如何修复?

[粘贴错误信息和相关代码]
\end{verbatim}

    \vspace{0.2cm}

    \textbf{AI诊断:}
    \begin{itemize}
        \item \textbf{问题}:图像深度不符合要求
        \item \textbf{原因}:threshold要求CV\_8U或CV\_32F类型
        \item \textbf{修复}:添加类型转换 \texttt{img.astype(np.uint8)}
    \end{itemize}
\end{frame}

%----------------------------------------------------------
% 本模块要点速查(新增)
%----------------------------------------------------------

\subsection{本模块要点速查}

\begin{frame}{模块02-工具环境:要点速查}
    \begin{columns}
        \begin{column}{0.48\textwidth}
            \textbf{开发工具选择}
            \begin{itemize}
                \item 大型项目:PyCharm
                \item 多语言/轻量:VS Code
                \item 数据分析:Jupyter
            \end{itemize}
        \end{column}
        \begin{column}{0.48\textwidth}
            \textbf{代码质量工具}
            \begin{itemize}
                \item 检查:flake8、pylint
                \item 格式化:black
                \item 类型检查:mypy
            \end{itemize}
        \end{column}
    \end{columns}

    \begin{block}{AI编程工具}
        \begin{itemize}
            \item Cursor:AI原生IDE(\texttt{Cmd+L}对话)
            \item Claude Code:命令行助手
            \item Copilot:代码补全
        \end{itemize}
    \end{block}
\end{frame}

%=============================================================================
% 模块三:Prompt工程学精讲 (约 15 页)
%=============================================================================

\section{Prompt工程精讲}

% -----------------------------------------------------------------------------
% 1. RTF 结构化Prompt框架
% -----------------------------------------------------------------------------

\begin{frame}{结构化Prompt框架:RTF 模式}
	\begin{center}
		\begin{tikzpicture}
			\node[draw, fill=red!20] (role) at (0,0) {\shortstack{\textbf{R}ole\\角色}};
			\node[draw, fill=green!20] (task) at (4,0) {\shortstack{\textbf{T}ask\\任务}};
			\node[draw, fill=blue!20] (format) at (8,0) {\shortstack{\textbf{F}ormat\\格式}};

			\draw[->, thick] (role) -- (task) node[midway, above] {定义};
			\draw[->, thick] (task) -- (format) node[midway, above] {指定};
		\end{tikzpicture}
	\end{center}

	\vspace{0.3cm}

	\begin{exampleblock}{RTF 模板示例}
		\textbf{Role (角色):} 你是一个资深计算机视觉算法专家,精通 OpenCV 和图像处理。

		\vspace{0.2cm}

		\textbf{Task (任务):} 实现试卷图像的二值化,要求能自适应处理光照不均的情况。

		\vspace{0.2cm}

		\textbf{Format (格式):}
		\begin{itemize}
			\item 返回带有详细 Docstring 的 Python 函数
			\item 包含输入输出示例
			\item 列出关键参数的调优建议
		\end{itemize}
	\end{exampleblock}
\end{frame}

\begin{frame}{RTF 框架实战对比}
	\begin{columns}
		\column{0.5\textwidth}
		\begin{alertblock}{❌ 不好的 Prompt}
			帮我写个代码处理图像。
		\end{alertblock}

		\vspace{0.3cm}

		\textbf{问题分析:}
		\begin{itemize}
			\item 没有定义角色
			\item 任务模糊
			\item 没有格式要求
		\end{itemize}

		\column{0.5\textwidth}
		\begin{exampleblock}{✅ 好的 Prompt (RTF)}
			\textbf{Role:}你是一位有10年经验的计算机视觉工程师,精通OpenCV。

			\vspace{0.2cm}

			\textbf{Task:}请编写一个Python函数,实现答题卡图像的自适应二值化。

			\vspace{0.2cm}

			\textbf{Format:}
			\begin{itemize}
				\item 提供完整的、可运行的代码
				\item 包含详细的函数文档
				\item 解释关键参数的设置依据
			\end{itemize}
		\end{exampleblock}
	\end{columns}
\end{frame}

% -----------------------------------------------------------------------------
% 2. Chain-of-Thought (思维链)
% -----------------------------------------------------------------------------

\begin{frame}{进阶技巧:Chain-of-Thought (思维链)}
	\begin{block}{什么是思维链?}
		让 AI \textbf{"一步一步思考"}(Step by Step),而不是直接给出答案。
	\end{block}

	\vspace{0.3cm}

	\begin{columns}
		\column{0.5\textwidth}
		\begin{alertblock}{❌ 直接给答案}
			\textbf{Prompt:} 用 OpenCV 实现透视变换矫正倾斜的试卷。

			\vspace{0.2cm}

			\textbf{问题:}
			\begin{itemize}
				\item AI 直接扔出代码
				\item 学生不理解原理
				\item 换个场景就不会了
			\end{itemize}
		\end{alertblock}

		\column{0.5\textwidth}
		\begin{exampleblock}{✅ 思维链引导}
			\textbf{Prompt:}

			请帮我实现试卷透视变换矫正。请按以下步骤思考:

			\textbf{Step 1:} 透视变换的数学原理是什么?需要哪些参数?

			\textbf{Step 2:} 如何从图像中自动找到试卷的四个角点?

			\textbf{Step 3:} 请写出完整的 Python 实现代码
		\end{exampleblock}
	\end{columns}
\end{frame}

\begin{frame}{思维链实战:准确率对比}
	\textbf{测试任务:} 用 OpenCV 实现自适应阈值处理

	\vspace{0.3cm}

	\begin{table}
		\centering
		\begin{tabular}{p{3cm}p{4cm}p{4cm}}
			\toprule
			\textbf{Prompt 类型} & \textbf{直接提问} & \textbf{思维链引导} \\
			\midrule
			代码完整度              & 70\%          & 95\%           \\
			参数解释清晰度            & 一般            & 详细             \\
			边界情况处理             & 很少提及          & 全面考虑           \\
			实际运行成功率            & 60\%          & 90\%           \\
			\bottomrule
		\end{tabular}
	\end{table}

	\vspace{0.3cm}

	\begin{exampleblock}{思维链 Prompt 模板}
		请帮我解决 [问题]。请按以下步骤思考:

		\textbf{Step 1:} 分析问题的核心要点是什么?

		\textbf{Step 2:} 有哪些可能的解决方案?各自的优缺点?

		\textbf{Step 3:} 请给出推荐的实现代码
	\end{exampleblock}
\end{frame}

% -----------------------------------------------------------------------------
% 3. Few-shot (少样本提示)
% -----------------------------------------------------------------------------

\begin{frame}{进阶技巧:Few-shot (少样本提示)}
	\begin{block}{什么是 Few-shot?}
		给 AI 几个 \textbf{正确的例子},让它模仿你的风格或格式处理新任务。
	\end{block}

	\vspace{0.3cm}

	\begin{columns}
		\column{0.5\textwidth}
		\textbf{示例场景:} OpenCV 图像转换

		\vspace{0.2cm}

		\begin{exampleblock}{Few-shot Prompt}
			请模仿以下示例的代码风格和注释规范:

			\vspace{0.2cm}

			\textbf{示例 1 - 灰度化:}

			\texttt{\# 读取图像并转换为灰度图}

			\texttt{\# 参数:image\_path: 图像文件路径}

			\texttt{\# 返回:gray\_image: 灰度图像数组}

			\texttt{def load\_and\_gray(image\_path):}

			\texttt{~~~img = cv2.imread(image\_path)}

			\texttt{~~~gray = cv2.cvtColor(img, cv2.COLOR\_BGR2GRAY)}

			\texttt{~~~return gray}
		\end{exampleblock}

		\column{0.5\textwidth}
		\textbf{待处理任务:}

		\vspace{0.2cm}

		\begin{exampleblock}{继续 Prompt}
			\textbf{示例 2 - 高斯模糊:}

			\texttt{\# 对图像进行高斯模糊}

			\texttt{\# 参数:image: 输入图像}

			\texttt{\# 返回:blurred: 模糊后的图像}

			\texttt{def gaussian\_smooth(image):}

			\texttt{~~~blurred = cv2.GaussianBlur(image, (5,5), 0)}

			\texttt{~~~return blurred}

			\vspace{0.3cm}

			\textbf{任务:}

			请按照上述风格和规范,实现一个图像边缘检测函数。
		\end{exampleblock}
	\end{columns}
\end{frame}

\begin{frame}{Few-shot 效果对比}
	\begin{table}
		\centering
		\begin{tabular}{p{3cm}p{5cm}p{5cm}}
			\toprule
			\textbf{Prompt 类型} & \textbf{无 Few-shot} & \textbf{有 Few-shot} \\
			\midrule
			代码风格一致性            & 随机,不稳定              & 与示例高度一致             \\
			注释完整度              & 较简略                 & 详细,符合示例规范           \\
			参数说明               & 缺失或不清晰              & 完整的 Docstring       \\
			错误处理               & 经常遗漏                & 按示例模式添加             \\
			代码可读性              & 一般                  & 优秀                  \\
			\bottomrule
		\end{tabular}
	\end{table}

	\vspace{0.3cm}

	\begin{block}{Few-shot 使用技巧}
		\begin{enumerate}
			\item \textbf{示例数量:} 2-3 个示例通常足够
			\item \textbf{示例质量:} 确保示例是正确的、高质量的
			\item \textbf{格式一致:} 示例之间保持风格一致
			\item \textbf{明确指令:} 告诉 AI "请按照示例的风格"
		\end{enumerate}
	\end{block}
\end{frame}

% -----------------------------------------------------------------------------
% 4. 上下文窗口管理
% -----------------------------------------------------------------------------

\begin{frame}{上下文窗口管理}
	\begin{alertblock}{问题:为什么代码太长了 AI 就会"忘掉"前面的内容?}
		\begin{itemize}
			\item 大模型有 \textbf{上下文窗口限制}(通常 4K-128K tokens)
			\item 超出限制后,模型会"遗忘"最早的内容
			\item 导致前后文不一致、回答质量下降
		\end{itemize}
	\end{alertblock}

	\vspace{0.3cm}

	\begin{columns}
		\column{0.5\textwidth}
		\textbf{各模型上下文限制:}
		\begin{table}
			\centering
			\small
			\begin{tabular}{lc}
				\toprule
				\textbf{模型} & \textbf{上下文} \\
				\midrule
				GPT-4       & 8K/32K       \\
				GPT-4o      & 128K         \\
				Claude 3.5  & 200K         \\
				DeepSeek    & 64K          \\
				通义千问        & 128K         \\
				\bottomrule
			\end{tabular}
		\end{table}

		\column{0.5\textwidth}
		\textbf{精简 Prompt 的技巧:}
		\begin{enumerate}
			\item \textbf{只提供必要代码:} 不要贴整个文件
			\item \textbf{使用摘要:} "前面我们讨论了X,现在要解决Y"
			\item \textbf{分段处理:} 长任务拆分成多个短任务
			\item \textbf{定期总结:} 让 AI 总结当前进度
		\end{enumerate}
	\end{columns}
\end{frame}

\input{modules/04_cv_practice.tex}
%=============================================================================
% 模块四:核心算法与实战演练
%=============================================================================

\section{图像滤镜原理}

\begin{frame}[fragile]{滤镜 1:灰度化 (Grayscale)}
	\textbf{为什么要灰度化?}
	\begin{itemize}
		\item 减少计算量(数据量降至 1/3)
		\item 识别试卷上的文字,颜色信息通常是不必要的
	\end{itemize}
	\textbf{原理:} $Gray = R \times 0.299 + G \times 0.587 + B \times 0.114$
	\\ (为什么绿色权重最高?因为人眼对绿色最敏感。)

	\begin{lstlisting}[language=Python, basicstyle=\ttfamily\small]
# 方法1:OpenCV 函数
gray = cv2.cvtColor(img, cv2.COLOR_BGR2GRAY)

# 方法2:手动计算(不推荐)
gray = 0.299 * r + 0.587 * g + 0.114 * b
\end{lstlisting}
\end{frame}

\begin{frame}[fragile]{滤镜 2:反色 (Inversion)}
	\textbf{原理:} $NewValue = 255 - OldValue$
	\begin{itemize}
		\item 黑色 (0) $\to$ 白色 (255)
		\item 白色 (255) $\to$ 黑色 (0)
	\end{itemize}
	\textbf{应用:} 增强暗背景下的试卷特征,或者扫描负片。

	\begin{lstlisting}[language=Python, basicstyle=\ttfamily\small]
# 方法1:NumPy 运算
inverted = 255 - img

# 方法2:OpenCV 位运算
inverted = cv2.bitwise_not(img)

# 方法3:NumPy 按位取反
inverted = np.bitwise_not(img)
\end{lstlisting}
\end{frame}

\begin{frame}[fragile]{滤镜 3:亮度调整与"溢出"陷阱}
	\textbf{错误做法:} \texttt{img + 50}
	\\ 如果像素值是 220,加 50 变成 270。而在 \texttt{uint8} 类型下,270 会变成 \highlight{14} (截断/绕回),导致图像出现难看的噪点。

	\begin{lstlisting}[title={安全写法}]
# 使用 numpy 的 clip 函数限制范围
bright_img = np.clip(img.astype(np.int32) + 50, 0, 255).astype(np.uint8)

# 或者使用 OpenCV 内置函数(推荐,速度更快)
bright_img = cv2.add(img, np.array([50.0]))
\end{lstlisting}
\end{frame}

% -----------------------------------------------------------------------------
% 代码实战环节
% -----------------------------------------------------------------------------

\section{代码实战}

\begin{frame}[fragile]{代码实战(1/5):图像翻转与旋转}
	\textbf{场景:}阅卷时试卷可能被倒置,需要自动旋转

	\begin{columns}
		\column{0.5\textwidth}
		\begin{lstlisting}[language=Python, basicstyle=\ttfamily\tiny]
import cv2
import numpy as np

img = cv2.imread('exam.jpg')

# 方法1:NumPy 数组切片
# 垂直翻转(上下颠倒)
flip_v = img[::-1, :, :]

# 水平翻转(左右颠倒)
flip_h = img[:, ::-1, :]

# 水平+垂直翻转(旋转180度)
flip_both = img[::-1, ::-1, :]
\end{lstlisting}

		\column{0.5\textwidth}
		\begin{lstlisting}[language=Python, basicstyle=\ttfamily\tiny]
# 方法2:OpenCV 函数(推荐)
flip_v = cv2.flip(img, 0)      # 垂直
flip_h = cv2.flip(img, 1)      # 水平
flip_both = cv2.flip(img, -1)  # 两者

# 显示对比
cv2.imshow('Original', img)
cv2.imshow('Flip V', flip_v)
cv2.imshow('Flip H', flip_h)
cv2.waitKey(0)
\end{lstlisting}
	\end{columns}

	\vspace{0.2cm}
	\textbf{性能对比:} NumPy 切片比 cv2.flip 快约 20\%,但 cv2.flip 更易读
\end{frame}

\begin{frame}[fragile]{代码实战(2/5):提取答题卡区域(ROI)}
	\textbf{场景:}从整张试卷中提取答题卡区域

	\begin{columns}
		\column{0.5\textwidth}
		\begin{lstlisting}[language=Python, basicstyle=\ttfamily\tiny]
import cv2
import numpy as np

exam = cv2.imread('exam.jpg')
h, w = exam.shape[:2]

# 假设答题卡在右下角
# 坐标:从宽度的60%到末尾,高度的50%到末尾
x1, x2 = int(w * 0.6), w
y1, y2 = int(h * 0.5), h

# 提取 ROI
roi = exam[y1:y2, x1:x2]

# 保存 ROI
cv2.imwrite('answer_sheet.jpg', roi)

print("原图大小:", exam.shape)
print("ROI 大小:", roi.shape)
\end{lstlisting}

		\column{0.5\textwidth}
		\textbf{坐标系统回顾:}
		\begin{itemize}
			\item 原点在左上角 (0, 0)
			\item \texttt{img[y1:y2, x1:x2]}
			\item y 是行(高度),x 是列(宽度)
		\end{itemize}

		\vspace{0.3cm}
		\begin{center}
			\begin{tikzpicture}[scale=0.4]
				\draw[thick, fill=blue!10] (0,0) rectangle (6,4);
				\node at (3,2) {整张试卷};
				\draw[thick, fill=red!30] (3.5,0) rectangle (6,2);
				\node at (4.75,1) {\tiny ROI};
				\draw[->] (3.5,2) -- (3.5,3) node[above] {\tiny y1};
				\draw[->] (3.5,0) -- (3.5,-1) node[below] {\tiny y2};
				\draw[->] (3.5,1) -- (2.5,1) node[left] {\tiny x1};
				\draw[->] (6,1) -- (7,1) node[right] {\tiny x2};
			\end{tikzpicture}
		\end{center}
	\end{columns}
\end{frame}

\begin{frame}[fragile]{代码实战(3/5):通道分离与合并}
	\textbf{场景:}提取特定颜色通道

	\begin{columns}
		\column{0.5\textwidth}
		\begin{lstlisting}[language=Python, basicstyle=\ttfamily\tiny]
import cv2

img = cv2.imread('exam.jpg')

# 方法1:使用 split 函数
b, g, r = cv2.split(img)

# 只保留红色通道,其他设为0
zeros = np.zeros_like(b)
img_r = cv2.merge([zeros, zeros, r])
\end{lstlisting}

		\vspace{0.2cm}
		\begin{lstlisting}[language=Python, basicstyle=\ttfamily\tiny]
# 方法2:直接索引(更快)
img_r = img.copy()
img_r[:, :, 0] = 0  # B通道
img_r[:, :, 1] = 0  # G通道
# R通道保持不变

cv2.imshow('Red Only', img_r)
cv2.waitKey(0)
\end{lstlisting}

		\column{0.5\textwidth}
		\textbf{通道顺序:}
		\begin{itemize}
			\item OpenCV: \textbf{BGR}
			\item matplotlib: \textbf{RGB}
			\item PIL: \textbf{RGB}
		\end{itemize}

		\vspace{0.3cm}
		\begin{alertblock}{常见错误}
			使用 \texttt{plt.imshow(img)} 显示 OpenCV 图像时,颜色会异常!
		\end{alertblock}

		\vspace{0.2cm}
		\textbf{解决方案:}
		\begin{lstlisting}[language=Python, basicstyle=\ttfamily\tiny]
img_rgb = cv2.cvtColor(img, cv2.COLOR_BGR2RGB)
plt.imshow(img_rgb)
\end{lstlisting}
	\end{columns}
\end{frame}

\begin{frame}[fragile]{代码实战(4/5):阅卷系统核心代码}
	\textbf{场景:}检测答题卡填涂位置

	\begin{columns}
		\column{0.5\textwidth}
		\begin{lstlisting}[language=Python, basicstyle=\ttfamily\tiny]
import cv2
import numpy as np

# 1. 读取答题卡区域
roi = cv2.imread('answer_sheet.jpg',
                 cv2.IMREAD_GRAYSCALE)

# 2. 二值化
_, binary = cv2.threshold(roi, 127, 255,
                          cv2.THRESH_BINARY)

# 3. 定义选项位置
positions = [
    (100, 100, 120, 120),  # A
    (100, 130, 120, 150),  # B
    (100, 160, 120, 180),  # C
    (100, 190, 120, 200)   # D
]
\end{lstlisting}

		\column{0.5\textwidth}
		\begin{lstlisting}[language=Python, basicstyle=\ttfamily\tiny]
# 4. 检测每个选项是否被填涂
answers = []
for (x1, y1, x2, y2) in positions:
    option = binary[y1:y2, x1:x2]

    # 计算黑色像素比例
    black_pixels = np.sum(option == 0)
    total_pixels = option.size
    ratio = black_pixels / total_pixels

    # 判断是否填涂(阈值30%)
    if ratio > 0.3:
        answers.append('填涂')
    else:
        answers.append('未填')

print(answers)
\end{lstlisting}
	\end{columns}

	\vspace{0.2cm}
	\textbf{核心思想:}填涂区域黑色像素占比显著高于未填涂区域
\end{frame}

\begin{frame}[fragile]{代码实战(5/5):图像增强对比}
	\textbf{场景:}答题卡光照不均,需要增强对比度

	\begin{columns}
		\column{0.5\textwidth}
		\begin{lstlisting}[language=Python, basicstyle=\ttfamily\tiny]
import cv2
import numpy as np

img = cv2.imread('exam.jpg')

# 方法1:线性对比度调整
# new = alpha * old + beta
enhanced = cv2.convertScaleAbs(
    img, alpha=1.5, beta=30
)
\end{lstlisting}

		\vspace{0.2cm}
		\begin{lstlisting}[language=Python, basicstyle=\ttfamily\tiny]
# 方法2:直方图均衡化
gray = cv2.cvtColor(img, cv2.COLOR_BGR2GRAY)
equalized = cv2.equalizeHist(gray)
\end{lstlisting}

		\column{0.5\textwidth}
		\begin{lstlisting}[language=Python, basicstyle=\ttfamily\tiny]
# 方法3:CLAHE(自适应)
clahe = cv2.createCLAHE(
    clipLimit=2.0,
    tileGridSize=(8,8)
)
enhanced_clahe = clahe.apply(gray)
\end{lstlisting}

		\vspace{0.2cm}
		\textbf{效果对比:}
		\begin{itemize}
			\item \textbf{线性调整}:简单但效果有限
			\item \textbf{直方图均衡化}:全局优化
			\item \textbf{CLAHE}:局部自适应,效果最好
		\end{itemize}

		\vspace{0.2cm}
		\textbf{阅卷推荐:} CLAHE 适合光照不均场景
	\end{columns}
\end{frame}

% -----------------------------------------------------------------------------
% 完整阅卷系统Live Coding
% -----------------------------------------------------------------------------

\begin{frame}[fragile]{Live Coding:完整的阅卷预处理流程}
	\textbf{目标:} 从照片到可识别的图像

	\begin{columns}
		\column{0.5\textwidth}
		\begin{lstlisting}[language=Python, basicstyle=\ttfamily\tiny]
def preprocess_exam(image_path):
    """试卷预处理完整流程"""

    # 1. 读取图像(支持中文路径)
    img = imread_chinese(image_path)

    # 2. 转为灰度
    gray = cv2.cvtColor(img,
                       cv2.COLOR_BGR2GRAY)

    # 3. 去噪
    denoised = cv2.GaussianBlur(
        gray, (5, 5), 0)

    # 4. 对比度增强(CLAHE)
    clahe = cv2.createCLAHE(2.0, (8, 8))
    enhanced = clahe.apply(denoised)

    # 5. 二值化
    binary = cv2.adaptiveThreshold(
        enhanced, 255,
        cv2.ADAPTIVE_THRESH_GAUSSIAN_C,
        cv2.THRESH_BINARY, 11, 2)

    return img, gray, enhanced, binary

# 使用
img, gray, enhanced, binary = \
    preprocess_exam('exam.jpg')
		\end{lstlisting}

		\column{0.5\textwidth}
		\textbf{流程图:}
		\begin{center}
			\begin{tikzpicture}[scale=0.6, node distance=0.8cm]
				\node[draw, rounded corners] (1) {原图};
				\node[draw, rounded corners, below of=1] (2) {灰度};
				\node[draw, rounded corners, below of=2] (3) {去噪};
				\node[draw, rounded corners, below of=3] (4) {增强};
				\node[draw, rounded corners, below of=4] (5) {二值};

				\draw[->] (1) -- (2);
				\draw[->] (2) -- (3);
				\draw[->] (3) -- (4);
				\draw[->] (4) -- (5);
			\end{tikzpicture}
		\end{center}

		\vspace{0.2cm}
		\textbf{展示结果:}
		\begin{itemize}
			\item 原始照片
			\item 预处理后图像
			\item 处理时间对比
		\end{itemize}
	\end{columns}
\end{frame}

\begin{frame}[fragile]{Live Coding:阅卷系统核心检测}
	\textbf{功能1:填涂检测}
	\begin{lstlisting}[language=Python, basicstyle=\ttfamily\tiny]
def detect_bubble(binary, position):
    """检测单个气泡的填涂状态"""
    x1, y1, x2, y2 = position

    # 提取气泡区域
    bubble = binary[y1:y2, x1:x2]

    # 计算填涂密度
    black_pixels = np.sum(bubble == 0)
    total_pixels = bubble.size
    fill_ratio = black_pixels / total_pixels

    # 判断状态
    if fill_ratio > 0.6:
        return 'filled'
    elif fill_ratio < 0.2:
        return 'empty'
    else:
        return 'uncertain'
	\end{lstlisting}

	\vspace{0.2cm}
	\textbf{功能2:多选检测与警告}
	\begin{lstlisting}[language=Python, basicstyle=\ttfamily\tiny]
def detect_multiple_choice(binary, positions):
    """检测多选并警告"""
    results = []
    for pos in positions:
        state = detect_bubble(binary, pos)
        results.append(state)

    # 统计填涂数量
    filled_count = sum(1 for r in results if r == 'filled')

    if filled_count > 1:
        print(f"警告:检测到多选({filled_count}个选项)")

    return results
	\end{lstlisting}
\end{frame}

\begin{frame}[fragile]{Live Coding:图像质量检测函数}
	\textbf{目标:} 自动判断试卷照片是否适合识别

	\begin{columns}
		\column{0.5\textwidth}
		\begin{lstlisting}[language=Python, basicstyle=\ttfamily\tiny]
def check_image_quality(img):
    """检测图像质量"""

    h, w = img.shape[:2]

    # 1. 分辨率检查
    if min(h, w) < 1000:
        return False, "分辨率过低"

    # 2. 曝光检查
    gray = cv2.cvtColor(img,
                       cv2.COLOR_BGR2GRAY)
    mean_brightness = np.mean(gray)

    if mean_brightness < 80:
        return False, "曝光不足"
    elif mean_brightness > 200:
        return False, "过曝"

    # 3. 清晰度检查
    laplacian_var = cv2.Laplacian(
        gray, cv2.CV_64F
    ).var()

    if laplacian_var < 100:
        return False, "图像模糊"

    return True, "质量合格"
		\end{lstlisting}

		\column{0.5\textwidth}
		\textbf{使用示例:}
		\begin{lstlisting}[language=Python, basicstyle=\ttfamily\tiny]
img = imread_chinese('exam.jpg')

is_good, msg = check_image_quality(img)

if is_good:
    print(f"图像质量:{msg}")
    # 继续处理
    result = process_image(img)
else:
    print(f"图像质量:{msg}")
    print("提示用户重新拍照")
		\end{lstlisting}

		\vspace{0.2cm}
		\textbf{质量标准:}
		\begin{itemize}
			\item 分辨率:≥1000px
			\item 曝光:80-200
			\item 清晰度:Laplacian方差 ≥100
		\end{itemize}
	\end{columns}
\end{frame}

\begin{frame}[fragile]{Live Coding:批量处理与结果输出}
	\textbf{批量处理函数:}
	\begin{lstlisting}[language=Python, basicstyle=\ttfamily\tiny]
import os
import json

def batch_process_exams(folder_path, output_path):
    """批量处理试卷"""
    results = []

    for filename in os.listdir(folder_path):
        if not filename.endswith(('.jpg', '.png')):
            continue

        input_path = os.path.join(folder_path, filename)

        # 1. 质量检查
        is_good, msg = check_image_quality(
            imread_chinese(input_path))
        if not is_good:
            print(f"X {filename}: {msg}")
            continue

        # 2. 预处理
        img, gray, enhanced, binary = \
            preprocess_exam(input_path)

        # 3. 检测答题
        answers = detect_all_answers(binary)

        # 4. 评分
        score, details = grade_answers(answers)

        # 5. 保存结果
        result = {
            'filename': filename,
            'score': score,
            'details': details,
            'quality': is_good
        }
        results.append(result)

        print(f"OK {filename}: {score}分")

    # 保存到JSON
    with open(output_path, 'w', encoding='utf-8') as f:
        json.dump(results, f, ensure_ascii=False, indent=2)

    return results
	\end{lstlisting}
\end{frame}

\input{modules/05_debug.tex}
%=============================================================================
% 模块七:案例分析与互动 (约 8 页)
% 根据 week2_reconstruct.md 新增
%=============================================================================

\section{案例分析与互动}

% -----------------------------------------------------------------------------
% 1. 优秀Prompt案例分析 (2-3页)
% -----------------------------------------------------------------------------

\begin{frame}{优秀Prompt案例分析:图像处理Prompt优化}
	\begin{columns}
		\column{0.5\textwidth}
		\begin{alertblock}{[\times] 原始Prompt(模糊)}
			帮我写个代码处理图像。
		\end{alertblock}

		\vspace{0.3cm}

		\textbf{问题分析:}
		\begin{itemize}
			\item 没有定义角色
			\item 任务模糊
			\item 没有格式要求
			\item 没有约束条件
		\end{itemize}

		\column{0.5\textwidth}
		\begin{exampleblock}{[\checkmark] 优化后的Prompt(具体)}
			\textbf{角色:}你是一位有10年经验的计算机视觉工程师,精通OpenCV。

			\vspace{0.2cm}

			\textbf{任务:}请编写一个Python函数,实现答题卡图像的自适应二值化。

			\vspace{0.2cm}

			\textbf{格式:}
			\begin{itemize}
				\item 提供完整的、可运行的代码
				\item 包含详细的函数文档
			\end{itemize}
		\end{exampleblock}
	\end{columns}
\end{frame}

\begin{frame}{优秀Prompt案例分析:算法实现Prompt优化}
	\begin{columns}
		\column{0.5\textwidth}
		\begin{alertblock}{[\times] 原始Prompt(缺少上下文)}
			实现一个排序算法。
		\end{alertblock}

		\vspace{0.3cm}

		\textbf{问题分析:}
		\begin{itemize}
			\item 没有指定算法类型
			\item 没有数据规模要求
			\item 没有性能约束
			\item 缺少示例
		\end{itemize}

		\column{0.5\textwidth}
		\begin{exampleblock}{[\checkmark] 优化后的Prompt(有示例)}
			\textbf{角色:}你是一位算法工程师。

			\vspace{0.2cm}

			\textbf{任务:}实现一个快速排序算法。

			\vspace{0.2cm}

			\textbf{约束:}
			\begin{itemize}
				\item 时间复杂度 O(n log n)
				\item 空间复杂度 O(log n)
			\end{itemize}

			\vspace{0.2cm}

			\textbf{示例:}
			\begin{itemize}
				\item 输入: [3, 6, 8, 10, 1, 2, 1]
				\item 输出: [1, 1, 2, 3, 6, 8, 10]
			\end{itemize}
		\end{exampleblock}
	\end{columns}
\end{frame}

% -----------------------------------------------------------------------------
% 2. 常见错误分析与解决 (2-3页)
% -----------------------------------------------------------------------------

\begin{frame}{常见错误分析与解决:Prompt太模糊}
	\begin{alertblock}{[\times] 问题Prompt}
		帮我写代码。
	\end{alertblock}

	\vspace{0.3cm}

	\textbf{分析原因:}
	\begin{itemize}
		\item 缺少角色设定
		\item 缺少任务描述
		\item 缺少约束条件
		\item AI无法确定具体需求
	\end{itemize}

	\vspace{0.3cm}

	\begin{exampleblock}{[\checkmark] 解决方案}
		\textbf{添加角色:}你是一位Python开发者。

		\vspace{0.2cm}

		\textbf{明确任务:}请帮我写一个读取CSV文件的Python函数。

		\vspace{0.2cm}

		\textbf{增加约束:}
		\begin{itemize}
			\item 使用pandas库
			\item 处理文件不存在的情况
			\item 返回DataFrame对象
		\end{itemize}
	\end{exampleblock}
\end{frame}

\begin{frame}{常见错误分析与解决:Prompt太复杂}
	\begin{alertblock}{[\times] 问题Prompt}
		你是一位有5年OpenCV经验、精通Python和C++、熟悉深度学习和机器学习的工程师,请用Python和OpenCV实现一个完整的智能阅卷系统,包括图像采集、预处理、版面分析、OCR识别、评分输出等所有模块,要求代码有详细注释、性能优化、异常处理、单元测试、文档生成...
	\end{alertblock}

	\vspace{0.3cm}

	\textbf{分析原因:}
	\begin{itemize}
		\item Prompt过长
		\item 信息过载
		\item AI容易混淆
		\item 一次性要求太多
	\end{itemize}

	\vspace{0.3cm}

	\begin{exampleblock}{[\checkmark] 解决方案}
		\textbf{拆分为多个子任务:}

		\vspace{0.2cm}

		\textbf{任务1:}请先设计智能阅卷系统的整体架构。

		\vspace{0.2cm}

		\textbf{任务2:}基于上述架构,实现图像预处理模块。

		\vspace{0.2cm}

		\textbf{任务3:}继续实现版面分析模块。

		\vspace{0.2cm}

		\textbf{逐步实现} $\rightarrow$ \textbf{逐步优化} $\rightarrow$ \textbf{最终完善}
	\end{exampleblock}
\end{frame}

% -----------------------------------------------------------------------------
% 3. 课堂互动与Quiz (1-2页)
% -----------------------------------------------------------------------------

\begin{frame}{实时互动环节}
	\begin{exampleblock}{互动方式建议}
		\begin{itemize}
			\item \textbf{问卷星实时投票}:使用手机扫码参与,实时显示统计结果
			\item \textbf{代码拼图游戏}:将打乱的代码片段重新排序,小组竞赛
			\item \textbf{错误找茬挑战}:展示有问题的代码,集体找出bug
		\end{itemize}
	\end{exampleblock}

	\vspace{0.3cm}

	\begin{columns}
		\column{0.5\textwidth}
		\textbf{问题1:好的Prompt应该包含哪些要素?}
		\begin{itemize}
			\item[A] 角色设定
			\item[B] 任务描述
			\item[C] 约束条件
			\item[D] \textbf{以上都是} [\checkmark]
		\end{itemize}

		\vspace{0.3cm}

		\textbf{问题2:以下哪种Prompt更适合调试?}
		\begin{itemize}
			\item[A] "代码错了,帮我修"
			\item[B] \textbf{"错误信息是xxx,请分析原因并提供解决方案"} [\checkmark]
		\end{itemize}

		\column{0.5\textwidth}
		\textbf{问题3:AI生成的代码一定正确吗?}
		\begin{itemize}
			\item[A] 一定正确
			\item[B] \textbf{不一定正确,需要验证和测试} [\checkmark]
		\end{itemize}

		\vspace{0.3cm}

		\textbf{问题4:以下哪种情况说明Prompt太模糊?}
		\begin{itemize}
			\item[A] AI生成了完全符合需求的代码
			\item[B] \textbf{AI反复询问具体需求,或生成的代码与期望差距很大} [\checkmark]
		\end{itemize}
	\end{columns}

	\vspace{0.5cm}

	\begin{center}
		\highlight{使用问卷星扫码投票,实时显示全班正确率}
	\end{center}
\end{frame}

\begin{frame}[fragile]{代码找错挑战}
	\begin{exampleblock}{找出以下Prompt中的3个问题}
		\begin{lstlisting}[language=Python, basicstyle=\ttfamily\small]
帮我写个代码。

实现一个系统。

要完整的。

快点。
		\end{lstlisting}
	\end{exampleblock}

	\vspace{0.3cm}

	\begin{block}{答案揭晓}
		\begin{enumerate}
			\item \textbf{问题1:缺少角色设定}
			\begin{itemize}
				\item 没有告诉AI它应该扮演什么角色
				\item 解决方案:添加"你是一位Python开发者"
			\end{itemize}

			\item \textbf{问题2:任务描述模糊}
			\begin{itemize}
				\item "实现一个系统"太笼统
				\item 解决方案:具体说明"实现一个智能阅卷系统,包含图像采集、预处理、OCR识别模块"
			\end{itemize}

			\item \textbf{问题3:缺少约束条件}
			\begin{itemize}
				\item 没有格式、技术栈、时间等约束
				\item 解决方案:添加"使用Python和OpenCV,代码要有详细注释,性能要优化"
			\end{itemize}
		\end{enumerate}
	\end{block}
\end{frame}

%=============================================================================
% 模块六:总结与作业 (约 3 页)
%=============================================================================

\section{总结与作业}

% -----------------------------------------------------------------------------
% 1. 总结梳理
% -----------------------------------------------------------------------------

\begin{frame}{建立自己的"AI 协作 SOP"}
    \begin{block}{什么是 SOP?}
        \textbf{S}tandard \textbf{O}perating \textbf{P}rocedure(标准操作流程)

        一套可重复、可优化的工作流程
    \end{block}

    \vspace{0.3cm}

    \begin{center}
        \begin{tikzpicture}
            \node[draw, fill=blue!20] (step1) at (0,0) {1. 问题定义};
            \node[draw, fill=green!20] (step2) at (2.5,0) {2. Prompt 设计};
            \node[draw, fill=yellow!20] (step3) at (5,0) {3. AI 生成};
            \node[draw, fill=orange!20] (step4) at (7.5,0) {4. 审查验证};
            \node[draw, fill=purple!20] (step5) at (10,0) {5. 迭代优化};

            \draw[->, thick] (step1) -- (step2);
            \draw[->, thick] (step2) -- (step3);
            \draw[->, thick] (step3) -- (step4);
            \draw[->, thick] (step4) -- (step5);
            \draw[->, thick, dashed] (step5.south) -- ++(0,-0.5) -| (step1.south);
        \end{tikzpicture}
    \end{center}

    \vspace{0.3cm}

    \begin{columns}
        \column{0.5\textwidth}
        \textbf{1. 问题定义}
        \begin{itemize}
            \item 明确要解决的问题
            \item 确定输入输出
            \item 列出约束条件
        \end{itemize}

        \textbf{2. Prompt 设计}
        \begin{itemize}
            \item 应用 RTF 框架
            \item 使用思维链
            \item 提供 Few-shot 示例
        \end{itemize}

        \column{0.5\textwidth}
        \textbf{3. AI 生成}
        \begin{itemize}
            \item 运行 Prompt
            \item 获取初始输出
            \item 记录生成时间
        \end{itemize}

        \textbf{4. 审查验证}
        \begin{itemize}
            \item 检查代码逻辑
            \item 运行测试用例
            \item 验证输出结果
        \end{itemize}

        \textbf{5. 迭代优化}
        \begin{itemize}
            \item 识别改进点
            \item 设计优化 Prompt
            \item 重复流程直到满意
        \end{itemize}
    \end{columns}
\end{frame}

% -----------------------------------------------------------------------------
% 2. 知识点总结
% -----------------------------------------------------------------------------

\begin{frame}{知识点总结:Prompt工程核心要点}
    \begin{columns}
        \column{0.5\textwidth}
        \begin{block}{1. Prompt工程三要素}
            \begin{enumerate}
                \item \textbf{Role(角色)}
                    \begin{itemize}
                        \item 明确AI的身份
                        \item 例如:资深CV工程师
                    \end{itemize}
                \item \textbf{Task(任务)}
                    \begin{itemize}
                        \item 清晰说明要做什么
                        \item 具体、明确、可执行
                    \end{itemize}
                \item \textbf{Format(格式)}
                    \begin{itemize}
                        \item 指定输出格式和要求
                        \item 代码、文档、表格等
                    \end{itemize}
            \end{enumerate}
        \end{block}

        \column{0.5\textwidth}
        \begin{block}{2. Prompt优化技巧}
            \begin{enumerate}
                \item \textbf{少样本提示(Few-Shot)}
                    \begin{itemize}
                        \item 提供1-3个示例
                        \item 模仿学习
                    \end{itemize}
                \item \textbf{思维链(Chain-of-Thought)}
                    \begin{itemize}
                        \item 分步思考
                        \item 推理过程
                    \end{itemize}
                \item \textbf{角色扮演(Role-Playing)}
                    \begin{itemize}
                        \item 设定专业身份
                        \item 专业术语
                    \end{itemize}
            \end{enumerate}
        \end{block}
    \end{columns}
\end{frame}

\begin{frame}{知识点总结:AI工具使用规范}
    \begin{columns}
        \column{0.33\textwidth}
        \begin{block}{\textcolor{green!70!black}{[\checkmark] 允许}}
            \begin{itemize}
                \item 用AI解释概念
                \item 调试错误
                \item 优化代码
                \item 学习新技术
            \end{itemize}
        \end{block}

        \column{0.33\textwidth}
        \begin{block}{\textcolor{blue!70!black}{[\textbf{+}] 鼓励}}
            \begin{itemize}
                \item 生成对比实验
                \item 可视化结果
                \item 探索不同方案
                \item 提高学习效率
            \end{itemize}
        \end{block}

        \column{0.33\textwidth}
        \begin{block}{\textcolor{red!70!black}{[\textbackslash times] 禁止}}
            \begin{itemize}
                \item 直接复制完整代码
                \item 用AI完成全部作业
                \item 不思考直接照搬
                \item 欺骗性使用
            \end{itemize}
        \end{block}
    \end{columns}

    \vspace{0.5cm}

    \begin{center}
        \begin{tikzpicture}
            \node[draw, fill=green!20, rounded corners] at (0,0) {\textbf{核心理念:理解原理 > 复制代码}};
            \node[draw, fill=blue!20, rounded corners] at (6,0) {\textbf{AI是助手,不是替代者}};
        \end{tikzpicture}
    \end{center}
\end{frame}

% -----------------------------------------------------------------------------
% 3. 延伸学习资源
% -----------------------------------------------------------------------------

\begin{frame}{延伸学习资源}
    \begin{columns}
        \column{0.5\textwidth}
        \begin{block}{官方文档}
            \begin{enumerate}
                \item \textbf{ChatGPT官方文档}
                    \begin{itemize}
                        \item \url{https://platform.openai.com/docs}
                        \item API使用指南
                        \item 最佳实践
                    \end{itemize}

                \item \textbf{Claude官方文档}
                    \begin{itemize}
                        \item \url{https://docs.anthropic.com}
                        \item Claude API文档
                        \item 提示工程指南
                    \end{itemize}

                \item \textbf{通义千问官方文档}
                    \begin{itemize}
                        \item \url{https://help.aliyun.com/qwen}
                        \item 快速入门指南
                        \item SDK文档
                    \end{itemize}
            \end{enumerate}
        \end{block}

        \column{0.5\textwidth}
        \begin{block}{推荐教程与资源}
            \begin{enumerate}
            \setcounter{enumi}{3}
                \item \textbf{Prompt工程教程}
                    \begin{itemize}
                        \item 《Prompt Engineering Guide》
                        \item \url{https://www.promptingguide.ai}
                        \item 中文版:提示工程指南
                    \end{itemize}

                \item \textbf{GitHub Copilot使用指南}
                    \begin{itemize}
                        \item \url{https://github.com/features/copilot}
                        \item 官方文档和教程
                        \item 最佳实践分享
                    \end{itemize}

                \item \textbf{Cursor官方教程}
                    \begin{itemize}
                        \item \url{https://cursor.sh/tutorials}
                        \item 视频教程
                        \item 快捷键指南
                    \end{itemize}

                \item \textbf{社区资源}
                    \begin{itemize}
                        \item Reddit: r/ChatGPT, r/ClaudeAI
                        \item 知乎AI编程话题
                        \item 掘金AI专栏
                    \end{itemize}
            \end{enumerate}
        \end{block}
    \end{columns}
\end{frame}

% -----------------------------------------------------------------------------
% 4. 本周作业详解
% -----------------------------------------------------------------------------

\begin{frame}{本周作业:答题卡边界检测}
    \begin{block}{作业题目}
        用 AI 辅助实现 \textbf{答题卡边界检测} 程序
    \end{block}

    \vspace{0.3cm}

    \begin{columns}
        \column{0.5\textwidth}
        \textbf{项目关联}

        这是 \textbf{AI 阅卷助手} 的第一步:
        \begin{enumerate}
            \item 图像采集与预处理
            \item \textbf{答题卡定位(当前任务)}
            \item 填涂检测与识别
            \item 手写文字 OCR
            \item 成绩统计与输出
        \end{enumerate}

        \vspace{0.3cm}

        \textbf{分层任务说明}

        \begin{itemize}
            \item \textcolor{blue}{\textbf{基础版}}:检测边界+透视矫正
            \item \textcolor{orange}{\textbf{进阶版}}:增加鲁棒性处理
            \item \textcolor{red}{\textbf{挑战版}}:封装成可复用类
        \end{itemize}

        \column{0.5\textwidth}
        \textbf{AI 对话要求}

        \vspace{0.2cm}

        至少 \textbf{3 轮}交互:
        \begin{enumerate}
            \item 第1轮:基础实现
            \item 第2轮:优化改进
            \item 第3轮:完善功能
        \end{enumerate}

        \vspace{0.3cm}

        \textbf{不同参与模式的建议}
        \begin{itemize}
            \item \textbf{观察者}:完成基础版,重点理解Prompt设计
            \item \textbf{使用者}:完成进阶版,重点调试运行
            \item \textbf{创造者}:完成挑战版,重点优化创新
        \end{itemize}
    \end{columns}
\end{frame}

\begin{frame}{作业提交内容与评分标准}
    \begin{columns}
        \column{0.5\textwidth}
        \textbf{提交内容}

        \vspace{0.2cm}

        \begin{enumerate}
            \item \textbf{AI 对话记录}
                  \begin{itemize}
                      \item 截图或文本
                      \item 标注关键Prompt
                      \item 说明迭代过程
                  \end{itemize}

            \item \textbf{最终代码}
                  \begin{itemize}
                      \item 完整可运行
                      \item 必要注释
                  \end{itemize}

            \item \textbf{测试结果}
                  \begin{itemize}
                      \item 测试图片
                      \item 处理结果
                  \end{itemize}

            \item \textbf{反思报告}
                  \begin{itemize}
                      \item Prompt设计思路
                      \item 问题与解决
                      \item 学习体会
                  \end{itemize}
        \end{enumerate}

        \column{0.5\textwidth}
        \textbf{详细评分标准(100分)}

        \vspace{0.2cm}

        \begin{table}
            \centering
            \small
            \begin{tabular}{cl}
                \toprule
                \textbf{分项} & \textbf{评分要点} \\
                \midrule
                \textbf{Prompt设计} & 结构清晰、迭代合理 \\
                (30分) & RTF框架应用恰当 \\
                \midrule
                \textbf{代码完整性} & 功能完整、可运行 \\
                (30分) & 注释清晰、结构合理 \\
                \midrule
                \textbf{测试覆盖} & 多场景测试 \\
                (20分) & 边界情况考虑 \\
                \midrule
                \textbf{反思深度} & 思考深入 \\
                (20分) & 有独特见解 \\
                \bottomrule
            \end{tabular}
        \end{table}
    \end{columns}
\end{frame}

\begin{frame}{分层任务详解}
    \begin{columns}
        \column{0.33\textwidth}
        \begin{block}{\textcolor{blue}{基础版}}
            \textbf{要求:}
            \begin{itemize}
                \item 读取答题卡图像
                \item Canny边缘检测
                \item 查找轮廓
                \item 绘制边界框
            \end{itemize}

            \vspace{0.2cm}

            \textbf{适合:}
            \begin{itemize}
                \item 观察者模式
                \item 编程基础较弱
            \end{itemize}

            \vspace{0.2cm}

            \textbf{目标:} 理解基本流程
        \end{block}

        \column{0.33\textwidth}
        \begin{block}{\textcolor{orange}{进阶版}}
            \textbf{要求:}
            \begin{itemize}
                \item 完成基础版所有功能
                \item 添加预处理(去噪)
                \item 自适应阈值
                \item 透视变换矫正
            \end{itemize}

            \vspace{0.2cm}

            \textbf{适合:}
            \begin{itemize}
                \item 使用者模式
                \item 有一定编程基础
            \end{itemize}

            \vspace{0.2cm}

            \textbf{目标:} 独立完成实用功能
        \end{block}

        \column{0.33\textwidth}
        \begin{block}{\textcolor{red}{挑战版}}
            \textbf{要求:}
            \begin{itemize}
                \item 完成进阶版所有功能
                \item 封装成类
                \item 处理极端情况
                \item 可视化调试信息
            \end{itemize}

            \vspace{0.2cm}

            \textbf{适合:}
            \begin{itemize}
                \item 创造者模式
                \item 编程基础较好
            \end{itemize}

            \vspace{0.2cm}

            \textbf{目标:} 生产级可用代码
        \end{block}
    \end{columns}

    \vspace{0.5cm}

    \begin{center}
        \textit{鼓励挑战更高版本,完成进阶版可获附加分5分,完成挑战版可获附加分10分}
    \end{center}
\end{frame}

% -----------------------------------------------------------------------------
% 5. 结束页
% -----------------------------------------------------------------------------

\begin{frame}
\centering
\Huge \textbf{谢谢!}

\vspace{1cm}

\Large
\textbf{第3周预告:图像预处理与增强}

\vspace{0.5cm}

\normalsize
\textcolor{blue}{故事问题:试卷拍照模糊怎么办?}

\vspace{0.3cm}

\begin{itemize}
    \item 图像去噪(高斯/中值滤波)
    \item 图像二值化(全局/Otsu/自适应)
    \item 透视矫正(透视变换)
\end{itemize}
\end{frame}


\end{document}
