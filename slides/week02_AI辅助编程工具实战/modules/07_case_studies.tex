%=============================================================================
% 模块七:案例分析与互动 (约 8 页)
% 根据 week2_reconstruct.md 新增
%=============================================================================

\section{案例分析与互动}

% -----------------------------------------------------------------------------
% 1. 优秀Prompt案例分析 (2-3页)
% -----------------------------------------------------------------------------

\begin{frame}{优秀Prompt案例分析:图像处理Prompt优化}
	\begin{columns}
		\column{0.5\textwidth}
		\begin{alertblock}{[\times] 原始Prompt(模糊)}
			帮我写个代码处理图像。
		\end{alertblock}

		\vspace{0.3cm}

		\textbf{问题分析:}
		\begin{itemize}
			\item 没有定义角色
			\item 任务模糊
			\item 没有格式要求
			\item 没有约束条件
		\end{itemize}

		\column{0.5\textwidth}
		\begin{exampleblock}{[\checkmark] 优化后的Prompt(具体)}
			\textbf{角色:}你是一位有10年经验的计算机视觉工程师,精通OpenCV。

			\vspace{0.2cm}

			\textbf{任务:}请编写一个Python函数,实现答题卡图像的自适应二值化。

			\vspace{0.2cm}

			\textbf{格式:}
			\begin{itemize}
				\item 提供完整的、可运行的代码
				\item 包含详细的函数文档
			\end{itemize}
		\end{exampleblock}
	\end{columns}
\end{frame}

\begin{frame}{优秀Prompt案例分析:算法实现Prompt优化}
	\begin{columns}
		\column{0.5\textwidth}
		\begin{alertblock}{[\times] 原始Prompt(缺少上下文)}
			实现一个排序算法。
		\end{alertblock}

		\vspace{0.3cm}

		\textbf{问题分析:}
		\begin{itemize}
			\item 没有指定算法类型
			\item 没有数据规模要求
			\item 没有性能约束
			\item 缺少示例
		\end{itemize}

		\column{0.5\textwidth}
		\begin{exampleblock}{[\checkmark] 优化后的Prompt(有示例)}
			\textbf{角色:}你是一位算法工程师。

			\vspace{0.2cm}

			\textbf{任务:}实现一个快速排序算法。

			\vspace{0.2cm}

			\textbf{约束:}
			\begin{itemize}
				\item 时间复杂度 O(n log n)
				\item 空间复杂度 O(log n)
			\end{itemize}

			\vspace{0.2cm}

			\textbf{示例:}
			\begin{itemize}
				\item 输入: [3, 6, 8, 10, 1, 2, 1]
				\item 输出: [1, 1, 2, 3, 6, 8, 10]
			\end{itemize}
		\end{exampleblock}
	\end{columns}
\end{frame}

% -----------------------------------------------------------------------------
% 2. 常见错误分析与解决 (2-3页)
% -----------------------------------------------------------------------------

\begin{frame}{常见错误分析与解决:Prompt太模糊}
	\begin{alertblock}{[\times] 问题Prompt}
		帮我写代码。
	\end{alertblock}

	\vspace{0.3cm}

	\textbf{分析原因:}
	\begin{itemize}
		\item 缺少角色设定
		\item 缺少任务描述
		\item 缺少约束条件
		\item AI无法确定具体需求
	\end{itemize}

	\vspace{0.3cm}

	\begin{exampleblock}{[\checkmark] 解决方案}
		\textbf{添加角色:}你是一位Python开发者。

		\vspace{0.2cm}

		\textbf{明确任务:}请帮我写一个读取CSV文件的Python函数。

		\vspace{0.2cm}

		\textbf{增加约束:}
		\begin{itemize}
			\item 使用pandas库
			\item 处理文件不存在的情况
			\item 返回DataFrame对象
		\end{itemize}
	\end{exampleblock}
\end{frame}

\begin{frame}{常见错误分析与解决:Prompt太复杂}
	\begin{alertblock}{[\times] 问题Prompt}
		你是一位有5年OpenCV经验、精通Python和C++、熟悉深度学习和机器学习的工程师,请用Python和OpenCV实现一个完整的智能阅卷系统,包括图像采集、预处理、版面分析、OCR识别、评分输出等所有模块,要求代码有详细注释、性能优化、异常处理、单元测试、文档生成...
	\end{alertblock}

	\vspace{0.3cm}

	\textbf{分析原因:}
	\begin{itemize}
		\item Prompt过长
		\item 信息过载
		\item AI容易混淆
		\item 一次性要求太多
	\end{itemize}

	\vspace{0.3cm}

	\begin{exampleblock}{[\checkmark] 解决方案}
		\textbf{拆分为多个子任务:}

		\vspace{0.2cm}

		\textbf{任务1:}请先设计智能阅卷系统的整体架构。

		\vspace{0.2cm}

		\textbf{任务2:}基于上述架构,实现图像预处理模块。

		\vspace{0.2cm}

		\textbf{任务3:}继续实现版面分析模块。

		\vspace{0.2cm}

		\textbf{逐步实现} $\rightarrow$ \textbf{逐步优化} $\rightarrow$ \textbf{最终完善}
	\end{exampleblock}
\end{frame}

% -----------------------------------------------------------------------------
% 3. 课堂互动与Quiz (1-2页)
% -----------------------------------------------------------------------------

\begin{frame}{实时互动环节}
	\begin{exampleblock}{互动方式建议}
		\begin{itemize}
			\item \textbf{问卷星实时投票}:使用手机扫码参与,实时显示统计结果
			\item \textbf{代码拼图游戏}:将打乱的代码片段重新排序,小组竞赛
			\item \textbf{错误找茬挑战}:展示有问题的代码,集体找出bug
		\end{itemize}
	\end{exampleblock}

	\vspace{0.3cm}

	\begin{columns}
		\column{0.5\textwidth}
		\textbf{问题1:好的Prompt应该包含哪些要素?}
		\begin{itemize}
			\item[A] 角色设定
			\item[B] 任务描述
			\item[C] 约束条件
			\item[D] \textbf{以上都是} [\checkmark]
		\end{itemize}

		\vspace{0.3cm}

		\textbf{问题2:以下哪种Prompt更适合调试?}
		\begin{itemize}
			\item[A] "代码错了,帮我修"
			\item[B] \textbf{"错误信息是xxx,请分析原因并提供解决方案"} [\checkmark]
		\end{itemize}

		\column{0.5\textwidth}
		\textbf{问题3:AI生成的代码一定正确吗?}
		\begin{itemize}
			\item[A] 一定正确
			\item[B] \textbf{不一定正确,需要验证和测试} [\checkmark]
		\end{itemize}

		\vspace{0.3cm}

		\textbf{问题4:以下哪种情况说明Prompt太模糊?}
		\begin{itemize}
			\item[A] AI生成了完全符合需求的代码
			\item[B] \textbf{AI反复询问具体需求,或生成的代码与期望差距很大} [\checkmark]
		\end{itemize}
	\end{columns}

	\vspace{0.5cm}

	\begin{center}
		\highlight{使用问卷星扫码投票,实时显示全班正确率}
	\end{center}
\end{frame}

\begin{frame}[fragile]{代码找错挑战}
	\begin{exampleblock}{找出以下Prompt中的3个问题}
		\begin{lstlisting}[language=Python, basicstyle=\ttfamily\small]
帮我写个代码。

实现一个系统。

要完整的。

快点。
		\end{lstlisting}
	\end{exampleblock}

	\vspace{0.3cm}

	\begin{block}{答案揭晓}
		\begin{enumerate}
			\item \textbf{问题1:缺少角色设定}
			\begin{itemize}
				\item 没有告诉AI它应该扮演什么角色
				\item 解决方案:添加"你是一位Python开发者"
			\end{itemize}

			\item \textbf{问题2:任务描述模糊}
			\begin{itemize}
				\item "实现一个系统"太笼统
				\item 解决方案:具体说明"实现一个智能阅卷系统,包含图像采集、预处理、OCR识别模块"
			\end{itemize}

			\item \textbf{问题3:缺少约束条件}
			\begin{itemize}
				\item 没有格式、技术栈、时间等约束
				\item 解决方案:添加"使用Python和OpenCV,代码要有详细注释,性能要优化"
			\end{itemize}
		\end{enumerate}
	\end{block}
\end{frame}
