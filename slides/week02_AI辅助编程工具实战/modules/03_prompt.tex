%=============================================================================
% 模块三:Prompt工程学精讲 (约 15 页)
%=============================================================================

\section{Prompt工程精讲}

% -----------------------------------------------------------------------------
% 1. RTF 结构化Prompt框架
% -----------------------------------------------------------------------------

\begin{frame}{结构化Prompt框架:RTF 模式}
	\begin{center}
		\begin{tikzpicture}
			\node[draw, fill=red!20] (role) at (0,0) {\shortstack{\textbf{R}ole\\角色}};
			\node[draw, fill=green!20] (task) at (4,0) {\shortstack{\textbf{T}ask\\任务}};
			\node[draw, fill=blue!20] (format) at (8,0) {\shortstack{\textbf{F}ormat\\格式}};

			\draw[->, thick] (role) -- (task) node[midway, above] {定义};
			\draw[->, thick] (task) -- (format) node[midway, above] {指定};
		\end{tikzpicture}
	\end{center}

	\vspace{0.3cm}

	\begin{exampleblock}{RTF 模板示例}
		\textbf{Role (角色):} 你是一个资深计算机视觉算法专家,精通 OpenCV 和图像处理。

		\vspace{0.2cm}

		\textbf{Task (任务):} 实现试卷图像的二值化,要求能自适应处理光照不均的情况。

		\vspace{0.2cm}

		\textbf{Format (格式):}
		\begin{itemize}
			\item 返回带有详细 Docstring 的 Python 函数
			\item 包含输入输出示例
			\item 列出关键参数的调优建议
		\end{itemize}
	\end{exampleblock}
\end{frame}

\begin{frame}{RTF 框架实战对比}
	\begin{columns}
		\column{0.5\textwidth}
		\begin{alertblock}{❌ 不好的 Prompt}
			帮我写个代码处理图像。
		\end{alertblock}

		\vspace{0.3cm}

		\textbf{问题分析:}
		\begin{itemize}
			\item 没有定义角色
			\item 任务模糊
			\item 没有格式要求
		\end{itemize}

		\column{0.5\textwidth}
		\begin{exampleblock}{✅ 好的 Prompt (RTF)}
			\textbf{Role:}你是一位有10年经验的计算机视觉工程师,精通OpenCV。

			\vspace{0.2cm}

			\textbf{Task:}请编写一个Python函数,实现答题卡图像的自适应二值化。

			\vspace{0.2cm}

			\textbf{Format:}
			\begin{itemize}
				\item 提供完整的、可运行的代码
				\item 包含详细的函数文档
				\item 解释关键参数的设置依据
			\end{itemize}
		\end{exampleblock}
	\end{columns}
\end{frame}

% -----------------------------------------------------------------------------
% 2. Chain-of-Thought (思维链)
% -----------------------------------------------------------------------------

\begin{frame}{进阶技巧:Chain-of-Thought (思维链)}
	\begin{block}{什么是思维链?}
		让 AI \textbf{"一步一步思考"}(Step by Step),而不是直接给出答案。
	\end{block}

	\vspace{0.3cm}

	\begin{columns}
		\column{0.5\textwidth}
		\begin{alertblock}{❌ 直接给答案}
			\textbf{Prompt:} 用 OpenCV 实现透视变换矫正倾斜的试卷。

			\vspace{0.2cm}

			\textbf{问题:}
			\begin{itemize}
				\item AI 直接扔出代码
				\item 学生不理解原理
				\item 换个场景就不会了
			\end{itemize}
		\end{alertblock}

		\column{0.5\textwidth}
		\begin{exampleblock}{✅ 思维链引导}
			\textbf{Prompt:}

			请帮我实现试卷透视变换矫正。请按以下步骤思考:

			\textbf{Step 1:} 透视变换的数学原理是什么?需要哪些参数?

			\textbf{Step 2:} 如何从图像中自动找到试卷的四个角点?

			\textbf{Step 3:} 请写出完整的 Python 实现代码
		\end{exampleblock}
	\end{columns}
\end{frame}

\begin{frame}{思维链实战:准确率对比}
	\textbf{测试任务:} 用 OpenCV 实现自适应阈值处理

	\vspace{0.3cm}

	\begin{table}
		\centering
		\begin{tabular}{p{3cm}p{4cm}p{4cm}}
			\toprule
			\textbf{Prompt 类型} & \textbf{直接提问} & \textbf{思维链引导} \\
			\midrule
			代码完整度              & 70\%          & 95\%           \\
			参数解释清晰度            & 一般            & 详细             \\
			边界情况处理             & 很少提及          & 全面考虑           \\
			实际运行成功率            & 60\%          & 90\%           \\
			\bottomrule
		\end{tabular}
	\end{table}

	\vspace{0.3cm}

	\begin{exampleblock}{思维链 Prompt 模板}
		请帮我解决 [问题]。请按以下步骤思考:

		\textbf{Step 1:} 分析问题的核心要点是什么?

		\textbf{Step 2:} 有哪些可能的解决方案?各自的优缺点?

		\textbf{Step 3:} 请给出推荐的实现代码
	\end{exampleblock}
\end{frame}

% -----------------------------------------------------------------------------
% 3. Few-shot (少样本提示)
% -----------------------------------------------------------------------------

\begin{frame}{进阶技巧:Few-shot (少样本提示)}
	\begin{block}{什么是 Few-shot?}
		给 AI 几个 \textbf{正确的例子},让它模仿你的风格或格式处理新任务。
	\end{block}

	\vspace{0.3cm}

	\begin{columns}
		\column{0.5\textwidth}
		\textbf{示例场景:} OpenCV 图像转换

		\vspace{0.2cm}

		\begin{exampleblock}{Few-shot Prompt}
			请模仿以下示例的代码风格和注释规范:

			\vspace{0.2cm}

			\textbf{示例 1 - 灰度化:}

			\texttt{\# 读取图像并转换为灰度图}

			\texttt{\# 参数:image\_path: 图像文件路径}

			\texttt{\# 返回:gray\_image: 灰度图像数组}

			\texttt{def load\_and\_gray(image\_path):}

			\texttt{~~~img = cv2.imread(image\_path)}

			\texttt{~~~gray = cv2.cvtColor(img, cv2.COLOR\_BGR2GRAY)}

			\texttt{~~~return gray}
		\end{exampleblock}

		\column{0.5\textwidth}
		\textbf{待处理任务:}

		\vspace{0.2cm}

		\begin{exampleblock}{继续 Prompt}
			\textbf{示例 2 - 高斯模糊:}

			\texttt{\# 对图像进行高斯模糊}

			\texttt{\# 参数:image: 输入图像}

			\texttt{\# 返回:blurred: 模糊后的图像}

			\texttt{def gaussian\_smooth(image):}

			\texttt{~~~blurred = cv2.GaussianBlur(image, (5,5), 0)}

			\texttt{~~~return blurred}

			\vspace{0.3cm}

			\textbf{任务:}

			请按照上述风格和规范,实现一个图像边缘检测函数。
		\end{exampleblock}
	\end{columns}
\end{frame}

\begin{frame}{Few-shot 效果对比}
	\begin{table}
		\centering
		\begin{tabular}{p{3cm}p{5cm}p{5cm}}
			\toprule
			\textbf{Prompt 类型} & \textbf{无 Few-shot} & \textbf{有 Few-shot} \\
			\midrule
			代码风格一致性            & 随机,不稳定              & 与示例高度一致             \\
			注释完整度              & 较简略                 & 详细,符合示例规范           \\
			参数说明               & 缺失或不清晰              & 完整的 Docstring       \\
			错误处理               & 经常遗漏                & 按示例模式添加             \\
			代码可读性              & 一般                  & 优秀                  \\
			\bottomrule
		\end{tabular}
	\end{table}

	\vspace{0.3cm}

	\begin{block}{Few-shot 使用技巧}
		\begin{enumerate}
			\item \textbf{示例数量:} 2-3 个示例通常足够
			\item \textbf{示例质量:} 确保示例是正确的、高质量的
			\item \textbf{格式一致:} 示例之间保持风格一致
			\item \textbf{明确指令:} 告诉 AI "请按照示例的风格"
		\end{enumerate}
	\end{block}
\end{frame}

% -----------------------------------------------------------------------------
% 4. 上下文窗口管理
% -----------------------------------------------------------------------------

\begin{frame}{上下文窗口管理}
	\begin{alertblock}{问题:为什么代码太长了 AI 就会"忘掉"前面的内容?}
		\begin{itemize}
			\item 大模型有 \textbf{上下文窗口限制}(通常 4K-128K tokens)
			\item 超出限制后,模型会"遗忘"最早的内容
			\item 导致前后文不一致、回答质量下降
		\end{itemize}
	\end{alertblock}

	\vspace{0.3cm}

	\begin{columns}
		\column{0.5\textwidth}
		\textbf{各模型上下文限制:}
		\begin{table}
			\centering
			\small
			\begin{tabular}{lc}
				\toprule
				\textbf{模型} & \textbf{上下文} \\
				\midrule
				GPT-4       & 8K/32K       \\
				GPT-4o      & 128K         \\
				Claude 3.5  & 200K         \\
				DeepSeek    & 64K          \\
				通义千问        & 128K         \\
				\bottomrule
			\end{tabular}
		\end{table}

		\column{0.5\textwidth}
		\textbf{精简 Prompt 的技巧:}
		\begin{enumerate}
			\item \textbf{只提供必要代码:} 不要贴整个文件
			\item \textbf{使用摘要:} "前面我们讨论了X,现在要解决Y"
			\item \textbf{分段处理:} 长任务拆分成多个短任务
			\item \textbf{定期总结:} 让 AI 总结当前进度
		\end{enumerate}
	\end{columns}
\end{frame}
