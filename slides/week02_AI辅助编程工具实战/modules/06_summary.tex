%=============================================================================
% 模块六:总结与作业 (约 3 页)
%=============================================================================

\section{总结与作业}

% -----------------------------------------------------------------------------
% 1. 总结梳理
% -----------------------------------------------------------------------------

\begin{frame}{建立自己的"AI 协作 SOP"}
	\begin{block}{什么是 SOP?}
		\textbf{S}tandard \textbf{O}perating \textbf{P}rocedure(标准操作流程)

		一套可重复、可优化的工作流程
	\end{block}

	\vspace{0.3cm}

	\begin{center}
		\begin{tikzpicture}
			\node[draw, fill=blue!20] (step1) at (0,0) {1. 问题定义};
			\node[draw, fill=green!20] (step2) at (2.5,0) {2. Prompt 设计};
			\node[draw, fill=yellow!20] (step3) at (5,0) {3. AI 生成};
			\node[draw, fill=orange!20] (step4) at (7.5,0) {4. 审查验证};
			\node[draw, fill=purple!20] (step5) at (10,0) {5. 迭代优化};

			\draw[->, thick] (step1) -- (step2);
			\draw[->, thick] (step2) -- (step3);
			\draw[->, thick] (step3) -- (step4);
			\draw[->, thick] (step4) -- (step5);
			\draw[->, thick, dashed] (step5.south) -- ++(0,-0.5) -| (step1.south);
		\end{tikzpicture}
	\end{center}

	\vspace{0.3cm}

	\begin{columns}
		\column{0.5\textwidth}
		\textbf{1. 问题定义}
		\begin{itemize}
			\item 明确要解决的问题
			\item 确定输入输出
			\item 列出约束条件
		\end{itemize}

		\textbf{2. Prompt 设计}
		\begin{itemize}
			\item 应用 RTF 框架
			\item 使用思维链
			\item 提供 Few-shot 示例
		\end{itemize}

		\column{0.5\textwidth}
		\textbf{3. AI 生成}
		\begin{itemize}
			\item 运行 Prompt
			\item 获取初始输出
			\item 记录生成时间
		\end{itemize}

		\textbf{4. 审查验证}
		\begin{itemize}
			\item 检查代码逻辑
			\item 运行测试用例
			\item 验证输出结果
		\end{itemize}

		\textbf{5. 迭代优化}
		\begin{itemize}
			\item 识别改进点
			\item 设计优化 Prompt
			\item 重复流程直到满意
		\end{itemize}
	\end{columns}
\end{frame}

% -----------------------------------------------------------------------------
% 2. 课后作业
% -----------------------------------------------------------------------------

\begin{frame}{课后作业:答题卡边界检测}
	\begin{block}{作业题目}
		用 AI 辅助实现 \textbf{答题卡边界检测} 程序
	\end{block}

	\vspace{0.3cm}

	\begin{columns}
		\column{0.5\textwidth}
		\textbf{项目关联}

		这是 \textbf{AI 阅卷助手} 的第一步:
		\begin{enumerate}
			\item 图像采集与预处理
			\item \textbf{答题卡定位(当前任务)}
			\item 填涂检测与识别
			\item 手写文字 OCR
			\item 成绩统计与输出
		\end{enumerate}

		\vspace{0.3cm}

		\textbf{作业要求}

		\begin{enumerate}
			\item \textbf{AI 对话:} 至少 3 轮交互
			      \begin{itemize}
				      \item 第1轮:基础实现
				      \item 第2轮:优化改进
				      \item 第3轮:完善功能
			      \end{itemize}

			\item \textbf{功能要求:}
			      \begin{itemize}
				      \item 检测答题卡四边形边界
				      \item 透视变换矫正
				      \item 显示处理结果
			      \end{itemize}
		\end{enumerate}

		\column{0.5\textwidth}
		\textbf{提交内容}

		\vspace{0.2cm}

		\begin{enumerate}
			\item \textbf{AI 对话记录}(必交)
			      \begin{itemize}
				      \item 截图或复制文本
				      \item 展示完整的交互过程
				      \item 标注关键的 Prompt 设计
			      \end{itemize}

			\item \textbf{最终代码}(必交)
			      \begin{itemize}
				      \item 完整可运行的 Python 代码
				      \item 包含必要的注释
			      \end{itemize}

			\item \textbf{测试结果}(必交)
			      \begin{itemize}
				      \item 测试用例图片
				      \item 处理结果图片
				      \item 标注检测到的边界
			      \end{itemize}

			\item \textbf{反思报告}(必交)
			      \begin{itemize}
				      \item 如何设计 Prompt
				      \item 遇到的问题及解决
				      \item 对 AI 辅助编程的思考
			      \end{itemize}
		\end{enumerate}

		\vspace{0.3cm}

		\begin{block}{评分重点}
			\begin{itemize}
				\item \textbf{过程 > 结果}
				\item \textbf{Prompt 设计}
				\item \textbf{反思深度}
			\end{itemize}
		\end{block}
	\end{columns}
\end{frame}

% -----------------------------------------------------------------------------
% 3. 结束页
% -----------------------------------------------------------------------------

\begin{frame}
	\begin{center}
		\Huge \textbf{谢谢!}

		\vspace{1cm}

		\Large
		\textbf{第3周预告:图像预处理与增强}

		\vspace{0.5cm}

		\normalsize
		\textcolor{blue}{故事问题:试卷拍照模糊怎么办?}

		\vspace{0.3cm}

		\begin{itemize}
			\item 图像去噪(高斯/中值滤波)
			\item 图像二值化(全局/Otsu/自适应)
			\item 透视矫正(透视变换)
		\end{itemize}
	\end{center}
\end{frame}
