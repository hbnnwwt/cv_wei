%=============================================================================
% 模块六:总结与作业 (约 3 页)
%=============================================================================

\section{总结与作业}

% -----------------------------------------------------------------------------
% 1. 总结梳理
% -----------------------------------------------------------------------------

\begin{frame}{建立自己的"AI 协作 SOP"}
    \begin{block}{什么是 SOP?}
        \textbf{S}tandard \textbf{O}perating \textbf{P}rocedure(标准操作流程)

        一套可重复、可优化的工作流程
    \end{block}

    \vspace{0.3cm}

    \begin{center}
        \begin{tikzpicture}
            \node[draw, fill=blue!20] (step1) at (0,0) {1. 问题定义};
            \node[draw, fill=green!20] (step2) at (2.5,0) {2. Prompt 设计};
            \node[draw, fill=yellow!20] (step3) at (5,0) {3. AI 生成};
            \node[draw, fill=orange!20] (step4) at (7.5,0) {4. 审查验证};
            \node[draw, fill=purple!20] (step5) at (10,0) {5. 迭代优化};

            \draw[->, thick] (step1) -- (step2);
            \draw[->, thick] (step2) -- (step3);
            \draw[->, thick] (step3) -- (step4);
            \draw[->, thick] (step4) -- (step5);
            \draw[->, thick, dashed] (step5.south) -- ++(0,-0.5) -| (step1.south);
        \end{tikzpicture}
    \end{center}

    \vspace{0.3cm}

    \begin{columns}
        \column{0.5\textwidth}
        \textbf{1. 问题定义}
        \begin{itemize}
            \item 明确要解决的问题
            \item 确定输入输出
            \item 列出约束条件
        \end{itemize}

        \textbf{2. Prompt 设计}
        \begin{itemize}
            \item 应用 RTF 框架
            \item 使用思维链
            \item 提供 Few-shot 示例
        \end{itemize}

        \column{0.5\textwidth}
        \textbf{3. AI 生成}
        \begin{itemize}
            \item 运行 Prompt
            \item 获取初始输出
            \item 记录生成时间
        \end{itemize}

        \textbf{4. 审查验证}
        \begin{itemize}
            \item 检查代码逻辑
            \item 运行测试用例
            \item 验证输出结果
        \end{itemize}

        \textbf{5. 迭代优化}
        \begin{itemize}
            \item 识别改进点
            \item 设计优化 Prompt
            \item 重复流程直到满意
        \end{itemize}
    \end{columns}
\end{frame}

% -----------------------------------------------------------------------------
% 2. 知识点总结
% -----------------------------------------------------------------------------

\begin{frame}{知识点总结:Prompt工程核心要点}
    \begin{columns}
        \column{0.5\textwidth}
        \begin{block}{1. Prompt工程三要素}
            \begin{enumerate}
                \item \textbf{Role(角色)}
                    \begin{itemize}
                        \item 明确AI的身份
                        \item 例如:资深CV工程师
                    \end{itemize}
                \item \textbf{Task(任务)}
                    \begin{itemize}
                        \item 清晰说明要做什么
                        \item 具体、明确、可执行
                    \end{itemize}
                \item \textbf{Format(格式)}
                    \begin{itemize}
                        \item 指定输出格式和要求
                        \item 代码、文档、表格等
                    \end{itemize}
            \end{enumerate}
        \end{block}

        \column{0.5\textwidth}
        \begin{block}{2. Prompt优化技巧}
            \begin{enumerate}
                \item \textbf{少样本提示(Few-Shot)}
                    \begin{itemize}
                        \item 提供1-3个示例
                        \item 模仿学习
                    \end{itemize}
                \item \textbf{思维链(Chain-of-Thought)}
                    \begin{itemize}
                        \item 分步思考
                        \item 推理过程
                    \end{itemize}
                \item \textbf{角色扮演(Role-Playing)}
                    \begin{itemize}
                        \item 设定专业身份
                        \item 专业术语
                    \end{itemize}
            \end{enumerate}
        \end{block}
    \end{columns}
\end{frame}

\begin{frame}{知识点总结:AI工具使用规范}
    \begin{columns}
        \column{0.33\textwidth}
        \begin{block}{\textcolor{green!70!black}{[\checkmark] 允许}}
            \begin{itemize}
                \item 用AI解释概念
                \item 调试错误
                \item 优化代码
                \item 学习新技术
            \end{itemize}
        \end{block}

        \column{0.33\textwidth}
        \begin{block}{\textcolor{blue!70!black}{[\textbf{+}] 鼓励}}
            \begin{itemize}
                \item 生成对比实验
                \item 可视化结果
                \item 探索不同方案
                \item 提高学习效率
            \end{itemize}
        \end{block}

        \column{0.33\textwidth}
        \begin{block}{\textcolor{red!70!black}{[\textbackslash times] 禁止}}
            \begin{itemize}
                \item 直接复制完整代码
                \item 用AI完成全部作业
                \item 不思考直接照搬
                \item 欺骗性使用
            \end{itemize}
        \end{block}
    \end{columns}

    \vspace{0.5cm}

    \begin{center}
        \begin{tikzpicture}
            \node[draw, fill=green!20, rounded corners] at (0,0) {\textbf{核心理念:理解原理 > 复制代码}};
            \node[draw, fill=blue!20, rounded corners] at (6,0) {\textbf{AI是助手,不是替代者}};
        \end{tikzpicture}
    \end{center}
\end{frame}

% -----------------------------------------------------------------------------
% 3. 延伸学习资源
% -----------------------------------------------------------------------------

\begin{frame}{延伸学习资源}
    \begin{columns}
        \column{0.5\textwidth}
        \begin{block}{官方文档}
            \begin{enumerate}
                \item \textbf{ChatGPT官方文档}
                    \begin{itemize}
                        \item \url{https://platform.openai.com/docs}
                        \item API使用指南
                        \item 最佳实践
                    \end{itemize}

                \item \textbf{Claude官方文档}
                    \begin{itemize}
                        \item \url{https://docs.anthropic.com}
                        \item Claude API文档
                        \item 提示工程指南
                    \end{itemize}

                \item \textbf{通义千问官方文档}
                    \begin{itemize}
                        \item \url{https://help.aliyun.com/qwen}
                        \item 快速入门指南
                        \item SDK文档
                    \end{itemize}
            \end{enumerate}
        \end{block}

        \column{0.5\textwidth}
        \begin{block}{推荐教程与资源}
            \begin{enumerate}
            \setcounter{enumi}{3}
                \item \textbf{Prompt工程教程}
                    \begin{itemize}
                        \item 《Prompt Engineering Guide》
                        \item \url{https://www.promptingguide.ai}
                        \item 中文版:提示工程指南
                    \end{itemize}

                \item \textbf{GitHub Copilot使用指南}
                    \begin{itemize}
                        \item \url{https://github.com/features/copilot}
                        \item 官方文档和教程
                        \item 最佳实践分享
                    \end{itemize}

                \item \textbf{Cursor官方教程}
                    \begin{itemize}
                        \item \url{https://cursor.sh/tutorials}
                        \item 视频教程
                        \item 快捷键指南
                    \end{itemize}

                \item \textbf{社区资源}
                    \begin{itemize}
                        \item Reddit: r/ChatGPT, r/ClaudeAI
                        \item 知乎AI编程话题
                        \item 掘金AI专栏
                    \end{itemize}
            \end{enumerate}
        \end{block}
    \end{columns}
\end{frame}

% -----------------------------------------------------------------------------
% 4. 本周作业详解
% -----------------------------------------------------------------------------

\begin{frame}{本周作业:答题卡边界检测}
    \begin{block}{作业题目}
        用 AI 辅助实现 \textbf{答题卡边界检测} 程序
    \end{block}

    \vspace{0.3cm}

    \begin{columns}
        \column{0.5\textwidth}
        \textbf{项目关联}

        这是 \textbf{AI 阅卷助手} 的第一步:
        \begin{enumerate}
            \item 图像采集与预处理
            \item \textbf{答题卡定位(当前任务)}
            \item 填涂检测与识别
            \item 手写文字 OCR
            \item 成绩统计与输出
        \end{enumerate}

        \vspace{0.3cm}

        \textbf{分层任务说明}

        \begin{itemize}
            \item \textcolor{blue}{\textbf{基础版}}:检测边界+透视矫正
            \item \textcolor{orange}{\textbf{进阶版}}:增加鲁棒性处理
            \item \textcolor{red}{\textbf{挑战版}}:封装成可复用类
        \end{itemize}

        \column{0.5\textwidth}
        \textbf{AI 对话要求}

        \vspace{0.2cm}

        至少 \textbf{3 轮}交互:
        \begin{enumerate}
            \item 第1轮:基础实现
            \item 第2轮:优化改进
            \item 第3轮:完善功能
        \end{enumerate}

        \vspace{0.3cm}

        \textbf{不同参与模式的建议}
        \begin{itemize}
            \item \textbf{观察者}:完成基础版,重点理解Prompt设计
            \item \textbf{使用者}:完成进阶版,重点调试运行
            \item \textbf{创造者}:完成挑战版,重点优化创新
        \end{itemize}
    \end{columns}
\end{frame}

\begin{frame}{作业提交内容与评分标准}
    \begin{columns}
        \column{0.5\textwidth}
        \textbf{提交内容}

        \vspace{0.2cm}

        \begin{enumerate}
            \item \textbf{AI 对话记录}
                  \begin{itemize}
                      \item 截图或文本
                      \item 标注关键Prompt
                      \item 说明迭代过程
                  \end{itemize}

            \item \textbf{最终代码}
                  \begin{itemize}
                      \item 完整可运行
                      \item 必要注释
                  \end{itemize}

            \item \textbf{测试结果}
                  \begin{itemize}
                      \item 测试图片
                      \item 处理结果
                  \end{itemize}

            \item \textbf{反思报告}
                  \begin{itemize}
                      \item Prompt设计思路
                      \item 问题与解决
                      \item 学习体会
                  \end{itemize}
        \end{enumerate}

        \column{0.5\textwidth}
        \textbf{详细评分标准(100分)}

        \vspace{0.2cm}

        \begin{table}
            \centering
            \small
            \begin{tabular}{cl}
                \toprule
                \textbf{分项} & \textbf{评分要点} \\
                \midrule
                \textbf{Prompt设计} & 结构清晰、迭代合理 \\
                (30分) & RTF框架应用恰当 \\
                \midrule
                \textbf{代码完整性} & 功能完整、可运行 \\
                (30分) & 注释清晰、结构合理 \\
                \midrule
                \textbf{测试覆盖} & 多场景测试 \\
                (20分) & 边界情况考虑 \\
                \midrule
                \textbf{反思深度} & 思考深入 \\
                (20分) & 有独特见解 \\
                \bottomrule
            \end{tabular}
        \end{table}
    \end{columns}
\end{frame}

\begin{frame}{分层任务详解}
    \begin{columns}
        \column{0.33\textwidth}
        \begin{block}{\textcolor{blue}{基础版}}
            \textbf{要求:}
            \begin{itemize}
                \item 读取答题卡图像
                \item Canny边缘检测
                \item 查找轮廓
                \item 绘制边界框
            \end{itemize}

            \vspace{0.2cm}

            \textbf{适合:}
            \begin{itemize}
                \item 观察者模式
                \item 编程基础较弱
            \end{itemize}

            \vspace{0.2cm}

            \textbf{目标:} 理解基本流程
        \end{block}

        \column{0.33\textwidth}
        \begin{block}{\textcolor{orange}{进阶版}}
            \textbf{要求:}
            \begin{itemize}
                \item 完成基础版所有功能
                \item 添加预处理(去噪)
                \item 自适应阈值
                \item 透视变换矫正
            \end{itemize}

            \vspace{0.2cm}

            \textbf{适合:}
            \begin{itemize}
                \item 使用者模式
                \item 有一定编程基础
            \end{itemize}

            \vspace{0.2cm}

            \textbf{目标:} 独立完成实用功能
        \end{block}

        \column{0.33\textwidth}
        \begin{block}{\textcolor{red}{挑战版}}
            \textbf{要求:}
            \begin{itemize}
                \item 完成进阶版所有功能
                \item 封装成类
                \item 处理极端情况
                \item 可视化调试信息
            \end{itemize}

            \vspace{0.2cm}

            \textbf{适合:}
            \begin{itemize}
                \item 创造者模式
                \item 编程基础较好
            \end{itemize}

            \vspace{0.2cm}

            \textbf{目标:} 生产级可用代码
        \end{block}
    \end{columns}

    \vspace{0.5cm}

    \begin{center}
        \textit{鼓励挑战更高版本,完成进阶版可获附加分5分,完成挑战版可获附加分10分}
    \end{center}
\end{frame}

% -----------------------------------------------------------------------------
% 5. 结束页
% -----------------------------------------------------------------------------

\begin{frame}
\centering
\Huge \textbf{谢谢!}

\vspace{1cm}

\Large
\textbf{第3周预告:图像预处理与增强}

\vspace{0.5cm}

\normalsize
\textcolor{blue}{故事问题:试卷拍照模糊怎么办?}

\vspace{0.3cm}

\begin{itemize}
    \item 图像去噪(高斯/中值滤波)
    \item 图像二值化(全局/Otsu/自适应)
    \item 透视矫正(透视变换)
\end{itemize}
\end{frame}
