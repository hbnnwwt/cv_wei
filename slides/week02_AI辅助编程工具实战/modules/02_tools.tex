%=============================================================================
% 模块二:工具链深度拆解——从对话框到 IDE (约 10 页)
%=============================================================================

\section{AI编程工具全景}

% -----------------------------------------------------------------------------
% 1. 工具全景图
% -----------------------------------------------------------------------------

\begin{frame}{AI编程工具全景图}
	\begin{center}
		\begin{tikzpicture}
			\node[draw, fill=blue!20] (web) at (0,0) {Web 界面};
			\node[draw, fill=green!20] (plugin) at (0,-1.5) {IDE 插件};
			\node[draw, fill=orange!20] (ide) at (0,-3) {智能 IDE};

			\draw[->, thick] (web) -- (plugin) node[midway, right] {集成度提升};
			\draw[->, thick] (plugin) -- (ide) node[midway, right] {智能化加深};
		\end{tikzpicture}
	\end{center}

	\vspace{0.3cm}

	\begin{block}{选择建议}
		\begin{itemize}
			\item \textbf{学习理解:} ChatGPT/Claude(对话深入)
			\item \textbf{实时编码:} GitHub Copilot/Cursor(IDE集成)
			\item \textbf{国内使用:} 通义千问/DeepSeek/CodeGeeX
		\end{itemize}
	\end{block}
\end{frame}

\begin{frame}{主流工具详细对比}
	\begin{table}
		\centering
		\small
		\begin{tabular}{p{2.5cm}p{3.5cm}p{3cm}p{2.5cm}}
			\toprule
			\textbf{工具}    & \textbf{特点}  & \textbf{适用场景} & \textbf{推荐指数} \\
			\midrule
			ChatGPT        & 对话能力强,代码生成准确 & 学习、调试、解释      & [star][star][star][star][star]         \\
			Claude         & 代码分析深入       & 代码审查、架构设计     & [star][star][star][star][star]         \\
			GitHub Copilot & IDE集成,实时补全   & 日常编码、快速开发     & [star][star][star][star][star-empty]         \\
			Cursor         & AI原生IDE      & 项目开发、重构       & [star][star][star][star][star]         \\
			通义千问           & 中文友好,国内可用    & 中文问题咨询        & [star][star][star][star][star-empty]         \\
			DeepSeek       & 编程能力强,开源     & 代码生成、算法实现     & [star][star][star][star][star]         \\
			\bottomrule
		\end{tabular}
	\end{table}
\end{frame}

\begin{frame}{AI编程工具发展史:从传统IDE到AI助手}
	\begin{center}
		\begin{tikzpicture}[scale=0.9]
			% 时间轴
			\draw[->, thick] (0,0) -- (12,0);

			% 时间节点
			\node[below] at (1,-0.3) {\small 2018};
			\node[below] at (4,-0.3) {\small 2020};
			\node[below] at (7,-0.3) {\small 2022};
			\node[below] at (10,-0.3) {\small 2024-26};

			% 事件
			\node[draw, fill=blue!20, align=center] at (1,1.5) {\small TabNine\\\small 代码补全};
			\node[draw, fill=green!20, align=center] at (4,1.5) {\small GPT-3发布\\\small 代码生成};
			\node[draw, fill=orange!20, align=center] at (7,1.5) {\small ChatGPT\\\small Copilot爆发};
			\node[draw, fill=red!20, align=center] at (10,1.5) {\small Claude/Cursor\\\small 原生AI IDE};
		\end{tikzpicture}
	\end{center}

	\begin{block}{里程碑事件}
		\begin{itemize}
			\item \textbf{2018}:TabNine首次将深度学习用于代码补全
			\item \textbf{2020}:GPT-3展示强大的代码生成能力
			\item \textbf{2022}:ChatGPT和GitHub Copilot引爆AI编程浪潮
			\item \textbf{2024-26}:Claude、Cursor等原生AI IDE重新定义编程体验
		\end{itemize}
	\end{block}
\end{frame}

% -----------------------------------------------------------------------------
% 2. Cursor 专题实战
% -----------------------------------------------------------------------------

\begin{frame}{Cursor 专题实战(重中之重)}
	\begin{center}
		\Huge \textbf{Cursor = VS Code + AI 原生集成}

		\vspace{0.5cm}

		\normalsize
		\textcolor{blue}{\url{https://cursor.sh}}
	\end{center}

	\vspace{0.3cm}

	\begin{columns}
		\column{0.5\textwidth}
		\textbf{核心快捷键:}
		\begin{itemize}
			\item \texttt{Ctrl+K}:原地编辑代码
			\item \texttt{Ctrl+L}:上下文对话
			\item \texttt{@Files}:引用项目文件
			\item \texttt{@Code}:引用代码片段
			\item \texttt{@Docs}:引用文档
		\end{itemize}

		\column{0.5\textwidth}
		\textbf{强大功能:}
		\begin{itemize}
			\item \textbf{代码生成:} 根据注释直接生成代码
			\item \textbf{代码解释:} 选中代码,一键解释
			\item \textbf{代码重构:} 智能重命名、提取函数
			\item \textbf{错误修复:} 自动检测并修复 bug
			\item \textbf{项目理解:} 全局搜索、跨文件分析
		\end{itemize}
	\end{columns}
\end{frame}

\begin{frame}{Cursor 实战演示}
	\textbf{场景1:原地编辑 (Ctrl+K)}
	\begin{enumerate}
		\item 选中要修改的代码
		\item 按 \texttt{Ctrl+K}
		\item 输入指令:"添加异常处理"
		\item AI 原地修改代码
	\end{enumerate}

	\vspace{0.3cm}

	\textbf{场景2:上下文对话 (Ctrl+L)}
	\begin{enumerate}
		\item 按 \texttt{Ctrl+L} 打开聊天面板
		\item 输入:"解释这段代码的作用"
		\item AI 结合上下文给出解释
	\end{enumerate}

	\vspace{0.3cm}

	\begin{block}{实战技巧}
		\begin{itemize}
			\item 多轮对话细化需求
			\item 结合 @Files 提供上下文
			\item 选中代码后再提问
			\item 不满意可以要求重写
		\end{itemize}
	\end{block}
\end{frame}

% -----------------------------------------------------------------------------
% 3. 本地大模型与隐私
% -----------------------------------------------------------------------------

\begin{frame}{本地大模型与隐私保护}
	\begin{alertblock}{真实场景}
		企业/学校要求:代码不能上传到外部服务器!
	\end{alertblock}

	\vspace{0.3cm}

	\begin{columns}
		\column{0.5\textwidth}
		\textbf{解决方案:本地部署}

		\vspace{0.3cm}

		\textbf{1. Ollama(推荐)}
		\begin{itemize}
			\item 一行命令部署本地模型
			\item 支持 Llama、DeepSeek、Qwen 等
			\item 完全离线运行
		\end{itemize}

		\column{0.5\textwidth}
		\textbf{2. LM Studio}
		\begin{itemize}
			\item 图形界面,易于使用
			\item 支持多种模型格式
			\item 内置聊天界面
		\end{itemize}

		\vspace{0.3cm}

		\textbf{3. vLLM}
		\begin{itemize}
			\item 高性能推理引擎
			\item 适合企业级部署
			\item 支持多卡并行
		\end{itemize}
	\end{columns}
\end{frame}

% -----------------------------------------------------------------------------
% 4. 模型对比测试
% -----------------------------------------------------------------------------

\begin{frame}{不同模型的对比测试}
	\textbf{测试任务:} 用 Python 和 OpenCV 实现答题卡边界检测

	\vspace{0.3cm}

	\begin{table}
		\centering
		\small
		\begin{tabular}{p{2.5cm}p{3cm}p{3cm}p{3cm}}
			\toprule
			\textbf{模型}    & \textbf{代码质量} & \textbf{中文理解} & \textbf{运行成功率} \\
			\midrule
			GPT-4o         & [star][star][star][star][star]         & [star][star][star][star][star-empty]         & 95\%           \\
			Claude 3.5     & [star][star][star][star][star]         & [star][star][star][star][star-empty]         & 93\%           \\
			DeepSeek-Coder & [star][star][star][star][star-empty]         & [star][star][star][star][star]         & 90\%           \\
			通义千问2.5        & [star][star][star][star][star-empty]         & [star][star][star][star][star]         & 88\%           \\
			GPT-3.5        & [star][star][star][star-empty][star-empty]         & [star][star][star][star-empty][star-empty]         & 75\%           \\
			\bottomrule
		\end{tabular}
	\end{table}

	\vspace{0.3cm}

	\begin{columns}
		\column{0.5\textwidth}
		\textbf{GPT-4o/Claude 3.5 优势:}
		\begin{itemize}
			\item 代码逻辑最严谨
			\item 边界情况处理完善
			\item 参数解释最详细
		\end{itemize}

		\column{0.5\textwidth}
		\textbf{DeepSeek/通义千问 优势:}
		\begin{itemize}
			\item 中文理解更自然
			\item 国内访问更稳定
			\item 价格更实惠
		\end{itemize}
	\end{columns}
\end{frame}

% -----------------------------------------------------------------------------
% 5. ChatGPT 详细使用教程
% -----------------------------------------------------------------------------

\begin{frame}{ChatGPT 使用教程(一):账号与界面}
	\textbf{1. 账号注册与登录}
	\begin{itemize}
		\item 访问 \url{https://chat.openai.com}
		\item 使用邮箱/Google/Microsoft 账号注册
		\item 免费版使用 GPT-4o mini,Plus 版使用 GPT-4o
	\end{itemize}

	\vspace{0.3cm}

	\textbf{2. 界面介绍}
	\begin{columns}
		\column{0.5\textwidth}
		\begin{block}{左侧边栏}
			\begin{itemize}
				\item 新建对话
				\item 历史记录
				\item GPTs 商店
			\end{itemize}
		\end{block}

		\column{0.5\textwidth}
		\begin{block}{输入区域}
			\begin{itemize}
				\item 文字输入框
				\item 文件上传
				\item 语音输入
			\end{itemize}
		\end{block}
	\end{columns}
\end{frame}

\begin{frame}{ChatGPT 使用教程(二):编程技巧}
	\textbf{核心功能:}
	\begin{enumerate}
		\item \textbf{代码生成}:描述功能需求,ChatGPT 生成代码
		\item \textbf{代码解释}:粘贴代码,逐行解释逻辑
		\item \textbf{错误调试}:粘贴错误信息,分析原因
		\item \textbf{代码优化}:要求优化性能/可读性
	\end{enumerate}

	\vspace{0.3cm}

	\textbf{高级功能(Plus 用户):}
	\begin{itemize}
		\item \textbf{GPTs}:创建专用助手(如 "OpenCV 专家")
		\item \textbf{文件分析}:上传 PDF/代码文件分析
		\item \textbf{联网搜索}:获取最新文档
		\item \textbf{数据分析}:上传 Excel/CSV,生成分析代码
	\end{itemize}
\end{frame}

\begin{frame}{ChatGPT 使用教程(三):实战示例}
	\textbf{示例:用 ChatGPT 实现人脸检测}

	\vspace{0.2cm}

	\textbf{Prompt 模板:}

	\begin{block}{推荐 Prompt}
		你是一位有 5 年 OpenCV 经验的工程师。请用 Python 和 OpenCV 实现人脸检测:

		\textbf{功能:} 从图片中检测所有人脸,并用绿色矩形框标注

		\textbf{输入:} 图片文件路径(字符串)

		\textbf{输出:} 显示标注后的图像,打印检测到的人脸数量

		\textbf{要求:}
		\begin{itemize}
			\item 使用 Haar 级联分类器
			\item 处理中文路径问题
			\item 代码有详细中文注释
		\end{itemize}
	\end{block}
\end{frame}

% -----------------------------------------------------------------------------
% 6. Claude 详细使用教程
% -----------------------------------------------------------------------------

\begin{frame}{Claude 使用教程(一):核心优势}
	\textbf{Claude vs ChatGPT:代码分析更深入}

	\vspace{0.3cm}

	\begin{columns}
		\column{0.5\textwidth}
		\textbf{Claude 独特优势:}
		\begin{itemize}
			\item \textbf{长文本处理}:支持 200K token
			\item \textbf{代码审查}:分析完整项目
			\item \textbf{安全意识}:避免不安全代码
			\item \textbf{诚实性}:不知道时会说明
		\end{itemize}

		\column{0.5\textwidth}
		\textbf{适用场景:}
		\begin{itemize}
			\item 架构设计咨询
			\item 复杂代码重构
			\item 代码审查与优化
			\item 长文档分析
		\end{itemize}
	\end{columns}

	\vspace{0.3cm}

	\begin{block}{访问方式}
		\begin{itemize}
			\item 网页版:\url{https://claude.ai}
			\item Claude Code:命令行工具
			\item API:开发者集成
		\end{itemize}
	\end{block}
\end{frame}

\begin{frame}{Claude 使用教程(二):Artifacts 功能}
	\textbf{Artifacts:Claude 的杀手级功能}

	\vspace{0.3cm}

	\begin{columns}
		\column{0.6\textwidth}
		\textbf{什么是 Artifacts?}
		\begin{itemize}
			\item 侧边栏预览生成内容
			\item 支持 HTML/React/Mermaid
			\item 实时编辑与交互
		\end{itemize}

		\vspace{0.2cm}

		\textbf{实战应用:}
		\begin{itemize}
			\item 生成可视化图表
			\item 创建交互式文档
			\item 快速原型开发
		\end{itemize}

		\column{0.4\textwidth}
		\begin{alertblock}{示例}
			让 Claude 生成:
			\begin{itemize}
				\item 数据处理流程图
				\item 算法架构图
				\item API 文档页面
			\end{itemize}
		\end{alertblock}
	\end{columns}
\end{frame}

\begin{frame}{Claude 使用教程(三):代码审查实战}
	\textbf{用 Claude 进行代码审查}

	\vspace{0.3cm}

	\textbf{步骤 1:提供上下文}

	\begin{alertblock}{输入}
		"这是我的人脸检测代码,请审查其质量和安全性"
	\end{alertblock}

	\vspace{0.2cm}

	\textbf{步骤 2:Claude 分析}
	\begin{itemize}
		\item 指出潜在的安全漏洞
		\item 建议性能优化方案
		\item 检查边界情况处理
		\item 评估代码可读性
	\end{itemize}

	\vspace{0.2cm}

	\textbf{步骤 3:迭代改进}

	\begin{exampleblock}{输出}
		"请按照你的建议重构这段代码"
	\end{exampleblock}
\end{frame}

% -----------------------------------------------------------------------------
% 7. 通义千问详细使用教程
% -----------------------------------------------------------------------------

\begin{frame}{通义千问使用教程(一):国内优势}
	\textbf{为什么选择通义千问?}

	\vspace{0.3cm}

	\begin{columns}
		\column{0.5\textwidth}
		\begin{block}{核心优势}
			\begin{itemize}
				\item \textbf{中文友好}:理解中文术语
				\item \textbf{国内可用}:无需特殊网络
				\item \textbf{免费额度}:每日百万 token
				\item \textbf{多模态}:支持图像理解
			\end{itemize}
		\end{block}

		\column{0.5\textwidth}
		\begin{block}{适用场景}
			\begin{itemize}
				\item 中文技术问题咨询
				\item 代码中文注释生成
				\item 中文文档撰写
				\item 图像识别与分析
			\end{itemize}
		\end{block}
	\end{columns}

	\vspace{0.3cm}

	\textbf{访问方式:}
	\begin{itemize}
		\item 网页版:\url{https://tongyi.aliyun.com}
		\item 通义灵码:IDE 插件(类似 Copilot)
		\item API:Python SDK 集成
	\end{itemize}
\end{frame}

\begin{frame}{通义千问使用教程(二):灵码插件}
	\textbf{通义灵码:阿里推出的智能编码助手}

	\vspace{0.3cm}

	\textbf{安装与配置:}
	\begin{enumerate}
		\item VS Code / JetBrains 搜索 "Tongyi Lingma"
		\item 使用阿里云账号登录
		\item 开始使用自动补全
	\end{enumerate}

	\vspace{0.3cm}

	\textbf{核心功能:}
	\begin{itemize}
		\item \textbf{智能补全}:根据上下文预测代码
		\item \textbf{注释生成}:一键添加中文注释
		\item \textbf{代码解释}:选中代码查看解释
		\item \textbf{单元测试}:自动生成测试用例
	\end{itemize}

	\vspace{0.2cm}

	\begin{block}{免费额度}
		个人用户每月 20 万次补全,完全免费!
	\end{block}
\end{frame}

\begin{frame}[fragile]{通义千问使用教程(三):Python SDK}
	\textbf{在代码中直接调用通义千问}

	\vspace{0.3cm}

	\textbf{安装 SDK:}
	\begin{verbatim}
	pip install dashscope
	\end{verbatim}

	\vspace{0.3cm}

	\textbf{代码示例:}
	\begin{lstlisting}[language=Python, basicstyle=\tiny\ttfamily]
import dashscope
from dashscope import Generation

# 设置 API Key
dashscope.api_key = "your-api-key"

# 调用模型
response = Generation.call(
    model='qwen-turbo',
    prompt='用 Python 和 OpenCV 实现图像灰度化',
    max_tokens=1500
)

print(response.output.text)
	\end{lstlisting}

	\vspace{0.2cm}

	\textbf{应用场景:} 自动化脚本、批量处理、集成开发
\end{frame}
