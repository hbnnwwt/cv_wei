%=============================================================================
% 模块一:认知升级——LLM 时代的编程新范式 (约 12 页)
%=============================================================================

\section{LLM时代的编程新范式}

% -----------------------------------------------------------------------------
% 1. 回顾与痛点
% -----------------------------------------------------------------------------

\begin{frame}[fragile]{回顾:第1周的挑战}
	\begin{columns}
		\column{0.5\textwidth}
		\textbf{还记得这些痛点吗?}
		\begin{itemize}
			\item \texttt{cv2.imread()} 返回 \texttt{None},找不到原因
			\item 图像显示颜色异常
			\item 数组越界、维度不匹配
			\item API 参数太多,记不住
		\end{itemize}

		\vspace{0.3cm}

		\begin{alertblock}{典型错误代码}
			\begin{lstlisting}[language=Python, basicstyle=\ttfamily\tiny]
# 死循环!
while img is not None:
    cv2.imshow('image', img)
    if cv2.waitKey(1) == 'q':
        break
\end{lstlisting}
		\end{alertblock}

		\column{0.5\textwidth}
		\textbf{传统解决方式:}
		\begin{enumerate}
			\item 翻文档(\texttt{docs.opencv.org})
			\item 搜索 StackOverflow
			\item 问同学/老师
			\item 试错(耗费大量时间)
		\end{enumerate}

		\vspace{0.3cm}

		\begin{exampleblock}{AI 辅助的新方式}
			\begin{itemize}
				\item 直接问 AI
				\item 解释错误原因
				\item 给出修改建议
				\item \textbf{降低学习门槛!}
			\end{itemize}
		\end{exampleblock}
	\end{columns}
\end{frame}

% -----------------------------------------------------------------------------
% 2. LLM 的本质
% -----------------------------------------------------------------------------

\begin{frame}{大模型写代码的本质}
	\begin{center}
		\Huge \textbf{Token 预测} \\[0.5cm]
		\Large + \\[0.5cm]
		\Huge \textbf{模式匹配}
	\end{center}

	\vspace{0.5cm}

	\begin{block}{为什么 AI 擅长语法但不一定懂逻辑?}
		\begin{itemize}
			\item \textbf{擅长:} 基于海量代码库的模式复刻
			\item \textbf{不擅长:} 理解你的具体业务逻辑
		\end{itemize}
	\end{block}
\end{frame}

% -----------------------------------------------------------------------------
% 3. 幻觉专题
% -----------------------------------------------------------------------------

\begin{frame}[fragile]{幻觉专题}
	\begin{alertblock}{什么是 AI 幻觉?}
		AI 自信满满地生成看似合理但实际上错误或不存在的信息。
	\end{alertblock}

	\vspace{0.3cm}

	\textbf{案例:AI 发明了一个不存在的 OpenCV 函数}

	\begin{exampleblock}{AI 的"创意"代码}
		\begin{lstlisting}[language=Python, basicstyle=\ttfamily\tiny]
import cv2

img = cv2.imread('exam.jpg')

# 不存在的函数!
fixed = cv2.auto_fix_exposure(img)
\end{lstlisting}
	\end{exampleblock}

	\begin{block}{如何识别幻觉?}
		\begin{itemize}
			\item 查官方文档 (docs.opencv.org)
			\item 在 Python 中 \texttt{dir(cv2)} 查看所有可用函数
			\item 多问一句:"这个函数真的存在吗?"
		\end{itemize}
	\end{block}
\end{frame}

% -----------------------------------------------------------------------------
% 4. 编程范式演进
% -----------------------------------------------------------------------------

\begin{frame}{编程范式的演进}
	\begin{columns}
		\column{0.33\textwidth}
		\textbf{1.0 阶段:查阅文档/StackOverflow}
		\begin{itemize}
			\item 手动搜索
			\item 阅读文档
			\item 试错调试
		\end{itemize}
		\textbf{效率:}两星

		\column{0.33\textwidth}
		\textbf{2.0 阶段:AI 辅助}
		\begin{itemize}
			\item 从"写作者"转变为"审查者"
			\item AI 生成初稿,人审查修改
			\item 快速原型验证
		\end{itemize}
		\textbf{效率:}四星

		\column{0.33\textwidth}
		\textbf{3.0 阶段:AI 原生?}
		\begin{itemize}
			\item AI 直接理解需求生成完整系统
			\item 人主要负责需求定义和架构设计
			\item 可能的未来方向
		\end{itemize}
		\textbf{效率:}五星?
	\end{columns}

	\vspace{0.5cm}

	\begin{block}{核心结论:程序员的核心竞争力转变}
		\begin{center}
			\Large 从 \textbf{``记忆语法''} 转向 \textbf{``问题定义''} 与 \textbf{``代码审查''}
		\end{center}
	\end{block}
\end{frame}

% =============================================================================
% 编程范式对跨专业学习者的启示(新增)
% =============================================================================

\begin{frame}{不同专业背景的学习路径}
	\begin{columns}
		\column{0.33\textwidth}
		\begin{block}{理工科背景}
			\textbf{优势:}编程基础较好
			\begin{itemize}
				\item 直接进入使用者/创造者模式
				\item 重点学习Prompt工程技巧
				\item 探索复杂场景的AI协作
			\end{itemize}
		\end{block}

		\column{0.33\textwidth}
		\begin{block}{文科/经管背景}
			\textbf{优势:}表达与逻辑能力
			\begin{itemize}
				\item 从观察者模式开始
				\item 发挥自然语言优势设计Prompt
				\item 逐步掌握基础代码理解
			\end{itemize}
		\end{block}

		\column{0.33\textwidth}
		\begin{block}{艺术/设计背景}
			\textbf{优势:}视觉敏感度高
			\begin{itemize}
				\item 适合图像处理任务
				\item 关注可视化结果
				\item AI辅助实现创意想法
			\end{itemize}
		\end{block}
	\end{columns}

	\vspace{0.5cm}

	\begin{center}
		\textit{AI辅助编程降低了技术门槛,不同专业都能找到适合自己的切入点}
	\end{center}
\end{frame}
