%=============================================================================
% 模块三:OpenCV 基础与进阶操作
%=============================================================================

\section{OpenCV 基础}

\begin{frame}[fragile]{OpenCV 核心函数详解}
	\begin{lstlisting}[title={图像读取的隐患}]
import cv2

# 路径千万不能有中文(新手常见错误)
img = cv2.imread('paper.jpg')

# 检查是否读取成功
if img is None:
    print("错误:文件不存在或路径含有中文!")

# 打印维度 (H, W, C)
print(img.shape)
	\end{lstlisting}

	\begin{block}{常用 Flag}
		\begin{itemize}
			\item \texttt{cv2.IMREAD\_COLOR}:加载彩色图(默认)
			\item \texttt{cv2.IMREAD\_GRAYSCALE}:直接以灰度模式加载
		\end{itemize}
	\end{block}
\end{frame}

\begin{frame}[fragile]{工程实战:处理中文路径的"终极方案"}
	\begin{alertblock}{真实场景}
		阅卷系统中,学生姓名经常是中文,文件名如:\texttt{张三\_试卷.jpg}\\
		\texttt{cv2.imread()} 直接读取会失败(返回 None)
	\end{alertblock}

	\begin{columns}
		\column{0.5\textwidth}
		\textbf{问题原因:}
		\begin{itemize}
			\item OpenCV 的 C++ 库不支持 Unicode 路径
			\item 中文路径会乱码
		\end{itemize}

		\vspace{0.3cm}
		\textbf{错误做法:}
		\begin{lstlisting}[language=Python, basicstyle=\ttfamily\tiny]
# 直接读取,返回 None
img = cv2.imread('张三_试卷.jpg')
if img is None:
    print("读取失败!")
		\end{lstlisting}

		\column{0.5\textwidth}
		\textbf{正确做法:}
		\begin{lstlisting}[language=Python, basicstyle=\ttfamily\tiny]
import cv2
import numpy as np

def imread_chinese(path):
    """读取中文路径的图片"""
    img_array = np.fromfile(path, 
                            dtype=np.uint8)
    img = cv2.imdecode(img_array, -1)
    return img

# 使用
img = imread_chinese('张三_试卷.jpg')
cv2.imshow('成功!', img)
		\end{lstlisting}
	\end{columns}
\end{frame}

\begin{frame}[fragile]{工程实战:保存中文路径图片}
	\textbf{配套技巧:}保存时也支持中文路径

	\begin{columns}
		\column{0.5\textwidth}
		\begin{lstlisting}[language=Python, basicstyle=\ttfamily\tiny]
import cv2
import numpy as np

def imwrite_chinese(path, img):
    """保存中文路径的图片"""
    is_success, img_buf = cv2.imencode(
        ".jpg", img)
    if is_success:
        img_buf.tofile(path)
        return True
    return False

# 使用
img = cv2.imread('exam.jpg')
imwrite_chinese('处理结果_张三.jpg', img)
print("保存成功!")
		\end{lstlisting}

		\column{0.5\textwidth}
		\textbf{原理说明:}
		\begin{enumerate}
			\item \textbf{imdecode}:从内存中的字节数组解码
			\item \textbf{imencode + tofile}:编码为字节数组后写入
		\end{enumerate}

		\vspace{0.3cm}
		\textbf{阅卷系统应用:}
		\begin{itemize}
			\item 批量处理学生试卷
			\item 生成带学生姓名的结果文件
		\end{itemize}
	\end{columns}
\end{frame}

\begin{frame}[fragile]{Matplotlib 显示与 BGR 转换}
	为什么用 \texttt{plt.imshow} 显示出来的人脸是青蓝色的?

	\begin{lstlisting}[language=Python, basicstyle=\ttfamily\small]
import matplotlib.pyplot as plt

# OpenCV 是 BGR,Matplotlib 是 RGB
img_rgb = cv2.cvtColor(img, cv2.COLOR_BGR2RGB)

plt.imshow(img_rgb)
plt.show()
	\end{lstlisting}

	\vspace{0.3cm}
	\textbf{注意:} 灰度图显示时需要设置 \texttt{cmap='gray'},否则会变成"原油色"。

	\begin{lstlisting}[language=Python, basicstyle=\ttfamily\small]
plt.imshow(gray_img, cmap='gray')
	\end{lstlisting}
\end{frame}

\begin{frame}[fragile]{OpenCV进阶:几何变换}
	\begin{columns}
		\column{0.5\textwidth}
		\textbf{1. 平移(Translation)}
		\begin{lstlisting}[language=Python, basicstyle=\ttfamily\tiny]
# 向右平移100,向下平移50
M = np.float32([
    [1, 0, 100],  # x位移
    [0, 1, 50]   # y位移
])
translated = cv2.warpAffine(
    img, M, (w, h)
)
		\end{lstlisting}

		\vspace{0.2cm}
		\textbf{2. 旋转(Rotation)}
		\begin{lstlisting}[language=Python, basicstyle=\ttfamily\tiny]
# 逆时针旋转30度
center = (w//2, h//2)  # 旋转中心
M = cv2.getRotationMatrix2D(
    center,
    30,     # 角度(度)
    1.0     # 缩放比例
)
rotated = cv2.warpAffine(
    img, M, (w, h)
)
		\end{lstlisting}

		\column{0.5\textwidth}
		\textbf{3. 缩放(Scaling)}
		\begin{lstlisting}[language=Python, basicstyle=\ttfamily\tiny]
# 放大2倍
scaled = cv2.resize(
    img,
    None,
    fx=2.0,
    fy=2.0,
    interpolation=cv2.INTER_CUBIC
)

# 指定尺寸缩放
resized = cv2.resize(
    img,
    (800, 600),  # 宽, 高
    interpolation=cv2.INTER_LINEAR
)
		\end{lstlisting}

		\vspace{0.2cm}
		\begin{block}{插值方法选择}
		\begin{itemize}
			\item 放大:\texttt{INTER\_CUBIC}(质量最好)
			\item 缩小:\texttt{INTER\_AREA}(抗锯齿)
			\item 快速:\texttt{INTER\_LINEAR}
		\end{itemize}
		\end{block}
	\end{columns}
\end{frame}

\begin{frame}[fragile]{OpenCV进阶:仿射变换与透视变换}
	\begin{columns}
		\column{0.5\textwidth}
		\textbf{仿射变换(Affine Transform)}
		\begin{itemize}
			\item 保持平行线的平行性
			\item 需要3个点对应
		\end{itemize}
		\begin{lstlisting}[language=Python, basicstyle=\ttfamily\tiny]
# 原图像的3个点
src_pts = np.float32([
    [50, 50],    # 左上
    [200, 50],   # 右上
    [50, 200]    # 左下
])

# 目标图像的3个点
dst_pts = np.float32([
    [10, 10],
    [200, 20],
    [10, 200]
])

# 计算变换矩阵
M = cv2.getAffineTransform(
    src_pts, dst_pts
)

# 执行变换
affine = cv2.warpAffine(
    img, M, (w, h)
)
		\end{lstlisting}

		\column{0.5\textwidth}
		\textbf{透视变换(Perspective Transform)}
		\begin{itemize}
			\item 不保持平行性
			\item 需要4个点对应
			\item \highlight{下周重点!}
		\end{itemize}
		\begin{lstlisting}[language=Python, basicstyle=\ttfamily\tiny]
# 原图像的4个角点
src_pts = np.float32([
    [100, 150],  # 左上
    [450, 120],  # 右上
    [480, 380],  # 右下
    [80, 400]    # 左下
])

# 目标矩形
width, height = 400, 300
dst_pts = np.float32([
    [0, 0],
    [width-1, 0],
    [width-1, height-1],
    [0, height-1]
])

# 计算变换矩阵
M = cv2.getPerspectiveTransform(
    src_pts, dst_pts
)

# 执行变换
warped = cv2.warpPerspective(
    img, M, (width, height)
)
		\end{lstlisting}
	\end{columns}
\end{frame}

\begin{frame}[fragile]{OpenCV进阶:形态学操作基础}
	\textbf{形态学操作:} 基于图像形状的变换,常用于二值图像

	\begin{columns}
		\column{0.5\textwidth}
		\textbf{1. 腐蚀(Erosion)}
		\begin{itemize}
			\item 膨胀白色,收缩黑色
			\item 去除小噪点
		\end{itemize}
		\begin{lstlisting}[language=Python, basicstyle=\ttfamily\tiny]
kernel = np.ones((3, 3), np.uint8)
eroded = cv2.erode(binary, kernel, 
                   iterations=1)
		\end{lstlisting}

		\vspace{0.2cm}
		\textbf{2. 膨胀(Dilation)}
		\begin{itemize}
			\item 收缩白色,膨胀黑色
			\item 填充小孔洞
		\end{itemize}
		\begin{lstlisting}[language=Python, basicstyle=\ttfamily\tiny]
dilated = cv2.dilate(binary, kernel, 
                     iterations=1)
		\end{lstlisting}

		\vspace{0.2cm}
		\textbf{3. 开运算(Opening)}
		\begin{itemize}
			\item 先腐蚀后膨胀
			\item 去除小物体,保持大物体
		\end{itemize}
		\begin{lstlisting}[language=Python, basicstyle=\ttfamily\tiny]
opening = cv2.morphologyEx(
    binary, cv2.MORPH_OPEN, kernel
)
		\end{lstlisting}

		\column{0.5\textwidth}
		\textbf{4. 闭运算(Closing)}
		\begin{itemize}
			\item 先膨胀后腐蚀
			\item 填充小孔洞,连接近邻物体
		\end{itemize}
		\begin{lstlisting}[language=Python, basicstyle=\ttfamily\tiny]
closing = cv2.morphologyEx(
    binary, cv2.MORPH_CLOSE, kernel
)
		\end{lstlisting}

		\vspace{0.2cm}
		\textbf{5. 形态学梯度}
		\begin{itemize}
			\item 膨胀 - 腐蚀
			\item 提取边缘
		\end{itemize}
		\begin{lstlisting}[language=Python, basicstyle=\ttfamily\tiny]
gradient = cv2.morphologyEx(
    binary, cv2.MORPH_GRADIENT, 
    kernel
)
		\end{lstlisting}

		\vspace{0.2cm}
		\begin{block}{结构元素(Kernel)}
		可自定义形状:矩形、十字形、椭圆形\\
		大小决定影响范围
		\end{block}
	\end{columns}
\end{frame}

\begin{frame}[fragile]{形态学操作在阅卷中的应用}
	\begin{columns}
		\column{0.5\textwidth}
		\textbf{场景1:去除填涂噪点}
		\begin{itemize}
			\item 问题:填涂边缘有细小噪点
			\item 解决:开运算去除小噪点
		\end{itemize}
		\begin{lstlisting}[language=Python, basicstyle=\ttfamily\tiny]
# 去除小噪点
kernel = np.ones((2, 2), np.uint8)
clean_bubble = cv2.morphologyEx(
    binary,
    cv2.MORPH_OPEN,
    kernel
)
		\end{lstlisting}

		\vspace{0.2cm}
		\textbf{场景2:连接断开的笔迹}
		\begin{itemize}
			\item 问题:手写数字断开
			\item 解决:闭运算连接
		\end{itemize}
		\begin{lstlisting}[language=Python, basicstyle=\ttfamily\tiny]
# 连接断开部分
kernel = np.ones((3, 3), np.uint8)
connected = cv2.morphologyEx(
    binary,
    cv2.MORPH_CLOSE,
    kernel
)
		\end{lstlisting}

		\column{0.5\textwidth}
		\textbf{场景3:提取轮廓边缘}
		\begin{itemize}
			\item 问题:需要清晰的轮廓
			\item 解决:形态学梯度
		\end{itemize}
		\begin{lstlisting}[language=Python, basicstyle=\ttfamily\tiny]
# 提取边缘
gradient = cv2.morphologyEx(
    binary,
    cv2.MORPH_GRADIENT,
    kernel
)
		\end{lstlisting}

		\vspace{0.2cm}
		\begin{block}{可视化对比}
		\begin{itemize}
			\item 原始:有噪点的填涂
			\item 开运算:干净的填涂
			\item 闭运算:连接的笔迹
			\item 梯度:清晰的边缘
		\end{itemize}
		\end{block}

		\vspace{0.2cm}
		\begin{center}
			\highlight{形态学操作是图像预处理的"外科手术刀"}
		\end{center}
	\end{columns}
\end{frame}
