%=============================================================================
% 模块一:导论与项目愿景
%=============================================================================

\section{计算机视觉导论}

\begin{frame}[fragile]{视觉:人类获取信息的最主要渠道}
    \begin{columns}
        \column{0.5\textwidth}
        \textbf{人类视觉:}
        \begin{itemize}
            \item 人脑约 50\% 的神经元参与视觉处理。
            \item \highlight{语义理解}:我们看到的不是像素,是"人"、"车"、"试卷"。
        \end{itemize}
        \vspace{0.5cm}
        \textbf{计算机视觉 (CV):}
        \begin{itemize}
            \item 目标:给机器安装"眼睛"和"大脑"。
            \item 挑战:图像在计算机眼中只是一组 \highlight{数字}。
        \end{itemize}

        \column{0.5\textwidth}
        \begin{figure}
            \centering
            \includegraphics[width=0.8\textwidth]{example-image}
            \caption{语义 vs 矩阵(请替换为实际图片)}
        \end{figure}
    \end{columns}
\end{frame}

\begin{frame}[fragile]{计算机视觉应用全景}
    \begin{columns}
        \column{0.33\textwidth}
        \begin{block}{自动驾驶}
            Tesla Autopilot、Waymo
            \begin{itemize}
                \item 车道检测
                \item 交通标志识别
                \item 行人检测
            \end{itemize}
        \end{block}

        \column{0.33\textwidth}
        \begin{block}{医疗影像}
            CT/MRI诊断
            \begin{itemize}
                \item 肿瘤检测
                \item 器官分割
                \item 病理分析
            \end{itemize}
        \end{block}

        \column{0.33\textwidth}
        \begin{block}{工业检测}
            产品质检
            \begin{itemize}
                \item 缺陷识别
                \item 尺寸测量
                \item 质量控制
            \end{itemize}
        \end{block}
    \end{columns}

    \vspace{0.5cm}

    \begin{columns}
        \column{0.33\textwidth}
        \begin{block}{人脸识别}
            支付宝、安防系统
            \begin{itemize}
                \item 身份验证
                \item 门禁系统
                \item 犯罪侦查
            \end{itemize}
        \end{block}

        \column{0.33\textwidth}
        \begin{block}{OCR文字识别}
            文档数字化
            \begin{itemize}
                \item 发票识别
                \item 车牌识别
                \item 手写转录
            \end{itemize}
        \end{block}

        \column{0.33\textwidth}
        \begin{block}{AR/VR}
            增强现实
            \begin{itemize}
                \item 虚拟试衣
                \item 游戏交互
                \item 远程协作
            \end{itemize}
        \end{block}
    \end{columns}

    \vspace{0.3cm}
    \begin{center}
        \highlight{所有这些应用的核心都是:将图像转化为可理解的信息}
    \end{center}
\end{frame}

\begin{frame}[fragile]{CV 的历史与现状}
    \begin{itemize}
        \item \textbf{1960s}:Larry Roberts (CV之父) 尝试让机器识别积木世界。
        \item \textbf{1970s-1980s}:提出边缘检测、Marr 视觉计算理论。
        \item \textbf{2012-至今}:\highlight{深度学习爆发},AlexNet 在 ImageNet 竞赛中夺冠。
    \end{itemize}
    \begin{block}{核心任务演变}
        分类 (是什么?) $\to$ 检测 (在哪儿?) $\to$ 分割 (形状如何?) $\to$ 生成 (画一个出来)
    \end{block}
\end{frame}

\begin{frame}[fragile]{贯穿本学期的项目:AI 阅卷助手}
    \begin{columns}
        \column{0.4\textwidth}
        \textbf{任务分解:}
        \begin{enumerate}
            \item \textbf{图像采集}:拍照、扫描。
            \item \textbf{预处理}:纠偏、增强(本周内容)。
            \item \textbf{定位}:找到答题卡、填空区。
            \item \textbf{识别}:OCR (光学字符识别)。
            \item \textbf{评分}:逻辑比对。
        \end{enumerate}
        \column{0.6\textwidth}
        \begin{center}
            \begin{tikzpicture}[scale=0.8]
                \draw[thick] (0,0) rectangle (4,5);
                \node at (2,4.5) {试卷图像};
                \draw[fill=blue!20] (0.5,3) rectangle (3.5,3.8); \node at (2,3.4) {学号区};
                \draw[fill=red!20] (0.5,1) rectangle (3.5,2.5); \node at (2,1.75) {答题区};
                \draw[->, ultra thick, red] (4.5, 2.5) -- (6, 2.5) node[right] {AI 识别};
            \end{tikzpicture}
        \end{center}
    \end{columns}
\end{frame}

\begin{frame}[fragile]{AI阅卷系统的技术挑战}
    \begin{columns}
        \column{0.5\textwidth}
        \textbf{挑战1:手写字迹识别难度}
        \begin{itemize}
            \item 每个人的字迹不同
            \item 连笔、潦草、抖动
            \item 相似字符混淆(如:0 vs O, 1 vs l)
        \end{itemize}

        \vspace{0.3cm}
        \textbf{挑战2:答题卡污渍处理}
        \begin{itemize}
            \item 涂改痕迹
            \item 折痕污损
            \item 水渍污染
        \end{itemize}

        \column{0.5\textwidth}
        \textbf{挑战3:多种笔迹类型识别}
        \begin{itemize}
            \item 钢笔、圆珠笔、铅笔
            \item 蓝色、黑色、红色
            \item 粗细不同、压力不同
        \end{itemize}

        \vspace{0.3cm}
        \textbf{挑战4:防作弊机制}
        \begin{itemize}
            \item 检测异常填涂
            \item 识别多选作弊
            \item 图像篡改检测
        \end{itemize}
    \end{columns}

    \vspace{0.5cm}
    \begin{block}{工程价值}
        阅卷系统将人工阅卷的准确率从\textbf{95\%}提升到\textbf{99.9\%},效率提升\textbf{100倍}
    \end{block}
\end{frame}

% -----------------------------------------------------------------------------
% 答题卡定位锚点(Timing Marks)
% -----------------------------------------------------------------------------

\begin{frame}[fragile]{项目深度:答题卡的"定位锚点" (Timing Marks)}
    \textbf{问题:}如果试卷皱了或者拍摄角度极度倾斜,如何准确定位答题卡?

    \begin{columns}
        \column{0.5\textwidth}
        \textbf{Timing Marks 的作用:}
        \begin{itemize}
            \item 答题卡边缘的黑色小方块
            \item 用于精确定位答题卡区域
            \item 类似二维码的定位图案
        \end{itemize}

        \vspace{0.3cm}
        \textbf{设计规范:}
        \begin{itemize}
            \item 位置:四个角或边缘
            \item 大小:固定的几何尺寸
            \item 对比度:黑色 vs 白纸
            \item 排列:特定模式(如L形)
        \end{itemize}

        \column{0.5\textwidth}
        \begin{center}
            \begin{tikzpicture}[scale=0.6]
                % 试卷边框
                \draw[thick] (0,0) rectangle (6,8);

                % 定位块(Timing Marks)
                \draw[fill=black] (0.2,0.2) rectangle (0.8,0.8);
                \draw[fill=black] (5.2,0.2) rectangle (5.8,0.8);
                \draw[fill=black] (0.2,7.2) rectangle (0.8,7.8);
                \draw[fill=black] (5.2,7.2) rectangle (5.8,7.8);

                % 答题区
                \draw[fill=blue!10] (1,2) rectangle (5,6);
                \node at (3,4) {答题区};

                % 箭头说明
                \draw[->, red, thick] (0.8,0.5) -- (2,0.5) node[right] {\tiny 检测黑块定位};
            \end{tikzpicture}
        \end{center}
    \end{columns}

    \begin{block}{工程价值}
        Timing Marks 是阅卷系统的"眼睛",有了它们,即使试卷皱了也能准确定位
    \end{block}
\end{frame}

\begin{frame}[fragile]{定位锚点检测:从轮廓到坐标}
    \textbf{核心思路:}检测黑色方块 $\to$ 计算中心 $\to$ 透视变换

    \begin{columns}
        \column{0.5\textwidth}
        \begin{lstlisting}
import cv2
import numpy as np

# 1. 读取答题卡
img = cv2.imread('answer_sheet.jpg')
gray = cv2.cvtColor(img, cv2.COLOR_BGR2GRAY)

# 2. 二值化
_, binary = cv2.threshold(gray, 127, 255, cv2.THRESH_BINARY)

# 3. 查找黑色方块轮廓
contours, _ = cv2.findContours(binary, cv2.RETR_EXTERNAL, cv2.CHAIN_APPROX_SIMPLE)

# 4. 筛选定位块(小方块)
marks = []
for cnt in contours:
    area = cv2.contourArea(cnt)
    if 100 < area < 1000:
        M = cv2.moments(cnt)
        if M['m00'] != 0:
            cx = int(M['m10'] / M['m00'])
            cy = int(M['m01'] / M['m00'])
            marks.append((cx, cy))

print(f"检测到 {len(marks)} 个定位点")
        \end{lstlisting}

        \column{0.5\textwidth}
        \textbf{检测流程:}
        \begin{enumerate}
            \item \textbf{二值化}:分离黑色定位块和白色背景
            \item \textbf{轮廓查找}:找到所有黑色区域
            \item \textbf{面积筛选}:定位块面积固定,排除过大或过小的轮廓
            \item \textbf{中心计算}:使用质心矩计算中心点
        \end{enumerate}

        \begin{block}{下周预告}
        透视变换:矫正倾斜;特征匹配:自动定位
        \end{block}
    \end{columns}
\end{frame}

\begin{frame}[fragile]{为什么需要AI编程助手?}
    \begin{columns}
        \column{0.5\textwidth}
        \textbf{传统编程的痛点:}
        \begin{itemize}
            \item API 参数复杂,记不住
            \item 报错信息看不懂
            \item 算法原理理解困难
            \item 编程基础薄弱
        \end{itemize}

        \vspace{0.3cm}
        \textbf{AI辅助的优势:}
        \begin{itemize}
            \item \highlight{快速生成代码框架}
            \item \highlight{解释错误原因}
            \item \highlight{提供优化建议}
            \item \highlight{降低学习门槛}
        \end{itemize}

        \column{0.5\textwidth}
        \textbf{本学期AI工具:}
        \begin{itemize}
            \item \textbf{Claude Code}:代码生成与解释
            \item \textbf{通义千问}:中文友好,国内可用
            \item \textbf{ChatGPT}:通用编程助手
        \end{itemize}

        \vspace{0.3cm}
        \begin{block}{课程特色}
        用AI工具辅助学习,聚焦\textbf{理解原理}而非\textbf{记忆API}
        \end{block}
    \end{columns}

    \vspace{0.5cm}
    \begin{center}
        \textbf{下周我们将专门学习如何用AI辅助编程!}
    \end{center}
\end{frame}

\begin{frame}[fragile]{本学期AI工具使用计划}
    \begin{columns}
        \column{0.5\textwidth}
        \textbf{AI工具应用场景:}
        \begin{enumerate}
            \item \textbf{Week 2}:AI辅助编程实战
            \begin{itemize}
                \item 学习Prompt工程
                \item 用AI生成人脸检测代码
            \end{itemize}
            \item \textbf{Week 3-4}:图像预处理与版面分析
            \begin{itemize}
                \item 用AI解释复杂算法
                \item 调试二值化参数
            \end{itemize}
            \item \textbf{Week 5-6}:选择题与判断题识别
            \begin{itemize}
                \item 用AI生成代码框架
                \item 优化识别算法
            \end{itemize}
            \item \textbf{Week 7-8}:OCR与手写识别
            \begin{itemize}
                \item 用AI理解深度学习模型
                \item 调试模型训练过程
            \end{itemize}
        \end{enumerate}

        \column{0.5\textwidth}
        \textbf{AI工具使用规范:}
        \begin{itemize}
            \item \textcolor{green!60!black}{\checkmark} 允许:用AI解释概念、调试错误、优化代码
            \item \textcolor{green!60!black}{\checkmark} 鼓励:用AI生成对比实验、可视化结果
            \item \textcolor{red}{\times} 禁止:直接复制完整代码、用AI完成全部作业
        \end{itemize}

        \vspace{0.3cm}
        \begin{alertblock}{重要原则}
        理解原理 > 复制代码\\
        AI是助手,不是替代者
        \end{alertblock}
    \end{columns}
\end{frame}

\begin{frame}[fragile]{引出下周:透视变换的力量}
    \textbf{场景:}检测到定位点后,如何矫正倾斜的试卷?

    \begin{columns}
        \column{0.5\textwidth}
        \textbf{当前问题:}
        \begin{itemize}
            \item 手机拍照难免倾斜
            \item 试卷可能有透视变形
            \item 直接处理会降低识别率
        \end{itemize}

        \textbf{下周解决方案:}
        \begin{enumerate}
            \item \textbf{透视变换}:将任意四边形矫正为矩形
            \item \textbf{特征匹配}:自动找到定位点
        \end{enumerate}

        \column{0.5\textwidth}
        \begin{center}
            \begin{tikzpicture}[scale=0.5]
                % 原始倾斜图像
                \begin{scope}[shift={(0,0)}]
                    \draw[thick, fill=gray!20] (0,0) -- (3,1) -- (4,5) -- (1,4) -- cycle;
                    \node at (2,2.5) {\tiny 倾斜试卷};
                    \fill[black] (0.2,0.3) circle (0.1);
                    \fill[black] (2.8,1.2) circle (0.1);
                    \fill[black] (3.8,4.7) circle (0.1);
                    \fill[black] (1.2,3.8) circle (0.1);
                \end{scope}

                \draw[->, ultra thick, blue] (4.5,2.5) -- (6.5,2.5) node[midway,above] {\tiny 透视变换};

                % 矫正后图像
                \begin{scope}[shift={(7,0)}]
                    \draw[thick, fill=white] (0,0) rectangle (4,5);
                    \node at (2,2.5) {\tiny 矫正面};
                    \fill[black] (0.2,0.2) rectangle (0.5,0.5);
                    \fill[black] (3.5,0.2) rectangle (3.8,0.5);
                    \fill[black] (0.2,4.5) rectangle (0.5,4.8);
                    \fill[black] (3.5,4.5) rectangle (3.8,4.8);
                \end{scope}
            \end{tikzpicture}
        \end{center}

        \begin{block}{关键技术}
        \texttt{cv2.getPerspectiveTransform()} + \texttt{cv2.warpPerspective()}
        \end{block}
    \end{columns}
\end{frame}
