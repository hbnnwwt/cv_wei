%=============================================================================
% 模块六:互动测验与Live Coding
%=============================================================================

\section{互动测验}

\begin{frame}[fragile]{小测验时间(1):NumPy 综合测试}
	\begin{block}{问题}
		给定形状为 (100, 100, 3) 的彩色图像,如何提取中心 50×50 的红色通道值?
		\begin{enumerate}[(A)]
			\item \texttt{img[25:75, 25:75, 0]}
			\item \texttt{img[25:75, 25:75, 2]}
			\item \texttt{img[50:100, 50:100, 2]}
			\item \texttt{img[25:75, 25:75]}
		\end{enumerate}
	\end{block}
	\pause
	\textbf{答案:} \highlight{B. img[25:75, 25:75, 2]} \\
	\vspace{0.2cm}
	\textbf{解析:}
	\begin{itemize}
		\item 100×100 图像的中心 50×50:索引从 25 到 75
		\item OpenCV 中 BGR 顺序,红色通道索引为 2
		\item \texttt{img[y1:y2, x1:x2, channel]} 格式
	\end{itemize}
\end{frame}

% -----------------------------------------------------------------------------
% Live Coding 闪电编程任务
% -----------------------------------------------------------------------------

\begin{frame}[fragile]{Live Coding(1/3):5分钟闪电编程任务}
	\begin{alertblock}{编程挑战}
		给定一张"脏"试卷图像(含噪声、光照不均)\\
		\textbf{任务:}在 5 分钟内,用 \highlight{3 行代码} 提取出学号区的均值
	\end{alertblock}

	\begin{columns}
		\column{0.5\textwidth}
		\textbf{提示:}
		\begin{enumerate}
			\item 读取图像(已有)
			\item 切片学号区
			\item 计算均值
		\end{enumerate}

		\vspace{0.3cm}
		\textbf{学号区位置:}
		\begin{itemize}
			\item 假设在左上角
			\item 坐标范围:\texttt{[50:150, 100:300]}
			\item 高度:100px,宽度:200px
		\end{itemize}

		\column{0.5\textwidth}
		\begin{lstlisting}[language=Python, basicstyle=\ttfamily\tiny]
# 给定代码(不要修改)
import cv2
import numpy as np

img = cv2.imread('dirty_exam.jpg')

# ===== 你的任务:补全这3行 =====
# 第1行:灰度化
gray = ???

# 第2行:切片学号区
id_region = ???

# 第3行:计算均值
mean_value = ???

# =============================

print(f"学号区均值: {mean_value}")
\end{lstlisting}
	\end{columns}
\end{frame}

\begin{frame}[fragile]{Live Coding(2/3):参考答案与解析}
	\textbf{参考答案:}

	\begin{lstlisting}[language=Python, basicstyle=\ttfamily\tiny]
import cv2
import numpy as np

img = cv2.imread('dirty_exam.jpg')

# 第1行:灰度化
gray = cv2.cvtColor(img, cv2.COLOR_BGR2GRAY)

# 第2行:切片学号区
id_region = gray[50:150, 100:300]

# 第3行:计算均值
mean_value = np.mean(id_region)

print(f"学号区均值: {mean_value:.2f}")
\end{lstlisting}

	\vspace{0.3cm}
	\textbf{代码解析:}
	\begin{itemize}
		\item \textbf{灰度化}:\texttt{cv2.cvtColor(img, cv2.COLOR\_BGR2GRAY)} 将三维彩色图转为二维灰度图
		\item \textbf{切片}:\texttt{gray[y1:y2, x1:x2]},y在前,x在后,提取高度方向 [50:150],宽度方向 [100:300]
		\item \textbf{均值}:\texttt{np.mean(array)} 返回所有像素的平均亮度,可用于判断学号区是否填涂
	\end{itemize}
\end{frame}

\begin{frame}[fragile]{Live Coding(3/3):即时反馈与扩展}
	\textbf{课堂互动:}分享你的切片坐标

	\begin{columns}
		\column{0.5\textwidth}
		\textbf{学生分享环节:}
		\begin{itemize}
			\item 你用的是哪个坐标范围?
			\item \texttt{img[50:150, 100:300]}?
			\item 还是 \texttt{img[0:100, 0:200]}?
			\item 坐标不同,结果如何?
		\end{itemize}

		\vspace{0.3cm}
		\textbf{观察要点:}
		\begin{enumerate}
			\item 不同坐标范围,均值会不同
			\item 越亮的区域,均值越大
			\item 越暗的区域,均值越小
		\end{enumerate}

		\column{0.5\textwidth}
		\textbf{进阶挑战(可选):}
		\begin{lstlisting}[language=Python, basicstyle=\ttfamily\tiny]
# 挑战1:计算标准差
std_value = np.std(id_region)
print(f"标准差: {std_value:.2f}")

# 挑战2:判断是否填涂
if mean_value < 100:
    print("学号区可能已填涂")
else:
    print("学号区未填涂")

# 挑战3:可视化切片
cv2.imshow('学号区', id_region)
cv2.waitKey(0)
\end{lstlisting}

		\vspace{0.2cm}
		\textbf{阅卷应用:}
		\begin{itemize}
			\item 均值:判断填涂密度
			\item 标准差:判断填涂均匀性
			\item 阈值:自动识别填涂状态
		\end{itemize}
	\end{columns}
\end{frame}

\begin{frame}[fragile]{小测验时间(5):综合应用}
	\begin{block}{问题}
		在阅卷系统中,要将答题卡的填涂区域(黑色)从白色纸张中分离出来,应该使用哪种阈值类型?
		\begin{enumerate}[(A)]
			\item \texttt{cv2.THRESH\_BINARY}
			\item \texttt{cv2.THRESH\_BINARY\_INV}
			\item \texttt{cv2.THRESH\_TRUNC}
			\item \texttt{cv2.THRESH\_TOZERO}
		\end{enumerate}
	\end{block}
	\pause
	\textbf{答案:} \highlight{A. cv2.THRESH\_BINARY} \\
	\vspace{0.2cm}
	\textbf{解析:}
	\begin{itemize}
		\item 填涂区域是黑色(低像素值)
		\item 纸张是白色(高像素值)
		\item \texttt{THRESH\_BINARY} 会将低于阈值的设为 0(黑),高于阈值的设为 255(白)
		\item 正好分离填涂和纸张
	\end{itemize}
\end{frame}
