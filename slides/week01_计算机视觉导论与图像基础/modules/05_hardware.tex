%=============================================================================
% 模块五:工业硬件知识
%=============================================================================

\section{硬件知识}

\begin{frame}[fragile]{硬件知识(1/3):卷帘快门 vs 全局快门}
	\textbf{快门:控制传感器曝光时间的机制}

	\begin{columns}
		\column{0.5\textwidth}
		\textbf{卷帘快门}
		\begin{itemize}
			\item 逐行曝光,像"扫描"一样
			\item 优点:便宜、功耗低
			\item 缺点:拍摄快速物体会变形
			\item \highlight{斜坡效应}
			      \begin{itemize}
				      \item 照片上下部分时间不同
				      \item 快速物体会倾斜
			      \end{itemize}
		\end{itemize}

		\vspace{0.3cm}
		\textbf{应用:}手机摄像头、普通相机

		\column{0.5\textwidth}
		\textbf{全局快门}
		\begin{itemize}
			\item 所有像素同时曝光
			\item 优点:无变形
			\item 缺点:昂贵、功耗高
			\item 工业相机标配
		\end{itemize}

		\vspace{0.3cm}
		\textbf{应用:}工业检测、高速摄影

		\vspace{0.3cm}
		\begin{center}
			\begin{tikzpicture}[scale=0.3]
				\node[draw, fill=blue!20] (rolling) at (0,0) {卷帘快门};
				\node[draw, fill=green!20] (global) at (5,0) {全局快门};
				\draw[->] (rolling) -- (3,-2) node[midway,right] {\tiny 倾斜};
				\draw[->] (global) -- (8,-2) node[midway,right] {\tiny 正常};
			\end{tikzpicture}
		\end{center}
	\end{columns}

	\vspace{0.3cm}
	\begin{block}{阅卷系统建议}
		如果试卷传输速度快,需使用全局快门相机,避免文字倾斜变形
	\end{block}
\end{frame}

\begin{frame}[fragile]{硬件知识(2/3):CMOS vs CCD 传感器}
	\textbf{图像传感器:将光信号转换为电信号的芯片}

	\begin{columns}
		\column{0.5\textwidth}
		\textbf{CMOS 传感器}
		\begin{itemize}
			\item 每个像素有独立的放大器
			\item 优点:
			      \begin{itemize}
				      \item 速度快(并行读出)
				      \item 功耗低
				      \item 成本低
				      \item 集成度高
			      \end{itemize}
			\item 缺点:
			      \begin{itemize}
				      \item 噪声较大
				      \item 填充因子低
			      \end{itemize}
		\end{itemize}

		\vspace{0.2cm}
		\textbf{应用:}手机、网络摄像头、消费级相机

		\column{0.5\textwidth}
		\textbf{CCD 传感器}
		\begin{itemize}
			\item 所有电荷通过一个放大器
			\item 优点:
			      \begin{itemize}
				      \item 图像质量高
				      \item 噪声低
				      \item 动态范围大
				      \item 一致性好
			      \end{itemize}
			\item 缺点:
			      \begin{itemize}
				      \item 速度慢(串行读出)
				      \item 功耗高
				      \item 成本高
			      \end{itemize}
		\end{itemize}

		\vspace{0.2cm}
		\textbf{应用:}天文摄影、科学成像、高端相机
	\end{columns}

	\vspace{0.3cm}
	\begin{block}{阅卷系统选择}
		高速扫描仪使用 CMOS(速度快),精密分析可考虑 CCD(质量高)
	\end{block}
\end{frame}

\begin{frame}[fragile]{硬件知识(3/3):图像传感器关键参数}
	\textbf{选择相机时需要关注的参数}

	\begin{columns}
		\column{0.5\textwidth}
		\textbf{1. 分辨率}
		\begin{itemize}
			\item 像素数量:1920×1080, 4K等
			\item \highlight{不是越高越好!}
			\item 阅卷:300dpi 扫描足够
		\end{itemize}

		\vspace{0.2cm}
		\textbf{2. 帧率}
		\begin{itemize}
			\item 每秒传输图像数
			\item 高速扫描:30-60 fps
		\end{itemize}

		\vspace{0.2cm}
		\textbf{3. 像素尺寸}
		\begin{itemize}
			\item 单个像素物理大小(μm)
			\item 越大 = 进光量越大 = 噪声越低
			\item 手机:1.4μm,单反:5-8μm
		\end{itemize}

		\column{0.5\textwidth}
		\textbf{4. 动态范围}
		\begin{itemize}
			\item 能同时捕捉的最亮和最暗细节
			\item 单位:dB 或 stops
			\item 高端:60-80 dB
			\item 普通:40-60 dB
		\end{itemize}

		\vspace{0.2cm}
		\textbf{5. 信噪比(SNR)}
		\begin{itemize}
			\item 信号与噪声的比值
			\item 越高越好
			\item >40dB 为优秀
		\end{itemize}

		\vspace{0.2cm}
		\textbf{6. ISO 感光度}
		\begin{itemize}
			\item 对光的敏感程度
			\item 高 ISO:噪点多
			\item 阅卷:低 ISO(100-400)
		\end{itemize}
	\end{columns}

	\vspace{0.3cm}
	\begin{block}{阅卷系统推荐配置}
		分辨率:≥5MP | 帧率:≥30 fps | 快门:全局快门 | 接口:GigE/USB3.0
	\end{block}
\end{frame}

\begin{frame}[fragile]{工业相机选型指南}
	\textbf{阅卷系统相机选型决策树}

	\begin{columns}
		\column{0.5\textwidth}
		\textbf{根据吞吐量选择:}
		\begin{itemize}
			\item \textbf{小型(<1000张/天)}:USB3.0相机
			\item \textbf{中型(1000-10000张/天)}:GigE相机
			\item \textbf{大型(>10000张/天)}:Camera Link高速相机
		\end{itemize}

		\vspace{0.3cm}
		\textbf{根据精度选择:}
		\begin{itemize}
			\item \textbf{普通阅卷}:300dpi ≈ 5MP
			\item \textbf{手写识别}:600dpi ≈ 10MP
			\item \textbf{图表分析}:1200dpi ≈ 20MP
		\end{itemize}

		\column{0.5\textwidth}
		\textbf{热门品牌对比:}
		\begin{table}
			\centering
			\small
			\begin{tabular}{lccc}
				\toprule
				品牌 & 价格 & 质量 & 支持 \\
				\midrule
				Basler & 中 & 高 & 好 \\
				IDS & 中 & 高 & 好 \\
				海康 & 低 & 中 & 中 \\
				大华 & 低 & 中 & 中 \\
				\bottomrule
			\end{tabular}
		\end{table}

		\vspace{0.2cm}
		\textbf{接口选择:}
		\begin{itemize}
			\item \textbf{USB3.0}:短距离(<3m),即插即用
			\item \textbf{GigE}:长距离(<100m),需交换机
			\item \textbf{Camera Link}:超高速,专用线缆
		\end{itemize}
	\end{columns}
\end{frame}

\begin{frame}[fragile]{阅卷系统硬件配置建议}
	\textbf{完整硬件方案示例}

	\begin{columns}
		\column{0.5\textwidth}
		\textbf{方案A:经济型(学校用)}
		\begin{itemize}
			\item 相机:USB3.0工业相机(5MP,30fps)
			\item 镜头:定焦镜头(8mm, f/2.8)
			\item 光源:LED环形光(可调亮度)
			\item 传输:USB3.0线缆(2米)
			\item 预算:约5000-8000元
		\end{itemize}

		\vspace{0.3cm}
		\textbf{方案B:专业型(考试中心)}
		\begin{itemize}
			\item 相机:GigE全局快门(10MP,60fps)
			\item 镜头:远心镜头(低畸变)
			\item 光源:条形光源阵列
			\item 传输:千兆以太网
			\item 预算:约20000-30000元
		\end{itemize}

		\column{0.5\textwidth}
		\textbf{关键配置要点:}
		\begin{enumerate}
			\item \textbf{全局快门}:避免文字倾斜
			\item \textbf{镜头畸变<1\%}:保证定位精度
			\item \textbf{均匀照明}:光照变化<10\%
			\item \textbf{固定焦距}:避免自动对焦抖动
			\item \textbf{稳定支架}:减少震动影响
		\end{enumerate}

		\vspace{0.3cm}
		\begin{alertblock}{避坑指南}
			\begin{itemize}
				\item 不要用手机摄像头(卷帘快门)
				\item 不要用网络摄像头(分辨率低)
				\item 不要用自动对焦镜头(会漂移)
			\end{itemize}
		\end{alertblock}
	\end{columns}
\end{frame}
