%=============================================================================
% 总结与作业
%=============================================================================

\section{完整流程演示}

\begin{frame}[fragile]{阅卷系统完整流程(1/3):图像预处理}
    \textbf{目标:}将拍摄的试卷图像转换为适合处理的格式

    \begin{lstlisting}
import cv2
import numpy as np

# 1. 读取图像
img = cv2.imread('exam_paper.jpg')

# 2. 灰度化
gray = cv2.cvtColor(img, cv2.COLOR_BGR2GRAY)

# 3. 去噪
denoised = cv2.GaussianBlur(gray, (5, 5), 0)

# 4. 对比度增强
clahe = cv2.createCLAHE(clipLimit=2.0, tileGridSize=(8,8))
enhanced = clahe.apply(denoised)

# 5. 二值化
binary = cv2.adaptiveThreshold(enhanced, 255,
    cv2.ADAPTIVE_THRESH_GAUSSIAN_C,
    cv2.THRESH_BINARY, 11, 2)
    \end{lstlisting}

    \textbf{预处理步骤:}
    \begin{enumerate}
        \item \textbf{灰度化}:去除颜色信息,减少数据量
        \item \textbf{去噪}:去除扫描噪点
        \item \textbf{对比度增强}:CLAHE 自适应处理
        \item \textbf{二值化}:分离填涂和纸张
    \end{enumerate}
\end{frame}

\begin{frame}[fragile]{阅卷系统完整流程(2/3):定位答题卡}
    \textbf{目标:}在试卷中找到答题卡区域

    \begin{lstlisting}
# 1. 边缘检测
edges = cv2.Canny(binary, 50, 150)

# 2. 查找轮廓
contours, _ = cv2.findContours(edges, cv2.RETR_EXTERNAL, cv2.CHAIN_APPROX_SIMPLE)

# 3. 筛选矩形轮廓
for cnt in contours:
    peri = cv2.arcLength(cnt, True)
    approx = cv2.approxPolyDP(cnt, 0.02 * peri, True)
    if len(approx) == 4:
        x, y, w, h = cv2.boundingRect(cnt)
        answer_sheet = binary[y:y+h, x:x+w]
        break
    \end{lstlisting}
\end{frame}

\begin{frame}[fragile]{阅卷系统完整流程(3/3):识别填涂}
    \textbf{目标:}识别每个选项是否被填涂

    \begin{lstlisting}
# 定义每个选项的位置
options = [
    (50, 30, 80, 60),   # 第1题 A
    (90, 30, 120, 60),  # 第1题 B
]

results = []
for (x1, y1, x2, y2) in options:
    option = answer_sheet[y1:y2, x1:x2]
    black_ratio = np.sum(option == 0) / option.size
    is_filled = black_ratio > 0.3  # 阈值30%
    results.append(is_filled)
    \end{lstlisting}

    \textbf{识别算法优化:}
    \begin{itemize}
        \item 阈值 30\% 是经验值,可根据实际情况调整
        \item 检测是否多选,给出警告
        \item 未填涂记录,后续人工复核
    \end{itemize}
\end{frame}

% -----------------------------------------------------------------------------
% 互动测验
% -----------------------------------------------------------------------------

\section{互动测验}

\begin{frame}[fragile]{快速问答环节}
    \begin{columns}
        \column{0.5\textwidth}
        \textbf{问题1:OpenCV默认读取的彩色图像是什么顺序?}
        \begin{itemize}
            \item[A] RGB
            \item[B] \highlight{BGR}(正确)
            \item[C] HSV
            \item[D] LAB
        \end{itemize}

        \vspace{0.3cm}
        \textbf{问题2:如何判断一个图像是否读取成功?}
        \begin{itemize}
            \item[A] if img != None
            \item[B] \highlight{if img is not None}(正确)
            \item[C] if img.exists()
            \item[D] if len(img) > 0
        \end{itemize}

        \column{0.5\textwidth}
        \textbf{问题3:uint8类型的像素值范围是?}
        \begin{itemize}
            \item[A] 0-1023
            \item[B] \highlight{0-255}(正确)
            \item[C] -128-127
            \item[D] 0-65535
        \end{itemize}

        \vspace{0.3cm}
        \textbf{问题4:图像像素相加时,如何避免溢出?}
        \begin{itemize}
            \item[A] 直接相加
            \item[B] \highlight{cv2.add() 或 np.clip()}(正确)
            \item[C] 转为float后相加
            \item[D] 无需处理
        \end{itemize}
    \end{columns}

    \vspace{0.5cm}
    \begin{center}
        \highlight{正确率:\uncover<2->{\textbf{100\%} 🎉}}
    \end{center}
\end{frame}

\begin{frame}[fragile]{小测验时间(2):代码找错挑战}
    \textbf{找出以下代码中的3个错误:}

    \begin{lstlisting}
import cv2
import numpy as np

# 读取图像
img = cv2.imread('张三试卷.jpg')  # 错误1

# 亮度增加50
bright_img = img + 50  # 错误2

# 显示
plt.imshow(img)  # 错误3
plt.show()
    \end{lstlisting}

    \vspace{0.3cm}
    \begin{block}{答案揭晓}
        \begin{enumerate}
            \item \textbf{错误1:} 中文路径问题。需要使用\texttt{imread\_chinese()}函数
            \item \textbf{错误2:} 直接相加会导致溢出。应该使用\texttt{cv2.add(img, np.array([50.0]))}
            \item \textbf{错误3:} OpenCV是BGR,matplotlib是RGB。应该先转换\texttt{img = cv2.cvtColor(img, cv2.COLOR\_BGR2RGB)}
        \end{enumerate}
    \end{block}

    \vspace{0.3cm}
    \begin{center}
        \textbf{修正后的代码:}
        \begin{lstlisting}
img = imread_chinese('张三试卷.jpg')
bright_img = cv2.add(img, np.array([50.0]))
img_rgb = cv2.cvtColor(img, cv2.COLOR_BGR2RGB)
plt.imshow(img_rgb)
        \end{lstlisting}
    \end{center}
\end{frame}

% -----------------------------------------------------------------------------
% 课后作业
% -----------------------------------------------------------------------------

\section{课后作业}

\begin{frame}[fragile]{课后作业:我的第一个图像处理器}
    \begin{exampleblock}{作业要求}
        编写一个 Python 脚本,读取一张照片并生成一张包含 4 张子图的对比图:
        \begin{enumerate}
            \item 原图
            \item 灰度图
            \item 亮度增强后的图
            \item 反色后的图
        \end{enumerate}
    \end{exampleblock}
    \textbf{提交方式:} 截图 + 代码
\end{frame}

\begin{frame}{知识点网络与下周预告}
    \begin{columns}
        \column{0.6\textwidth}
        \textbf{本周核心知识点:}
        \begin{itemize}
            \item 计算机视觉基本概念
            \item 图像的数字表示(矩阵、RGB)
            \item OpenCV基础操作
            \item 图像预处理(灰度、二值化、去噪)
            \item 阅卷系统入门
        \end{itemize}

        \vspace{0.3cm}
        \textbf{下周预告(Week 2):AI辅助编程工具实战}
        \begin{itemize}
            \item \textbf{ChatGPT/Claude}:学习编程的AI助手
            \item \textbf{Prompt工程}:如何让AI帮我们写代码
            \item \textbf{实战演练}:用AI辅助实现人脸检测
        \end{itemize}

        \column{0.4\textwidth}
        \begin{block}{跨周链接}
            \begin{itemize}
                \item Week 1:图像基础 ⚙️
                \item Week 2:AI工具 🤖
                \item Week 3:图像预处理(深度) 🖼️
                \item Week 4:版面分析 📄
                \item Week 5:选择题识别 ⭕
            \end{itemize}
        \end{block}

        \vspace{0.3cm}
        \begin{alertblock}{重点提示}
        Week 2我们将学习如何用\textbf{AI工具}来加速Week 1学到的OpenCV代码开发!
        \end{alertblock}
    \end{columns}
\end{frame}

\begin{frame}{总结与问答}
    \begin{center}
        \Huge \textbf{Q \& A}
        \vspace{1cm}
        \Large 准备好进入计算机视觉的世界了吗?
    \end{center}
\end{frame}
