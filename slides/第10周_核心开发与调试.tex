\documentclass[aspectratio=169, 12pt]{beamer}
\usepackage[UTF8]{ctex}
\usepackage{graphicx}
\usepackage{booktabs}
\usepackage{listings}
\usepackage{xcolor}
\usepackage{tikz}
\usepackage{hyperref}

\usetheme{Madrid}
\usecolortheme{whale}
\usefonttheme{professionalfonts}

\lstset{
    language=Python,
    basicstyle=\ttfamily\small,
    keywordstyle=\color{blue},
    commentstyle=\color{green!60!black},
    stringstyle=\color{orange},
    breaklines=true,
    frame=single,
    showstringspaces=false,
    backgroundcolor=\color{gray!10}
}

\title[核心开发与调试]{第10周:核心开发与调试}
\subtitle{让系统真正跑起来}
\author{计算机视觉课程组}
\institute{通选课}
\date{}

\begin{document}

\begin{frame}
    \titlepage
\end{frame}

\begin{frame}{课程概览}
    \tableofcontents
\end{frame}

\section{主程序框架}

\begin{frame}[fragile]{主程序结构}
    \begin{lstlisting}[basicstyle=\ttfamily\scriptsize]
class AutoGradingSystem:
    """自动阅卷系统"""

    def __init__(self):
        # 初始化各模块
        self.preprocessor = Preprocessor()
        self.layout_analyzer = LayoutAnalyzer()
        self.choice_recognizer = ChoiceRecognizer()
        self.judge_recognizer = JudgeRecognizer()
        self.essay_recognizer = EssayRecognizer()
        self.grading_module = GradingModule()

    def process(self, image_path):
        """处理试卷"""
        # 1. 预处理
        preprocessed = self.preprocessor.process(image)

        # 2. 版面分析
        layout = self.layout_analyzer.analyze(image)

        # 3. 识别
        choices = self.choice_recognizer.recognize_all(...)
        judges = self.judge_recognizer.recognize_all(...)
        essays = self.essay_recognizer.recognize_all(...)

        # 4. 评分
        result = self.grading_module.grade(...)

        return result
    \end{lstlisting}
\end{frame}

\section{集成指导}

\begin{frame}{集成步骤}
    \begin{enumerate}
        \item \textbf{搭建框架}
            \begin{itemize}
                \item 创建项目结构
                \item 配置开发环境
                \item 编写基础代码
            \end{itemize}

        \item \textbf{模块开发}
            \begin{itemize}
                \item 从选择题开始
                \item 编写单元测试
                \item 调试优化
            \end{itemize}

        \item \textbf{模块集成}
            \begin{itemize}
                \item 整合各模块
                \item 测试整体流程
                \item 修复接口问题
            \end{itemize}

        \item \textbf{测试优化}
            \begin{itemize}
                \item 使用测试集验证
                \item 优化识别准确率
            \end{itemize}
    \end{enumerate}
\end{frame}

\section{测试策略}

\begin{frame}{测试集准备}
    \textbf{测试数据:}
    \begin{itemize}
        \item 基础选择题:单独测试选择题识别
        \item 基础判断题:单独测试判断题识别
        \item 完整试卷:测试整体流程
    \end{itemize}

    \vspace{0.3cm}

    \textbf{测试目标:}
    \begin{itemize}
        \item 选择题准确率 $>$80\%
        \item 判断题准确率 $>$80\%
        \item 简答题能提取文字
    \end{itemize}
\end{frame}

\begin{frame}{单元测试示例}
    \begin{lstlisting}[basicstyle=\ttfamily\scriptsize]
import unittest

class TestChoiceRecognizer(unittest.TestCase):
    def test_calculate_density(self):
        """测试密度计算"""
        test_img = create_test_image(density=0.5)
        density = self.recognizer.calculate_density(test_img)
        self.assertAlmostEqual(density, 0.5, places=1)

    def test_recognize_single(self):
        """测试单题识别"""
        result = self.recognizer.recognize_question(img, positions)
        self.assertIn(result, ['A', 'B', 'C', 'D', None])

if __name__ == '__main__':
    unittest.main()
    \end{lstlisting}
\end{frame}

\section{思考题}

\begin{frame}{课堂思考题}
    \begin{block}{问题1:集成调试}
        \begin{itemize}
            \item 模块集成时常见问题有哪些?
            \item 如何快速定位错误来源?
        \end{itemize}
    \end{block}

    \vspace{0.3cm}

    \begin{block}{问题2:性能优化}
        \begin{itemize}
            \item 如何提升识别准确率?
            \item 如何优化代码执行速度?
        \end{itemize}
    \end{block}
\end{frame}

\section{开发实践}

\begin{frame}{开发检查点}
    \textbf{检查点1:选择题模块完成}
    \begin{itemize}
        \item 能检测填涂
        \item 能识别A/B/C/D
        \item 输出格式正确
    \end{itemize}

    \vspace{0.2cm}

    \textbf{检查点2:判断题模块完成}
    \begin{itemize}
        \item 能检测$\checkmark$/$\times$
        \item 能区分正误
        \item 输出格式正确
    \end{itemize}

    \vspace{0.2cm}

    \textbf{检查点3:模块集成完成}
    \begin{itemize}
        \item 主程序能调用各模块
        \item 数据能正确流转
        \item 无致命错误
    \end{itemize}
\end{frame}

\section{常见问题}

\begin{frame}{常见问题与解决}
    \begin{table}
        \centering
        \small
        \begin{tabular}{lp{6cm}}
            \toprule
            \textbf{问题} & \textbf{解决方案} \\
            \midrule
            填涂识别不准 & 调整阈值,检查二值化 \\
            符号识别失败 & 检查轮廓,用模板匹配后备 \\
            OCR结果乱码 & 检查预处理,调整分辨率 \\
            模块导入错误 & 检查sys.path,确保\_\_init\_\_.py \\
            数组越界 & 使用safe\_crop,检查边界 \\
            \bottomrule
        \end{tabular}
    \end{table}
\end{frame}

\begin{frame}{本周任务}
    \begin{alertblock}{核心任务}
        \begin{enumerate}
            \item 完成所有模块开发
            \item 完成模块集成
            \item 通过基本功能测试
            \item 准备演示材料
        \end{enumerate}
    \end{alertblock}

    \vspace{0.2cm}

    \textbf{时间紧迫,务必确保系统基本可用!}
\end{frame}

\begin{frame}{下节预告}
    \begin{center}
        \Large \textbf{第11周:成果展示与总结}

        \vspace{0.5cm}

        \normalsize
        每组5分钟演示 + 2分钟答辩

        \vspace{0.3cm}

        \textbf{准备好你们的展示!}
    \end{center}
\end{frame}

\begin{frame}
    \begin{center}
        \Huge \textbf{加油!}

        \vspace{1cm}

        \Large 最后一周冲刺
    \end{center}
\end{frame}

\end{document}
