%===========================================================
% preamble.tex - Beamer 配置文件
%===========================================================

% 中文支持
\usepackage[UTF8]{ctex}

% 图形与表格
\usepackage{graphicx}
\usepackage{booktabs}

% 代码高亮
\usepackage{listings}
\lstset{
    language=Python,
    basicstyle=\ttfamily\small,
    keywordstyle=\color{blue},
    commentstyle=\color{green!60!black},
    stringstyle=\color{orange},
    breaklines=true,
    frame=single,
    showstringspaces=false,
    backgroundcolor=\color{gray!10}
}

% 颜色与图形
\usepackage{xcolor}
\usepackage{tikz}
\usetikzlibrary{shapes, arrows, positioning, calc}

% 数学公式
\usepackage{amsmath}
\usepackage{amssymb}

% 超链接
\usepackage{hyperref}

%===========================================================
% 主题设置
%===========================================================
\usetheme{Madrid}
\usecolortheme{whale}
\usefonttheme{professionalfonts}

% 页脚logo(缩小显示)
\logo{\includegraphics[height=0.8cm]{../xiaohui.png}}

%===========================================================
% 课程信息
%===========================================================
\title[判断题识别(符号匹配)]{第6周:判断题识别(符号匹配)}
\subtitle{怎么看到是$\checkmark$还是$\times$?}
\author{北京石油化工学院\textbackslash 人工智能研究院\textbackslash 王文通}
\institute{通选课}
\date{2025-2026 学年}

%===========================================================
% 自定义命令
%===========================================================
% 高亮命令
\newcommand{\highlight}[1]{\textcolor{red}{\textbf{#1}}}

% TODO标记
\newcommand{\todo}[1]{\textcolor{red}{\textbf{[TODO: #1]}}}

% AI提示标记
\newcommand{\aihint}{\textcolor{blue}{\textbf{[AI提示]}}}
