%===========================================================
% summary.tex - 总结与作业
%===========================================================
\section{总结}

\begin{frame}{本周知识要点}
    \textbf{判断题识别核心技术:}
    \begin{columns}
        \column{0.5\textwidth}
        \begin{block}{轮廓特征法}
            \begin{itemize}
                \item 面积、周长
                \item 圆度、凸性
                \item Hu矩
                \item 规则分类
            \end{itemize}
        \end{block}

        \column{0.5\textwidth}
        \begin{block}{模板匹配法}
            \begin{itemize}
                \item 相似度度量
                \item 多尺度匹配
                \item 模板库构建
                \item 阈值判断
            \end{itemize}
        \end{block}
    \end{columns}

    \vspace{0.5cm}

    \textbf{完整流程:}
    预处理 $\to$ 定位 $\to$ 提取 $\to$ 特征 $\to$ 分类 $\to$ 输出
\end{frame}

\begin{frame}{方法对比总结}
    \begin{table}
        \centering
        \small
        \begin{tabular}{l c c c}
        \toprule
        \textbf{方法} & \textbf{准确率} & \textbf{速度} & \textbf{适用场景} \\
        \midrule
        轮廓特征 & 中 & 快 & 规范符号 \\
        模板匹配 & 高 & 中 & 标准符号 \\
        机器学习 & 高 & 慢 & 手写符号 \\
        \bottomrule
        \end{tabular}
    \end{table}

    \vspace{0.3cm}

    \textbf{实践建议:}
    \begin{itemize}
        \item 先用特征法快速实现
        \item 效果不好再加模板匹配
        \item 复杂场景考虑机器学习
    \end{itemize}
\end{frame}

\section{课后作业}

\begin{frame}{课后作业}
    \begin{block}{题目:判断题符号识别模块}
        实现一个能够识别对号和错号的程序
    \end{block}

    \textbf{基础要求(60分):}
    \begin{itemize}
        \item 实现轮廓特征提取函数(20分)
        \item 实现基于规则的分类器(20分)
        \item 能够区分标准对号和错号(20分)
    \end{itemize}

    \textbf{进阶要求(30分):}
    \begin{itemize}
        \item 支持圆圈识别(10分)
        \item 实现模板匹配方法(10分)
        \item 输出识别置信度(10分)
    \end{itemize}

    \textbf{加分项(10分):}
    \begin{itemize}
        \item 可视化展示识别过程(10分)
    \end{itemize}
\end{frame}

\begin{frame}{作业提交要求}
    \textbf{提交内容:}
    \begin{enumerate}
        \item Python代码(.py文件)
        \item 测试结果截图
        \item 简要说明文档
    \end{enumerate}

    \vspace{0.3cm}

    \textbf{评分标准:}
    \begin{itemize}
        \item 代码规范性(20分)
        \item 功能完整性(40分)
        \item 识别准确率(30分)
        \item 文档与展示(10分)
    \end{itemize}

    \vspace{0.3cm}

    \textbf{截止时间:} 下周上课前
\end{frame}

\section{下节预告}

\begin{frame}{下节预告}
    \begin{center}
        \Large \textbf{第7周:OCR基础与文字识别}

        \vspace{0.5cm}

        \normalsize
        故事问题:\textcolor{blue}{怎么让机器"阅读"文字?}

        \vspace{0.5cm}

        \begin{columns}
            \column{0.5\textwidth}
            \textbf{你将学会:}
            \begin{itemize}
                \item OCR技术原理
                \item PaddleOCR使用
                \item 印刷文字识别
                \item 手写文字识别
            \end{itemize}

            \column{0.5\textwidth}
            \textbf{实践项目:}
            \begin{itemize}
                \item 学号识别
                \item 姓名识别
                \item 简答题识别
            \end{itemize}
        \end{columns}
    \end{center}
\end{frame}

\begin{frame}
    \begin{center}
        \Huge \textbf{谢谢!}

        \vspace{1cm}

        \Large \textbf{下节课见!}
    \end{center}
\end{frame}
