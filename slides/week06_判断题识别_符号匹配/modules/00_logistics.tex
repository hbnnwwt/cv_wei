%===========================================================
% 00_logistics.tex - 教学组织与准备
%===========================================================
\section{教学组织}

\begin{frame}{预备知识回顾}
    \textbf{上周内容:选择题填涂检测}
    \begin{itemize}
        \item OMR技术:基于像素密度判断填涂状态
        \item 形态学操作:腐蚀、膨胀、开闭运算
        \item 连通域分析:findContours与contourArea
        \item 填涂检测:阈值判断 + 位置定位
    \end{itemize}

    \vspace{0.5cm}

    \textbf{本周不同之处:}
    \begin{itemize}
        \item 选择题:\textbf{密度}判断(填涂 vs 空白)
        \item 判断题:\textbf{形状}判断(对号 vs 错号 vs 圆圈)
    \end{itemize}
\end{frame}

\begin{frame}{学习目标}
    \begin{block}{知识目标}
        \begin{itemize}
            \item 理解判断题识别与选择题识别的区别
            \item 掌握轮廓特征的提取与计算方法
            \item 理解模板匹配的基本原理
            \item 了解机器学习在符号识别中的应用
        \end{itemize}
    \end{block}

    \begin{block}{能力目标}
        \begin{itemize}
            \item 能够使用OpenCV提取符号的形状特征
            \item 能够实现基于特征的符号分类器
            \item 能够实现模板匹配算法
            \item 能够构建完整的判断题识别流程
        \end{itemize}
    \end{block}
\end{frame}

%===========================================================
% 根据审查报告新增的内容
%===========================================================

\begin{frame}{预备知识(课前5分钟视频)}
    \begin{columns}
        \column{0.5\textwidth}
        \textbf{相似度度量方法:}
        \begin{itemize}
            \item 欧氏距离(Euclidean Distance)
            \item 余弦相似度(Cosine Similarity)
            \item 归一化相关系数
        \end{itemize}

        \column{0.5\textwidth}
        \textbf{轮廓特征提取基础:}
        \begin{itemize}
            \item 轮廓查找的基本概念
            \item 轮廓层级结构
            \item 基础轮廓特征(面积、周长、长宽比)
        \end{itemize}
    \end{columns}

    \vspace{0.5cm}

    \begin{alertblock}{观看要求}
    请在课前观看预备知识视频,为本周学习做好准备
    \end{alertblock}
\end{frame}

\begin{frame}{分组策略与角色分工}
    \textbf{分组原则:}
    \begin{itemize}
        \item 每4人为一组
        \item 确保不同专业背景混合
        \item 建议包含:理工科、文科、无编程基础、有编程基础
    \end{itemize}

    \vspace{0.3cm}

    \textbf{角色分工:}
    \begin{table}
        \centering
        \small
        \begin{tabular}{lp{6cm}l}
            \toprule
            \textbf{角色} & \textbf{职责} & \textbf{适合} \\
            \midrule
            组长 & 统筹协调、进度管理 & 组织能力强的 \\
            算法实现者 & 实现轮廓特征、模板匹配 & 有编程基础的 \\
            特征调优者 & 调整圆度阈值、凸性阈值 & 细心负责的 \\
            测试者 & 收集测试用例、报告问题 & 细心负责的 \\
            \bottomrule
        \end{tabular}
    \end{table}

    \vspace{0.3cm}

    \begin{block}{本周协作任务}
        用轮廓特征或模板匹配实现判断题识别,区分对号和错号
    \end{block}
\end{frame}

\begin{frame}{并行学习路径}
    \textbf{观察者路径:}
    \begin{itemize}
        \item 理解判断题识别原理
        \item 看老师演示轮廓特征提取、模板匹配
        \item 完成基础任务:运行示例代码
    \end{itemize}

    \vspace{0.3cm}

    \textbf{使用者路径:}
    \begin{itemize}
        \item 使用示例代码处理自己的判断题图像
        \item 调整圆度阈值、凸性阈值
        \item 完成核心任务:识别对号和错号
    \end{itemize}

    \vspace{0.3cm}

    \textbf{创造者路径:}
    \begin{itemize}
        \item 设计自己的符号分类器
        \item 处理不同手写风格的符号
        \item 完成挑战任务:实现自适应符号识别
    \end{itemize}
\end{frame}

\begin{frame}{多屏协同设计}
    \textbf{双屏协作:}
    \begin{itemize}
        \item \textbf{主屏}: 显示PPT理论和讲解
        \item \textbf{侧屏}: 实时演示代码运行效果
        \item \textbf{移动设备}: 互动答题、查看代码
    \end{itemize}

    \vspace{0.5cm}

    \textbf{Live Coding演示流程:}
    \begin{enumerate}
        \item 教师展示问题需求
        \item 用AI生成代码框架
        \item 师生共同完善关键代码
        \item 实时运行验证效果
    \end{enumerate}

    \vspace{0.3cm}

    \begin{exampleblock}{互动方式}
        使用手机扫码参与实时投票,反馈理解情况
    \end{exampleblock}
\end{frame}
