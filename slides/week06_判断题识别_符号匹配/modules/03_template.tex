%===========================================================
% 03_template.tex - 模板匹配方法
%===========================================================
\section{模板匹配方法}

\begin{frame}{模板匹配原理}
    \textbf{基本思想:}
    \begin{itemize}
        \item 在图像中滑动模板窗口
        \item 计算每个位置的相似度
        \item 找到相似度最高的位置
    \end{itemize}

    \vspace{0.3cm}

    \textbf{工作流程:}
    \begin{enumerate}
        \item 准备标准符号模板图像
        \item 将模板在待识别图像上滑动
        \item 计算每个位置的匹配分数
        \item 选择分数最高的位置和符号类型
    \end{enumerate}
\end{frame}

\begin{frame}{相似度度量方法}
    \textbf{OpenCV提供6种匹配方法:}
    \begin{table}
        \centering
        \small
        \begin{tabular}{l|l}
        \toprule
        \textbf{方法} & \textbf{特点} \\
        \midrule
        TM\_SQDIFF & 差值平方和,越小越匹配 \\
        TM\_SQDIFF\_NORMED & 归一化差值平方和,范围[0,1] \\
        TM\_CCORR & 相关性,越大越匹配 \\
        TM\_CCORR\_NORMED & 归一化相关性,范围[0,1] \\
        TM\_CCOEFF & 相关系数,范围[-1,1] \\
        TM\_CCOEFF\_NORMED & 归一化相关系数,范围[-1,1] \\
        \bottomrule
        \end{tabular}
    \end{table}

    \begin{alertblock}{推荐使用}
    TM\_CCOEFF\_NORMED:1表示完美匹配,-1表示完全不匹配
    \end{alertblock}
\end{frame}

\begin{frame}{模板匹配演示}
    \begin{center}
    \begin{tikzpicture}[scale=0.8]
        % 待识别图像
        \draw[step=0.5, gray!30, thin] (0,0) grid (4,4);
        \draw[thick] (0,0) rectangle (4,4);
        \node at (2,-0.5) {待识别图像};

        % 模板窗口
        \draw[red, thick] (1,1) rectangle (2,2);
        \node[red] at (1.5,2.3) {模板};

        % 匹配结果示意
        \draw[->, thick, blue] (4.5,2) -- (5.5,2);
        \node at (6,2) {匹配分数};

        % 模板库
        \draw[step=0.5, gray!30, thin] (7,0) grid (8,1);
        \draw[thick] (7,0) rectangle (8,1);
        \node at (7.5,-0.3) {模板1: $\checkmark$};

        \draw[step=0.5, gray!30, thin] (7,1.5) grid (8,2.5);
        \draw[thick] (7,1.5) rectangle (8,2.5);
        \node at (7.5,1.2) {模板2: $\times$};

        \node at (7.5,3) {模板库};
    \end{tikzpicture}
    \end{center}
\end{frame}

\begin{frame}{模板匹配的优势与局限}
    \textbf{优势:}
    \begin{itemize}
        \item 简单直观,易于理解和实现
        \item 无需训练过程
        \item 识别速度快
    \end{itemize}

    \vspace{0.3cm}

    \textbf{局限性:}
    \begin{itemize}
        \item 对尺度变化敏感(需要resize)
        \item 对旋转敏感(需要多角度模板)
        \item 对形变敏感(手写符号变化大)
        \item 需要准备足够的模板
    \end{itemize}
\end{frame}

\begin{frame}{多尺度模板匹配}
    \textbf{问题:} 手写符号大小不一

    \textbf{解决方案:图像金字塔}
    \begin{itemize}
        \item 构建多尺度图像金字塔
        \item 在每个尺度上进行模板匹配
        \item 选择所有尺度中最佳匹配结果
    \end{itemize}

    \vspace{0.3cm}

    \textbf{实现思路:}
    \begin{enumerate}
        \item 生成不同尺度的图像(0.8倍递减)
        \item 对每个尺度执行模板匹配
        \item 记录最佳匹配的尺度、位置、分数
        \item 返回最高分数对应的符号类型
    \end{enumerate}
\end{frame}

\begin{frame}{模板库构建策略}
    \textbf{模板采集原则:}
    \begin{itemize}
        \item 覆盖不同书写风格
        \item 包含不同大小和角度
        \item 数量适中(每类10-20个)
    \end{itemize}

    \vspace{0.3cm}

    \textbf{模板标准化:}
    \begin{enumerate}
        \item 统一尺寸(如32$\times$32像素)
        \item 统一粗细(形态学处理)
        \item 去除噪声(滤波处理)
        \item 归一化颜色(转灰度、二值化)
    \end{enumerate}
\end{frame}
