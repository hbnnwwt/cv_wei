%===========================================================
% 09_cases.tex - 案例分析
%===========================================================
\section{案例分析}

\begin{frame}{案例1:标准符号识别}
    \textbf{场景:} 考试答题卡上的标准打印符号

    \vspace{0.3cm}

    \textbf{特点:}
    \begin{itemize}
        \item 符号清晰规范
        \item 位置准确
        \item 无噪声干扰
    \end{itemize}

    \vspace{0.3cm}

    \textbf{推荐方案:} 轮廓特征法

    \textbf{识别率:} 99\%+
\end{frame}

\begin{frame}{案例2:手写符号识别}
    \textbf{场景:} 学生手写的判断题符号

    \vspace{0.3cm}

    \textbf{挑战:}
    \begin{itemize}
        \item 符号大小不一
        \item 形状变化较大
        \item 可能有修改痕迹
    \end{itemize}

    \vspace{0.3cm}

    \textbf{推荐方案:}
    \begin{itemize}
        \item 优先:多尺度模板匹配
        \item 备选:SVM分类器
    \end{itemize}

    \textbf{识别率:} 85-95\%
\end{frame}

\begin{frame}{案例3:模糊符号处理}
    \textbf{场景:} 擦除修改后留下的模糊符号

    \vspace{0.3cm}

    \textbf{挑战:}
    \begin{itemize}
        \item 多个符号重叠
        \item 符号不完整
        \item 特征不明显
    \end{itemize}

    \vspace{0.3cm}

    \textbf{处理策略:}
    \begin{enumerate}
        \item 检测多个轮廓
        \item 对每个轮廓分别识别
        \item 根据置信度选择结果
        \item 低置信度标记为"人工审核"
    \end{enumerate}
\end{frame}

\begin{frame}{识别失败案例分析}
    \begin{columns}
        \column{0.5\textwidth}
        \begin{alertblock}{常见失败原因}
            \begin{itemize}
                \item 符号太轻(阈值问题)
                \item 符号被遮挡
                \item 多个符号重叠
                \item 扫描质量问题
            \end{itemize}
        \end{alertblock}

        \column{0.5\textwidth}
        \begin{block}{解决方案}
            \begin{itemize}
                \item 自适应阈值
                \item 形态学处理
                \item 多轮廓检测
                \item 预处理增强
            \end{itemize}
        \end{block}
    \end{columns}
\end{frame}
