%===========================================================
% 06_ai_guide.tex - AI辅助编程指导
%===========================================================
\section{AI辅助编程}

\begin{frame}{AI辅助编程:本周使用指南}
    \begin{block}{AI工具推荐}
        \begin{itemize}
            \item \textbf{Cursor}: 强烈推荐,AI辅助编程IDE
            \item \textbf{ChatGPT/Claude}: 通用AI编程助手
            \item \textbf{通义灵码}: 国内可用的AI编程工具
        \end{itemize}
    \end{block}

    \vspace{0.3cm}

    \textbf{本周AI辅助任务:}
    \begin{itemize}
        \item 用AI解释圆度和凸性的数学原理
        \item 用AI生成轮廓特征提取代码框架
        \item 用AI调试特征阈值参数
        \item 用AI优化模板匹配性能
    \end{itemize}
\end{frame}

\begin{frame}[fragile]{Prompt工程:RTF框架}
    \textbf{RTF框架示例:}

    \begin{exampleblock}{Prompt模板}
    \texttt{[角色] 你是OpenCV和图像处理专家 \\
    [任务] 帮我解释什么是Hu矩,以及它如何用于符号识别 \\
    [格式] 用通俗语言配合具体示例说明}
    \end{exampleblock}

    \vspace{0.3cm}

    \textbf{场景示例:}
    \begin{lstlisting}[basicstyle=\ttfamily\scriptsize]
# 示例1:理解概念
请解释圆度的概念,并用简单的比喻帮助我理解
为什么对号的圆度较低而圆圈的圆度较高。

# 示例2:代码生成
请用Python和OpenCV实现计算轮廓的圆度、凸性特征,
并解释每个参数的含义。
    \end{lstlisting}
\end{frame}

\begin{frame}[fragile]{AI辅助调试:问题诊断}
    \textbf{常见问题与AI求助方式:}

    \begin{alertblock}{问题:符号识别错误}
    \textbf{Prompt模板:}
    \texttt{我的判断题符号识别代码将对号误识别为错号。\\
    运行环境:Windows 11, Python 3.9, OpenCV 4.8\\
    特征值:圆度=0.65, 凸性=0.85\\
    代码:[粘贴关键代码片段]\\
    请帮我分析问题并提供解决方案。}
    \end{alertblock}

    \vspace{0.3cm}

    \textbf{AI调试三部曲(回顾Week 2):}
    \begin{enumerate}
        \item \textbf{问题定位}:向AI描述现象,请求可能原因
        \item \textbf{代码分析}:粘贴代码,请求AI检查bug
        \item \textbf{解决方案}:请AI提供修复代码和解释
    \end{enumerate}
\end{frame}

\begin{frame}[fragile]{AI辅助学习:代码脚手架使用}
    \begin{block}{代码脚手架说明}
        以下是带有TODO标记的代码框架,请使用AI助手完成TODO部分。
    \end{block}

    \begin{lstlisting}[basicstyle=\ttfamily\scriptsize]
def extract_features(contour):
    """提取轮廓特征"""
    features = {}

    # TODO: 使用AI助手完成以下代码
    # Prompt: 请用Python和OpenCV实现轮廓特征提取

    # TODO: 计算轮廓面积
    # 提示:使用cv2.contourArea()
    features['area'] = ______

    # TODO: 计算轮廓周长
    # 提示:使用cv2.arcLength(contour, True)
    features['perimeter'] = ______

    # TODO: 计算圆度
    # 提示:圆度公式 C = 4π × Area / Perimeter²
    features['circularity'] = ______

    return features
    \end{lstlisting}

    \vspace{0.2cm}

    \begin{center}
        \small \textbf{练习}:使用Cursor/Claude完成上述TODO,并解释每个步骤
    \end{center}
\end{frame}

\begin{frame}[fragile]{AI辅助学习:高级Prompt技巧}
    \textbf{场景1:优化模板匹配}
    \begin{exampleblock}{Prompt示例}
        我的模板匹配识别准确率不高,特别是对于手写差异较大的符号。
        当前方法:cv2.matchTemplate(roi, template, cv2.TM\_CCOEFF\_NORMED)
        问题:手写对号和标准模板的匹配度较低。
        请提供优化方案的实现代码。
    \end{exampleblock}

    \vspace{0.3cm}

    \textbf{场景2:设计自适应阈值}
    \begin{exampleblock}{Prompt示例}
        我想设计一个自适应阈值系统,能够根据不同的手写风格自动调整圆度和凸性的阈值。
        请提供设计思路和Python实现代码。
    \end{exampleblock}
\end{frame}
