%===========================================================
% 04_ml_methods.tex - 机器学习方法
%===========================================================
\section{机器学习方法}

\begin{frame}{为什么需要机器学习?}
    \textbf{特征法的局限:}
    \begin{itemize}
        \item 需要手工设计特征和规则
        \item 对复杂符号难以设计有效规则
        \item 泛化能力有限
    \end{itemize}

    \vspace{0.3cm}

    \textbf{机器学习的优势:}
    \begin{itemize}
        \item 自动学习特征组合
        \item 泛化能力强
        \item 可以处理更复杂的符号
    \end{itemize}
\end{frame}

\begin{frame}{传统机器学习方法}
    \textbf{1. KNN(K近邻分类器)}
    \begin{itemize}
        \item 简单直观
        \item 无需训练过程
        \item 适合小数据集
    \end{itemize}

    \textbf{2. SVM(支持向量机)}
    \begin{itemize}
        \item 适合高维特征
        \item 泛化能力强
        \item 对小样本效果好
    \end{itemize}

    \textbf{3. 决策树}
    \begin{itemize}
        \item 可解释性强
        \item 类似手工规则的自动化
        \item 适合特征工程
    \end{itemize}
\end{frame}

\begin{frame}{深度学习方法}
    \textbf{CNN(卷积神经网络):}
    \begin{itemize}
        \item 自动提取特征
        \item 准确率最高
        \item 需要大量训练数据
    \end{itemize}

    \vspace{0.3cm}

    \textbf{轻量级网络设计:}
    \begin{itemize}
        \item LeNet-5:经典小型CNN
        \item MobileNet:移动端优化
        \item 自定义浅层CNN
    \end{itemize}

    \vspace{0.3cm}

    \begin{alertblock}{实际应用建议}
    对于判断题识别,传统方法已足够。深度学习适合更复杂的符号识别场景。
    \end{alertblock}
\end{frame}

\begin{frame}{方法选择建议}
    \begin{table}
        \centering
        \small
        \begin{tabular}{l l l}
        \toprule
        \textbf{场景} & \textbf{推荐方法} & \textbf{原因} \\
        \midrule
        标准印刷符号 & 模板匹配 & 简单高效 \\
        手写规范符号 & 轮廓特征+规则 & 特征明显 \\
        手写多样符号 & SVM/KNN & 泛化能力强 \\
        复杂手写符号 & 深度学习 & 准确率最高 \\
        \bottomrule
        \end{tabular}
        \caption{不同场景的方法选择}
    \end{table}
\end{frame}
