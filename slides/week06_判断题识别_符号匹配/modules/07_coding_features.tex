%===========================================================
% 07_coding_features.tex - Live Coding: 特征提取实战
%===========================================================
\section{Live Coding: 特征提取}

\begin{frame}[fragile]{AI辅助编程:特征提取}
    \begin{block}{AI辅助提示}
        你可以使用Cursor、ChatGPT、Claude等AI工具来帮助你实现轮廓特征提取。

        \textbf{推荐Prompt:}
        \texttt{请用Python和OpenCV实现轮廓特征提取函数,并解释每个特征的含义。}
    \end{block}

    \vspace{0.3cm}

    \begin{lstlisting}[basicstyle=\ttfamily\scriptsize]
import cv2
import numpy as np

def extract_features(contour):
    """提取轮廓特征"""
    features = {}

    # TODO: 使用AI助手完成以下代码
    # 提示:使用cv2.contourArea()计算轮廓面积
    features['area'] = cv2.contourArea(contour)

    # TODO: 计算轮廓周长
    # 提示:使用cv2.arcLength(contour, True),True表示闭合轮廓
    features['perimeter'] = cv2.arcLength(contour, True)

    # TODO: 计算长宽比
    # 提示:先获取边界矩形 x,y,w,h = cv2.boundingRect(contour)
    x, y, w, h = cv2.boundingRect(contour)
    features['aspect_ratio'] = float(w) / h if h > 0 else 0

    # TODO: 计算圆度
    # 提示:公式 C = 4π × Area / Perimeter²
    features['circularity'] = 4 * np.pi * features['area'] / (features['perimeter'] ** 2) if features['perimeter'] > 0 else 0

    # TODO: 计算凸性
    # 提示:先获取凸包 hull = cv2.convexHull(contour)
    hull = cv2.convexHull(contour)
    hull_area = cv2.contourArea(hull)
    features['convexity'] = features['area'] / hull_area if hull_area > 0 else 0

    return features
    \end{lstlisting}
\end{frame}

\begin{frame}[fragile]{特征可视化}
    \textbf{任务:} 可视化展示特征提取结果

    \begin{lstlisting}[basicstyle=\ttfamily\scriptsize]
def visualize_features(image, contour, features):
    """可视化特征"""
    result = image.copy()

    # TODO: 绘制轮廓(绿色)
    # 提示:cv2.drawContours(图像, 轮廓列表, 轮廓索引, 颜色, 线宽)
    cv2.drawContours(result, [contour], -1, (0, 255, 0), 2)

    # TODO: 绘制边界矩形(蓝色)
    # 提示:cv2.rectangle(图像, 左上角, 右下角, 颜色, 线宽)
    x, y, w, h = cv2.boundingRect(contour)
    cv2.rectangle(result, (x, y), (x+w, y+h), (255, 0, 0), 2)

    # TODO: 绘制凸包(红色)
    # 提示:先获取凸包,再绘制
    hull = cv2.convexHull(contour)
    cv2.drawContours(result, [hull], -1, (0, 0, 255), 2)

    # TODO: 显示特征值文本
    # 提示:使用cv2.putText()在图像上添加文字
    text = f"C={features['circularity']:.2f} Conv={features['convexity']:.2f}"
    cv2.putText(result, text, (x, y-10), cv2.FONT_HERSHEY_SIMPLEX, 0.5, (0, 0, 0), 2)

    return result
    \end{lstlisting}
\end{frame}

\begin{frame}[fragile]{基于特征的分类器}
    \textbf{任务:} 实现基于特征的符号分类器

    \begin{lstlisting}[basicstyle=\ttfamily\scriptsize]
def classify_by_features(features):
    """基于特征的符号分类"""
    # TODO: 提取特征值
    circularity = features['circularity']
    convexity = features['convexity']

    # TODO: 实现分类规则
    # 规则1:圆度接近1 -> 圆圈
    if circularity > 0.8:
        return 'circle'

    # TODO: 添加凸性判断规则
    # 提示:对号的凸性明显小于1(因为有凹陷)
    if convexity < 0.85:
        return 'check'

    # TODO: 添加错号判断规则
    # 提示:错号的圆度较低
    if circularity < 0.5:
        return 'cross'

    # 默认返回未知
    return 'unknown'

# 测试代码
# TODO: 运行并测试你的分类器
features = extract_features(contour)
symbol_type = classify_by_features(features)
print(f"识别结果: {symbol_type}")
    \end{lstlisting}
\end{frame}
