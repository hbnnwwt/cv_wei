%===========================================================
% 10_quiz.tex - 课堂Quiz
%===========================================================
\section{课堂Quiz}

\begin{frame}{课堂Quiz}
    \textbf{问题1:} 判断题识别和选择题识别的核心区别是什么?

    \begin{itemize}
        \item[A] 处理速度不同
        \item[B] 密度 vs 形状
        \item[C] 图像大小不同
        \item[D] 算法复杂度不同
    \end{itemize}

    \pause

    \textbf{答案:} B\\
    选择题关注填涂密度,判断题关注符号形状
\end{frame}

\begin{frame}{课堂Quiz}
    \textbf{问题2:} 圆度(Circularity)公式正确的是?

    \begin{itemize}
        \item[A] $C = \frac{Perimeter^2}{4\pi \times Area}$
        \item[B] $C = \frac{4\pi \times Area}{Perimeter^2}$
        \item[C] $C = \frac{Area}{Perimeter}$
        \item[D] $C = \frac{Perimeter}{Area}$
    \end{itemize}

    \pause

    \textbf{答案:} B\\
    圆形的圆度=1,其他形状<1
\end{frame}

\begin{frame}{课堂Quiz}
    \textbf{问题3:} 凸性为什么可以区分$\checkmark$和$\times$?

    \pause

    \textbf{答案:}
    \begin{itemize}
        \item $\checkmark$有明显凹陷(checkmark的钩)
        \item $\times$近似凸(两直线交叉)
        \item 凸性 = 轮廓面积/凸包面积
        \item $\checkmark$的凸性明显小于1,$\times$的凸性接近1
    \end{itemize}
\end{frame}

\begin{frame}{课堂Quiz}
    \textbf{问题4:} 模板匹配的主要缺点是什么?

    \pause

    \textbf{答案:}
    \begin{itemize}
        \item 对尺度变化敏感
        \item 对旋转敏感
        \item 对形变敏感(手写差异)
        \item 需要准备足够多的模板
    \end{itemize}
\end{frame}
