%===========================================================
% 02_features.tex - 轮廓特征提取理论
%===========================================================
\section{轮廓特征提取}

\begin{frame}{为什么选择轮廓特征?}
    \textbf{轮廓(Contour)的定义:}
    \begin{itemize}
        \item 连接所有连续边界点的曲线
        \item 具有相同的颜色或强度
        \item 是形状分析的基础
    \end{itemize}

    \vspace{0.3cm}

    \textbf{轮廓特征的优点:}
    \begin{itemize}
        \item 直观反映形状的几何特性
        \item 对光照变化相对鲁棒
        \item 计算效率高
        \item OpenCV提供完善的API支持
    \end{itemize}
\end{frame}

\begin{frame}{基础轮廓特征}
    \begin{table}
        \centering
        \small
        \begin{tabular}{l|l|l}
        \toprule
        \textbf{特征} & \textbf{物理意义} & \textbf{OpenCV函数} \\
        \midrule
        面积(Area) & 轮廓所围区域大小 & cv2.contourArea() \\
        周长(Perimeter) & 轮廓边界长度 & cv2.arcLength() \\
        边界矩形 & 包围轮廓的最小矩形 & cv2.boundingRect() \\
        最小外接矩形 & 旋转的最小矩形 & cv2.minAreaRect() \\
        最小外接圆 & 包围轮廓的最小圆 & cv2.minEnclosingCircle() \\
        凸包 & 包围轮廓的凸多边形 & cv2.convexHull() \\
        \bottomrule
        \end{tabular}
    \end{table}
\end{frame}

\begin{frame}{长宽比与占空比}
    \textbf{1. 长宽比(Aspect Ratio)}
    $$AR = \frac{Width}{Height}$$

    \begin{itemize}
        \item 细长形状:AR较大或较小
        \item 方形/圆形:AR接近1
        \item 对号的AR通常大于错号的AR
    \end{itemize}

    \vspace{0.3cm}

    \textbf{2. 占空比(Extent)}
    $$Extent = \frac{ContourArea}{BoundingBoxArea}$$

    \begin{itemize}
        \item 反映轮廓填充边界矩形的程度
        \item 圆形:接近$\pi/4 \approx 0.785$
        \item 稀疏形状:值较小
    \end{itemize}
\end{frame}

\begin{frame}{圆度(Circularity)}
    \textbf{定义:}
    $$C = \frac{4\pi \times Area}{Perimeter^2}$$

    \textbf{物理意义:}
    \begin{itemize}
        \item 形状接近圆的程度
        \item 圆形的圆度 = 1(周长$2\pi r$,面积$\pi r^2$)
        \item 其他形状的圆度 < 1
    \end{itemize}

    \vspace{0.3cm}

    \textbf{判断题符号的圆度特征:}
    \begin{itemize}
        \item $\bigcirc$:接近1
        \item $\checkmark$:较低(开口形状,周长大)
        \item $\times$:更低(两线交叉,周长更大)
    \end{itemize}
\end{frame}

%===========================================================
% 三个理解层级:圆度概念
%===========================================================
\begin{frame}{三个理解层级:圆度概念}
    \begin{columns}
        \column{0.33\textwidth}
        \begin{block}{基础概念}
        \begin{itemize}
            \item 什么是圆度?
            \item 圆度的公式是什么?
            \item 为什么圆度能区分不同符号?
        \end{itemize}
        \end{block}

        \column{0.33\textwidth}
        \begin{block}{可视化演示}
        \begin{itemize}
            \item 对比对号、错号、圆圈的圆度值
            \item 观察圆度值的差异
            \item 调整阈值观察分类效果
        \end{itemize}
        \end{block}

        \column{0.33\textwidth}
        \begin{block}{扩展应用}
        \begin{itemize}
            \item 设计自适应圆度阈值
            \item 处理不同手写风格的符号
            \item 优化符号分类准确率
        \end{itemize}
        \end{block}
    \end{columns}

    \vspace{0.5cm}

    \begin{center}
    \small
    \textbf{理解层级建议}:观察者掌握基础概念,使用者完成可视化演示,创造者探索扩展应用
    \end{center}
\end{frame}

\begin{frame}{凸性(Convexity)}
    \textbf{凸包(Convex Hull):}
    \begin{itemize}
        \item 包围轮廓的最小凸多边形
        \item 类似"橡皮筋"包裹形状
    \end{itemize}

    \vspace{0.3cm}

    \textbf{凸性定义:}
    $$Convexity = \frac{ContourArea}{ConvexHullArea}$$

    \textbf{判断题符号的凸性特征:}
    \begin{itemize}
        \item $\bigcirc$:接近1(本身就是凸的)
        \item $\checkmark$:明显小于1(有凹陷)
        \item $\times$:接近1(近似凸的)
    \end{itemize}

    \vspace{0.3cm}

    \begin{exampleblock}{关键区分点}
    凸性可以有效区分对号和错号!
    \end{exampleblock}
\end{frame}

\begin{frame}[fragile]{Hu矩(Hu Moments)}
    \textbf{什么是矩?}
    \begin{itemize}
        \item 描述图像分布的统计特征
        \item 类似于物理学中的"矩"
        \item 具有旋转、缩放、平移不变性
    \end{itemize}

    \vspace{0.3cm}

    \textbf{Hu矩的特点:}
    \begin{itemize}
        \item 7个不变量
        \item 对形状变换高度鲁棒
        \item 适合识别不同角度/大小的符号
    \end{itemize}

    \vspace{0.3cm}

    \textbf{OpenCV使用:}
    \begin{lstlisting}[basicstyle=\ttfamily\scriptsize]
moments = cv2.moments(contour)
hu_moments = cv2.HuMoments(moments)
    \end{lstlisting}
\end{frame}

\begin{frame}{特征选择策略}
    \textbf{区分对号和错号的特征组合:}
    \begin{columns}
        \column{0.5\textwidth}
        \begin{block}{圆度优先}
            \begin{itemize}
                \item 对号:中等圆度
                \item 错号:低圆度
            \end{itemize}
        \end{block}

        \column{0.5\textwidth}
        \begin{block}{凸性辅助}
            \begin{itemize}
                \item 对号:明显凹陷
                \item 错号:近似凸
            \end{itemize}
        \end{block}
    \end{columns}

    \vspace{0.5cm}

    \textbf{特征工程建议:}
    \begin{itemize}
        \item 计算多个特征,构建特征向量
        \item 使用决策树或规则组合判断
        \item 通过实验确定最优特征组合和阈值
    \end{itemize}
\end{frame}
