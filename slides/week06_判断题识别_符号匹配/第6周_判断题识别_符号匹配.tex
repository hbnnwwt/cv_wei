\documentclass[aspectratio=169, 12pt]{beamer}
\usepackage[UTF8]{ctex}
\usepackage{graphicx}
\usepackage{booktabs}
\usepackage{listings}
\usepackage{xcolor}
\usepackage{tikz}
\usepackage{hyperref}
\usepackage{amssymb}

\usetheme{Madrid}
\usecolortheme{whale}
\usefonttheme{professionalfonts}

% 页脚logo(缩小显示)
\logo{\includegraphics[height=0.8cm]{../xiaohui.png}}

\lstset{
    language=Python,
    basicstyle=\ttfamily\small,
    keywordstyle=\color{blue},
    commentstyle=\color{green!60!black},
    stringstyle=\color{orange},
    breaklines=true,
    frame=single,
    showstringspaces=false,
    backgroundcolor=\color{gray!10}
}

\title[判断题识别(符号匹配)]{第6周:判断题识别(符号匹配)}
\subtitle{怎么看到是$\checkmark$还是$\times$?}
\author{北京石油化工学院\\人工智能研究院\\王文通}
\institute{通选课}
\date{2025-2026 学年}
\titlegraphic{
    \includegraphics[height=1.2cm]{../xiaohui.png}\hspace{2cm}
    \includegraphics[height=1.2cm]{../name.png}
}

\begin{document}

\begin{frame}
    \titlepage
\end{frame}

\begin{frame}{课程概览}
    \tableofcontents
\end{frame}

\section{判断题识别概述}

\begin{frame}{判断题的特点}
    \textbf{常见符号:}
    \begin{itemize}
        \item $\checkmark$(正确/对)
        \item $\times$(错误/错)
        \item $\sqrt{}$(正确)
        \item $\bigcirc$(正确)
    \end{itemize}

    \vspace{0.5cm}

    \textbf{与选择题的区别:}
    \begin{itemize}
        \item 选择题:关注\textbf{填涂密度}
        \item 判断题:关注\textbf{符号形状}
    \end{itemize}
\end{frame}

\begin{frame}{识别方案}
    \textbf{方案1:轮廓特征法}
    \begin{itemize}
        \item 提取符号轮廓
        \item 计算形状特征
        \item 根据特征判断符号类型
    \end{itemize}

    \vspace{0.3cm}

    \textbf{方案2:模板匹配法}
    \begin{itemize}
        \item 准备标准符号模板
        \item 与模板进行匹配
        \item 选择最佳匹配结果
    \end{itemize}

    \vspace{0.3cm}

    \textbf{推荐:先尝试特征法,效果不好再用模板法}
\end{frame}

\section{轮廓特征提取}

\begin{frame}{基础轮廓特征}
    \begin{table}
        \centering
        \begin{tabular}{lp{5cm}l}
            \toprule
            \textbf{特征} & \textbf{说明} & \textbf{OpenCV函数} \\
            \midrule
            面积 & 轮廓所围区域大小 & cv2.contourArea() \\
            周长 & 轮廓长度 & cv2.arcLength() \\
            长宽比 & 宽度/高度 & boundingRect \\
            占空比 & 面积/边界矩形面积 & area / (w*h) \\
            \bottomrule
        \end{tabular}
    \end{table}
\end{frame}

\begin{frame}{高级形状特征}
    \textbf{1. 圆度(Circularity)}
    $$C = \frac{4\pi \times Area}{Perimeter^2}$$

    \begin{itemize}
        \item 圆形:接近1
        \item $\checkmark$:较低(开口形状)
        \item $\times$:更低(两线交叉)
    \end{itemize}

    \vspace{0.3cm}

    \textbf{2. 凸性(Convexity)}
    $$Convexity = \frac{Area}{ConvexHullArea}$$

    \begin{itemize}
        \item $\bigcirc$(凸):接近1
        \item $\checkmark$(凹):小于1
    \end{itemize}
\end{frame}

\section{基于特征的分类}

\begin{frame}[fragile]{特征提取与分类}
    \begin{lstlisting}[basicstyle=\ttfamily\scriptsize]
def extract_features(contour):
    """提取轮廓特征"""
    features = {}

    # 基础特征
    features['area'] = cv2.contourArea(contour)
    features['perimeter'] = cv2.arcLength(contour, True)

    # 形状特征
    features['circularity'] = 4 * np.pi * features['area'] / (features['perimeter'] ** 2)

    # 凸性
    hull = cv2.convexHull(contour)
    hull_area = cv2.contourArea(hull)
    features['convexity'] = features['area'] / hull_area if hull_area > 0 else 0

    return features

def classify_symbol(features):
    """根据特征分类"""
    if features['circularity'] > 0.8:
        return 'circle'  # ○
    elif features['convexity'] < 0.8:
        return 'check'   # ✓
    else:
        return 'cross'   # ×
    \end{lstlisting}
\end{frame}

\section{模板匹配法}

\begin{frame}{模板匹配原理}
    \textbf{思想:} 在图像中滑动模板,计算相似度

    \vspace{0.3cm}

    \textbf{相似度度量:}
    \begin{itemize}
        \item TM\_CCOEFF\_NORMED:归一化相关系数
        \item 1 = 完美匹配
        \item -1 = 完全不匹配
    \end{itemize}

    \vspace{0.3cm}

    \begin{center}
        \begin{tikzpicture}
            \node[draw] (template) {模板};
            \node[draw, right=2cm of template] (image) {图像};
            \node[below=0.5cm of template] (slide) {滑动};
            \node[below=0.5cm of image] (match) {匹配};

            \draw[->, dashed] (template) -- (image);
            \draw[->] (slide) -- (match);
        \end{tikzpicture}
    \end{center}
\end{frame}

\begin{frame}[fragile]{模板匹配实现}
    \begin{lstlisting}[basicstyle=\ttfamily\scriptsize]
def match_template(roi, templates, threshold=0.7):
    """模板匹配"""
    best_match = None
    best_value = -float('inf')

    for symbol_type, template in templates.items():
        # 调整大小
        if template.shape != roi.shape:
            template = cv2.resize(template, (roi.shape[1], roi.shape[0]))

        # 模板匹配
        result = cv2.matchTemplate(roi, template, cv2.TM_CCOEFF_NORMED)
        match_value = result[0, 0]

        if match_value > best_value:
            best_value = match_value
            best_match = symbol_type

    return best_match if best_value >= threshold else 'unknown'
    \end{lstlisting}
\end{frame}

\section{思考题}

\begin{frame}{课堂思考题}
    \begin{block}{问题1:形状特征}
        \begin{itemize}
            \item 如何区分$\checkmark$和$\times$的形状?
            \item 凸性为什么能判断凹凸形状?
        \end{itemize}
    \end{block}

    \vspace{0.3cm}

    \begin{block}{问题2:模板匹配}
        \begin{itemize}
            \item 模板匹配适用于什么场景?
            \item 如果手写符号形变严重,怎么办?
        \end{itemize}
    \end{block}
\end{frame}

\section{课后作业}

\begin{frame}{课后作业}
    \begin{block}{题目}
        实现判断题符号识别模块
    \end{block}

    \textbf{要求:}
    \begin{enumerate}
        \item 实现轮廓特征提取
        \item 实现基于规则的分类器
        \item 识别$\checkmark$和$\times$符号
        \item 可视化标注识别结果
    \end{enumerate}

    \vspace{0.2cm}

    \textbf{评分标准:}
    \begin{itemize}
        \item 特征提取:30分
        \item 分类实现:35分
        \item 识别效果:25分
        \item 可视化:10分
    \end{itemize}
\end{frame}

\begin{frame}{下节预告}
    \begin{center}
        \Large \textbf{第7周:OCR基础与文字识别}

        \vspace{0.5cm}

        \normalsize
        故事问题:\textcolor{blue}{怎么让机器"阅读"文字?}

        \vspace{0.3cm}

        你将学会:
        \begin{itemize}
            \item OCR技术原理
            \item PaddleOCR使用
            \item 印刷文字识别
        \end{itemize}
    \end{center}
\end{frame}

\begin{frame}
    \begin{center}
        \Huge \textbf{谢谢!}
    \end{center}
\end{frame}

\end{document}
