\documentclass[aspectratio=169, 12pt]{beamer}
\usepackage[UTF8]{ctex}
\usepackage{graphicx}
\usepackage{booktabs}
\usepackage{listings}
\usepackage{xcolor}
\usepackage{tikz}
\usepackage{hyperref}

\usetheme{Madrid}
\usecolortheme{whale}
\usefonttheme{professionalfonts}

\lstset{
    language=Python,
    basicstyle=\ttfamily\small,
    keywordstyle=\color{blue},
    commentstyle=\color{green!60!black},
    stringstyle=\color{orange},
    breaklines=true,
    frame=single,
    showstringspaces=false,
    backgroundcolor=\color{gray!10}
}

\title[AI辅助编程工具实战]{第2周:AI辅助编程工具实战}
\subtitle{怎么让AI帮我写代码?}
\author{计算机视觉课程组}
\institute{通选课}
\date{}

\begin{document}

\begin{frame}
    \titlepage
\end{frame}

\begin{frame}{课程概览}
    \tableofcontents
\end{frame}

\section{为什么需要AI辅助编程}

\begin{frame}{传统编程的痛点}
    \begin{columns}
        \column{0.5\textwidth}
        \begin{alertblock}{常见问题}
            \begin{itemize}
                \item API参数复杂,记不住
                \item 报错信息看不懂
                \item 算法原理理解困难
                \item 编程基础薄弱
            \end{itemize}
        \end{alertblock}

        \column{0.5\textwidth}
        \begin{exampleblock}{AI辅助的优势}
            \begin{itemize}
                \item 快速生成代码框架
                \item 解释错误原因
                \item 提供优化建议
                \item 降低学习门槛
            \end{itemize}
        \end{exampleblock}
    \end{columns}
\end{frame}

\section{主流AI工具对比}

\begin{frame}{主流AI编程工具}
    \begin{table}
        \centering
        \begin{tabular}{lp{5cm}l}
            \toprule
            \textbf{工具} & \textbf{特点} & \textbf{适用场景} \\
            \midrule
            ChatGPT & 对话能力强,代码生成准确 & 学习、调试、解释 \\
            Claude & 代码分析深入,长文本处理好 & 代码审查、架构设计 \\
            GitHub Copilot & IDE集成,实时补全 & 日常编码 \\
            通义千问 & 中文友好,国内可用 & 中文问题咨询 \\
            \bottomrule
        \end{tabular}
    \end{table}

    \vspace{0.3cm}

    \textbf{选择建议:}
    \begin{itemize}
        \item 学习理解:ChatGPT/Claude
        \item 实时编码:GitHub Copilot/Cursor
        \item 国内使用:通义千问/DeepSeek
    \end{itemize}
\end{frame}

\section{Prompt工程实战}

\begin{frame}{什么是Prompt?}
    \begin{block}{定义}
        Prompt是给AI的指令或提示词
    \end{block}

    \vspace{0.3cm}

    \textbf{好Prompt的标准:}
    \begin{enumerate}
        \item \textbf{具体明确}:不说模糊的话
        \item \textbf{有上下文}:提供足够的背景信息
        \item \textbf{有约束}:明确输出格式要求
    \end{enumerate}

    \vspace{0.3cm}

    \begin{center}
        \begin{tikzpicture}
            \node[draw, rounded corners, fill=blue!10] (vague) {模糊:"帮我写代码"};
            \node[draw, rounded corners, fill=green!10, right=2cm of vague] (clear) {具体:"用OpenCV实现人脸检测"};
            \draw[->, thick] (vague) -- node[above]{\textcolor{red}{不要}} (clear);
        \end{tikzpicture}
    \end{center}
\end{frame}

\begin{frame}{CV领域专用Prompt模板}
    \textbf{模板1:代码生成}
    \begin{exampleblock}{}
        请用Python和OpenCV实现以下功能:\\
        功能描述:[详细描述要实现的功能]\\
        输入:[描述输入数据格式]\\
        输出:[描述期望的输出格式]\\
        要求:
        \begin{itemize}
            \item 使用OpenCV库
            \item 代码有详细注释
            \item 包含使用示例
        \end{itemize}
    \end{exampleblock}
\end{frame}

\begin{frame}{CV领域专用Prompt模板(续)}
    \textbf{模板2:代码解释}
    \begin{exampleblock}{}
        请解释以下OpenCV代码的含义:\\
        代码:[粘贴代码]\\
        要求:
        \begin{itemize}
            \item 逐行解释代码功能
            \item 说明关键参数的作用
            \item 指出可能的错误用法
        \end{itemize}
    \end{exampleblock}

    \vspace{0.2cm}

    \textbf{模板3:调试求助}
    \begin{exampleblock}{}
        我的OpenCV代码运行出错,请帮我分析:\\
        代码:[粘贴代码]\\
        错误信息:[粘贴报错信息]\\
        预期行为:[描述期望的结果]
    \end{exampleblock}
\end{frame}

\section{实战:人脸检测}

\begin{frame}[fragile]{实战任务:用AI辅助实现人脸检测}
    \textbf{Prompt示例:}
    \begin{verbatim}
请用Python和OpenCV实现一个人脸检测程序:
功能:从图片中检测所有人脸,并用矩形框标注
输入:图片文件路径
输出:标注了人脸框的图片
要求:
- 使用OpenCV的Haar级联分类器
- 在每个人脸周围绘制绿色矩形框
- 显示检测到的人脸数量
- 代码有详细中文注释
    \end{verbatim}

    \vspace{0.2cm}

    \textbf{预期输出:完整的可运行代码}
\end{frame}

\begin{frame}[fragile]{AI生成的代码示例}
    \begin{lstlisting}[basicstyle=\ttfamily\scriptsize]
import cv2

def detect_faces(image_path):
    """检测图像中的人脸并标注"""
    # 加载预训练模型
    face_cascade = cv2.CascadeClassifier(
        cv2.data.haarcascades + 'haarcascade_frontalface_default.xml'
    )

    # 读取图像
    img = cv2.imread(image_path)
    gray = cv2.cvtColor(img, cv2.COLOR_BGR2GRAY)

    # 检测人脸
    faces = face_cascade.detectMultiScale(
        gray, scaleFactor=1.1, minNeighbors=5
    )

    # 绘制检测框
    for (x, y, w, h) in faces:
        cv2.rectangle(img, (x, y), (x+w, y+h), (0, 255, 0), 2)

    print(f"检测到 {len(faces)} 个人脸")
    return img

result = detect_faces('test.jpg')
    \end{lstlisting}
\end{frame}

\section{代码调试技巧}

\begin{frame}{AI辅助调试流程}
    \begin{enumerate}
        \item \textbf{复制错误信息}
            \begin{itemize}
                \item 完整复制报错的traceback
                \item 注意错误类型和行号
            \end{itemize}

        \item \textbf{准备上下文}
            \begin{itemize}
                \item 粘贴相关代码片段
                \item 说明输入数据
            \end{itemize}

        \item \textbf{向AI提问}
            \begin{itemize}
                \item 使用模板化Prompt
                \item 明确预期vs实际行为
            \end{itemize}

        \item \textbf{验证解决方案}
            \begin{itemize}
                \item 不要盲目信任AI
                \item 理解修改原理
                \item 测试修改效果
            \end{itemize}
    \end{enumerate}
\end{frame}

\begin{frame}{常见错误案例}
    \begin{table}
        \centering
        \small
        \begin{tabular}{lp{6cm}l}
            \toprule
            \textbf{错误} & \textbf{原因} & \textbf{AI建议} \\
            \midrule
            img = None & 图片路径不存在 & 检查路径,用os.path.exists验证 \\
            \midrule
            颜色异常 & OpenCV是BGR,matplotlib是RGB & 用cv2.cvtColor转换 \\
            \midrule
            数据溢出 & uint8范围是0-255 & 用np.clip或cv2.add \\
            \bottomrule
        \end{tabular}
    \end{table}
\end{frame}

\section{思考题}

\begin{frame}{课堂思考题}
    \begin{block}{问题1:AI工具的使用}
        \begin{itemize}
            \item 在什么情况下适合使用AI辅助编程?
            \item AI生成的代码一定正确吗?如何验证?
        \end{itemize}
    \end{block}

    \vspace{0.3cm}

    \begin{block}{问题2:Prompt工程}
        \begin{itemize}
            \item 什么样的Prompt是好的Prompt?
            \item 如何向AI描述一个模糊的错误信息?
        \end{itemize}
    \end{block}
\end{frame}

\section{课后作业}

\begin{frame}{课后作业}
    \begin{block}{题目}
        用AI辅助实现答题卡边界检测程序
    \end{block}

    \textbf{项目关联:} 这是自动阅卷系统的第一步 —— 定位答题卡在图像中的位置

    \textbf{要求:}
    \begin{enumerate}
        \item 用Prompt向AI询问答题卡边界检测的实现方法
        \item 记录完整的AI对话过程(至少3轮交互)
        \item 运行并测试代码(可使用试卷图片)
        \item 撰写反思报告
    \end{enumerate}

    \vspace{0.2cm}

    \textbf{提交内容:}
    \begin{itemize}
        \item AI对话记录(截图或复制文本)
        \item 最终代码
        \item 测试结果图片(标注出检测到的边界)
        \item 反思报告
    \end{itemize}
\end{frame}

\begin{frame}{下节预告}
    \begin{center}
        \Large \textbf{第3周:图像预处理与增强}

        \vspace{0.5cm}

        \normalsize
        故事问题:\textcolor{blue}{试卷拍照模糊怎么办?}

        \vspace{0.3cm}

        你将学会:
        \begin{itemize}
            \item 图像去噪(高斯/中值滤波)
            \item 图像二值化
            \item 透视矫正
        \end{itemize}
    \end{center}
\end{frame}

\begin{frame}
    \begin{center}
        \Huge \textbf{谢谢!}

        \vspace{1cm}

        \Large 有问题随时交流
    \end{center}
\end{frame}

\end{document}
