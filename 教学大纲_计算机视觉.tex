\documentclass[12pt,a4paper]{article}
\usepackage[UTF8]{ctex}
\usepackage{geometry}
\usepackage{amsmath}
\usepackage{graphicx}
\usepackage{booktabs}
\usepackage{array}
\usepackage{enumitem}
\usepackage{xcolor}
\usepackage{tcolorbox}
\usepackage{longtable}
\usepackage{multirow}
\usepackage{hyperref}

\geometry{left=2.5cm,right=2.5cm,top=2.5cm,bottom=2.5cm}

\hypersetup{
    colorlinks=true,
    linkcolor=blue,
    filecolor=magenta,
    urlcolor=cyan,
}

\title{\textbf{\huge 计算机视觉通选课教学大纲}}
\author{}
\date{}

\begin{document}

\maketitle
\tableofcontents
\newpage

% ============================================================================
% 第一部分:课程定位
% ============================================================================
\section{课程定位}

\begin{table}[htbp]
\centering
\renewcommand{\arraystretch}{1.5}
\begin{tabular}{|l|p{10cm}|}
\hline
\textbf{属性} & \textbf{说明} \\
\hline
课程性质 & 通选课,面向多专业本科生 \\
\hline
学时安排 & 32学时(前10周$\times$3学时 + 第11周$\times$2学时) \\
\hline
设计理念 & OBE成果导向 + 学生中心 + 项目驱动 \\
\hline
最终项目 & \textbf{自动阅卷系统}(支持选择题、判断题、简答题) \\
\hline
\end{tabular}
\end{table}

% ============================================================================
% 第二部分:OBE能力矩阵
% ============================================================================
\section{OBE能力矩阵}

\begin{table}[htbp]
\centering
\renewcommand{\arraystretch}{1.5}
\begin{tabular}{|l|p{6cm}|p{4cm}|}
\hline
\textbf{能力维度} & \textbf{具体描述} & \textbf{可验证产出} \\
\hline
A1-图像处理 & 能对图像进行预处理、增强、分割 & 每周实验代码 \\
\hline
A2-版面分析 & 能定位试卷中各题型区域 & 第4周作业 \\
\hline
A3-特征识别 & 能识别填涂、符号、手写文字 & 第5-8周作业 \\
\hline
A4-系统集成 & 能将各模块整合为完整系统 & 最终项目 \\
\hline
A5-AI协作 & 能用AI工具辅助编程开发 & 全过程体现 \\
\hline
\end{tabular}
\end{table}

% ============================================================================
% 第三部分:课程故事线
% ============================================================================
\section{课程故事线}

本课程以\textbf{“造一个能改卷子的AI助教”}为主线,通过11周的渐进式学习,让学生从零开始构建一个完整的自动阅卷系统。

\begin{center}
\begin{tikzpicture}[scale=0.9, every node/.style={transform shape}]
\node[draw, rectangle, minimum width=3cm, minimum height=1cm, fill=blue!20] (goal) at (0,3) {\textbf{目标:AI阅卷助教}};

\node[draw, rectangle, minimum width=2.5cm, minimum height=0.8cm, fill=green!20] (phase1) at (-4,1) {\textbf{第1-2周}\\装上眼睛+AI武器};
\node[draw, rectangle, minimum width=2.5cm, minimum height=0.8cm, fill=yellow!20] (phase2) at (0,1) {\textbf{第3-8周}\\学会识别三种题型};
\node[draw, rectangle, minimum width=2.5cm, minimum height=0.8cm, fill=red!20] (phase3) at (4,1) {\textbf{第9-11周}\\成为助教改完整卷};

\draw[->, thick] (goal) -- (phase1);
\draw[->, thick] (goal) -- (phase2);
\draw[->, thick] (goal) -- (phase3);
\end{tikzpicture}
\end{center}

\textbf{故事线章节:}
\begin{enumerate}[label=第\chinese*章:]
\item \textbf{装上眼睛} —— 机器怎么看见试卷?(第1-2周)
\item \textbf{让试卷更清晰} —— 拍照模糊怎么办?(第3周)
\item \textbf{找到题目位置} —— 选择题、简答题在哪里?(第4周)
\item \textbf{识别填涂答案} —— 怎么知道选了A还是B?(第5周)
\item \textbf{识别对错符号} —— 怎么看到是$\checkmark$还是$\times$?(第6周)
\item \textbf{阅读手写文字} —— 能看懂学生写的答案吗?(第7-8周)
\item \textbf{整合成AI助教} —— 让它改完一整张卷子(第9-11周)
\end{enumerate}

\newpage

% ============================================================================
% 第四部分:详细周次设计
% ============================================================================
\section{详细周次设计(32学时)}

% ============================================================================
% 第一阶段:基础准备
% ============================================================================
\subsection{第一阶段:基础准备(第1-2周,6学时)}

% ---------------- 第1周 ----------------
\subsubsection{第1周:计算机视觉导论与图像基础}

\begin{tcolorbox}[colback=blue!5!white,colframe=blue!75!black,title=\textbf{故事问题:机器是怎么“看见”试卷的?}]
\end{tcolorbox}

\begin{table}[htbp]
\centering
\renewcommand{\arraystretch}{1.4}
\begin{tabular}{|c|p{5cm}|p{3cm}|p{2.5cm}|}
\hline
\textbf{时间} & \textbf{内容} & \textbf{活动} & \textbf{产出} \\
\hline
40min & CV导论:从人脸识别到阅卷系统 & 案例展示 & 理解课程目标 \\
\hline
50min & 图像的数字表示:像素$\to$RGB$\to$矩阵 & Jupyter实验 & 能用代码显示图像 \\
\hline
40min & OpenCV入门:读取、显示、保存 & 跟着做 & 读取试卷图片 \\
\hline
30min & 实验:给试卷加滤镜 & 动手实践 & 提交实验结果 \\
\hline
\end{tabular}
\end{table}

\textbf{课后作业:} 用OpenCV实现3种图像滤镜效果

% ---------------- 第2周 ----------------
\subsubsection{第2周:AI辅助编程工具实战}

\begin{tcolorbox}[colback=blue!5!white,colframe=blue!75!black,title=\textbf{故事问题:怎么让AI帮我写代码?}]
\end{tcolorbox}

\begin{table}[htbp]
\centering
\renewcommand{\arraystretch}{1.4}
\begin{tabular}{|c|p{5cm}|p{3cm}|p{2.5cm}|}
\hline
\textbf{时间} & \textbf{内容} & \textbf{活动} & \textbf{产出} \\
\hline
30min & AI编程工具介绍 & 演示 & 了解工具 \\
\hline
40min & Prompt工程:CV领域专用模板 & 互动练习 & 能写有效Prompt \\
\hline
50min & 实战:用AI辅助实现人脸检测 & 分组实践 & 跑通第一个CV项目 \\
\hline
40min & 代码调试与优化技巧 & 实战演示 & 学会调试方法 \\
\hline
\end{tabular}
\end{table}

\textbf{课后作业:} 用AI辅助实现一个手势识别程序

\newpage

% ============================================================================
% 第二阶段:核心技能
% ============================================================================
\subsection{第二阶段:核心技能(第3-8周,18学时)}

% ---------------- 第3周 ----------------
\subsubsection{第3周:图像预处理与增强}

\begin{tcolorbox}[colback=yellow!5!white,colframe=yellow!75!black,title=\textbf{故事问题:试卷拍照模糊怎么办?}]
\end{tcolorbox}

\begin{table}[htbp]
\centering
\renewcommand{\arraystretch}{1.4}
\begin{tabular}{|c|p{5cm}|p{3.5cm}|p{2.5cm}|}
\hline
\textbf{时间} & \textbf{内容} & \textbf{项目贡献} & \textbf{产出} \\
\hline
40min & 图像去噪(高斯/中值滤波) & 清理试卷噪点 & \\
\hline
50min & 二值化与阈值分割 & 分离填涂与背景 & \\
\hline
40min & 透视矫正(修正拍照角度) & 矫正倾斜试卷 & \\
\hline
30min & 实验:预处理流水线 & 动手实践 & 提交对比效果图 \\
\hline
\end{tabular}
\end{table}

\textbf{课后作业:} 实现试卷图像预处理完整流程

% ---------------- 第4周 ----------------
\subsubsection{第4周:试卷版面分析}

\begin{tcolorbox}[colback=yellow!5!white,colframe=yellow!75!black,title=\textbf{故事问题:怎么知道选择题、简答题在哪里?}]
\end{tcolorbox}

\begin{table}[htbp]
\centering
\renewcommand{\arraystretch}{1.4}
\begin{tabular}{|c|p{5cm}|p{3.5cm}|p{2.5cm}|}
\hline
\textbf{时间} & \textbf{内容} & \textbf{项目贡献} & \textbf{产出} \\
\hline
40min & 边缘检测与轮廓查找 & 找到题目边界 & \\
\hline
50min & 连通域分析与区域分割 & 分隔各题型区域 & \\
\hline
40min & 文本行检测(PaddleLayout) & 定位简答题位置 & \\
\hline
30min & 实验:版面结构可视化 & 动手实践 & 提交标注结果 \\
\hline
\end{tabular}
\end{table}

\textbf{课后作业:} 实现试卷版面分析,标注三种题型区域

% ---------------- 第5周 ----------------
\subsubsection{第5周:选择题识别(填涂检测)}

\begin{tcolorbox}[colback=yellow!5!white,colframe=yellow!75!black,title=\textbf{故事问题:怎么知道选了A还是B?}]
\end{tcolorbox}

\begin{table}[htbp]
\centering
\renewcommand{\arraystretch}{1.4}
\begin{tabular}{|c|p{5cm}|p{3.5cm}|p{2.5cm}|}
\hline
\textbf{时间} & \textbf{内容} & \textbf{项目贡献} & \textbf{产出} \\
\hline
30min & OMR原理:光学标记识别 & 理解填涂检测 & \\
\hline
60min & 像素密度统计算法 & 核心识别逻辑 & \\
\hline
50min & 多选项处理(A/B/C/D) & 完整选择题识别 & \\
\hline
20min & 实验:识别一张选择题答卷 & 动手实践 & 提交识别结果 \\
\hline
\end{tabular}
\end{table}

\textbf{课后作业:} 实现选择题填涂识别模块

\newpage

% ---------------- 第6周 ----------------
\subsubsection{第6周:判断题识别(符号匹配)}

\begin{tcolorbox}[colback=yellow!5!white,colframe=yellow!75!black,title=\textbf{故事问题:怎么看到是$\checkmark$还是$\times$?}]
\end{tcolorbox}

\begin{table}[htbp]
\centering
\renewcommand{\arraystretch}{1.4}
\begin{tabular}{|c|p{5cm}|p{3.5cm}|p{2.5cm}|}
\hline
\textbf{时间} & \textbf{内容} & \textbf{项目贡献} & \textbf{产出} \\
\hline
40min & 形状匹配算法原理 & 理解符号识别 & \\
\hline
50min & 轮廓特征提取(圆度、凸性) & 区分$\checkmark$和$\times$ & \\
\hline
40min & 模板匹配实战 & 判断题识别 & \\
\hline
30min & 实验:识别判断题答案 & 动手实践 & 提交识别结果 \\
\hline
\end{tabular}
\end{table}

\textbf{课后作业:} 实现判断题符号识别模块

% ---------------- 第7周 ----------------
\subsubsection{第7周:OCR基础与文字识别}

\begin{tcolorbox}[colback=yellow!5!white,colframe=yellow!75!black,title=\textbf{故事问题:怎么让机器“阅读”文字?}]
\end{tcolorbox}

\begin{table}[htbp]
\centering
\renewcommand{\arraystretch}{1.4}
\begin{tabular}{|c|p{5cm}|p{3.5cm}|p{2.5cm}|}
\hline
\textbf{时间} & \textbf{内容} & \textbf{项目贡献} & \textbf{产出} \\
\hline
40min & OCR技术原理与发展 & 理解文字识别 & \\
\hline
60min & PaddleOCR快速上手 & 中文识别工具 & \\
\hline
40min & 印刷体文字识别实战 & 识别试卷题号 & \\
\hline
20min & 实验:识别试卷标题 & 动手实践 & 提交识别结果 \\
\hline
\end{tabular}
\end{table}

\textbf{课后作业:} 用OCR识别试卷中的印刷文字

% ---------------- 第8周 ----------------
\subsubsection{第8周:手写简答题识别}

\begin{tcolorbox}[colback=yellow!5!white,colframe=yellow!75!black,title=\textbf{故事问题:能看懂学生写的答案吗?}]
\end{tcolorbox}

\begin{table}[htbp]
\centering
\renewcommand{\arraystretch}{1.4}
\begin{tabular}{|c|p{5cm}|p{3.5cm}|p{2.5cm}|}
\hline
\textbf{时间} & \textbf{内容} & \textbf{项目贡献} & \textbf{产出} \\
\hline
40min & 手写识别的挑战 & 理解技术难点 & \\
\hline
60min & 端到端手写识别(TrOCR) & 手写文字识别 & \\
\hline
40min & 识别结果后处理 & 优化识别效果 & \\
\hline
20min & 实验:识别手写简答 & 动手实践 & 提交识别结果 \\
\hline
\end{tabular}
\end{table}

\textbf{课后作业:} 实现简答题手写识别模块

\newpage

% ============================================================================
% 第三阶段:项目实战
% ============================================================================
\subsection{第三阶段:项目实战(第9-11周,8学时)}

% ---------------- 第9周 ----------------
\subsubsection{第9周:系统架构与分组开发}

\begin{tcolorbox}[colback=red!5!white,colframe=red!75!black,title=\textbf{故事问题:把所有模块组合起来}]
\end{tcolorbox}

\begin{table}[htbp]
\centering
\renewcommand{\arraystretch}{1.4}
\begin{tabular}{|c|p{12cm}|}
\hline
\textbf{时间} & \textbf{内容} \\
\hline
40min & 项目架构讲解与模块划分 \\
\hline
80min & 分组开发 + 教师巡回指导 \\
\hline
20min & 进度检查与问题解答 \\
\hline
\end{tabular}
\end{table}

\textbf{系统框架(已提供):}
\begin{verbatim}
auto_grading_system/
├── input/                  # 测试试卷
├── output/                 # 识别结果
├── modules/
│   ├── preprocess.py       # 预处理(已实现)
│   ├── layout.py           # 版面分析(已实现)
│   ├── choice.py           # 选择题识别(学生实现)
│   ├── judge.py            # 判断题识别(学生实现)
│   ├── essay.py            # 简答题识别(学生实现)
│   └── grading.py          # 评分逻辑(学生实现)
└── main.py                 # 主程序(部分实现)
\end{verbatim}

\textbf{分组要求:}
\begin{itemize}
\item 3-4人/组
\item 包含不同专业背景
\item 明确分工:前端展示、后端整合、算法优化
\end{itemize}

% ---------------- 第10周 ----------------
\subsubsection{第10周:核心开发与调试}

\begin{tcolorbox}[colback=red!5!white,colframe=red!75!black,title=\textbf{故事问题:让系统真正跑起来}]
\end{tcolorbox}

\begin{table}[htbp]
\centering
\renewcommand{\arraystretch}{1.4}
\begin{tabular}{|c|p{12cm}|}
\hline
\textbf{时间} & \textbf{内容} \\
\hline
30min & 集成指导:如何拼接各模块 \\
\hline
120min & 分组开发 + 教师一对一指导 \\
\hline
30min & 测试集验证(10张标准试卷) \\
\hline
\end{tabular}
\end{table}

\textbf{开发要求:}

\begin{table}[htbp]
\centering
\renewcommand{\arraystretch}{1.3}
\begin{tabular}{|l|p{5cm}|p{3cm}|}
\hline
\textbf{等级} & \textbf{完成度} & \textbf{分数段} \\
\hline
基础版 & 完成一种题型识别 & 60-75 \\
\hline
进阶版 & 完成两种题型识别 & 76-88 \\
\hline
完整版 & 完成三种题型识别 & 89-100 \\
\hline
\end{tabular}
\end{table}

\newpage

% ---------------- 第11周 ----------------
\subsubsection{第11周:成果展示与总结}

\begin{tcolorbox}[colback=red!5!white,colframe=red!75!black,title=\textbf{故事问题:展示你的AI助教}]
\end{tcolorbox}

\begin{table}[htbp]
\centering
\renewcommand{\arraystretch}{1.4}
\begin{tabular}{|c|p{12cm}|}
\hline
\textbf{时间} & \textbf{内容} \\
\hline
30min & 分组展示(每组5分钟演示 + 2分钟答辩) \\
\hline
40min & 互评与教师点评 \\
\hline
20min & 课程总结与CV技术展望 \\
\hline
10min & 优秀项目颁奖 \\
\hline
\end{tabular}
\end{table}

\newpage

% ============================================================================
% 第五部分:课程设计详细方案
% ============================================================================
\section{课程设计详细方案}

\subsection{项目名称:智能阅卷系统}

\subsection{功能需求矩阵}

\begin{table}[htbp]
\centering
\renewcommand{\arraystretch}{1.5}
\begin{tabular}{|l|p{3cm}|p{4cm}|p{2.5cm}|p{3cm}|}
\hline
\textbf{题型} & \textbf{输入} & \textbf{处理} & \textbf{输出} & \textbf{技术方案} \\
\hline
\textbf{选择题} & 填涂区域图像 & 像素密度统计 & 选项(A/B/C/D) & OpenCV countNonZero \\
\hline
\textbf{判断题} & 符号区域图像 & 形状匹配 & 符号($\checkmark$/$\times$) & 轮廓特征 + 模板匹配 \\
\hline
\textbf{简答题} & 手写文字区域图像 & 端到端OCR & 文字内容 & PaddleOCR / TrOCR \\
\hline
\end{tabular}
\end{table}

\subsection{系统流程}

\begin{center}
\textbf{试卷图片} $\to$ \textbf{图像预处理} $\to$ \textbf{版面分析} $\to$
\begin{tabular}{c}
\textbf{选择题识别} \\
\textbf{判断题识别} \\
\textbf{简答题识别}
\end{tabular}
$\to$ \textbf{答案匹配} $\to$ \textbf{评分报告}
\end{center}

\subsection{分组策略}

\begin{table}[htbp]
\centering
\renewcommand{\arraystretch}{1.4}
\begin{tabular}{|l|p{6cm}|p{5cm}|}
\hline
\textbf{角色} & \textbf{职责} & \textbf{适合专业} \\
\hline
组长 & 统筹协调、进度管理 & 所有专业 \\
\hline
算法负责人 & 识别算法实现与调优 & CS/EE \\
\hline
前端负责人 & 界面展示、可视化 & 设计/媒体 \\
\hline
测试负责人 & 测试用例、文档撰写 & 所有专业 \\
\hline
\end{tabular}
\end{table}

\subsection{评估标准}

\begin{table}[htbp]
\centering
\renewcommand{\arraystretch}{1.4}
\begin{tabular}{|l|p{3cm}|p{8cm}|}
\hline
\textbf{维度} & \textbf{占比} & \textbf{评分标准} \\
\hline
功能完整性 & 40\% &
\begin{tabular}{@{}l@{}}
三种题型全实现(40分) / 两种题型(30分) / 一种题型(20分)
\end{tabular} \\
\hline
识别准确率 & 25\% &
\begin{tabular}{@{}l@{}}
测试集准确率$>$90\%(25分) / $>$80\%(20分) / $>$70\%(15分)
\end{tabular} \\
\hline
代码质量 & 15\% & 结构清晰、注释完整、可复用 \\
\hline
展示效果 & 10\% & 演示流畅、问题回答清晰 \\
\hline
团队协作 & 10\% & 分工明确、组内互评 \\
\hline
\end{tabular}
\end{table}

\newpage

% ============================================================================
% 第六部分:教学资源设计
% ============================================================================
\section{教学资源设计}

\subsection{代码脚手架(三层支持)}

\textbf{Layer 1: 完全封装版(面向零基础)}
\begin{verbatim}
from cv_toolkit import ChoiceRecognizer

recognizer = ChoiceRecognizer()
result = recognizer.detect(image)  # 一行调用
\end{verbatim}

\textbf{Layer 2: 参数可调版(面向进阶)}
\begin{verbatim}
from cv_toolkit import ChoiceRecognizer

recognizer = ChoiceRecognizer(
    threshold=127,      # 可调参数
    option_count=4,
    min_density=0.3
)
result = recognizer.detect(image)
\end{verbatim}

\textbf{Layer 3: 算法实现版(面向挑战)}
\begin{verbatim}
# 学生需要实现核心算法
def detect_answers(image, threshold=127):
    binary = cv2.threshold(image, threshold, 255, cv2.THRESH_BINARY)[1]
    # TODO: 实现密度统计算法
    pass
\end{verbatim}

\subsection{实验数据集}

\begin{verbatim}
datasets/
├── test_papers/          # 测试试卷(20张,三种题型)
├── templates/            # 标准答案模板
├── samples/
│   ├── choice/           # 选择题样本
│   ├── judge/            # 判断题样本
│   └── essay/            # 简答题样本
└── results/              # 标注结果(用于验证)
\end{verbatim}

% ============================================================================
% 第七部分:差异化教学策略
% ============================================================================
\section{差异化教学策略}

\subsection{跨专业适配方案}

\begin{table}[htbp]
\centering
\renewcommand{\arraystretch}{1.4}
\begin{tabular}{|l|p{5cm}|p{7cm}|}
\hline
\textbf{学生类型} & \textbf{特点} & \textbf{支持策略} \\
\hline
CS专业 & 编程基础好 & 提供算法实现版,鼓励深入优化 \\
\hline
EE专业 & 数学基础好 & 强调算法原理,提供数学推导 \\
\hline
文科专业 & 编程薄弱 & 提供封装API,强调应用场景 \\
\hline
\end{tabular}
\end{table}

\subsection{AI辅助编程Prompt模板}

\begin{table}[htbp]
\centering
\renewcommand{\arraystretch}{1.3}
\begin{tabular}{|l|p{11cm}|}
\hline
\textbf{场景} & \textbf{Prompt模板} \\
\hline
理解代码 & “这段OpenCV代码是什么意思?请用通俗语言解释:\{代码\}” \\
\hline
调试问题 & “我的填涂识别结果不准确,这是代码和输入输出,请帮我分析问题” \\
\hline
优化算法 & “如何提高这张试卷的OCR识别准确率?当前使用PaddleOCR” \\
\hline
\end{tabular}
\end{table}

\newpage

% ============================================================================
% 第八部分:考核方案
% ============================================================================
\section{考核方案}

\begin{table}[htbp]
\centering
\renewcommand{\arraystretch}{1.5}
\begin{tabular}{|l|p{4cm}|p{7cm}|}
\hline
\textbf{考核环节} & \textbf{占比} & \textbf{说明} \\
\hline
平时作业 & 30\% & 每周实验(8次$\times$3-4分) \\
\hline
课程项目 & 50\% & 功能完整度 + 代码质量 + 识别准确率 \\
\hline
展示报告 & 15\% & 演示效果 + 文档质量 + 答辩表现 \\
\hline
团队互评 & 5\% & 组内贡献度评价 \\
\hline
\end{tabular}
\end{table}

% ============================================================================
% 第九部分:时间投入建议
% ============================================================================
\section{时间投入建议}

\begin{table}[htbp]
\centering
\renewcommand{\arraystretch}{1.4}
\begin{tabular}{|c|c|c|p{6cm}|}
\hline
\textbf{周次} & \textbf{课内学时} & \textbf{课外学时} & \textbf{累计产出} \\
\hline
1-2周 & 6h & 4h/周 & 能用OpenCV + AI工具 \\
\hline
3-4周 & 6h & 6h/周 & 预处理 + 版面分析 \\
\hline
5-6周 & 6h & 6h/周 & 选择题 + 判断题识别 \\
\hline
7-8周 & 6h & 8h/周 & OCR + 手写识别 \\
\hline
9-11周 & 8h & 12h/周 & 完整阅卷系统 \\
\hline
\end{tabular}
\end{table}

% ============================================================================
% 第十部分:设计总结
% ============================================================================
\section{设计总结}

\begin{tcolorbox}[colback=green!5!white,colframe=green!75!black,title=\textbf{课程设计核心优势}]
\begin{itemize}
\item \textbf{零浪费}:每周产出直接用于最终项目
\item \textbf{零门槛}:AI工具 + 三层脚手架
\item \textbf{零脱节}:所有内容服务于三种题型识别
\item \textbf{故事化}:从“看见”到“理解”再到“评分”的完整叙事
\item \textbf{可扩展}:基础版$\to$进阶版$\to$挑战版,满足不同层次需求
\end{itemize}
\end{tcolorbox}

\end{document}
