\documentclass[12pt,a4paper]{article}
\usepackage[UTF8]{ctex}
\usepackage{geometry}
\usepackage{amsmath}
\usepackage{graphicx}
\usepackage{booktabs}
\usepackage{array}
\usepackage{enumitem}
\usepackage{xcolor}
\usepackage{tcolorbox}
\usepackage{listings}
\usepackage{hyperref}

\geometry{left=2.5cm,right=2.5cm,top=2.5cm,bottom=2.5cm}

\lstset{
    language=Python,
    basicstyle=\ttfamily\small,
    keywordstyle=\color{blue},
    commentstyle=\color{green!60!black},
    stringstyle=\color{orange},
    breaklines=true,
    frame=single,
    showstringspaces=false
}

\title{\textbf{\large 第4周教案:试卷版面分析}}
\author{计算机视觉课程组}
\date{}

\begin{document}

\maketitle

\section{基本信息}

\begin{tabular}{|l|p{12cm}|}
\hline
\textbf{周次} & 第4周 \\
\hline
\textbf{主题} & 试卷版面分析 \\
\hline
\textbf{学时} & 3学时(160分钟) \\
\hline
\textbf{故事问题} & 怎么知道选择题、简答题在哪里? \\
\hline
\textbf{OBE目标} & A2-版面分析:能定位试卷中各题型区域 \\
\hline
\textbf{项目贡献} & 找到题目边界,分隔各题型区域,为后续识别打基础 \\
\hline
\end{tabular}

\section{教学目标}

\begin{enumerate}
\item \textbf{知识目标}:
    \begin{itemize}
    \item 理解边缘检测的原理
    \item 掌握轮廓检测与形状分析
    \item 了解连通域分析方法
    \end{itemize}

\item \textbf{能力目标}:
    \begin{itemize}
    \item 能够使用Canny边缘检测
    \item 能够检测和绘制轮廓
    \item 能够根据形状特征筛选目标轮廓
    \end{itemize}

\item \textbf{素养目标}:
    \begin{itemize}
    \item 理解版面分析在文档处理中的重要性
    \item 培养从复杂图像中提取结构化信息的能力
    \end{itemize}
\end{enumerate}

\section{教学重点与难点}

\begin{tcolorbox}[colback=red!5!white,colframe=red!75!black,title=\textbf{教学重点}]
\begin{itemize}
\item Canny边缘检测
\item 轮廓检测与特征提取
\item 形状筛选(面积、长宽比、封闭性)
\end{itemize}
\end{tcolorbox}

\begin{tcolorbox}[colback=yellow!5!white,colframe=yellow!75!black,title=\textbf{教学难点}]
\begin{itemize}
\item 边缘检测参数调节
\item 从多个轮廓中识别目标区域
\end{itemize}
\end{tcolorbox}

\section{教学过程设计}

\subsection{环节一:版面分析概述(10分钟)}

\subsubsection{1.1 什么是版面分析?(5分钟)}

\textbf{定义:} 从文档图像中识别和定位不同区域(标题、正文、表格等)

\textbf{在阅卷系统中的作用:}
\begin{itemize}
\item 找到试卷边界
\item 定位选择题区域
\item 定位判断题区域
\item 定位简答题区域
\end{itemize}

\subsubsection{1.2 版面分析流程(5分钟)}

\begin{center}
\includegraphics[width=0.8\textwidth]{layout_pipeline.png}
\\
\footnotesize{输入 $\to$ 预处理 $\to$ 边缘检测 $\to$ 轮廓检测 $\to$ 区域筛选}
\end{center}

\subsection{环节二:边缘检测(40分钟)}

\subsubsection{2.1 什么是边缘?(10分钟)}

\textbf{边缘定义:} 图像中像素值发生剧烈变化的位置

\textbf{边缘类型:}
\begin{itemize}
\item 阶梯边缘:像素值突变
\item 屋顶边缘:像素值先增后减
\end{itemize}

\subsubsection{2.2 Canny边缘检测(20分钟)}

\textbf{为什么选择Canny?}
\begin{itemize}
\item 最优边缘检测算法
\item 检测准确、定位精确、单边缘响应
\end{itemize}

\begin{lstlisting}
import cv2
import numpy as np

# Canny边缘检测
# 参数:(图像, 低阈值, 高阈值)
edges = cv2.Canny(image, threshold1, threshold2)

# 阈值说明:
# - 小于threshold1的像素不是边缘
# - 大于threshold2的像素肯定是边缘
# - 介于两者之间的像素,如果与强边缘相连则认为是边缘
\end{lstlisting}

\textbf{参数选择经验:}
\begin{itemize}
\item 高阈值通常是低阈值的2-3倍
\item 低阈值:50-150
\item 高阈值:100-300
\end{itemize}

\begin{lstlisting}
# 示例:检测试卷边缘
img = cv2.imread('exam.jpg')
gray = cv2.cvtColor(img, cv2.COLOR_BGR2GRAY)
blur = cv2.GaussianBlur(gray, (5, 5), 0)

# Canny边缘检测
edges = cv2.Canny(blur, 50, 150)

# 显示结果
cv2.imshow('Edges', edges)
cv2.waitKey(0)
\end{lstlisting}

\subsubsection{2.3 其他边缘检测算子(10分钟)}

\begin{table}[htbp]
\centering
\begin{tabular}{|l|p{5cm}|p{5cm}|}
\hline
\textbf{算子} & \textbf{特点} & \textbf{适用场景} \\
\hline
Sobel & 基于导数,计算简单 & 快速检测 \\
\hline
Laplacian & 二阶导数,对噪声敏感 & 理论研究 \\
\hline
Canny & 多阶段,效果最好 & 实际应用,推荐 \\
\hline
\end{tabular}
\end{table}

\subsection{环节三:轮廓检测(50分钟)}

\subsubsection{3.1 什么是轮廓?(10分钟)}

\textbf{定义:** 连续的边缘点构成的曲线

\textbf{与边缘的区别:}
\begin{itemize}
\item 边缘:不连续的像素集合
\item 轮廓:封闭的、有序的曲线
\end{itemize}

\textbf{应用:** 物体识别、形状分析、区域定位}

\subsubsection{3.2 轮廓检测API(15分钟)}

\begin{lstlisting}
# 查找轮廓
# 参数:(图像, 轮廓模式, 轮廓逼近方法)
contours, hierarchy = cv2.findContours(
    image,           # 输入图像(二值图)
    cv2.RETR_EXTERNAL,  # 轮廓模式
    cv2.CHAIN_APPROX_SIMPLE  # 逼近方法
)
\end{lstlisting}

\textbf{轮廓模式:}

\begin{table}[htbp]
\centering
\begin{tabular}{|l|p{8cm}|}
\hline
\textbf{模式} & \textbf{说明} \\
\hline
cv2.RETR\_EXTERNAL & 只检测最外层轮廓 \\
\hline
cv2.RETR\_LIST & 检测所有轮廓,不建立层级关系 \\
\hline
cv2.RETR\_TREE & 检测所有轮廓,建立完整层级树 \\
\hline
\end{tabular}
\end{table}

\textbf{逼近方法:}

\begin{table}[htbp]
\centering
\begin{tabular}{|l|p{8cm}|}
\hline
\textbf{方法} & \textbf{说明} \\
\hline
cv2.CHAIN\_APPROX\_NONE & 存储所有轮廓点 \\
\hline
cv2.CHAIN\_APPROX\_SIMPLE & 压缩垂直、水平、对角线段,只保留端点(推荐) \\
\hline
\end{tabular}
\end{table}

\subsubsection{3.3 绘制轮廓(10分钟)}

\begin{lstlisting}
# 绘制轮廓
# 参数:(图像, 轮廓列表, 轮廓索引, 颜色, 线宽)
output = cv2.drawContours(
    image.copy(),      # 目标图像
    contours,          # 轮廓列表
    -1,                # 轮廓索引(-1表示绘制所有轮廓)
    (0, 255, 0),       # 颜色(BGR格式)
    2                  # 线宽
)
\end{lstlisting}

\subsubsection{3.4 轮廓特征(15分钟)}

\textbf{常用特征:}

\begin{lstlisting}
# 遍历所有轮廓
for i, contour in enumerate(contours):
    # 1. 面积
    area = cv2.contourArea(contour)

    # 2. 周长
    perimeter = cv2.arcLength(contour, True)  # True表示封闭

    # 3. 边界矩形
    x, y, w, h = cv2.boundingRect(contour)

    # 4. 最小外接矩形(可能旋转)
    rect = cv2.minAreaRect(contour)
    box = cv2.boxPoints(rect)
    box = np.int32(box)

    # 5. 最小外接圆
    (x, y), radius = cv2.minEnclosingCircle(contour)

    # 6. 凸包
    hull = cv2.convexHull(contour)
\end{lstlisting}

\textbf{筛选示例:** 找到试卷轮廓}

\begin{lstlisting}
def find_paper_contour(contours, image_area):
    """
    从轮廓列表中找到试卷轮廓

    参数:
        contours: 所有轮廓
        image_area: 图像总面积

    返回:
        试卷轮廓,未找到则返回None
    """
    for contour in contours:
        # 计算轮廓面积
        area = cv2.contourArea(contour)

        # 面积筛选:应该是图像面积的一定比例
        if area > image_area * 0.5:
            # 计算轮廓周长
            peri = cv2.arcLength(contour, True)

            # 多边形逼近(四边形)
            approx = cv2.approxPolyDP(contour, 0.02 * peri, True)

            # 如果是四边形(4个顶点)
            if len(approx) == 4:
                return approx

    return None
\end{lstlisting}

\subsection{环节四:区域定位(40分钟)}

\subsubsection{4.1 投影法(15分钟)}

\textbf{原理:** 通过投影检测文本行和空白区域}

\begin{lstlisting}
def horizontal_projection(binary_image):
    """
    水平投影:统计每行的白色像素数量

    用于检测文本行之间的空白
    """
    # 水平方向求和
    proj = np.sum(binary_image, axis=1)

    return proj

def vertical_projection(binary_image):
    """
    垂直投影:统计每列的白色像素数量

    用于检测字符之间的空白
    """
    # 垂直方向求和
    proj = np.sum(binary_image, axis=0)

    return proj

# 可视化投影
import matplotlib.pyplot as plt

proj = horizontal_projection(binary)

plt.figure(figsize=(10, 4))
plt.plot(proj)
plt.title('Horizontal Projection')
plt.xlabel('Row')
plt.ylabel('White pixel count')
plt.show()
\end{lstlisting}

\textbf{应用:** 检测题目之间的分隔}

\begin{lstlisting}
def find_divider_lines(projection, threshold=10):
    """
    找到投影值低于阈值的行(分隔线)

    参数:
        projection: 投影数组
        threshold: 阈值

    返回:
        分隔线的行索引列表
    """
    dividers = []
    for i, value in enumerate(projection):
        if value < threshold:
            dividers.append(i)

    return dividers
\end{lstlisting}

\subsubsection{4.2 连通域分析(15分钟)}

\textbf{什么是连通域?** 相邻的相同像素值构成的区域}

\begin{lstlisting}
# 连通域分析
# 参数:(图像, 连通性类型)
# 连通性:
#   - cv2.CV_8UC1: 4连通(上下左右)
#   - cv2.CV_8UC1: 8连通(加上对角线)
num_labels, labels, stats, centroids = cv2.connectedComponentsWithStats(
    binary_image,
    connectivity=8
)

# stats: 每个连通域的统计信息
# stats[i] = [left, top, width, height, area]
\end{lstlisting}

\textbf{应用:** 识别填涂区域}

\begin{lstlisting}
def find_bubble_regions(binary_image, min_area=100, max_area=1000):
    """
    找到可能的填涂区域

    参数:
        binary_image: 二值图像
        min_area: 最小面积
        max_area: 最大面积

    返回:
        候选区域列表
    """
    num_labels, labels, stats, centroids = cv2.connectedComponentsWithStats(
        binary_image
    )

    bubbles = []
    for i in range(1, num_labels):  # 跳过背景(0)
        area = stats[i, cv2.CC_STAT_AREA]

        # 面积筛选
        if min_area < area < max_area:
            # 获取边界矩形
            x = stats[i, cv2.CC_STAT_LEFT]
            y = stats[i, cv2.CC_STAT_TOP]
            w = stats[i, cv2.CC_STAT_WIDTH]
            h = stats[i, cv2.CC_STAT_HEIGHT]

            bubbles.append((x, y, w, h))

    return bubbles
\end{lstlisting}

\subsubsection{4.3 完整版面分析示例(10分钟)}

\begin{lstlisting}
def analyze_exam_layout(image_path):
    """
    分析试卷版面,定位各题型区域

    参数:
        image_path: 试卷图像路径

    返回:
        版面分析结果
    """
    # 1. 读取和预处理
    img = cv2.imread(image_path)
    gray = cv2.cvtColor(img, cv2.COLOR_BGR2GRAY)
    blur = cv2.GaussianBlur(gray, (5, 5), 0)
    _, binary = cv2.threshold(blur, 0, 255, cv2.THRESH_BINARY + cv2.THRESH_OTSU)

    # 2. 边缘检测
    edges = cv2.Canny(binary, 50, 150)

    # 3. 查找轮廓
    contours, _ = cv2.findContours(
        edges,
        cv2.RETR_EXTERNAL,
        cv2.CHAIN_APPROX_SIMPLE
    )

    # 4. 找到试卷轮廓
    paper_contour = find_paper_contour(contours, img.size)

    # 5. 投影分析
    h_proj = horizontal_projection(binary)
    dividers = find_divider_lines(h_proj)

    return {
        'paper_contour': paper_contour,
        'dividers': dividers
    }
\end{lstlisting}

\subsection{环节五:实验(20分钟)}

\textbf{任务:** 实现试卷版面结构可视化}

\begin{lstlisting}
import cv2
import numpy as np
import matplotlib.pyplot as plt

def visualize_layout(image_path):
    """可视化版面分析结果"""
    img = cv2.imread(image_path)
    gray = cv2.cvtColor(img, cv2.COLOR_BGR2GRAY)

    # 预处理
    blur = cv2.GaussianBlur(gray, (5, 5), 0)
    edges = cv2.Canny(blur, 50, 150)

    # 查找轮廓
    contours, _ = cv2.findContours(
        edges.copy(),
        cv2.RETR_EXTERNAL,
        cv2.CHAIN_APPROX_SIMPLE
    )

    # 绘制所有轮廓
    result = img.copy()
    for contour in contours:
        # 筛选大轮廓
        area = cv2.contourArea(contour)
        if area > 100:
            cv2.drawContours(result, [contour], -1, (0, 255, 0), 2)

            # 绘制边界矩形
            x, y, w, h = cv2.boundingRect(contour)
            cv2.rectangle(result, (x, y), (x+w, y+h), (255, 0, 0), 2)

    # 显示结果
    fig, axes = plt.subplots(1, 3, figsize=(15, 5))

    axes[0].imshow(cv2.cvtColor(img, cv2.COLOR_BGR2RGB))
    axes[0].set_title('原图')
    axes[0].axis('off')

    axes[1].imshow(edges, cmap='gray')
    axes[1].set_title('边缘检测')
    axes[1].axis('off')

    axes[2].imshow(cv2.cvtColor(result, cv2.COLOR_BGR2RGB))
    axes[2].set_title('轮廓检测')
    axes[2].axis('off')

    plt.tight_layout()
    plt.show()

# 使用
visualize_layout('exam.jpg')
\end{lstlisting}

\section{课后作业}

\subsection{作业内容}

\textbf{题目:** 实现试卷版面分析,标注三种题型区域}

\textbf{要求:**
\begin{enumerate}
\item 使用轮廓检测找到试卷边界
\item 使用投影法分析试卷结构
\item 标注出选择题、判断题、简答题的大致区域
\item 提交标注结果的可视化图像
\end{enumerate}

\subsection{评分标准}

\begin{table}[htbp]
\centering
\begin{tabular}{|l|p{8cm}|p{2cm}|}
\hline
\textbf{评分项} & \textbf{标准} & \textbf{分值} \\
\hline
轮廓检测 & 正确检测到试卷边界 & 25分 \\
\hline
区域定位 & 合理标注各题型区域 & 35分 \\
\hline
可视化 & 标注清晰可见 & 25分 \\
\hline
代码质量 & 结构清晰,有注释 & 15分 \\
\hline
\textbf{合计} & & \textbf{100分} \\
\hline
\end{tabular}
\end{table}

\section{教学反思}

\subsection{重点提醒}

\begin{itemize}
\item 版面分析是后续题型识别的基础
\item 实际应用中可能需要结合深度学习方法(如PaddleLayout)
\item 参数调节需要根据具体试卷版面调整
\end{itemize}

\subsection{下节预告}

下周将学习选择题的填涂识别,会用到本周的轮廓和连通域分析技术。

\end{document}
