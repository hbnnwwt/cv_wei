\documentclass[12pt,a4paper]{article}
\usepackage[UTF8]{ctex}
\usepackage{geometry}
\usepackage{amsmath}
\usepackage{graphicx}
\usepackage{booktabs}
\usepackage{array}
\usepackage{enumitem}
\usepackage{xcolor}
\usepackage{tcolorbox}
\usepackage{listings}
\usepackage{hyperref}

\geometry{left=2.5cm,right=2.5cm,top=2.5cm,bottom=2.5cm}

\lstset{
    language=Python,
    basicstyle=\ttfamily\small,
    keywordstyle=\color{blue},
    commentstyle=\color{green!60!black},
    stringstyle=\color{orange},
    breaklines=true,
    frame=single,
    showstringspaces=false
}

\title{\textbf{\large 第2周教案:AI辅助编程工具实战}}
\author{计算机视觉课程组}
\date{}

\begin{document}

\maketitle

\section{基本信息}

\begin{tabular}{|l|p{12cm}|}
\hline
\textbf{周次} & 第2周 \\
\hline
\textbf{主题} & AI辅助编程工具实战 \\
\hline
\textbf{学时} & 3学时(160分钟) \\
\hline
\textbf{故事问题} & 怎么让AI帮我写代码? \\
\hline
\textbf{OBE目标} & A5-AI协作:能用AI工具辅助编程开发 \\
\hline
\end{tabular}

\section{教学目标}

\begin{enumerate}
\item \textbf{知识目标}:
    \begin{itemize}
    \item 了解主流AI编程工具(ChatGPT、Claude、Copilot等)
    \item 理解Prompt工程的基本原理
    \end{itemize}

\item \textbf{能力目标}:
    \begin{itemize}
    \item 能够编写有效的CV领域Prompt
    \item 能够用AI辅助调试和优化代码
    \item 能够利用AI快速实现CV功能原型
    \end{itemize}

\item \textbf{素养目标}:
    \begin{itemize}
    \item 建立AI协作意识,提高学习效率
    \item 培养批判性思维,验证AI生成代码的正确性
    \end{itemize}
\end{enumerate}

\section{教学重点与难点}

\begin{tcolorbox}[colback=red!5!white,colframe=red!75!black,title=\textbf{教学重点}]
\begin{itemize}
\item Prompt工程:如何向AI提问CV问题
\item AI辅助代码调试技巧
\item AI生成代码的验证与优化
\end{itemize}
\end{tcolorbox}

\begin{tcolorbox}[colback=yellow!5!white,colframe=yellow!75!black,title=\textbf{教学难点}]
\begin{itemize}
\item 编写精准有效的Prompt
\item 判断AI生成代码的正确性
\item 将AI生成的代码整合到项目中
\end{itemize}
\end{tcolorbox}

\section{教学过程设计}

\subsection{环节一:AI编程工具介绍(30分钟)}

\subsubsection{1.1 为什么需要AI辅助编程?(10分钟)}

\textbf{传统编程的痛点:}
\begin{itemize}
\item API参数复杂,记不住
\item 报错信息看不懂
\item 算法原理理解困难
\end{itemize}

\textbf{AI辅助的优势:}
\begin{itemize}
\item 快速生成代码框架
\item 解释错误原因
\item 提供优化建议
\end{itemize}

\subsubsection{1.2 主流AI工具对比(15分钟)}

\begin{table}[htbp]
\centering
\begin{tabular}{|l|p{5cm}|p{4cm}|}
\hline
\textbf{工具} & \textbf{特点} & \textbf{适用场景} \\
\hline
ChatGPT & 对话能力强,代码生成准确 & 学习、调试、解释 \\
\hline
Claude & 代码分析深入,长文本处理好 & 代码审查、架构设计 \\
\hline
GitHub Copilot & IDE集成,实时补全 & 日常编码 \\
\hline
通义千问/文心一言 & 中文友好,国内可用 & 中文问题咨询 \\
\hline
\end{tabular}
\end{table}

\textbf{工具选择建议:}
\begin{itemize}
\item 学习理解:ChatGPT/Claude
\item 实时编码:GitHub Copilot/Cursor
\item 国内使用:通义千问/DeepSeek
\end{itemize}

\subsubsection{1.3 演示:AI工具的基本使用(5分钟)}

现场演示:用ChatGPT/Claude生成一个简单的OpenCV代码

\subsection{环节二:Prompt工程实战(40分钟)}

\subsubsection{2.1 什么是Prompt?(5分钟)}

\textbf{定义:} Prompt是给AI的指令或提示词

\textbf{好Prompt的标准:}
\begin{enumerate}
\item \textbf{具体明确}:不说模糊的话
\item \textbf{有上下文}:提供足够的背景信息
\item \textbf{有约束}:明确输出格式要求
\end{enumerate}

\subsubsection{2.2 CV领域专用Prompt模板(30分钟)}

\textbf{模板1:代码生成}
\begin{tcolorbox}[colback=blue!5!white,colframe=blue!75!black]
\textbf{请用Python和OpenCV实现以下功能:}\\
\textbf{功能描述:} [详细描述要实现的功能]\\
\textbf{输入:} [描述输入数据格式]\\
\textbf{输出:} [描述期望的输出格式]\\
\textbf{要求:}\\
- 使用OpenCV库\\
- 代码有详细注释\\
- 包含使用示例
\end{tcolorbox}

\textbf{示例:}
\begin{lstlisting}[language=bash]
请用Python和OpenCV实现以下功能:
功能描述:将彩色图像转换为灰度图像
输入:一张JPG格式的彩色图像文件
输出:灰度图像,并保存为PNG格式
要求:
- 使用OpenCV库
- 代码有详细注释
- 包含使用示例
\end{lstlisting}

\textbf{模板2:代码解释}
\begin{tcolorbox}[colback=green!5!white,colframe=green!75!black]
\textbf{请解释以下OpenCV代码的含义:}\\
\textbf{代码:}\\
\texttt{[粘贴代码]}\\
\textbf{要求:}\\
- 逐行解释代码功能\\
- 说明关键参数的作用\\
- 指出可能的错误用法
\end{tcolorbox}

\textbf{模板3:调试求助}
\begin{tcolorbox}[colback=yellow!5!white,colframe=yellow!75!black]
\textbf{我的OpenCV代码运行出错,请帮我分析:}\\
\textbf{代码:}\\
\texttt{[粘贴代码]}\\
\textbf{错误信息:}\\
\texttt{[粘贴报错信息]}\\
\textbf{预期行为:} [描述期望的结果]\\
\textbf{实际行为:} [描述实际发生的情况]
\end{tcolorbox}

\textbf{模板4:性能优化}
\begin{tcolorbox}[colback=orange!5!white,colframe=orange!75!black]
\textbf{如何优化以下OpenCV代码的性能?}\\
\textbf{当前实现:}\\
\texttt{[粘贴代码]}\\
\textbf{问题:} [描述性能瓶颈]\\
\textbf{目标:} [期望达到的性能指标]
\end{tcolorbox}

\subsubsection{2.3 互动练习(5分钟)}

\textbf{练习:} 给学生一个OpenCV代码片段,让学生尝试编写Prompt让AI解释

\subsection{环节三:AI辅助实战——人脸检测(50分钟)}

\subsubsection{3.1 任务说明(5分钟)}

\textbf{任务目标:} 用AI辅助实现一个人脸检测程序

\textbf{实现步骤:}
\begin{enumerate}
\item 用Prompt生成人脸检测代码
\item 理解AI生成的代码
\item 运行并测试代码
\item 优化和改进
\end{enumerate}

\subsubsection{3.2 步骤1:生成代码(10分钟)}

\textbf{Prompt示例:}
\begin{lstlisting}[language=bash]
请用Python和OpenCV实现一个人脸检测程序:
功能:从图片中检测所有人脸,并用矩形框标注
输入:图片文件路径
输出:标注了人脸框的图片
要求:
- 使用OpenCV的Haar级联分类器
- 在每个人脸周围绘制绿色矩形框
- 显示检测到的人脸数量
- 代码有详细中文注释
\end{lstlisting}

\textbf{预期AI输出:}
\begin{lstlisting}
import cv2

def detect_faces(image_path):
    """
    检测图像中的人脸并标注

    参数:
        image_path: 输入图像路径

    返回:
        标注后的图像
    """
    # 加载预训练的人脸检测模型
    face_cascade = cv2.CascadeClassifier(
        cv2.data.haarcascades + 'haarcascade_frontalface_default.xml'
    )

    # 读取图像
    img = cv2.imread(image_path)
    if img is None:
        print(f"错误:无法读取图像 {image_path}")
        return None

    # 转换为灰度图(Haar分类器需要灰度输入)
    gray = cv2.cvtColor(img, cv2.COLOR_BGR2GRAY)

    # 检测人脸
    faces = face_cascade.detectMultiScale(
        gray,
        scaleFactor=1.1,    # 图像缩放比例
        minNeighbors=5,     # 每个检测框的最小邻居数
        minSize=(30, 30)    # 最小人脸尺寸
    )

    # 绘制检测框
    for (x, y, w, h) in faces:
        cv2.rectangle(img, (x, y), (x+w, y+h), (0, 255, 0), 2)

    print(f"检测到 {len(faces)} 个人脸")

    return img

# 使用示例
result = detect_faces('test.jpg')
if result is not None:
    cv2.imwrite('faces_detected.jpg', result)
    cv2.imshow('Faces', result)
    cv2.waitKey(0)
    cv2.destroyAllWindows()
\end{lstlisting}

\subsubsection{3.3 步骤2:代码理解(15分钟)}

\textbf{关键概念解析:}
\begin{enumerate}
\item \textbf{Haar级联分类器}
    \begin{itemize}
    \item 一种基于机器学习的目标检测方法
    \item OpenCV提供预训练模型文件(.xml)
    \end{itemize}

\item \textbf{detectMultiScale参数}
    \begin{itemize}
    \item scaleFactor: 图像金字塔缩放比例
    \item minNeighbors: 检测框合并的阈值
    \item minSize: 最小目标尺寸
    \end{itemize}
\end{enumerate}

\textbf{学生活动:}
\begin{itemize}
\item 运行代码,观察效果
\item 修改参数,观察检测效果变化
\item 用自己的照片测试
\end{itemize}

\subsubsection{3.4 步骤3:调试与优化(20分钟)}

\textbf{常见问题与AI辅助解决:}

\textbf{问题1:检测不到人脸}
\begin{lstlisting}[language=bash]
我的代码检测不到侧脸,请帮我分析:
[粘贴代码]
\end{lstlisting}
\textbf{AI建议:}
\begin{itemize}
\item 调整scaleFactor和minNeighbors参数
\item 尝试使用更先进的模型(如DNN人脸检测)
\end{itemize}

\textbf{问题2:误检测(非人脸被识别成人脸)}
\begin{lstlisting}[language=bash]
我的代码把背景中的圆形也识别成人脸了:
[粘贴代码]
\end{lstlisting}
\textbf{AI建议:}
\begin{itemize}
\item 增加minNeighbors值
\item 增加minSize,过滤小目标
\item 使用眼睛检测作为二次验证
\end{itemize}

\textbf{学生活动:}
\begin{enumerate}
\item 记录遇到的问题
\item 用AI工具寻求解决方案
\item 验证AI建议的有效性
\end{enumerate}

\subsection{环节四:代码调试技巧(30分钟)}

\subsubsection{4.1 AI辅助调试流程(10分钟)}

\begin{enumerate}
\item \textbf{复制错误信息}
    \begin{itemize}
    \item 完整复制报错的traceback
    \item 注意错误类型和行号
    \end{itemize}

\item \textbf{准备上下文}
    \begin{itemize}
    \item 粘贴相关代码片段
    \item 说明输入数据
    \end{itemize}

\item \textbf{向AI提问}
    \begin{itemize}
    \item 使用模板化Prompt
    \item 明确预期vs实际行为
    \end{itemize}

\item \textbf{验证解决方案}
    \begin{itemize}
    \item 不要盲目信任AI
    \item 理解修改原理
    \item 测试修改效果
    \end{itemize}
\end{enumerate}

\subsubsection{4.2 常见错误案例(15分钟)}

\textbf{案例1:图片路径错误}
\begin{lstlisting}
# 错误代码
img = cv2.imread('test.jpg')  # 返回None

# AI诊断:路径不存在或格式错误
# 解决方案:
import os
path = 'test.jpg'
if os.path.exists(path):
    img = cv2.imread(path)
else:
    print(f"文件不存在:{path}")
\end{lstlisting}

\textbf{案例2:颜色顺序混乱}
\begin{lstlisting}
# 错误:OpenCV是BGR,matplotlib是RGB
plt.imshow(img)  # 颜色异常

# AI诊断:颜色空间不匹配
# 解决方案:
img_rgb = cv2.cvtColor(img, cv2.COLOR_BGR2RGB)
plt.imshow(img_rgb)
\end{lstlisting}

\textbf{案例3:数据类型溢出}
\begin{lstlisting}
# 错误:uint8溢出
result = img + 50  # 大于255的值会循环

# AI诊断:uint8范围是0-255
# 解决方案:
result = cv2.add(img, 50)  # 自动截断
# 或者
result = np.clip(img.astype(int) + 50, 0, 255).astype(np.uint8)
\end{lstlisting}

\subsubsection{4.3 练习(5分钟)}

给学生一段有bug的代码,让学生用AI辅助找出并修复bug

\subsection{环节五:AI工具使用规范(10分钟)}

\subsubsection{5.1 使用原则}

\begin{enumerate}
\item \textbf{验证优先}
    \begin{itemize}
    \item AI生成代码必须测试
    \item 理解代码再使用
    \end{itemize}

\item \textbf{学习导向}
    \begin{itemize}
    \item 不仅问"怎么写"
    \item 更要问"为什么"
    \end{itemize}

\item \textbf{安全意识}
    \begin{itemize}
    \item 不要输入敏感信息
    \item 注意代码安全性
    \end{itemize}
\end{enumerate}

\subsubsection{5.2 最佳实践}

\begin{itemize}
\item 从简单Prompt开始,逐步迭代优化
\item 保持对话的连贯性
\item 保存有用的对话记录
\item 建立个人Prompt库
\end{itemize}

\section{课后作业}

\subsection{作业内容}

\textbf{题目:} 用AI辅助实现答题卡边界检测程序

\textbf{项目关联:} 这是自动阅卷系统的第一步 —— 定位答题卡在图像中的位置。通过完成此作业,学生将为后续的版面分析和题目识别模块奠定基础。

\textbf{要求:}
\begin{enumerate}
\item 用Prompt向AI询问答题卡边界检测的实现方法
\item 记录完整的AI对话过程(至少3轮交互)
\item 运行并测试代码(可使用试卷图片)
\item 撰写反思报告
\end{enumerate}

\subsection{提交内容}

\begin{enumerate}
\item \textbf{AI对话记录}(截图或复制文本)
\item \textbf{最终代码}
\item \textbf{测试结果图片}(标注出检测到的边界)
\item \textbf{反思报告}(包含以下内容)
    \begin{itemize}
    \item AI给你的帮助有哪些?
    \item 你遇到的问题如何解决的?
    \item AI生成的代码有哪些需要改进的地方?
    \item 此功能对自动阅卷系统的作用是什么?
    \end{itemize}
\end{enumerate}

\subsection{评分标准}

\begin{table}[htbp]
\centering
\begin{tabular}{|l|p{8cm}|p{2cm}|}
\hline
\textbf{评分项} & \textbf{标准} & \textbf{分值} \\
\hline
AI对话质量 & Prompt清晰有效,交互充分 & 20分 \\
\hline
代码实现 & 能正确检测答题卡边界 & 30分 \\
\hline
测试完整 & 有多场景测试记录,可视化结果清晰 & 20分 \\
\hline
反思深度 & 深入思考AI辅助的利与弊,理解项目关联 & 30分 \\
\hline
\textbf{合计} & & \textbf{100分} \\
\hline
\end{tabular}
\end{table}

\section{教学反思}

\subsection{预期挑战}

\begin{itemize}
\item 学生可能完全依赖AI,不理解代码
    \begin{itemize}
    \item 应对:设置代码理解考核环节
    \end{itemize}

\item AI工具访问问题(网络限制)
    \begin{itemize}
    \item 应对:推荐国内可用工具,准备备用方案
    \end{itemize}

\item Prompt编写能力差异大
    \begin{itemize}
    \item 应对:提供模板库,多练习
    \end{itemize}
\end{itemize}

\end{document}
