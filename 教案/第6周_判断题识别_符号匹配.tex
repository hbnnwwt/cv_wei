\documentclass[12pt,a4paper]{article}
\usepackage[UTF8]{ctex}
\usepackage{geometry}
Sepackage{amsmath}
Sepackage{amssymb}
\usepackage{graphicx}
\usepackage{booktabs}
\usepackage{array}
\usepackage{enumitem}
\usepackage{xcolor}
\usepackage{tcolorbox}
\usepackage{listings}
\usepackage{hyperref}

\geometry{left=2.5cm,right=2.5cm,top=2.5cm,bottom=2.5cm}

\lstset{
    language=Python,
    basicstyle=\ttfamily\small,
    keywordstyle=\color{blue},
    commentstyle=\color{green!60!black},
    stringstyle=\color{orange},
    breaklines=true,
    frame=single,
    showstringspaces=false
}

\title{\textbf{\large 第6周教案:判断题识别(符号匹配)}}
\author{计算机视觉课程组}
\date{}

\begin{document}

\maketitle

\section{基本信息}

\begin{tabular}{|l|p{12cm}|}
\hline
\textbf{周次} & 第6周 \\
\hline
\textbf{主题} & 判断题识别(符号匹配) \\
\hline
\textbf{学时} & 3学时(160分钟) \\
\hline
\textbf{故事问题} & 怎么看到是$\checkmark$还是$\times$? \\
\hline
\textbf{OBE目标} & A3-特征识别:能识别判断题符号($\checkmark$/$\times$) \\
\hline
\textbf{项目贡献} & 实现判断题自动识别模块 \\
\hline
\end{tabular}

\section{教学目标}

\begin{enumerate}
\item \textbf{知识目标}:
    \begin{itemize}
    \item 理解形状匹配算法原理
    \item 掌握轮廓特征提取方法
    \item 了解模板匹配技术
    \end{itemize}

\item \textbf{能力目标}:
    \begin{itemize}
    \item 能够提取符号的轮廓特征
    \item 能够使用轮廓特征区分$\checkmark$和$\times$
    \item 能够实现判断题答案识别
    \end{itemize}

\item \textbf{素养目标}:
    \begin{itemize}
    \item 理解形状特征在识别中的应用
    \item 培养抽象思维能力
    \end{itemize}
\end{enumerate}

\section{教学重点与难点}

\begin{tcolorbox}[colback=red!5!white,colframe=red!75!black,title=\textbf{教学重点}]
\begin{itemize}
\item 轮廓特征提取(圆度、凸性、长宽比)
\item 形状匹配算法
\item 判断题识别流程
\end{itemize}
\end{tcolorbox}

\begin{tcolorbox}[colback=yellow!5!white,colframe=yellow!75!black,title=\textbf{教学难点}]
\begin{itemize}
\item 形状特征的几何意义
\item 手写符号的形变处理
\end{itemize}
\end{tcolorbox}

\section{教学过程设计}

\subsection{环节一:判断题识别概述(15分钟)}

\subsubsection{1.1 判断题的特点(5分钟)}

\textbf{常见符号:}
\begin{itemize}
\item $\checkmark$(正确/对)
\item $\times$(错误/错)
\item $\sqrt{}$(正确)
\item $\bigcirc$(正确)
\end{itemize}

\textbf{与选择题的区别:}
\begin{itemize}
\item 选择题:关注填涂密度
\item 判断题:关注符号形状
\end{itemize}

\subsubsection{1.2 识别方案(10分钟)}

\textbf{方案1:轮廓特征法}
\begin{itemize}
\item 提取符号轮廓
\item 计算形状特征
\item 根据特征判断符号类型
\end{itemize}

\textbf{方案2:模板匹配法}
\begin{itemize}
\item 准备标准符号模板
\item 与模板进行匹配
\item 选择最佳匹配结果
\end{itemize}

\subsection{环节二:轮廓特征提取(45分钟)}

\subsubsection{2.1 基础轮廓特征(20分钟)}

\begin{lstlisting}
import cv2
import numpy as np

def extract_basic_features(contour):
    """
    提取基础轮廓特征

    参数:
        contour: 轮廓

    返回:
        特征字典
    """
    features = {}

    # 1. 面积
    features['area'] = cv2.contourArea(contour)

    # 2. 周长
    features['perimeter'] = cv2.arcLength(contour, True)

    # 3. 边界矩形
    x, y, w, h = cv2.boundingRect(contour)
    features['bbox'] = (x, y, w, h)

    # 4. 长宽比
    features['aspect_ratio'] = float(w) / h if h > 0 else 0

    # 5. 占空比(面积/边界矩形面积)
    bbox_area = w * h
    features['extent'] = features['area'] / bbox_area if bbox_area > 0 else 0

    return features
\end{lstlisting}

\subsubsection{2.2 高级形状特征(25分钟)}

\textbf{1. 圆度(Circularity)}

\begin{lstlisting}
def calculate_circularity(contour):
    """
    计算圆度

    公式: 4 * pi * area / perimeter^2
    完美圆形的圆度接近1
    """
    area = cv2.contourArea(contour)
    perimeter = cv2.arcLength(contour, True)

    if perimeter == 0:
        return 0

    circularity = 4 * np.pi * area / (perimeter ** 2)
    return circularity
\end{lstlisting}

\textbf{圆度特征:}
\begin{itemize}
\item 圆形:接近1
\item 正方形:约0.785
\item $\checkmark$:较低(开口形状)
\item $\times$:更低(两线交叉)
\end{itemize}

\textbf{2. 凸性(Convexity)}

\begin{lstlisting}
def calculate_convexity(contour):
    """
    计算凸性

    凸性 = 轮廓面积 / 凸包面积
    """
    area = cv2.contourArea(contour)
    hull = cv2.convexHull(contour)
    hull_area = cv2.contourArea(hull)

    if hull_area == 0:
        return 0

    convexity = area / hull_area
    return convexity
\end{lstlisting}

\textbf{凸性特征:}
\begin{itemize}
\item 凸图形:接近1(如$\bigcirc$)
\item 凹图形:小于1(如$\checkmark$)
\end{itemize}

\textbf{3. 固性(Solidity)}

\begin{lstlisting}
def calculate_solidity(contour):
    """
    计算固性(与凸性类似)

    固性 = 轮廓面积 / 凸包面积
    """
    area = cv2.contourArea(contour)
    hull = cv2.convexHull(contour)
    hull_area = cv2.contourArea(hull)

    if hull_area == 0:
        return 0

    solidity = float(area) / hull_area
    return solidity
\end{lstlisting}

\textbf{4. 偏心率(Eccentricity)}

\begin{lstlisting}
def calculate_eccentricity(contour):
    """
    计算偏心率

    使用椭圆拟合,偏心率 = sqrt(1 - (b/a)^2)
    a: 长半轴, b: 短半轴
    """
    if len(contour) < 5:
        return 0

    # 拟合椭圆
    ellipse = cv2.fitEllipse(contour)

    # 获取长短轴
    center, axes, angle = ellipse
    major_axis = max(axes) / 2
    minor_axis = min(axes) / 2

    if major_axis == 0:
        return 0

    eccentricity = np.sqrt(1 - (minor_axis / major_axis) ** 2)
    return eccentricity
\end{lstlisting}

\textbf{5. 拓扑特征:端点数}

\begin{lstlisting}
def count_endpoints(binary_image):
    """
    计算端点数量

    用于区分符号类型:
    - √: 2个端点
    - ×: 4个端点
    """
    # 定义端点模式(黑色像素周围只有一个白色像素)
    kernel = np.array([
        [0, 0, 0],
        [0, 1, 0],
        [0, 0, 0]
    ])

    # 膨胀操作
    dilated = cv2.dilate(binary_image, kernel, iterations=1)

    # 端点 = 原图 - 膨胀后的图
    endpoints = cv2.absdiff(binary_image, dilated)

    # 统计端点
    endpoint_count = cv2.countNonZero(endpoints)

    return endpoint_count
\end{lstlisting}

\subsection{环节三:基于特征的分类(35分钟)}

\subsubsection{3.1 特征提取完整函数(15分钟)}

\begin{lstlisting}
def extract_symbol_features(roi):
    """
    提取符号的完整特征集

    参数:
        roi: 符号区域图像(灰度图)

    返回:
        特征字典
    """
    # 二值化
    _, binary = cv2.threshold(roi, 127, 255, cv2.THRESH_BINARY_INV)

    # 查找轮廓
    contours, _ = cv2.findContours(
        binary,
        cv2.RETR_EXTERNAL,
        cv2.CHAIN_APPROX_SIMPLE
    )

    if len(contours) == 0:
        return None

    # 使用最大轮廓(假设符号是最大的对象)
    contour = max(contours, key=cv2.contourArea)

    features = {}

    # 基础特征
    features['area'] = cv2.contourArea(contour)
    features['perimeter'] = cv2.arcLength(contour, True)

    # 形状特征
    features['circularity'] = calculate_circularity(contour)
    features['convexity'] = calculate_convexity(contour)
    features['solidity'] = calculate_solidity(contour)

    # 边界矩形
    x, y, w, h = cv2.boundingRect(contour)
    features['aspect_ratio'] = float(w) / h if h > 0 else 0
    features['extent'] = features['area'] / (w * h) if w * h > 0 else 0

    # 端点数量
    features['endpoints'] = count_endpoints(binary)

    return features
\end{lstlisting}

\subsubsection{3.2 基于规则的分类器(20分钟)}

\begin{lstlisting}
def classify_by_rules(features):
    """
    基于规则判断符号类型

    参数:
        features: 特征字典

    返回:
        符号类型 ('check', 'cross', 'circle', 'unknown')
    """
    if features is None:
        return 'unknown'

    # 规则1:根据圆度判断
    # 圆形(○)圆度最高
    if features['circularity'] > 0.8:
        return 'circle'

    # 规则2:根据凸性判断
    # √ 是凹图形,凸性较低
    # × 的凸性介于 √ 和 ○ 之间
    if features['convexity'] < 0.8:
        return 'check'  # 可能是 √

    # 规则3:根据端点数量
    # × 有4个端点,√ 有2个端点
    if features['endpoints'] >= 3:
        return 'cross'
    elif features['endpoints'] >= 1:
        return 'check'

    # 规则4:根据长宽比
    # √ 通常是斜向的,长宽比较大
    if features['aspect_ratio'] > 1.5:
        return 'check'

    return 'unknown'
\end{lstlisting}

\textbf{训练集特征统计:}

\begin{table}[htbp]
\centering
\begin{tabular}{|l|c|c|c|c|}
\hline
\textbf{符号} & \textbf{圆度} & \textbf{凸性} & \textbf{固性} & \textbf{长宽比} \\
\hline
$\checkmark$ & 0.3-0.6 & 0.6-0.8 & 0.6-0.8 & 1.2-2.0 \\
\hline
$\times$ & 0.2-0.5 & 0.8-0.95 & 0.6-0.8 & 0.8-1.5 \\
\hline
$\bigcirc$ & 0.7-1.0 & 0.95-1.0 & 0.9-1.0 & 0.8-1.2 \\
\hline
\end{tabular}
\end{table}

\subsection{环节四:模板匹配法(35分钟)}

\subsubsection{4.1 模板匹配原理(10分钟)}

\textbf{原理:** 在图像中滑动模板,计算相似度}

\textbf{相似度度量方法:}

\begin{table}[htbp]
\centering
\begin{tabular}{|l|p{8cm}|}
\hline
\textbf{方法} & \textbf{说明} \\
\hline
TM\_SQDIFF & 平方差匹配,值越小越相似 \\
\hline
TM\_SQDIFF\_NORMED & 归一化平方差匹配,0表示完美匹配 \\
\hline
TM\_CCORR & 相关匹配,值越大越相似 \\
\hline
TM\_CCOEFF & 系数匹配,值越大越相似 \\
\hline
TM\_CCOEFF\_NORMED & 归一化系数匹配,1表示完美匹配,-1完全不相似 \\
\hline
\end{tabular}
\end{table}

\subsubsection{4.2 模板匹配实现(15分钟)}

\begin{lstlisting}
def match_template(roi, templates, threshold=0.7):
    """
    模板匹配

    参数:
        roi: 待匹配区域
        templates: 模板字典 {'check': img_check, 'cross': img_cross, ...}
        threshold: 匹配阈值

    返回:
        最佳匹配的符号类型
    """
    best_match = None
    best_value = -float('inf')

    for symbol_type, template in templates.items():
        # 调整模板大小以匹配roi
        if template.shape[:2] != roi.shape[:2]:
            template = cv2.resize(template, (roi.shape[1], roi.shape[0]))

        # 转灰度
        if len(roi.shape) == 3:
            roi_gray = cv2.cvtColor(roi, cv2.COLOR_BGR2GRAY)
        else:
            roi_gray = roi

        if len(template.shape) == 3:
            template_gray = cv2.cvtColor(template, cv2.COLOR_BGR2GRAY)
        else:
            template_gray = template

        # 模板匹配
        result = cv2.matchTemplate(roi_gray, template_gray, cv2.TM_CCOEFF_NORMED)

        # 获取匹配值
        match_value = result[0, 0]

        if match_value > best_value:
            best_value = match_value
            best_match = symbol_type

    # 阈值判断
    if best_value >= threshold:
        return best_match, best_value
    else:
        return 'unknown', best_value
\end{lstlisting}

\textbf{创建模板:}

\begin{lstlisting}
# 创建标准符号模板
import cv2
import numpy as np

def create_checkmark_template(size=50):
    """创建√模板"""
    template = np.zeros((size, size), dtype=np.uint8)

    # 绘制√
    pts = np.array([
        [size*0.2, size*0.5],
        [size*0.4, size*0.7],
        [size*0.8, size*0.3]
    ], np.int32)

    cv2.polylines(template, [pts], False, 255, 3)

    return template

def create_cross_template(size=50):
    """创建×模板"""
    template = np.zeros((size, size), dtype=np.uint8)

    # 绘制×
    center = (size//2, size//2)
    radius = size//3

    cv2.line(template, (center[0]-radius, center[1]-radius),
             (center[0]+radius, center[1]+radius), 255, 3)
    cv2.line(template, (center[0]+radius, center[1]-radius),
             (center[0]-radius, center[1]+radius), 255, 3)

    return template

# 创建模板字典
templates = {
    'check': create_checkmark_template(),
    'cross': create_cross_template(),
    # 'circle': create_circle_template()
}
\end{lstlisting}

\subsubsection{4.3 轮廓匹配(10分钟)}

\begin{lstlisting}
def match_contour_shape(contour, template_contours):
    """
    轮廓形状匹配

    参数:
        contour: 待匹配轮廓
        template_contours: 模板轮廓字典

    返回:
        最佳匹配的符号类型
    """
    best_match = None
    best_value = float('inf')

    for symbol_type, template_contour in template_contours.items():
        # Hu矩匹配
        match_value = cv2.matchShapes(contour, template_contour,
                                       cv2.CONTOURS_MATCH_I1, 0.0)

        if match_value < best_value:
            best_value = match_value
            best_match = symbol_type

    return best_match, best_value
\end{lstlisting}

\subsection{环节五:完整判断题识别模块(20分钟)}

\begin{lstlisting}
class JudgeRecognizer:
    """判断题识别器"""

    def __init__(self, method='feature'):
        """
        初始化

        参数:
            method: 识别方法 ('feature' 或 'template')
        """
        self.method = method
        self.templates = None
        self.template_contours = None

        if method == 'template':
            self._init_templates()

    def _init_templates(self):
        """初始化模板"""
        self.templates = {
            'check': create_checkmark_template(),
            'cross': create_cross_template()
        }

        # 提取模板轮廓
        self.template_contours = {}
        for symbol_type, template in self.templates.items():
            contours, _ = cv2.findContours(
                template, cv2.RETR_EXTERNAL, cv2.CHAIN_APPROX_SIMPLE
            )
            if len(contours) > 0:
                self.template_contours[symbol_type] = contours[0]

    def preprocess(self, image):
        """预处理"""
        gray = cv2.cvtColor(image, cv2.COLOR_BGR2GRAY)
        blur = cv2.GaussianBlur(gray, (5, 5), 0)
        _, binary = cv2.threshold(blur, 0, 255, cv2.THRESH_BINARY + cv2.THRESH_OTSU)
        return gray, binary

    def recognize(self, roi):
        """
        识别单个判断题

        参数:
            roi: 符号区域图像

        返回:
            (符号类型, 置信度)
        """
        gray, binary = self.preprocess(roi)

        if self.method == 'feature':
            # 特征法
            features = extract_symbol_features(gray)
            symbol = classify_by_rules(features)
            confidence = features.get('circularity', 0) if features else 0

        elif self.method == 'template':
            # 模板匹配法
            symbol, confidence = match_template(binary, self.templates)

        else:
            symbol = 'unknown'
            confidence = 0

        # 转换为布尔答案
        if symbol == 'check' or symbol == 'circle':
            answer = True
        elif symbol == 'cross':
            answer = False
        else:
            answer = None

        return answer, (symbol, confidence)

    def recognize_all(self, image, symbol_positions):
        """
        识别所有判断题

        参数:
            image: 完整图像
            symbol_positions: 符号位置列表 [(x,y,w,h), ...]

        返回:
            答案列表
        """
        results = []

        for i, (x, y, w, h) in enumerate(symbol_positions):
            roi = image[y:y+h, x:x+w]
            answer, details = self.recognize(roi)

            results.append({
                'question': i + 1,
                'answer': answer,
                'symbol': details[0],
                'confidence': details[1]
            })

        return results
\end{lstlisting}

\subsection{环节六:实验(10分钟)}

\begin{lstlisting}
# 使用示例
recognizer = JudgeRecognizer(method='feature')

# 读取图像
image = cv2.imread('judge_questions.jpg')

# 定义符号位置(实际需要自动检测)
positions = [
    (100, 200, 50, 50),
    (200, 200, 50, 50),
    # ...
]

# 识别
results = recognizer.recognize_all(image, positions)

# 打印结果
for r in results:
    print(f"题目{r['question']}: {r['answer']}")
\end{lstlisting}

\section{课后作业}

\subsection{作业内容}

\textbf{题目:** 实现判断题符号识别模块}

\textbf{要求:**
\begin{enumerate}
\item 实现轮廓特征提取
\item 实现基于规则的分类器
\item 识别$\checkmark$和$\times$符号
\item 可视化标注识别结果
\end{enumerate}

\subsection{评分标准}

\begin{table}[htbp]
\centering
\begin{tabular}{|l|p{8cm}|p{2cm}|}
\hline
\textbf{评分项} & \textbf{标准} & \textbf{分值} \\
\hline
特征提取 & 正确提取形状特征 & 30分 \\
\hline
分类实现 & 实现符号分类逻辑 & 35分 \\
\hline
识别效果 & 正确识别判断题答案 & 25分 \\
\hline
可视化 & 清晰标注结果 & 10分 \\
\hline
\textbf{合计} & & \textbf{100分} \\
\hline
\end{tabular}
\end{table}

\section{教学反思}

\subsection{重点提醒}

\begin{itemize}
\item 形状特征对符号形变敏感
\item 手写符号差异大,可能需要收集训练数据
\item 下周将学习OCR技术,用于简答题识别
\end{itemize}

\end{document}
