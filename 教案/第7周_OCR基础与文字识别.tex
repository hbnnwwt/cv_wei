\documentclass[12pt,a4paper]{article}
\usepackage[UTF8]{ctex}
\usepackage{geometry}
\usepackage{amsmath}
\usepackage{graphicx}
\usepackage{booktabs}
\usepackage{array}
\usepackage{enumitem}
\usepackage{xcolor}
\usepackage{tcolorbox}
\usepackage{listings}
\usepackage{hyperref}

\geometry{left=2.5cm,right=2.5cm,top=2.5cm,bottom=2.5cm}

\lstset{
    language=Python,
    basicstyle=\ttfamily\small,
    keywordstyle=\color{blue},
    commentstyle=\color{green!60!black},
    stringstyle=\color{orange},
    breaklines=true,
    frame=single,
    showstringspaces=false
}

\title{\textbf{\large 第7周教案:OCR基础与文字识别}}
\author{计算机视觉课程组}
\date{}

\begin{document}

\maketitle

\section{基本信息}

\begin{tabular}{|l|p{12cm}|}
\hline
\textbf{周次} & 第7周 \\
\hline
\textbf{主题} & OCR基础与文字识别 \\
\hline
\textbf{学时} & 3学时(160分钟) \\
\hline
\textbf{故事问题} & 怎么让机器"阅读"文字? \\
\hline
\textbf{OBE目标} & A3-特征识别:能使用OCR识别文字 \\
\hline
\textbf{项目贡献} & 为简答题识别和题号识别提供技术基础 \\
\hline
\end{tabular}

\section{教学目标}

\begin{enumerate}
\item \textbf{知识目标}:
    \begin{itemize}
    \item 理解OCR技术原理与发展
    \item 了解主流OCR工具和库
    \item 掌握PaddleOCR的基本使用
    \end{itemize}

\item \textbf{能力目标}:
    \begin{itemize}
    \item 能够安装和配置OCR工具
    \item 能够识别图像中的印刷文字
    \item 能够处理OCR识别结果
    \end{itemize}

\item \textbf{素养目标}:
    \begin{itemize}
    \item 理解OCR在实际应用中的价值
    \item 了解OCR技术的局限性和改进方向
    \end{itemize}
\end{enumerate}

\section{教学重点与难点}

\begin{tcolorbox}[colback=red!5!white,colframe=red!75!black,title=\textbf{教学重点}]
\begin{itemize}
\item PaddleOCR的安装与使用
\item OCR结果后处理
\item 中文文字识别
\end{itemize}
\end{tcolorbox}

\begin{tcolorbox}[colback=yellow!5!white,colframe=yellow!75!black,title=\textbf{教学难点}]
\begin{itemize}
\item OCR结果准确率优化
\item 多语言识别配置
\end{itemize}
\end{tcolorbox}

\section{教学过程设计}

\subsection{环节一:OCR技术概述(40分钟)}

\subsubsection{1.1 什么是OCR?(10分钟)}

\textbf{定义:** Optical Character Recognition,光学字符识别}

\textbf{功能:** 将图像中的文字转换为计算机可编辑的文本

\textbf{应用场景:}
\begin{itemize}
\item 文档数字化
\item 车牌识别
\item 身份证/护照识别
\item 证件/票据识别
\item \textbf{试卷文字识别}
\end{itemize}

\subsubsection{1.2 OCR技术的发展(15分钟)}

\begin{table}[htbp]
\centering
\begin{tabular}{|l|p{6cm}|p{5cm}|}
\hline
\textbf{阶段} & \textbf{技术} & \textbf{特点} \\
\hline
第一代 & 模板匹配 & 只能识别特定字体,准确率低 \\
\hline
第二代 & 特征提取 & 能识别多种字体,但仍需规则 \\
\hline
第三代 & 机器学习 & SVM、神经网络,准确率提升 \\
\hline
第四代 & 深度学习 & CNN、RNN、Transformer,准确率高 \\
\hline
\textbf{当前} & \textbf{端到端模型} & \textbf{PaddleOCR、Tesseract、TrOCR} \\
\hline
\end{tabular}
\end{table}

\subsubsection{1.3 主流OCR工具对比(15分钟)}

\begin{table}[htbp]
\centering
\begin{tabular}{|l|p{4cm}|p{4cm}|p{3cm}|}
\hline
\textbf{工具} & \textbf{优点} & \textbf{缺点} & \textbf{适用} \\
\hline
Tesseract & 开源免费,支持多语言 & 中文准确率一般 & 英文文档 \\
\hline
PaddleOCR & 中文准确率高,易用 & 需要联网下载模型 & 中文场景 \\
\hline
EasyOCR & 多语言支持好 & 速度较慢 & 多语言混合 \\
\hline
TrOCR & 手写识别强 & 资源占用大 & 手写文字 \\
\hline
\end{tabular}
\end{table}

\subsection{环节二:PaddleOCR快速上手(50分钟)}

\subsubsection{2.1 安装PaddleOCR(10分钟)}

\begin{lstlisting}
# 安装PaddlePaddle(CPU版本)
pip install paddlepaddle -i https://mirror.baidu.com/pypi/simple

# 安装PaddleOCR
pip install paddleocr -i https://mirror.baidu.com/pypi/simple

# 安装其他依赖
pip install opencv-python pillow
\end{lstlisting}

\subsubsection{2.2 基础使用(20分钟)}

\begin{lstlisting}
from paddleocr import PaddleOCR

# 初始化OCR
# use_angle_cls=True: 启用文字方向分类
# lang='ch': 中文模型('en'为英文)
ocr = PaddleOCR(use_angle_cls=True, lang='ch')

# 识别图像
result = ocr.ocr('exam_paper.jpg', cls=True)

# 结果格式
# [
#   [[[左上x, 左上y], [右上x, 右上y], [右下x, 右下y], [左下x, 左下y]], ('文字', 置信度)],
#   ...
# ]

# 打印识别结果
for line in result:
    boxes = line[0]  # 文字框坐标
    text = line[1][0]  # 文字内容
    confidence = line[1][1]  # 置信度

    print(f"文字: {text}, 置信度: {confidence:.4f}")
\end{lstlisting}

\subsubsection{2.3 结果可视化(15分钟)}

\begin{lstlisting}
import cv2
import numpy as np
from PIL import Image, ImageDraw, ImageFont

def visualize_ocr_result(image_path, ocr_result, output_path='result.jpg'):
    """
    可视化OCR识别结果

    参数:
        image_path: 输入图像路径
        ocr_result: OCR识别结果
        output_path: 输出图像路径
    """
    # 读取图像
    img = cv2.imread(image_path)
    img_pil = Image.fromarray(cv2.cvtColor(img, cv2.COLOR_BGR2RGB))
    draw = ImageDraw.Draw(img_pil)

    # 尝试加载中文字体
    try:
        font = ImageFont.truetype("simhei.ttf", 20)
    except:
        font = ImageFont.load_default()

    # 绘制每个文字框
    for line in ocr_result:
        boxes = line[0]
        text = line[1][0]
        confidence = line[1][1]

        # 绘制文字框
        pts = np.array(boxes).astype(np.int32)
        cv2.polylines(img, [pts], True, (0, 255, 0), 2)

        # 绘制文字和置信度
        x, y = boxes[0]
        label = f"{text} ({confidence:.2f})"

        # 使用PIL绘制中文
        img_pil_text = Image.fromarray(cv2.cvtColor(img, cv2.COLOR_BGR2RGB))
        draw_text = ImageDraw.Draw(img_pil_text)
        draw_text.text((x, y-30), label, fill=(0, 255, 0), font=font)

        img = cv2.cvtColor(np.array(img_pil_text), cv2.COLOR_RGB2BGR)

    # 保存结果
    cv2.imwrite(output_path, img)
    print(f"结果已保存到 {output_path}")

# 使用示例
ocr = PaddleOCR(use_angle_cls=True, lang='ch')
result = ocr.ocr('exam_paper.jpg', cls=True)
visualize_ocr_result('exam_paper.jpg', result)
\end{lstlisting}

\subsubsection{2.4 结果提取(5分钟)}

\begin{lstlisting}
def extract_text_lines(ocr_result):
    """
    提取所有识别的文字行

    参数:
        ocr_result: OCR识别结果

    返回:
        文字行列表
    """
    text_lines = []

    for line in ocr_result:
        if line is None:
            continue
        text = line[1][0]
        confidence = line[1][1]

        # 过滤低置信度结果
        if confidence > 0.5:
            text_lines.append(text)

    return text_lines

def extract_full_text(ocr_result):
    """
    提取完整文本

    参数:
        ocr_result: OCR识别结果

    返回:
        完整文本字符串
    """
    text_lines = extract_text_lines(ocr_result)
    full_text = '\n'.join(text_lines)
    return full_text

# 使用
text_lines = extract_text_lines(result)
full_text = extract_full_text(result)

print("识别的文字行:")
for line in text_lines:
    print(line)

print("\n完整文本:")
print(full_text)
\end{lstlisting}

\subsection{环节三:试卷文字识别实战(40分钟)}

\subsubsection{3.1 识别试卷标题(10分钟)}

\begin{lstlisting}
def extract_exam_title(image_path):
    """
    提取试卷标题

    策略: 识别图像上方最大字号的文字
    """
    ocr = PaddleOCR(use_angle_cls=True, lang='ch')
    result = ocr.ocr(image_path, cls=True)

    if not result or not result[0]:
        return None

    # 计算每个文字框的高度(字号)
    boxes_with_height = []
    for line in result:
        boxes = line[0]
        text = line[1][0]

        # 计算高度(取左右两侧的平均)
        left_height = np.linalg.norm(np.array(boxes[0]) - np.array(boxes[3]))
        right_height = np.linalg.norm(np.array(boxes[1]) - np.array(boxes[2]))
        avg_height = (left_height + right_height) / 2

        boxes_with_height.append((text, avg_height, boxes))

    # 按高度排序,取最大的
    boxes_with_height.sort(key=lambda x: x[1], reverse=True)

    # 取前3个最大的文字作为候选标题
    title_candidates = boxes_with_height[:3]
    title = ' '.join([t[0] for t in title_candidates])

    return title

# 使用
title = extract_exam_title('exam_paper.jpg')
print(f"试卷标题: {title}")
\end{lstlisting}

\subsubsection{3.2 识别题号(15分钟)}

\begin{lstlisting}
import re

def extract_question_numbers(ocr_result):
    """
    从OCR结果中提取题号

    参数:
        ocr_result: OCR识别结果

    返回:
        题号列表 [(题号, 坐标), ...]
    """
    questions = []

    for line in ocr_result:
        if line is None:
            continue

        text = line[1][0]
        boxes = line[0]

        # 匹配题号模式: "1." "1、" "(1)" 等
        patterns = [
            r'^(\d+)[.、.]',  # 1. 1、1.
            r'^[((](\d+)[))]',  # (1) (1)
            r'^第?(\d+)题',  # 第1题 1题
        ]

        for pattern in patterns:
            match = re.match(pattern, text.strip())
            if match:
                question_num = int(match.group(1))
                # 计算中心点坐标
                center_x = int(np.mean([box[0] for box in boxes]))
                center_y = int(np.mean([box[1] for box in boxes]))
                questions.append((question_num, (center_x, center_y), boxes))
                break

    return questions

# 使用
questions = extract_question_numbers(result)
for num, center, boxes in questions:
    print(f"题号: {num}, 位置: {center}")
\end{lstlisting}

\subsubsection{3.3 按题号分割题目(15分钟)}

\begin{lstlisting}
def split_by_questions(image, ocr_result):
    """
    按题号分割试卷图像

    参数:
        image: 试卷图像
        ocr_result: OCR识别结果

    返回:
        题目图像列表
    """
    questions = extract_question_numbers(ocr_result)

    if len(questions) < 2:
        return [image]

    # 计算分割线(相邻题号的中点)
    split_lines = []
    for i in range(len(questions) - 1):
        curr_center = questions[i][1]
        next_center = questions[i+1][1]

        # Y方向的中点作为分割线
        split_y = (curr_center[1] + next_center[1]) // 2
        split_lines.append(split_y)

    # 分割图像
    height, width = image.shape[:2]
    question_images = []

    # 第一题:从顶部到第一个分割线
    y_prev = 0
    for split_y in split_lines:
        question_img = image[y_prev:split_y, 0:width]
        question_images.append(question_img)
        y_prev = split_y

    # 最后一题:从最后一个分割线到底部
    question_img = image[y_prev:height, 0:width]
    question_images.append(question_img)

    return question_images

# 使用
img = cv2.imread('exam_paper.jpg')
question_images = split_by_questions(img, result)

for i, q_img in enumerate(question_images):
    cv2.imwrite(f'question_{i+1}.jpg', q_img)
\end{lstlisting}

\subsection{环节四:OCR优化技巧(20分钟)}

\subsubsection{4.1 图像预处理优化(10分钟)}

\begin{lstlisting}
def preprocess_for_ocr(image):
    """
    专门为OCR优化的预处理

    参数:
        image: 输入图像

    返回:
        优化后的图像
    """
    # 转灰度
    if len(image.shape) == 3:
        gray = cv2.cvtColor(image, cv2.COLOR_BGR2GRAY)
    else:
        gray = image

    # 去噪
    denoised = cv2.fastNlMeansDenoising(gray)

    # 二值化(针对文档)
    _, binary = cv2.threshold(
        denoised, 0, 255,
        cv2.THRESH_BINARY + cv2.THRESH_OTSU
    )

    # 或者使用自适应阈值(针对光照不均)
    # binary = cv2.adaptiveThreshold(
    #     denoised, 255,
    #     cv2.ADAPTIVE_THRESH_GAUSSIAN_C,
    #     cv2.THRESH_BINARY, 11, 2
    # )

    return binary

# 使用
img = cv2.imread('exam_paper.jpg')
processed = preprocess_for_ocr(img)
cv2.imwrite('processed_for_ocr.jpg', processed)
\end{lstlisting}

\subsubsection{4.2 参数调优(10分钟)}

\begin{lstlisting}
# PaddleOCR参数说明
ocr = PaddleOCR(
    # 基础参数
    use_angle_cls=True,      # 启用文字方向分类
    lang='ch',               # 语言: ch(中文), en(英文), fr(法文)等

    # 模型选择
    det_model_dir=None,      # 检测模型路径(None使用默认)
    rec_model_dir=None,      # 识别模型路径
    cls_model_dir=None,      # 方向分类模型路径

    # 高级参数
    use_gpu=False,           # 是否使用GPU
    show_log=False,          # 是否显示日志

    # 检测参数
    det_db_thresh=0.3,       # 检测阈值
    det_db_box_thresh=0.6,   # 框选阈值

    # 识别参数
    rec_batch_num=6,         # 批处理大小

    # 方向分类参数
    cls_batch_num=6,         # 批处理大小
    cls_thresh=0.9           # 分类阈值
)

# 针对手写文字
ocr_handwrite = PaddleOCR(
    use_angle_cls=True,
    lang='ch',               # 中文手写
    det_model_dir=None,      # 可以加载专门的手写模型
)

# 针对英文
ocr_english = PaddleOCR(
    use_angle_cls=True,
    lang='en'
)
\end{lstlisting}

\subsection{环节五:综合实验(10分钟)}

\begin{lstlisting}
# 完整示例:识别试卷中的所有印刷文字
from paddleocr import PaddleOCR
import cv2

def recognize_exam_paper(image_path):
    """识别试卷中的所有文字"""
    # 初始化OCR
    ocr = PaddleOCR(use_angle_cls=True, lang='ch', show_log=False)

    # 读取图像
    img = cv2.imread(image_path)

    # 预处理
    processed = preprocess_for_ocr(img)

    # 识别
    result = ocr.ocr(processed, cls=True)

    # 提取信息
    title = extract_exam_title(processed)
    questions = extract_question_numbers(result)
    text_lines = extract_text_lines(result)

    # 输出结果
    print(f"试卷标题: {title}")
    print(f"\n共识别到 {len(questions)} 道题目")
    print(f"\n文字内容:")
    for i, line in enumerate(text_lines, 1):
        print(f"{i}. {line}")

    # 可视化
    visualize_ocr_result(image_path, result, 'ocr_result.jpg')

    return {
        'title': title,
        'questions': questions,
        'text_lines': text_lines,
        'full_result': result
    }

# 使用
result = recognize_exam_paper('exam_paper.jpg')
\end{lstlisting}

\section{课后作业}

\subsection{作业内容}

\textbf{题目:** 用OCR识别试卷中的印刷文字}

\textbf{要求:}
\begin{enumerate}
\item 安装配置PaddleOCR
\item 识别试卷标题和题号
\item 提取所有文字内容
\item 可视化标注识别结果
\end{enumerate}

\subsection{评分标准}

\begin{table}[htbp]
\centering
\begin{tabular}{|l|p{8cm}|p{2cm}|}
\hline
\textbf{评分项} & \textbf{标准} & \textbf{分值} \\
\hline
环境配置 & 成功安装运行PaddleOCR & 20分 \\
\hline
识别效果 & 正确识别试卷文字 & 40分 \\
\hline
信息提取 & 准确提取题号、标题 & 25分 \\
\hline
可视化 & 清晰标注识别结果 & 15分 \\
\hline
\textbf{合计} & & \textbf{100分} \\
\hline
\end{tabular}
\end{table}

\section{教学反思}

\subsection{重点提醒}

\begin{itemize}
\item OCR需要联网下载模型,首次运行较慢
\item 印刷文字识别准确率较高,手写文字是下周内容
\item 图像质量对OCR结果影响很大
\end{itemize}

\subsection{下节预告}

下周将学习手写文字识别,使用TrOCR等专门的手写识别模型。

\end{document}
