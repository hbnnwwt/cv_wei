\documentclass[12pt,a4paper]{article}
\usepackage[UTF8]{ctex}
\usepackage{geometry}
\usepackage{amsmath}
\usepackage{graphicx}
\usepackage{booktabs}
\usepackage{array}
\usepackage{enumitem}
\usepackage{xcolor}
\usepackage{tcolorbox}
\usepackage{listings}
\usepackage{hyperref}

\geometry{left=2.5cm,right=2.5cm,top=2.5cm,bottom=2.5cm}

\lstset{
    language=Python,
    basicstyle=\ttfamily\small,
    keywordstyle=\color{blue},
    commentstyle=\color{green!60!black},
    stringstyle=\color{orange},
    breaklines=true,
    frame=single,
    showstringspaces=false
}

\title{\textbf{\large 第5周教案:选择题识别(填涂检测)}}
\author{计算机视觉课程组}
\date{}

\begin{document}

\maketitle

\section{基本信息}

\begin{tabular}{|l|p{12cm}|}
\hline
\textbf{周次} & 第5周 \\
\hline
\textbf{主题} & 选择题识别(填涂检测) \\
\hline
\textbf{学时} & 3学时(160分钟) \\
\hline
\textbf{故事问题} & 怎么知道选了A还是B? \\
\hline
\textbf{OBE目标} & A3-特征识别:能识别填涂答案 \\
\hline
\textbf{项目贡献} & 实现选择题自动识别模块 \\
\hline
\end{tabular}

\section{教学目标}

\begin{enumerate}
\item \textbf{知识目标}:
    \begin{itemize}
    \item 理解OMR(光学标记识别)原理
    \item 掌握像素密度统计方法
    \item 了解多选项处理策略
    \end{itemize}

\item \textbf{能力目标}:
    \begin{itemize}
    \item 能够定位填涂选项位置
    \item 能够判断选项是否被填涂
    \item 能够识别单选和多选题型
    \end{itemize}

\item \textbf{素养目标}:
    \begin{itemize}
    \item 理解阈值对识别结果的影响
    \item 培养参数调优的实验能力
    \end{itemize}
\end{enumerate}

\section{教学重点与难点}

\begin{tcolorbox}[colback=red!5!white,colframe=red!75!black,title=\textbf{教学重点}]
\begin{itemize}
\item OMR原理:基于像素密度的填涂检测
\item 选项区域定位与分割
\item 填涂判断与答案识别
\end{itemize}
\end{tcolorbox}

\begin{tcolorbox}[colback=yellow!5!white,colframe=yellow!75!black,title=\textbf{教学难点}]
\begin{itemize}
\item 阈值选择(区分填涂与未填涂)
\item 多选题识别
\end{itemize}
\end{tcolorbox}

\section{教学过程设计}

\subsection{环节一:OMR原理介绍(30分钟)}

\subsubsection{1.1 什么是OMR?(10分钟)}

\textbf{定义:** Optical Mark Recognition,光学标记识别}

\textbf{应用场景:**
\begin{itemize}
\item 标准化考试答题卡
\item 问卷调查
\item 彩票选号
\end{itemize}

\textbf{核心思想:** 填涂区域与空白区域的像素密度差异}

\begin{center}
填涂区域 $\to$ 深色像素多 $\to$ 密度高 \\
空白区域 $\to$ 浅色像素多 $\to$ 密度低
\end{center}

\subsubsection{1.2 填涂识别流程(10分钟)}

\begin{enumerate}
\item \textbf{定位选项区域}:找到每个选项的位置
\item \textbf{提取选项图像}:裁剪出单个选项区域
\item \textbf{二值化处理}:转换为黑白图像
\item \textbf{统计像素密度}:计算深色像素占比
\item \textbf{阈值判断}:密度超过阈值则认为已填涂
\end{enumerate}

\subsubsection{1.3 数据准备(10分钟)}

\textbf{标准答题卡格式:**
\begin{itemize}
\item 选项排列:横向或纵向
\item 选项数量:通常4个(A/B/C/D)
\item 选项间距:固定距离
\item 选项尺寸:固定大小
\end{itemize}

\textbf{示例答题卡:}

\begin{verbatim}
题目1: [A] [B] [C] [D]
题目2: [A] [B] [C] [D]
题目3: [A] [B] [C] [D]
\end{verbatim}

\subsection{环节二:选项区域定位(30分钟)}

\subsubsection{2.1 固定位置法(10分钟)}

\textbf{适用场景:** 标准答题卡,位置固定}

\begin{lstlisting}
# 假设选项位置固定,预先定义
OPTION_SIZE = 30  # 选项大小(像素)
OPTION_SPACING = 50  # 选项间距(像素)
START_X = 100  # 起始X坐标
START_Y = 200  # 起始Y坐标

def get_option_positions(num_questions=10, num_options=4):
    """
    获取所有选项的位置(固定格式)

    参数:
        num_questions: 题目数量
        num_options: 每题选项数量

    返回:
        选项位置列表 [[(x1,y1,w1,h1), (x2,y2,w2,h2), ...], ...]
    """
    positions = []
    for q in range(num_questions):
        question_options = []
        for o in range(num_options):
            x = START_X + o * OPTION_SPACING
            y = START_Y + q * OPTION_SPACING
            question_options.append((x, y, OPTION_SIZE, OPTION_SIZE))
        positions.append(question_options)

    return positions
\end{lstlisting}

\subsubsection{2.2 基于轮廓的定位(20分钟)}

\textbf{适用场景:** 选项位置不完全固定}

\begin{lstlisting}
def find_option_bubbles(binary_image, min_area=100, max_area=1000):
    """
    通过连通域分析找到选项气泡

    参数:
        binary_image: 二值图像
        min_area: 最小面积
        max_area: 最大面积

    返回:
        选项区域列表 [(x, y, w, h), ...]
    """
    # 连通域分析
    num_labels, labels, stats, centroids = cv2.connectedComponentsWithStats(
        binary_image
    )

    bubbles = []
    for i in range(1, num_labels):
        area = stats[i, cv2.CC_STAT_AREA]

        # 面积筛选
        if min_area <= area <= max_area:
            # 获取边界矩形
            x = stats[i, cv2.CC_STAT_LEFT]
            y = stats[i, cv2.CC_STAT_TOP]
            w = stats[i, cv2.CC_STAT_WIDTH]
            h = stats[i, cv2.CC_STAT_HEIGHT]

            # 长宽比筛选(气泡通常接近圆形)
            aspect_ratio = float(w) / h if h > 0 else 0
            if 0.5 < aspect_ratio < 2.0:
                bubbles.append((x, y, w, h))

    return bubbles

def group_bubbles_by_question(bubbles, questions=10, options=4):
    """
    将气泡按题目分组

    参数:
        bubbles: 所有气泡位置
        questions: 题目数量
        options: 每题选项数

    返回:
        分组后的气泡位置
    """
    # 按Y坐标排序(题目顺序)
    bubbles_sorted = sorted(bubbles, key=lambda b: b[1])

    # 简单分组:假设每题的气泡Y坐标相近
    groups = []
    for i in range(questions):
        group = bubbles_sorted[i*options : (i+1)*options]
        # 每组内按X坐标排序(A/B/C/D顺序)
        group = sorted(group, key=lambda b: b[0])
        groups.append(group)

    return groups
\end{lstlisting}

\subsection{环节三:填涂检测核心算法(50分钟)}

\subsubsection{3.1 像素密度统计(20分钟)}

\textbf{方法1:非零像素计数}

\begin{lstlisting}
def calculate_density_1(option_roi, threshold=127):
    """
    方法1:统计非零像素数量

    参数:
        option_roi: 选项区域图像(灰度图)
        threshold: 二值化阈值

    返回:
        填涂密度(0-1之间)
    """
    # 二值化(反色:填涂变白,空白变黑)
    _, binary = cv2.threshold(
        option_roi,
        threshold,
        255,
        cv2.THRESH_BINARY_INV
    )

    # 统计非零像素(白色像素)
    non_zero = cv2.countNonZero(binary)

    # 计算密度
    total_pixels = binary.shape[0] * binary.shape[1]
    density = float(non_zero) / total_pixels

    return density
\end{lstlisting}

\textbf{方法2:像素均值}

\begin{lstlisting}
def calculate_density_2(option_roi):
    """
    方法2:计算像素均值

    参数:
        option_roi: 选项区域图像(灰度图)

    返回:
        填涂密度(0-1之间)
    """
    # 填涂区域像素值低(接近0)
    # 空白区域像素值高(接近255)

    # 计算均值
    mean_value = cv2.mean(option_roi)[0]

    # 转换为密度(填涂越多,均值越低)
    density = 1.0 - (mean_value / 255.0)

    return density
\end{lstlisting}

\textbf{方法3:中心区域统计(更鲁棒)}

\begin{lstlisting}
def calculate_density_3(option_roi, center_ratio=0.6):
    """
    方法3:只统计中心区域(边缘可能不整齐)

    参数:
        option_roi: 选项区域图像
        center_ratio: 中心区域比例

    返回:
        填涂密度
    """
    h, w = option_roi.shape

    # 计算中心区域
    margin_h = int(h * (1 - center_ratio) / 2)
    margin_w = int(w * (1 - center_ratio) / 2)

    center = option_roi[margin_h:h-margin_h, margin_w:w-margin_w]

    # 统计中心区域
    _, binary = cv2.threshold(center, 127, 255, cv2.THRESH_BINARY_INV)
    non_zero = cv2.countNonZero(binary)
    total_pixels = center.shape[0] * center.shape[1]

    density = float(non_zero) / total_pixels

    return density
\end{lstlisting}

\subsubsection{3.2 阈值确定(15分钟)}

\textbf{固定阈值法:**

\begin{lstlisting}
def is_filled_fixed(density, threshold=0.3):
    """
    固定阈值判断

    参数:
        density: 填涂密度
        threshold: 阈值(通常0.2-0.4)

    返回:
        是否填涂
    """
    return density >= threshold
\end{lstlisting}

\textbf{自适应阈值法:**

\begin{lstlisting}
def determine_adaptive_threshold(densities):
    """
    自适应确定阈值(基于所有选项的密度分布)

    参数:
        densities: 所有选项的密度列表

    返回:
        最佳阈值
    """
    densities = np.array(densities)

    # 方法1:使用均值
    threshold = np.mean(densities)

    # 方法2:使用大津法思想
    # 找到密度分布的"谷底"
    hist, bins = np.histogram(densities, bins=50)
    # 找到直方图的最低点作为阈值
    threshold = bins[np.argmin(hist) + 1]

    return threshold
\end{lstlisting}

\subsubsection{3.3 单选题识别(15分钟)}

\begin{lstlisting}
def recognize_single_choice(question_image, option_positions):
    """
    识别单选题答案

    参数:
        question_image: 题目区域图像
        option_positions: 选项位置列表 [(x,y,w,h), ...]

    返回:
        答案('A', 'B', 'C', 'D' 或 None)
    """
    densities = []

    # 计算每个选项的密度
    for i, (x, y, w, h) in enumerate(option_positions):
        roi = question_image[y:y+h, x:x+w]
        density = calculate_density_1(roi)
        densities.append(density)

    # 自适应确定阈值
    threshold = determine_adaptive_threshold(densities)

    # 找到密度最高的选项
    max_density = max(densities)

    # 只有最高密度超过阈值才认为有效
    if max_density >= threshold:
        answer_idx = densities.index(max_density)
        return chr(ord('A') + answer_idx)

    return None  # 未填涂
\end{lstlisting}

\subsection{环节四:完整选择题识别模块(30分钟)}

\begin{lstlisting}
class ChoiceRecognizer:
    """选择题识别器"""

    def __init__(self, threshold=0.3):
        """
        初始化

        参数:
            threshold: 填涂判断阈值
        """
        self.threshold = threshold
        self.option_labels = ['A', 'B', 'C', 'D']

    def preprocess(self, image):
        """预处理图像"""
        gray = cv2.cvtColor(image, cv2.COLOR_BGR2GRAY)
        blur = cv2.GaussianBlur(gray, (5, 5), 0)
        return blur

    def calculate_density(self, roi):
        """计算填涂密度"""
        _, binary = cv2.threshold(roi, 127, 255, cv2.THRESH_BINARY_INV)
        non_zero = cv2.countNonZero(binary)
        total = roi.shape[0] * roi.shape[1]
        return float(non_zero) / total

    def recognize_question(self, question_img, option_positions):
        """
        识别单道选择题

        参数:
            question_img: 题目图像
            option_positions: 选项位置列表

        返回:
            (答案, 密度列表)
        """
        densities = []
        filled_options = []

        for x, y, w, h in option_positions:
            roi = question_img[y:y+h, x:x+w]
            density = self.calculate_density(roi)
            densities.append(density)

            if density >= self.threshold:
                filled_options.append(self.option_labels[len(filled_options)])

        # 单选题:只能有一个选项
        if len(filled_options) == 1:
            return filled_options[0], densities
        elif len(filled_options) > 1:
            # 多选,取密度最高的
            max_idx = densities.index(max(densities))
            return self.option_labels[max_idx], densities
        else:
            return None, densities

    def recognize_all(self, image, question_groups):
        """
        识别所有选择题

        参数:
            image: 完整答题卡图像
            question_groups: 所有题目的选项位置分组

        返回:
            答案列表
        """
        processed = self.preprocess(image)
        answers = []

        for i, option_positions in enumerate(question_groups):
            answer, densities = self.recognize_question(processed, option_positions)
            answers.append({
                'question': i + 1,
                'answer': answer,
                'densities': densities
            })

        return answers

    def visualize(self, image, answers, question_groups):
        """可视化识别结果"""
        result = image.copy()

        for i, (ans, positions) in enumerate(zip(answers, question_groups)):
            for j, (x, y, w, h) in enumerate(positions):
                label = chr(ord('A') + j)

                # 判断是否填涂
                if ans['answer'] == label:
                    color = (0, 255, 0)  # 绿:正确答案
                    thickness = 3
                else:
                    color = (0, 0, 255)  # 红:未填涂
                    thickness = 1

                cv2.rectangle(result, (x, y), (x+w, y+h), color, thickness)

                # 标签
                cv2.putText(
                    result, label, (x, y-5),
                    cv2.FONT_HERSHEY_SIMPLEX, 0.5, color, 1
                )

        return result
\end{lstlisting}

\subsection{环节五:实验与测试(20分钟)}

\begin{lstlisting}
# 使用示例
import cv2

# 读取答题卡图像
image = cv2.imread('answer_sheet.jpg')

# 创建识别器
recognizer = ChoiceRecognizer(threshold=0.3)

# 定义选项位置(实际需要自动检测)
# 这里简化为固定位置
option_groups = get_option_positions(num_questions=10, num_options=4)

# 识别所有答案
answers = recognizer.recognize_all(image, option_groups)

# 打印结果
for ans in answers:
    print(f"题目{ans['question']}: {ans['answer']}")

# 可视化
result = recognizer.visualize(image, answers, option_groups)
cv2.imwrite('result.jpg', result)
\end{lstlisting}

\section{课后作业}

\subsection{作业内容}

\textbf{题目:** 实现选择题填涂识别模块}

\textbf{要求:**
\begin{enumerate}
\item 实现选项区域定位(固定或自动)
\item 实现填涂密度计算
\item 识别单选题答案
\item 可视化标注识别结果
\end{enumerate}

\subsection{评分标准}

\begin{table}[htbp]
\centering
\begin{tabular}{|l|p{8cm}|p{2cm}|}
\hline
\textbf{评分项} & \textbf{标准} & \textbf{分值} \\
\hline
选项定位 & 准确定位所有选项区域 & 25分 \\
\hline
密度计算 & 正确计算填涂密度 & 25分 \\
\hline
答案识别 & 正确识别填涂答案 & 30分 \\
\hline
可视化 & 清晰标注识别结果 & 10分 \\
\hline
代码质量 & 结构清晰,有注释 & 10分 \\
\hline
\textbf{合计} & & \textbf{100分} \\
\hline
\end{tabular}
\end{table}

\section{教学反思}

\subsection{重点提醒}

\begin{itemize}
\item 阈值选择是关键,需要根据实际答题卡调整
\item 中心区域统计比全区域统计更鲁棒
\item 下周将学习判断题识别,技术类似但关注形状特征
\end{itemize}

\end{document}
