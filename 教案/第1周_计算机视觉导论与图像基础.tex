\documentclass[12pt,a4paper]{article}
\usepackage[UTF8]{ctex}
\usepackage{geometry}
\usepackage{amsmath}
\usepackage{graphicx}
\usepackage{booktabs}
\usepackage{array}
\usepackage{enumitem}
\usepackage{xcolor}
\usepackage{tcolorbox}
\usepackage{listings}
\usepackage{hyperref}

\geometry{left=2.5cm,right=2.5cm,top=2.5cm,bottom=2.5cm}

\lstset{
    language=Python,
    basicstyle=\ttfamily\small,
    keywordstyle=\color{blue},
    commentstyle=\color{green!60!black},
    stringstyle=\color{orange},
    breaklines=true,
    frame=single,
    showstringspaces=false
}

\title{\textbf{\large 第1周教案:计算机视觉导论与图像基础}}
\author{计算机视觉课程组}
\date{}

\begin{document}

\maketitle

% ============================================================================
% 基本信息
% ============================================================================
\section{基本信息}

\begin{tabular}{|l|p{12cm}|}
\hline
\textbf{周次} & 第1周 \\
\hline
\textbf{主题} & 计算机视觉导论与图像基础 \\
\hline
\textbf{学时} & 3学时(160分钟) \\
\hline
\textbf{故事问题} & 机器是怎么“看见”试卷的? \\
\hline
\textbf{OBE目标} & A1-图像处理:能对图像进行读取、显示、保存操作 \\
\hline
\end{tabular}

\section{教学目标}

\begin{enumerate}
\item \textbf{知识目标}:
    \begin{itemize}
    \item 理解计算机视觉的定义与应用领域
    \item 掌握图像的数字表示方法(像素、RGB、灰度图)
    \end{itemize}

\item \textbf{能力目标}:
    \begin{itemize}
    \item 能够使用OpenCV读取、显示、保存图像
    \item 能够访问和修改图像的像素值
    \item 能够实现基础的图像滤镜效果
    \end{itemize}

\item \textbf{素养目标}:
    \begin{itemize}
    \item 建立对计算机视觉课程的兴趣和期待
    \item 理解从图像到数字的转换思维
    \end{itemize}
\end{enumerate}

\section{教学重点与难点}

\begin{tcolorbox}[colback=red!5!white,colframe=red!75!black,title=\textbf{教学重点}]
\begin{itemize}
\item 图像的数字表示:像素与RGB色彩空间
\item OpenCV基础操作:imread、imshow、imwrite
\item 图像数据结构:numpy数组
\end{itemize}
\end{tcolorbox}

\begin{tcolorbox}[colback=yellow!5!white,colframe=yellow!75!black,title=\textbf{教学难点}]
\begin{itemize}
\item RGB通道的理解与分离
\item 图像数组索引与切片操作
\end{itemize}
\end{tcolorbox}

% ============================================================================
% 教学过程
% ============================================================================
\section{教学过程设计}

\subsection{环节一:课程导论(40分钟)}

\subsubsection{1.1 什么是计算机视觉?(10分钟)}

\textbf{引入问题:}
\begin{itemize}
\item 人眼是如何看到东西的?
\item 机器能不能像人一样“看见”世界?
\end{itemize}

\textbf{CV定义:}
\begin{quote}
    计算机视觉是一门研究如何使机器“看”的科学,指用摄影机和电脑代替人眼对目标进行识别、跟踪和测量等机器视觉,并进一步做图形处理。
\end{quote}

\subsubsection{1.2 CV应用案例展示(20分钟)}

\begin{table}[htbp]
\centering
\begin{tabular}{|l|p{8cm}|}
\hline
\textbf{领域} & \textbf{应用案例} \\
\hline
安防监控 & 人脸识别、行为分析、异常检测 \\
\hline
自动驾驶 & 车道检测、障碍物识别、交通标志识别 \\
\hline
医疗影像 & 肿瘤检测、病灶定位、影像辅助诊断 \\
\hline
工业检测 & 缺陷检测、尺寸测量、质量分拣 \\
\hline
\textbf{教育领域} & \textbf{自动阅卷系统、作业批改} \\
\hline
\end{tabular}
\end{table}

\textbf{演示:} 播放CV应用短视频,展示人脸识别、自动驾驶等实际应用

\subsubsection{1.3 课程介绍(10分钟)}

\textbf{课程目标:} 造一个能改卷子的AI助教

\textbf{课程路线图:}
\begin{enumerate}
\item 装上眼睛 —— 图像基础
\item 学会识别 —— 三种题型(选择、判断、简答)
\item 成为助教 —— 完整阅卷系统
\end{enumerate}

\subsection{环节二:图像的数字表示(50分钟)}

\subsubsection{2.1 图像是什么?(10分钟)}

\textbf{物理图像 $\to$ 数字图像}

\begin{itemize}
\item 物理图像:连续的光信号分布
\item 数字图像:离散的像素点阵
\end{itemize}

\textbf{类比:} 像马赛克拼图,每个小格子就是一个像素

\subsubsection{2.2 像素与分辨率(10分钟)}

\textbf{概念:}
\begin{itemize}
\item 像素(Pixel):图像的最小单位
\item 分辨率:图像的尺寸(宽$\times$高)
    \begin{itemize}
    \item 例:1920$\times$1080 表示宽1920像素,高1080像素
    \end{itemize}
\end{itemize}

\textbf{互动问题:} 手机拍照的像素越高越好吗?

\subsubsection{2.3 RGB色彩空间(20分钟)}

\textbf{三原色原理:}
\begin{itemize}
\item R(Red):红色通道
\item G(Green):绿色通道
\item B(Blue):蓝色通道
\end{itemize}

\textbf{数值范围:} 0-255(8位无符号整数)
\begin{itemize}
\item 0 = 最暗
\item 255 = 最亮
\end{itemize}

\textbf{混合示例:}
\begin{table}[htbp]
\centering
\begin{tabular}{|c|c|c|c|}
\hline
R & G & B & 颜色 \\
\hline
255 & 0 & 0 & 红色 \\
\hline
0 & 255 & 0 & 绿色 \\
\hline
0 & 0 & 255 & 蓝色 \\
\hline
255 & 255 & 255 & 白色 \\
\hline
0 & 0 & 0 & 黑色 \\
\hline
255 & 255 & 0 & 黄色 \\
\hline
\end{tabular}
\end{table}

\textbf{灰度图:} 单通道图像,0=黑,255=白

\subsubsection{2.4 Jupyter实验:查看图像数据(10分钟)}

\begin{lstlisting}[caption=查看图像基本信息]
import cv2
import numpy as np

# 读取图像
img = cv2.imread('test.jpg')

# 查看图像信息
print(f"图像尺寸: {img.shape}")  # (height, width, channels)
print(f"数据类型: {img.dtype}")  # uint8
print(f"总像素数: {img.size}")

# 访问单个像素
pixel = img[100, 100]  # [B, G, R]
print(f"像素值: {pixel}")
\end{lstlisting}

\subsection{环节三:OpenCV入门(40分钟)}

\subsubsection{3.1 OpenCV简介(5分钟)}

\begin{itemize}
\item 全称:Open Source Computer Vision Library
\item 特点:开源、跨平台、功能强大
\item 语言支持:C++、Python、Java等
\end{itemize}

\subsubsection{3.2 环境搭建(10分钟)}

\textbf{安装OpenCV:}
\begin{lstlisting}
pip install opencv-python
pip install opencv-python-headless  # 无GUI环境
pip install numpy matplotlib
\end{lstlisting}

\textbf{Jupyter环境配置:}
\begin{lstlisting}
# 在Jupyter中显示图像
import cv2
import matplotlib.pyplot as plt

# OpenCV读取的是BGR格式,需要转换为RGB
img = cv2.imread('test.jpg')
img_rgb = cv2.cvtColor(img, cv2.COLOR_BGR2RGB)

plt.imshow(img_rgb)
plt.axis('off')
plt.show()
\end{lstlisting}

\subsubsection{3.3 基础操作(25分钟)}

\textbf{1. 读取图像(10分钟):}
\begin{lstlisting}
# 读取彩色图像
img_color = cv2.imread('test.jpg', cv2.IMREAD_COLOR)

# 读取灰度图像
img_gray = cv2.imread('test.jpg', cv2.IMREAD_GRAYSCALE)

# 读取原始图像(包含alpha通道)
img_alpha = cv2.imread('test.png', cv2.IMREAD_UNCHANGED)
\end{lstlisting}

\textbf{2. 显示图像(10分钟):}
\begin{lstlisting}
# 方法1:使用cv2.imshow(本地运行)
cv2.imshow('Image', img)
cv2.waitKey(0)  # 等待按键
cv2.destroyAllWindows()

# 方法2:使用matplotlib(Jupyter推荐)
import matplotlib.pyplot as plt

plt.figure(figsize=(10, 6))
plt.subplot(1, 2, 1)
plt.imshow(cv2.cvtColor(img, cv2.COLOR_BGR2RGB))
plt.title('Color Image')

plt.subplot(1, 2, 2)
plt.imshow(img_gray, cmap='gray')
plt.title('Gray Image')
plt.show()
\end{lstlisting}

\textbf{3. 保存图像(5分钟):}
\begin{lstlisting}
# 保存图像
cv2.imwrite('output.jpg', img)
cv2.imwrite('output.png', img)  # PNG格式支持透明通道
\end{lstlisting}

\subsection{环节四:动手实验——给试卷加滤镜(30分钟)}

\subsubsection{4.1 任务说明(5分钟)}

\textbf{任务目标:}
\begin{enumerate}
\item 读取一张试卷图片
\item 实现3种滤镜效果
\item 保存并展示结果
\end{enumerate}

\subsubsection{4.2 滤镜实现(20分钟)}

\textbf{滤镜1:灰度化}
\begin{lstlisting}
def to_gray(image):
    """将图像转换为灰度图"""
    return cv2.cvtColor(image, cv2.COLOR_BGR2GRAY)
\end{lstlisting}

\textbf{滤镜2:反色效果}
\begin{lstlisting}
def invert_color(image):
    """反色效果:255 - 原值"""
    return 255 - image
\end{lstlisting}

\textbf{滤镜3:亮度调整}
\begin{lstlisting}
def adjust_brightness(image, value=50):
    """调整亮度:加上固定值"""
    # 确保值在0-255范围内
    new_image = image.astype(int) + value
    new_image = np.clip(new_image, 0, 255).astype(uint8)
    return new_image
\end{lstlisting}

\textbf{完整示例:}
\begin{lstlisting}
import cv2
import numpy as np
import matplotlib.pyplot as plt

# 读取试卷图像
img = cv2.imread('exam_paper.jpg')

# 应用滤镜
gray = to_gray(img)
inverted = invert_color(img)
bright = adjust_brightness(img, 50)

# 展示结果
fig, axes = plt.subplots(2, 2, figsize=(12, 10))

axes[0, 0].imshow(cv2.cvtColor(img, cv2.COLOR_BGR2RGB))
axes[0, 0].set_title('原图')
axes[0, 0].axis('off')

axes[0, 1].imshow(gray, cmap='gray')
axes[0, 1].set_title('灰度图')
axes[0, 1].axis('off')

axes[1, 0].imshow(cv2.cvtColor(inverted, cv2.COLOR_BGR2RGB))
axes[1, 0].set_title('反色')
axes[1, 0].axis('off')

axes[1, 1].imshow(cv2.cvtColor(bright, cv2.COLOR_BGR2RGB))
axes[1, 1].set_title('提亮')
axes[1, 1].axis('off')

plt.tight_layout()
plt.savefig('filter_results.png', dpi=150)
plt.show()
\end{lstlisting}

\subsubsection{4.3 学生练习(5分钟)}

\textbf{练习任务:}
\begin{enumerate}
\item 修改亮度调整的值,观察效果变化
\item 尝试实现"变暗"滤镜
\end{enumerate}

% ============================================================================
% 课后作业
% ============================================================================
\section{课后作业}

\subsection{作业内容}

\textbf{题目:} 用OpenCV实现3种图像滤镜效果

\textbf{项目关联:} 图像滤镜是自动阅卷系统中预处理的基础。例如:灰度化可以简化后续处理,反色可以增强某些特征,对比度调整可以提高识别准确率。通过本作业,学生将为后续的图像预处理模块打下基础。

\textbf{要求:}
\begin{enumerate}
\item 必须包含:灰度化、反色
\item 自选一种:亮度调整、对比度调整、模糊等
\item 提交:代码 + 处理前后对比图
\end{enumerate}

\subsection{评分标准}

\begin{table}[htbp]
\centering
\begin{tabular}{|l|p{8cm}|p{2cm}|}
\hline
\textbf{评分项} & \textbf{标准} & \textbf{分值} \\
\hline
代码正确性 & 能正常运行无错误 & 30分 \\
\hline
滤镜实现 & 实现3种滤镜,效果明显 & 40分 \\
\hline
代码规范 & 有注释、结构清晰 & 15分 \\
\hline
结果展示 & 对比图清晰可见 & 15分 \\
\hline
\textbf{合计} & & \textbf{100分} \\
\hline
\end{tabular}
\end{table}

\subsection{提交方式}

\begin{itemize}
\item 截止时间:下周上课前
\item 提交格式:学号\_姓名\_week1.zip
\item 包含内容:Python代码文件 + 结果图片
\end{itemize}

% ============================================================================
% 教学反思
% ============================================================================
\section{教学反思与改进}

\subsection{预期问题}

\begin{enumerate}
\item \textbf{环境问题:} 学生可能安装OpenCV失败
    \begin{itemize}
    \item 应对:提供详细安装文档和常见问题FAQ
    \item 备用:准备在线运行环境(如Google Colab)
    \end{itemize}

\item \textbf{编程基础差异:} 部分学生Python基础薄弱
    \begin{itemize}
    \item 应对:提供完整可运行代码,学生只需修改参数
    \item 准备:Python基础速查手册
    \end{itemize}

\item \textbf{时间控制:} 实验环节可能超时
    \begin{itemize}
    \item 应对:控制演示时间,确保学生有足够动手时间
    \item 预留:课后答疑时间
    \end{itemize}
\end{enumerate}

\subsection{改进记录}

\begin{itemize}
\item[$\square$] 学生反馈收集
\item[$\square$] 作业完成情况统计
\item[$\square$] 下周教案调整
\end{itemize}

\end{document}
