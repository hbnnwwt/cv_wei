\documentclass[12pt,a4paper]{article}
\usepackage[UTF8]{ctex}
\usepackage{geometry}
\usepackage{amsmath}
\usepackage{graphicx}
\usepackage{booktabs}
\usepackage{array}
\usepackage{enumitem}
\usepackage{xcolor}
\usepackage{tcolorbox}
\usepackage{listings}
\usepackage{hyperref}

\geometry{left=2.5cm,right=2.5cm,top=2.5cm,bottom=2.5cm}

\lstset{
    language=Python,
    basicstyle=\ttfamily\small,
    keywordstyle=\color{blue},
    commentstyle=\color{green!60!black},
    stringstyle=\color{orange},
    breaklines=true,
    frame=single,
    showstringspaces=false
}

\title{\textbf{\large 第9周教案:系统架构与分组开发}}
\author{计算机视觉课程组}
\date{}

\begin{document}

\maketitle

\section{基本信息}

\begin{tabular}{|l|p{12cm}|}
\hline
\textbf{周次} & 第9周 \\
\hline
\textbf{主题} & 系统架构与分组开发 \\
\hline
\textbf{学时} & 3学时(160分钟) \\
\hline
\textbf{故事问题} & 把所有模块组合起来 \\
\hline
\textbf{OBE目标} & A4-系统集成:能将各模块整合为完整系统 \\
\hline
\textbf{项目阶段} & 架构设计 + 分组开发启动 \\
\hline
\end{tabular}

\section{教学目标}

\begin{enumerate}
\item \textbf{知识目标}:
    \begin{itemize}
    \item 理解软件系统的模块化架构
    \item 掌握系统集成的方法
    \item 了解项目开发流程
    \end{itemize}

\item \textbf{能力目标}:
    \begin{itemize}
    \item 能够设计系统架构
    \item 能够整合各功能模块
    \item 能够进行团队协作开发
    \end{itemize}

\item \textbf{素养目标}:
    \begin{itemize}
    \item 培养工程化思维
    \item 提升团队协作能力
    \item 建立项目管理意识
    \end{itemize}
\end{enumerate}

\section{教学重点与难点}

\begin{tcolorbox}[colback=red!5!white,colframe=red!75!black,title=\textbf{教学重点}]
\begin{itemize}
\item 系统架构设计
\item 模块接口定义
\item 代码集成方法
\end{itemize}
\end{tcolorbox}

\begin{tcolorbox}[colback=yellow!5!white,colframe=yellow!75!black,title=\textbf{教学难点}]
\begin{itemize}
\item 模块间的数据流转
\item 错误处理与调试
\end{itemize}
\end{tcolorbox}

\section{教学过程设计}

\subsection{环节一:项目回顾与架构设计(40分钟)}

\subsubsection{1.1 已学模块回顾(15分钟)}

\textbf{我们已实现的模块:}

\begin{table}[htbp]
\centering
\begin{tabular}{|l|l|p{6cm}|}
\hline
\textbf{周次} & \textbf{模块} & \textbf{功能} \\
\hline
第1-2周 & 基础工具 & OpenCV基础、AI辅助编程 \\
\hline
第3周 & 图像预处理 & 去噪、二值化、透视矫正 \\
\hline
第4周 & 版面分析 & 边缘检测、轮廓检测、区域定位 \\
\hline
第5周 & 选择题识别 & 填涂检测、OMR识别 \\
\hline
第6周 & 判断题识别 & 符号匹配、形状识别 \\
\hline
第7-8周 & 文字识别 & OCR印刷识别、手写识别 \\
\hline
\end{tabular}
\end{table}

\subsubsection{1.2 系统架构设计(25分钟)}

\textbf{自动阅卷系统架构:}

\begin{center}
\begin{tikzpicture}[node distance=2cm]
\node[draw, rectangle] (input) {输入:试卷图像};
\node[draw, rectangle, below of=input] (preprocess) {预处理模块};
\node[draw, rectangle, below of=preprocess] (layout) {版面分析模块};
\node[draw, rectangle, below of=layout, align=center] (recognize) {识别模块\\选择题\\判断题\\简答题};
\node[draw, rectangle, below of=recognize] (grade) {评分模块};
\node[draw, rectangle, below of=grade] (output) {输出:评分报告};

\draw[->] (input) -- (preprocess);
\draw[->] (preprocess) -- (layout);
\draw[->] (layout) -- (recognize);
\draw[->] (recognize) -- (grade);
\draw[->] (grade) -- (output);
\end{tikzpicture}
\end{center}

\textbf{模块接口设计:}

\begin{table}[htbp]
\centering
\begin{tabular}{|l|p{4cm}|p{4cm}|p{3cm}|}
\hline
\textbf{模块} & \textbf{输入} & \textbf{输出} & \textbf{接口} \\
\hline
预处理 & 原始图像 & 预处理图像 & \texttt{preprocess(img)} \\
\hline
版面分析 & 预处理图像 & 区域坐标 & \texttt{analyze(img)} \\
\hline
选择题 & 选项区域图像 & 答案(A/B/C/D) & \texttt{recognize\_choice(roi)} \\
\hline
判断题 & 符号区域图像 & 答案(T/F) & \texttt{recognize\_judge(roi)} \\
\hline
简答题 & 答题区域图像 & 文字内容 & \texttt{recognize\_essay(roi)} \\
\hline
评分 & 所有答案 & 分数报告 & \texttt{grade(answers)} \\
\hline
\end{tabular}
\end{table}

\subsection{环节二:项目框架说明(40分钟)}

\subsubsection{2.1 代码结构(15分钟)}

\begin{lstlisting}
auto_grading_system/
├── README.md                 # 项目说明
├── requirements.txt          # 依赖包
├── config.py                 # 配置文件
├── main.py                   # 主程序入口
├── modules/                  # 功能模块
│   ├── __init__.py
│   ├── preprocess.py         # 预处理模块(已实现)
│   ├── layout.py             # 版面分析模块(已实现)
│   ├── choice_recognizer.py  # 选择题识别(学生实现)
│   ├── judge_recognizer.py   # 判断题识别(学生实现)
│   ├── essay_recognizer.py   # 简答题识别(学生实现)
│   └── grading.py            # 评分模块(学生实现)
├── utils/                    # 工具函数
│   ├── __init__.py
│   ├── visualization.py      # 可视化工具
│   └── logger.py             # 日志工具
├── data/                     # 数据目录
│   ├── input/                # 输入试卷
│   ├── output/               # 输出结果
│   ├── templates/            # 标准答案模板
│   └── test/                 # 测试数据
└── tests/                    # 单元测试
    ├── test_preprocess.py
    ├── test_choice.py
    ├── test_judge.py
    └── test_essay.py
\end{lstlisting}

\subsubsection{2.2 已实现模块说明(10分钟)}

\textbf{预处理模块(modules/preprocess.py):}

\begin{lstlisting}
"""
图像预处理模块
已实现:去噪、二值化、透视矫正
"""

import cv2
import numpy as np

class Preprocessor:
    """图像预处理器"""

    def __init__(self, config=None):
        self.config = config or {}

    def denoise(self, image):
        """去噪"""
        return cv2.medianBlur(image, 5)

    def binarize(self, image, method='otsu'):
        """二值化"""
        gray = cv2.cvtColor(image, cv2.COLOR_BGR2GRAY) if len(image.shape) == 3 else image

        if method == 'otsu':
            _, binary = cv2.threshold(gray, 0, 255, cv2.THRESH_BINARY + cv2.THRESH_OTSU)
        else:
            _, binary = cv2.threshold(gray, 127, 255, cv2.THRESH_BINARY)

        return binary

    def process(self, image):
        """完整预处理流程"""
        # 去噪
        denoised = self.denoise(image)

        # 二值化
        binary = self.binarize(denoised)

        return {
            'gray': cv2.cvtColor(denoised, cv2.COLOR_BGR2GRAY) if len(denoised.shape) == 3 else denoised,
            'binary': binary,
            'denoised': denoised
        }
\end{lstlisting}

\textbf{版面分析模块(modules/layout.py):}

\begin{lstlisting}
"""
版面分析模块
已实现:检测试卷边界、定位题型区域
"""

import cv2
import numpy as np

class LayoutAnalyzer:
    """版面分析器"""

    def __init__(self, config=None):
        self.config = config or {}

    def detect_paper_boundary(self, binary_image):
        """检测试卷边界"""
        contours, _ = cv2.findContours(
            binary_image,
            cv2.RETR_EXTERNAL,
            cv2.CHAIN_APPROX_SIMPLE
        )

        # 找到最大的轮廓(假设是试卷)
        if contours:
            paper_contour = max(contours, key=cv2.contourArea)
            return cv2.boundingRect(paper_contour)

        return None

    def locate_question_areas(self, binary_image):
        """
        定位题目区域

        返回: {
            'choice': [(x,y,w,h), ...],  # 选择题区域
            'judge': [(x,y,w,h), ...],   # 判断题区域
            'essay': [(x,y,w,h), ...]    # 简答题区域
        }
        """
        # 简化实现:基于投影法的区域分割
        # 学生需要根据实际试卷版面调整

        h, w = binary_image.shape

        # 示例:固定位置分区
        # 实际应该通过版面分析自动检测
        return {
            'choice': [(50, 200, w-100, 300)],    # 选择题区域
            'judge': [(50, 520, w-100, 200)],     # 判断题区域
            'essay': [(50, 740, w-100, 300)]      # 简答题区域
        }

    def analyze(self, image):
        """完整版面分析"""
        # 转换为二值图
        gray = cv2.cvtColor(image, cv2.COLOR_BGR2GRAY) if len(image.shape) == 3 else image
        _, binary = cv2.threshold(gray, 0, 255, cv2.THRESH_BINARY + cv2.THRESH_OTSU)

        # 检测边界
        boundary = self.detect_paper_boundary(binary)

        # 定位区域
        areas = self.locate_question_areas(binary)

        return {
            'boundary': boundary,
            'areas': areas,
            'binary': binary
        }
\end{lstlisting}

\subsubsection{2.3 待实现模块说明(15分钟)}

\textbf{选择题识别模块模板:}

\begin{lstlisting}
"""
选择题识别模块
学生需要实现填涂检测和答案识别
"""

import cv2
import numpy as np

class ChoiceRecognizer:
    """选择题识别器"""

    def __init__(self, config=None):
        self.config = config or {}
        self.threshold = self.config.get('threshold', 0.3)

    def calculate_density(self, roi):
        """
        计算填涂密度

        TODO: 学生需要实现
        """
        pass

    def recognize_question(self, question_img, option_positions):
        """
        识别单道选择题

        TODO: 学生需要实现

        参数:
            question_img: 题目图像
            option_positions: 选项位置列表 [(x,y,w,h), ...]

        返回:
            答案 ('A', 'B', 'C', 'D' 或 None)
        """
        pass

    def recognize_all(self, image, choice_areas):
        """
        识别所有选择题

        TODO: 学生需要实现

        参数:
            image: 完整图像
            choice_areas: 选择题区域列表

        返回:
            答案列表 [{'question': 1, 'answer': 'A'}, ...]
        """
        pass
\end{lstlisting}

\subsection{环节三:分组与任务分配(30分钟)}

\subsubsection{3.1 分组原则(10分钟)}

\textbf{分组建议:}
\begin{itemize}
\item 每组3-4人
\item 包含不同专业背景
\item 设立组长和技术负责人
\end{itemize}

\textbf{角色分工:}

\begin{table}[htbp]
\centering
\begin{tabular}{|l|p{6cm}|p{5cm}|}
\hline
\textbf{角色} & \textbf{职责} & \textbf{适合学生} \\
\hline
组长 & 统筹协调、进度管理、对外沟通 & 组织能力强的 \\
\hline
技术负责人 & 架构设计、核心算法、代码审查 & CS/EE专业 \\
\hline
模块开发A & 选择题+判断题模块实现 & 有编程基础的 \\
\hline
模块开发B & 简答题+评分模块实现 & 有编程基础的 \\
\hline
前端/测试 & 界面展示、测试用例、文档编写 & 设计/文科专业 \\
\hline
\end{tabular}
\end{table}

\subsubsection{3.2 任务分解(20分钟)}

\textbf{开发任务清单:}

\begin{table}[htbp]
\centering
\begin{tabular}{|l|l|p{5cm}|}
\hline
\textbf{任务} & \textbf{优先级} & \textbf{预计工时} \\
\hline
搭建开发环境 & 高 & 1h \\
\hline
实现选择题识别 & 高 & 4h \\
\hline
实现判断题识别 & 高 & 3h \\
\hline
实现简答题识别 & 中 & 4h \\
\hline
实现评分模块 & 中 & 2h \\
\hline
模块集成测试 & 高 & 2h \\
\hline
界面开发 & 低 & 2h \\
\hline
文档编写 & 中 & 2h \\
\hline
\end{tabular}
\end{table}

\subsection{环节四:开发指导(40分钟)}

\subsubsection{4.1 开发环境搭建(10分钟)}

\begin{lstlisting}
# 1. 克隆/创建项目
mkdir auto_grading_system
cd auto_grading_system

# 2. 创建虚拟环境
python -m venv venv
source venv/bin/activate  # Linux/Mac
# 或 venv\Scripts\activate  # Windows

# 3. 安装依赖
pip install opencv-python
pip install paddlepaddle
pip install paddleocr
pip install numpy pillow matplotlib

# 4. 创建项目结构
mkdir -p modules utils data/{input,output,templates,test} tests
\end{lstlisting}

\subsubsection{4.2 开发工作流(10分钟)}

\textbf{推荐开发流程:}

\begin{enumerate}
\item \textbf{搭建框架}
    \begin{itemize}
    \item 创建项目结构
    \item 配置开发环境
    \item 编写基础代码框架
    \end{itemize}

\item \textbf{模块开发}
    \begin{itemize}
    \item 选择一个模块开始(建议从选择题开始)
    \item 编写单元测试
    \item 调试优化
    \end{itemize}

\item \textbf{模块集成}
    \begin{itemize}
    \item 整合各模块
    \item 测试整体流程
    \item 修复接口问题
    \end{itemize}

\item \textbf{测试优化}
    \begin{itemize}
    \item 使用测试集验证
    \item 优化识别准确率
    \item 完善错误处理
    \end{itemize}
\end{enumerate}

\subsubsection{4.3 调试技巧(10分钟)}

\begin{lstlisting}
"""
调试工具函数
"""

import cv2
import numpy as np
from datetime import datetime

def save_debug_image(image, name, prefix='debug'):
    """保存调试图像"""
    timestamp = datetime.now().strftime('%Y%m%d_%H%M%S')
    filename = f"data/output/{prefix}_{name}_{timestamp}.jpg"
    cv2.imwrite(filename, image)
    print(f"Debug image saved: {filename}")

def draw_debug_boxes(image, boxes, labels=None, color=(0, 255, 0)):
    """绘制调试框"""
    debug_img = image.copy()

    for i, box in enumerate(boxes):
        x, y, w, h = box
        cv2.rectangle(debug_img, (x, y), (x+w, y+h), color, 2)

        if labels and i < len(labels):
            cv2.putText(debug_img, labels[i], (x, y-5),
                       cv2.FONT_HERSHEY_SIMPLEX, 0.5, color, 1)

    return debug_img

def print_debug_info(info_dict, title="Debug Info"):
    """打印调试信息"""
    print(f"\n{'='*50}")
    print(f"{title}")
    print(f"{'='*50}")
    for key, value in info_dict.items():
        print(f"{key}: {value}")
    print(f"{'='*50}\n")
\end{lstlisting}

\subsubsection{4.4 常见问题与解决方案(10分钟)}

\textbf{Q1: 模块导入错误}

\begin{lstlisting}
# 错误: ModuleNotFoundError: No module named 'modules'

# 解决方案: 确保项目根目录在Python路径中
import sys
sys.path.append('.')  # 或项目的绝对路径

from modules.choice_recognizer import ChoiceRecognizer
\end{lstlisting}

\textbf{Q2: 图像路径问题}

\begin{lstlisting}
import os

# 使用相对路径时,注意当前工作目录
base_dir = os.path.dirname(os.path.abspath(__file__))
image_path = os.path.join(base_dir, 'data', 'input', 'exam.jpg')

img = cv2.imread(image_path)
if img is None:
    print(f"Error: Cannot load image from {image_path}")
\end{lstlisting}

\textbf{Q3: 数组索引越界}

\begin{lstlisting}
# 安全地裁剪ROI
def safe_crop(image, x, y, w, h):
    """安全裁剪,防止越界"""
    img_h, img_w = image.shape[:2]

    # 确保坐标在图像范围内
    x = max(0, min(x, img_w - 1))
    y = max(0, min(y, img_h - 1))
    w = min(w, img_w - x)
    h = min(h, img_h - y)

    return image[y:y+h, x:x+w]
\end{lstlisting}

\subsection{环节五:进度检查与答疑(10分钟)}

\textbf{本周进度要求:}
\begin{enumerate}
\item 完成分组
\item 确定开发环境
\item 明确任务分工
\item 至少完成一个模块的框架代码
\end{enumerate}

\textbf{下周目标:}
\begin{itemize}
\item 完成所有模块开发
\item 完成模块集成
\item 基本功能可用
\end{itemize}

\section{课后任务}

\subsection{分组任务}

\begin{enumerate}
\item \textbf{确定分组}(3-4人/组)
\item \textbf{选定组长和技术负责人}
\item \textbf{制定开发计划}
\item \textbf{搭建开发环境}
\item \textbf}开始模块开发(至少完成选择题识别框架)
\end{enumerate}

\subsection{提交内容}

\begin{itemize}
\item 分组名单(组长、成员、角色分工)
\item 开发计划(时间表、任务分配)
\item 项目初始化代码(GitHub仓库链接或代码压缩包)
\end{itemize}

\section{教学反思}

\subsection{预期挑战}

\begin{itemize}
\item 分组合作可能出现沟通问题
    \begin{itemize}
    \item 建议:建立定期沟通机制
    \end{itemize}

\item 编程能力差异大
    \begin{itemize}
    \item 建议:提供更多代码模板和示例
    \end{itemize}

\item 时间紧迫
    \begin{itemize}
    \item 建议:聚焦MVP功能,确保基本可用
    \end{itemize}
\end{itemize}

\end{document}
