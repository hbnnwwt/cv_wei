\documentclass[12pt,a4paper]{article}
\usepackage[UTF8]{ctex}
\usepackage{geometry}
\usepackage{amsmath}
\usepackage{graphicx}
\usepackage{booktabs}
\usepackage{array}
\usepackage{enumitem}
\usepackage{xcolor}
\usepackage{tcolorbox}
\usepackage{listings}
\usepackage{hyperref}

\geometry{left=2.5cm,right=2.5cm,top=2.5cm,bottom=2.5cm}

\lstset{
    language=Python,
    basicstyle=\ttfamily\small,
    keywordstyle=\color{blue},
    commentstyle=\color{green!60!black},
    stringstyle=\color{orange},
    breaklines=true,
    frame=single,
    showstringspaces=false
}

\title{\textbf{\large 第3周教案:图像预处理与增强}}
\author{计算机视觉课程组}
\date{}

\begin{document}

\maketitle

\section{基本信息}

\begin{tabular}{|l|p{12cm}|}
\hline
\textbf{周次} & 第3周 \\
\hline
\textbf{主题} & 图像预处理与增强 \\
\hline
\textbf{学时} & 3学时(160分钟) \\
\hline
\textbf{故事问题} & 试卷拍照模糊怎么办? \\
\hline
\textbf{OBE目标} & A1-图像处理:能对图像进行去噪、二值化、矫正操作 \\
\hline
\textbf{项目贡献} & 为所有题型识别提供清晰、规范的输入图像 \\
\hline
\end{tabular}

\section{教学目标}

\begin{enumerate}
\item \textbf{知识目标}:
    \begin{itemize}
    \item 理解图像噪声的类型与来源
    \item 掌握图像滤波的基本原理
    \item 理解二值化与阈值分割的原理
    \end{itemize}

\item \textbf{能力目标}:
    \begin{itemize}
    \item 能够实现图像去噪(高斯/中值滤波)
    \item 能够实现图像二值化
    \item 能够进行透视矫正
    \end{itemize}

\item \textbf{素养目标}:
    \begin{itemize}
    \item 理解预处理对后续识别的重要性
    \item 培养参数调优的实验思维
    \end{itemize}
\end{enumerate}

\section{教学重点与难点}

\begin{tcolorbox}[colback=red!5!white,colframe=red!75!black,title=\textbf{教学重点}]
\begin{itemize}
\item 图像滤波:高斯滤波、中值滤波
\item 图像二值化:全局阈值、自适应阈值
\item 透视变换:四点变换矫正
\end{itemize}
\end{tcolorbox}

\begin{tcolorbox}[colback=yellow!5!white,colframe=yellow!75!black,title=\textbf{教学难点}]
\begin{itemize}
\item 阈值的选择与优化
\item 透视变换矩阵的理解
\end{itemize}
\end{tcolorbox}

\section{教学过程设计}

\subsection{环节一:图像预处理概述(10分钟)}

\subsubsection{1.1 为什么需要预处理?(5分钟)}

\textbf{现实问题:}
\begin{itemize}
\item 拍摄角度不正
\item 光照不均匀
\item 纸张有折痕
\item 背景有杂物
\end{itemize}

\textbf{预处理的目标:}
\begin{enumerate}
\item 去除噪声干扰
\item 增强目标特征
\item 规范图像格式
\item 提升识别准确率
\end{enumerate}

\subsubsection{1.2 预处理流水线(5分钟)}

\begin{center}
原图 $\to$ 去噪 $\to$ 二值化 $\to$ 矫正 $\to$ 增强输出
\end{center}

\subsection{环节二:图像去噪(40分钟)}

\subsubsection{2.1 噪声类型(10分钟)}

\textbf{常见噪声类型:}

\begin{table}[htbp]
\centering
\begin{tabular}{|l|p{6cm}|p{4cm}|}
\hline
\textbf{噪声类型} & \textbf{特征} & \textbf{典型场景} \\
\hline
高斯噪声 & 随机分布的亮度变化 & 传感器噪声、低光拍摄 \\
\hline
椒盐噪声 & 随机的黑点或白点 & 传输错误、老化的传感器 \\
\hline
周期噪声 & 规则的干扰条纹 & 电气干扰 \\
\hline
\end{tabular}
\end{table}

\subsubsection{2.2 滤波方法(20分钟)}

\textbf{1. 高斯滤波(Gaussian Blur)}

\begin{itemize}
\item 原理:用高斯分布权重进行卷积
\item 适合:高斯噪声、自然去噪
\end{itemize}

\begin{lstlisting}
import cv2
import numpy as np

# 高斯滤波
# 参数:(图像, 核大小, 标准差)
blur = cv2.GaussianBlur(img, (5, 5), 0)

# 核大小必须是正奇数:3, 5, 7, 9...
# 核越大,模糊效果越强
# 标准差为0时自动计算
\end{lstlisting}

\textbf{2. 中值滤波(Median Blur)}

\begin{itemize}
\item 原理:用邻域像素的中值替换中心像素
\item 适合:椒盐噪声、保持边缘
\end{itemize}

\begin{lstlisting}
# 中值滤波
# 参数:(图像, 核大小)
median = cv2.medianBlur(img, 5)

# 中值滤波对椒盐噪声效果更好
# 能更好地保持边缘清晰
\end{lstlisting}

\textbf{3. 双边滤波(Bilateral Filter)}

\begin{itemize}
\item 原理:同时考虑空间距离和像素值差异
\item 适合:保边去噪
\end{itemize}

\begin{lstlisting}
# 双边滤波
# 参数:(图像, d, sigmaColor, sigmaSpace)
bilateral = cv2.bilateralFilter(img, 9, 75, 75)

# d: 像素邻域直径
# sigmaColor: 颜色空间的标准差
# sigmaSpace: 坐标空间的标准差
\end{lstlisting}

\subsubsection{2.3 效果对比实验(10分钟)}

\begin{lstlisting}
import cv2
import matplotlib.pyplot as plt

# 读取带噪声的图像
img = cv2.imread('noisy_exam.jpg')

# 应用不同滤波方法
gaussian = cv2.GaussianBlur(img, (5, 5), 0)
median = cv2.medianBlur(img, 5)
bilateral = cv2.bilateralFilter(img, 9, 75, 75)

# 可视化对比
fig, axes = plt.subplots(2, 2, figsize=(12, 10))

axes[0, 0].imshow(cv2.cvtColor(img, cv2.COLOR_BGR2RGB))
axes[0, 0].set_title('原始噪声图像')
axes[0, 0].axis('off')

axes[0, 1].imshow(cv2.cvtColor(gaussian, cv2.COLOR_BGR2RGB))
axes[0, 1].set_title('高斯滤波')
axes[0, 1].axis('off')

axes[1, 0].imshow(cv2.cvtColor(median, cv2.COLOR_BGR2RGB))
axes[1, 0].set_title('中值滤波')
axes[1, 0].axis('off')

axes[1, 1].imshow(cv2.cvtColor(bilateral, cv2.COLOR_BGR2RGB))
axes[1, 1].set_title('双边滤波')
axes[1, 1].axis('off')

plt.tight_layout()
plt.show()
\end{lstlisting}

\subsection{环节三:图像二值化(50分钟)}

\subsubsection{3.1 什么是二值化?(10分钟)}

\textbf{定义:} 将灰度图像转换为只有黑白两种颜色的图像

\begin{itemize}
\item 像素值 > 阈值 $\to$ 白色(255)
\item 像素值 $\le$ 阈值 $\to$ 黑色(0)
\end{itemize}

\textbf{应用场景:}
\begin{itemize}
\item 文档扫描处理
\item 填涂识别(OMR)
\item 边缘检测预处理
\end{itemize}

\subsubsection{3.2 全局阈值(15分钟)}

\textbf{方法:} 对整幅图像使用同一个阈值

\begin{lstlisting}
# 先转灰度图
gray = cv2.cvtColor(img, cv2.COLOR_BGR2GRAY)

# 全局阈值
# 参数:(图像, 阈值, 最大值, 阈值类型)
ret, binary = cv2.threshold(gray, 127, 255, cv2.THRESH_BINARY)

# ret: 实际使用的阈值
# binary: 二值化结果
\end{lstlisting}

\textbf{阈值类型:}

\begin{table}[htbp]
\centering
\begin{tabular}{|l|l|p{6cm}|}
\hline
\textbf{类型} & \textbf{公式} & \textbf{效果} \\
\hline
THRESH\_BINARY & $dst(x,y) = maxval$ if $src(x,y) > thresh$ & 大于阈值变白 \\
\hline
THRESH\_BINARY\_INV & $dst(x,y) = 0$ if $src(x,y) > thresh$ & 大于阈值变黑(反色) \\
\hline
THRESH\_TRUNC & $dst(x,y) = threshold$ if $src(x,y) > thresh$ & 超过阈值截断 \\
\hline
THRESH\_TOZERO & $dst(x,y) = src(x,y)$ if $src(x,y) > thresh$ & 小于阈值变0 \\
\hline
\end{tabular}
\end{table}

\textbf{选择题识别的常用设置:}
\begin{lstlisting}
# 填涂识别:使用反色二值化
# 填涂部分(深色)变为白色,未填涂部分(浅色)变为黑色
ret, binary_inv = cv2.threshold(
    gray,
    127,           # 阈值
    255,           # 最大值
    cv2.THRESH_BINARY_INV
)
\end{lstlisting}

\subsubsection{3.3 自适应阈值(15分钟)}

\textbf{问题:} 全局阈值对光照不均的图像效果差

\textbf{解决方案:} 自适应阈值 —— 根据局部区域计算阈值

\begin{lstlisting}
# 自适应阈值
# 参数:(图像, 最大值, 方法, 阈值类型, 邻域大小, 常数C)
adaptive = cv2.adaptiveThreshold(
    gray, 255,
    cv2.ADAPTIVE_THRESH_GAUSSIAN_C,  # 方法:高斯加权平均
    cv2.THRESH_BINARY,                # 阈值类型
    11,                               # 邻域大小(奇数)
    2                                 # 常数C(从均值减去)
)
\end{lstlisting}

\textbf{两种方法对比:}

\begin{table}[htbp]
\centering
\begin{tabular}{|l|p{6cm}|p{4cm}|}
\hline
\textbf{方法} & \textbf{特点} & \textbf{适用场景} \\
\hline
ADAPTIVE\_THRESH\_MEAN\_C & 邻域均值计算阈值 & 计算快 \\
\hline
ADAPTIVE\_THRESH\_GAUSSIAN\_C & 高斯加权计算阈值 & 效果好,推荐 \\
\hline
\end{tabular}
\end{table}

\subsubsection{3.4 Otsu自动阈值(10分钟)}

\textbf{原理:} 自动寻找最佳阈值,使类间方差最大

\begin{lstlisting}
# Otsu自动阈值
# 阈值设为0,添加cv2.THRESH_OTSU标志
ret, otsu = cv2.threshold(
    gray,
    0,                              # 阈值(会被忽略)
    255,
    cv2.THRESH_BINARY + cv2.THRESH_OTSU
)

print(f"Otsu自动选择的阈值: {ret}")
\end{lstlisting}

\subsection{环节四:透视矫正(40分钟)}

\subsubsection{4.1 透视变换原理(10分钟)}

\textbf{问题:} 拍照时试卷角度不正

\textbf{解决:} 将梯形区域变换为矩形

\begin{center}
歪斜的试卷 $\to$ 矫正后的标准矩形
\end{center}

\textbf{四点变换:}
\begin{enumerate}
\item 检测试卷的四个角点
\item 定义目标矩形的四个角点
\item 计算透视变换矩阵
\item 应用变换得到矫正图像
\end{enumerate}

\subsubsection{4.2 实现代码(20分钟)}

\begin{lstlisting}
import cv2
import numpy as np

def four_point_transform(image, pts):
    """
    四点透视变换

    参数:
        image: 输入图像
        pts: 四个角点坐标 [tl, tr, br, bl]

    返回:
        变换后的图像
    """
    # 获取四个点坐标
    rect = order_points(pts)
    (tl, tr, br, bl) = rect

    # 计算新图像的宽度
    widthA = np.sqrt(((br[0] - bl[0])**2) + ((br[1] - bl[1])**2))
    widthB = np.sqrt(((tr[0] - tl[0])**2) + ((tr[1] - tl[1])**2))
    maxWidth = max(int(widthA), int(widthB))

    # 计算新图像的高度
    heightA = np.sqrt(((tr[0] - br[0])**2) + ((tr[1] - br[1])**2))
    heightB = np.sqrt(((tl[0] - bl[0])**2) + ((tl[1] - bl[1])**2))
    maxHeight = max(int(heightA), int(heightB))

    # 目标点(矩形)
    dst = np.array([
        [0, 0],
        [maxWidth - 1, 0],
        [maxWidth - 1, maxHeight - 1],
        [0, maxHeight - 1]
    ], dtype="float32")

    # 计算透视变换矩阵
    M = cv2.getPerspectiveTransform(rect, dst)

    # 应用变换
    warped = cv2.warpPerspective(image, M, (maxWidth, maxHeight))

    return warped

def order_points(pts):
    """对四个点排序:左上、右上、右下、左下"""
    rect = np.zeros((4, 2), dtype="float32")

    # 计算点的和(左上和最小,右下和最大)
    s = pts.sum(axis=1)
    rect[0] = pts[np.argmin(s)]
    rect[2] = pts[np.argmax(s)]

    # 计算点的差(右上差最小,左下差最大)
    diff = np.diff(pts, axis=1)
    rect[1] = pts[np.argmin(diff)]
    rect[3] = pts[np.argmax(diff)]

    return rect
\end{lstlisting}

\subsubsection{4.3 完整示例(10分钟)}

\begin{lstlisting}
# 读取图像
img = cv2.imread('skewed_exam.jpg')

# 假设已经检测到四个角点(实际需要用轮廓检测)
pts = np.array([
    [100, 150],   # 左上
    [450, 120],   # 右上
    [480, 380],   # 右下
    [80, 400]     # 左下
], dtype="float32")

# 应用透视变换
warped = four_point_transform(img, pts)

# 显示结果
fig, axes = plt.subplots(1, 2, figsize=(12, 5))

axes[0].imshow(cv2.cvtColor(img, cv2.COLOR_BGR2RGB))
axes[0].set_title('原始倾斜图像')
axes[0].axis('off')

axes[1].imshow(cv2.cvtColor(warped, cv2.COLOR_BGR2RGB))
axes[1].set_title('透视矫正后')
axes[1].axis('off')

plt.show()
\end{lstlisting}

\subsection{环节五:综合实验(20分钟)}

\textbf{任务:} 实现完整的试卷预处理流水线

\begin{lstlisting}
def preprocess_exam_paper(image_path):
    """
    试卷图像预处理流水线

    参数:
        image_path: 试卷图像路径

    返回:
        预处理后的图像
    """
    # 1. 读取图像
    img = cv2.imread(image_path)

    # 2. 去噪
    denoised = cv2.medianBlur(img, 5)

    # 3. 转灰度
    gray = cv2.cvtColor(denoised, cv2.COLOR_BGR2GRAY)

    # 4. 二值化(用于后续处理)
    _, binary = cv2.threshold(
        gray,
        127,
        255,
        cv2.THRESH_BINARY
    )

    # 5. 透视矫正(需要先检测角点)
    # warped = perspective_transform(binary)

    return gray, binary

# 使用示例
gray, binary = preprocess_exam_paper('exam.jpg')
\end{lstlisting}

\section{课后作业}

\subsection{作业内容}

\textbf{题目:} 实现试卷图像预处理完整流程

\textbf{要求:}
\begin{enumerate}
\item 对试卷图像进行去噪处理
\item 实现二值化(至少两种方法对比)
\item 提交处理前后对比图
\item 分析不同参数对结果的影响
\end{enumerate}

\subsection{评分标准}

\begin{table}[htbp]
\centering
\begin{tabular}{|l|p{8cm}|p{2cm}|}
\hline
\textbf{评分项} & \textbf{标准} & \textbf{分值} \\
\hline
代码实现 & 功能完整,无错误 & 40分 \\
\hline
效果对比 & 清晰展示处理前后差异 & 30分 \\
\hline
参数分析 & 分析参数对结果的影响 & 20分 \\
\hline
代码规范 & 注释完整,结构清晰 & 10分 \\
\hline
\textbf{合计} & & \textbf{100分} \\
\hline
\end{tabular}
\end{table}

\section{教学反思}

\subsection{重点提醒}

\begin{itemize}
\item 预处理是后续识别的基础,必须重视
\item 阈值选择需要根据具体图像调整
\item 透视变换需要准确检测四个角点(下周内容)
\end{itemize}

\end{document}
