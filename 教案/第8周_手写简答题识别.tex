\documentclass[12pt,a4paper]{article}
\usepackage[UTF8]{ctex}
\usepackage{geometry}
\usepackage{amsmath}
\usepackage{graphicx}
\usepackage{booktabs}
\usepackage{array}
\usepackage{enumitem}
\usepackage{xcolor}
\usepackage{tcolorbox}
\usepackage{listings}
\usepackage{hyperref}

\geometry{left=2.5cm,right=2.5cm,top=2.5cm,bottom=2.5cm}

\lstset{
    language=Python,
    basicstyle=\ttfamily\small,
    keywordstyle=\color{blue},
    commentstyle=\color{green!60!black},
    stringstyle=\color{orange},
    breaklines=true,
    frame=single,
    showstringspaces=false
}

\title{\textbf{\large 第8周教案:手写简答题识别}}
\author{计算机视觉课程组}
\date{}

\begin{document}

\maketitle

\section{基本信息}

\begin{tabular}{|l|p{12cm}|}
\hline
\textbf{周次} & 第8周 \\
\hline
\textbf{主题} & 手写简答题识别 \\
\hline
\textbf{学时} & 3学时(160分钟) \\
\hline
\textbf{故事问题} & 能看懂学生写的答案吗? \\
\hline
\textbf{OBE目标} & A3-特征识别:能识别手写文字 \\
\hline
\textbf{项目贡献} & 实现简答题手写识别模块 \\
\hline
\end{tabular}

\section{教学目标}

\begin{enumerate}
\item \textbf{知识目标}:
    \begin{itemize}
    \item 理解手写识别的挑战
    \item 了解端到端手写识别模型
    \item 掌握TrOCR/PaddleOCR手写识别
    \end{itemize}

\item \textbf{能力目标}:
    \begin{itemize}
    \item 能够使用手写识别API
    \item 能够处理手写识别结果
    \item 能够优化识别准确率
    \end{itemize}

\item \textbf{素养目标}:
    \begin{itemize}
    \item 理解手写识别在实际应用中的价值
    \item 了解AI技术的局限性
    \end{itemize}
\end{enumerate}

\section{教学重点与难点}

\begin{tcolorbox}[colback=red!5!white,colframe=red!75!black,title=\textbf{教学重点}]
\begin{itemize}
\item 手写识别的挑战与特点
\item TrOCR/PaddleOCR手写模型使用
\item 识别结果后处理
\end{itemize}
\end{tcolorbox}

\begin{tcolorbox}[colback=yellow!5!white,colframe=yellow!75!black,title=\textbf{教学难点}]
\begin{itemize}
\item 手写字迹形变处理
\item 低质量图像识别
\end{itemize}
\end{tcolorbox}

\section{教学过程设计}

\subsection{环节一:手写识别概述(40分钟)}

\subsubsection{1.1 手写识别的挑战(15分钟)}

\textbf{为什么手写识别比印刷识别难?}

\begin{table}[htbp]
\centering
\begin{tabular}{|l|p{5cm}|p{5cm}|}
\hline
\textbf{因素} & \textbf{印刷文字} & \textbf{手写文字} \\
\hline
字体规范 & 统一标准字体 & 每人不同 \\
\hline
字形稳定 & 字形固定 & 形状变化大 \\
\hline
间距规范 & 固定间距 & 随意 \\
\hline
倾斜角度 & 水平排列 & 可能倾斜 \\
\hline
连笔情况 & 无连笔 & 常见连笔 \\
\hline
\end{tabular}
\end{table}

\textbf{手写识别的难点:}
\begin{enumerate}
\item \textbf{字形差异大}:不同人的书写风格完全不同
\item \textbf{连笔问题}:字与字可能连在一起
\item \textbf{倾斜旋转}:书写方向可能不正
\item \textbf{涂改修改}:可能有涂改痕迹
\item \textbf{图像质量}:拍照可能有模糊、阴影
\end{enumerate}

\subsubsection{1.2 手写识别技术发展(15分钟)}

\begin{table}[htbp]
\centering
\begin{tabular}{|l|p{6cm}|p{5cm}|}
\hline
\textbf{阶段} & \textbf{技术} & \textbf{准确率} \\
\hline
传统方法 & 特征工程 + SVM/HMM & 60-70\% \\
\hline
CNN时代 & LeNet, AlexNet等CNN & 80-85\% \\
\hline
RNN+Attention & RNN+Attention机制 & 90-92\% \\
\hline
Transformer & TrOCR, Donut & 95\%+ \\
\hline
\textbf{当前} & \textbf{端到端大模型} & \textbf{95\%+} \\
\hline
\end{tabular}
\end{table}

\subsubsection{1.3 主流手写识别工具(10分钟)}

\begin{table}[htbp]
\centering
\begin{tabular}{|l|p{4cm}|p{4cm}|p{3cm}|}
\hline
\textbf{工具} & \textbf{优点} & \textbf{缺点} & \textbf{适用} \\
\hline
TrOCR & 准确率最高 & 速度慢,需GPU & 高精度场景 \\
\hline
PaddleOCR & 速度快,中文好 & 手写略弱于TrOCR & 实时应用 \\
\hline
Tesseract & 免费开源 & 中文手写差 & 英文手写 \\
\hline
EasyOCR & 多语言 & 中文一般 & 多语言混合 \\
\hline
\end{tabular}
\end{table}

\subsection{环节二:PaddleOCR手写识别(40分钟)}

\subsubsection{2.1 手写模型配置(10分钟)}

\begin{lstlisting}
from paddleocr import PaddleOCR

# PaddleOCR支持手写识别
# 使用中文模型时,默认已包含手写识别能力

# 初始化手写识别OCR
ocr_handwrite = PaddleOCR(
    use_angle_cls=True,    # 启用方向分类(手写文字可能有倾斜)
    lang='ch',             # 中文
    det_model_dir=None,    # 使用默认检测模型
    rec_model_dir=None,    # 使用默认识别模型
    use_gpu=False,         # 是否使用GPU
    show_log=False
)

# 识别手写文字
result = ocr_handwrite.ocr('handwriting.jpg', cls=True)

# 打印结果
for line in result:
    text = line[1][0]
    confidence = line[1][1]
    print(f"文字: {text}, 置信度: {confidence:.4f}")
\end{lstlisting}

\subsubsection{2.2 手写图像预处理(15分钟)}

\textbf{手写图像的特殊处理:}

\begin{lstlisting}
import cv2
import numpy as np

def preprocess_handwriting(image):
    """
    手写图像预处理

    参数:
        image: 输入图像

    返回:
        预处理后的图像
    """
    # 转灰度
    if len(image.shape) == 3:
        gray = cv2.cvtColor(image, cv2.COLOR_BGR2GRAY)
    else:
        gray = image.copy()

    # 去噪(手写可能有不规则噪声)
    denoised = cv2.fastNlMeansDenoising(gray, None, 10, 7, 21)

    # 对比度增强(手写可能对比度不足)
    clahe = cv2.createCLAHE(clipLimit=2.0, tileGridSize=(8, 8))
    enhanced = clahe.apply(denoised)

    # 二值化
    _, binary = cv2.threshold(
        enhanced, 0, 255,
        cv2.THRESH_BINARY + cv2.THRESH_OTSU
    )

    # 形态学操作(去除小噪点)
    kernel = np.ones((2, 2), np.uint8)
    cleaned = cv2.morphologyEx(binary, cv2.MORPH_CLOSE, kernel)

    return cleaned

# 使用
img = cv2.imread('handwriting.jpg')
processed = preprocess_handwriting(img)
cv2.imwrite('handwriting_processed.jpg', processed)

# 使用处理后的图像进行识别
result = ocr_handwrite.ocr('handwriting_processed.jpg', cls=True)
\end{lstlisting}

\subsubsection{2.3 手写数字识别(15分钟)}

\textbf{简答题常涉及数字答案:}

\begin{lstlisting}
def recognize_handwritten_digits(image_path):
    """
    识别手写数字

    适用于填空题、简答题中的数字答案
    """
    ocr = PaddleOCR(use_angle_cls=True, lang='ch', show_log=False)

    # 预处理
    img = cv2.imread(image_path)
    processed = preprocess_handwriting(img)

    # 识别
    result = ocr.ocr(processed, cls=True)

    # 提取数字
    import re
    digits = []

    for line in result:
        if line is None:
            continue
        text = line[1][0]
        confidence = line[1][1]

        # 提取所有数字
        found_digits = re.findall(r'\d+\.?\d*', text)
        digits.extend(found_digits)

    return digits

# 使用
digits = recognize_handwritten_digits('handwriting_answer.jpg')
print(f"识别到的数字: {digits}")
\end{lstlisting}

\subsection{环节三:TrOCR高精度识别(40分钟)}

\subsubsection{3.1 TrOCR简介(10分钟)}

\textbf{TrOCR:** Transformer-based Optical Character Recognition}

\begin{itemize}
\item \textbf{开发者:** Microsoft
\item \textbf{架构:** ViT (图像编码器) + GPT-2 (文本解码器)
\item \textbf{优势:** 端到端训练,准确率最高
\item \textbf{模型:**
    \begin{itemize}
    \item trocr-base-handwritten
    \item trocr-large-handwritten
    \end{itemize}
\end{itemize}

\subsubsection{3.2 安装和使用(20分钟)}

\begin{lstlisting}
# 安装transformers和torch
pip install transformers torch pillow

# 安装加速库(可选)
pip install accelerate

# 使用TrOCR
from transformers import TrOCRProcessor, VisionEncoderDecoderModel

# 加载模型
processor = TrOCRProcessor.from_pretrained('microsoft/trocr-base-handwritten')
model = VisionEncoderDecoderModel.from_pretrained('microsoft/trocr-base-handwritten')

# 读取图像
from PIL import Image
image = Image.open('handwriting.jpg').convert("RGB")

# 预处理
pixel_values = processor(images=image, return_tensors="pt").pixel_values

# 生成文本
generated_ids = model.generate(pixel_values)
generated_text = processor.batch_decode(generated_ids, skip_special_tokens=True)[0]

print(f"识别结果: {generated_text}")
\end{lstlisting}

\textbf{批量处理:}

\begin{lstlisting}
def recognize_handwrite_trocr(image_paths):
    """
    使用TrOCR批量识别手写图像

    参数:
        image_paths: 图像路径列表

    返回:
        识别结果列表
    """
    # 加载模型
    processor = TrOCRProcessor.from_pretrained('microsoft/trocr-base-handwritten')
    model = VisionEncoderDecoderModel.from_pretrained('microsoft/trocr-base-handwritten')

    results = []

    for image_path in image_paths:
        # 读取图像
        image = Image.open(image_path).convert("RGB")

        # 预处理
        pixel_values = processor(images=image, return_tensors="pt").pixel_values

        # 生成
        generated_ids = model.generate(pixel_values)
        generated_text = processor.batch_decode(generated_ids, skip_special_tokens=True)[0]

        results.append({
            'image': image_path,
            'text': generated_text
        })

    return results

# 使用
image_paths = ['handwriting_1.jpg', 'handwriting_2.jpg']
results = recognize_handwrite_trocr(image_paths)

for r in results:
    print(f"{r['image']}: {r['text']}")
\end{lstlisting}

\subsubsection{3.3 TrOCR优化(10分钟)}

\begin{lstlisting}
# 使用更大的模型(更高精度,但速度更慢)
processor_large = TrOCRProcessor.from_pretrained('microsoft/trocr-large-handwritten')
model_large = VisionEncoderDecoderModel.from_pretrained('microsoft/trocr-large-handwritten')

# 调整生成参数
generated_ids = model.generate(
    pixel_values,
    max_length=50,          # 最大生成长度
    min_length=1,           # 最小生成长度
    num_beams=5,            # beam search大小(越大越准确)
    temperature=1.0,        # 温度参数
    top_k=50,               # top-k采样
    top_p=0.95,             # top-p采样
    repetition_penalty=1.0  # 重复惩罚
)
\end{lstlisting}

\subsection{环节四:简答题识别完整流程(30分钟)}

\subsubsection{4.1 定位简答区域(10分钟)}

\begin{lstlisting}
def locate_essay_area(image, ocr_result):
    """
    定位简答题答题区域

    策略: 找到"简答"关键词后的空白区域
    """
    # 查找"简答"关键词
    for i, line in enumerate(ocr_result):
        if line is None:
            continue

        text = line[1][0]
        if '简答' in text or '回答' in text:
            # 获取该文字框的底部坐标
            boxes = line[0]
            bottom_y = int(np.mean([box[1] for box in boxes]))

            return bottom_y

    # 如果没找到,返回图像下半部分
    height = image.shape[0]
    return height // 2
\end{lstlisting}

\subsubsection{4.2 手写文字行分割(10分钟)}

\begin{lstlisting}
def segment_handwriting_lines(binary_image):
    """
    分割手写文字行

    参数:
        binary_image: 二值图像

    返回:
        文字行图像列表
    """
    # 水平投影
    h_proj = np.sum(binary_image == 0, axis=1)

    # 找到行间分隔(投影值低于阈值)
    threshold = np.mean(h_proj) * 0.3
    lines = []
    in_line = False
    line_start = 0

    for i, value in enumerate(h_proj):
        if value > threshold and not in_line:
            in_line = True
            line_start = i
        elif value <= threshold and in_line:
            in_line = False
            line_end = i
            lines.append((line_start, line_end))

    # 提取每行图像
    line_images = []
    for start, end in lines:
        if end - start > 10:  # 过滤太短的行
            line_img = binary_image[start:end, :]
            line_images.append(line_img)

    return line_images
\end{lstlisting}

\subsubsection{4.3 完整简答题识别(10分钟)}

\begin{lstlisting}
class EssayRecognizer:
    """简答题识别器"""

    def __init__(self, method='paddle'):
        """
        初始化

        参数:
            method: 识别方法 ('paddle' 或 'trocr')
        """
        self.method = method

        if method == 'paddle':
            self.ocr = PaddleOCR(use_angle_cls=True, lang='ch', show_log=False)
        elif method == 'trocr':
            from transformers import TrOCRProcessor, VisionEncoderDecoderModel
            self.processor = TrOCRProcessor.from_pretrained('microsoft/trocr-base-handwritten')
            self.model = VisionEncoderDecoderModel.from_pretrained('microsoft/trocr-base-handwritten')

    def preprocess(self, image):
        """预处理"""
        return preprocess_handwriting(image)

    def recognize(self, image):
        """
        识别简答题答案

        参数:
            image: 简答题区域图像

        返回:
            识别的文字
        """
        processed = self.preprocess(image)

        if self.method == 'paddle':
            result = self.ocr.ocr(processed, cls=True)

            text_lines = []
            for line in result:
                if line is not None:
                    text = line[1][0]
                    confidence = line[1][1]
                    if confidence > 0.5:
                        text_lines.append(text)

            return '\n'.join(text_lines)

        elif self.method == 'trocr':
            from PIL import Image
            img_pil = Image.fromarray(processed).convert("RGB")

            pixel_values = self.processor(images=img_pil, return_tensors="pt").pixel_values
            generated_ids = self.model.generate(pixel_values)
            generated_text = self.processor.batch_decode(generated_ids, skip_special_tokens=True)[0]

            return generated_text

# 使用
recognizer = EssayRecognizer(method='paddle')
answer = recognizer.recognize(essay_image)
print(f"简答答案: {answer}")
\end{lstlisting}

\subsection{环节五:实验(10分钟)}

\begin{lstlisting}
# 完整示例:识别试卷中的简答题
from paddleocr import PaddleOCR
import cv2

def recognize_essay_questions(image_path):
    """识别试卷中的所有简答题"""
    # 初始化
    ocr = PaddleOCR(use_angle_cls=True, lang='ch', show_log=False)
    essay_recognizer = EssayRecognizer(method='paddle')

    # 读取图像
    img = cv2.imread(image_path)

    # 定位简答区域
    ocr_result = ocr.ocr(img, cls=True)
    essay_y = locate_essay_area(img, ocr_result)

    # 裁剪简答区域
    height, width = img.shape[:2]
    essay_area = img[essay_y:height, 0:width]

    # 识别
    answer = essay_recognizer.recognize(essay_area)

    print(f"简答题答案:\n{answer}")

    return answer

# 使用
recognize_essay_questions('exam_with_essay.jpg')
\end{lstlisting}

\section{课后作业}

\subsection{作业内容}

\textbf{题目:** 实现简答题手写识别模块}

\textbf{要求:}
\begin{enumerate}
\item 使用PaddleOCR或TrOCR识别手写文字
\item 实现图像预处理优化
\item 识别手写答案内容
\item 分析识别准确率
\end{enumerate}

\subsection{评分标准}

\begin{table}[htbp]
\centering
\begin{tabular}{|l|p{8cm}|p{2cm}|}
\hline
\textbf{评分项} & \textbf{标准} & \textbf{分值} \\
\hline
模型配置 & 正确配置和使用识别模型 & 25分 \\
\hline
预处理 & 实现有效的预处理优化 & 25分 \\
\hline
识别效果 & 能识别出手写文字 & 30分 \\
\hline
分析报告 & 分析识别结果和准确率 & 20分 \\
\hline
\textbf{合计} & & \textbf{100分} \\
\hline
\end{tabular}
\end{table}

\section{教学反思}

\subsection{重点提醒}

\begin{itemize}
\item 手写识别准确率受字迹质量影响大
\item 实际应用中可能需要人工校对
\item 下周开始课程设计,整合所有模块
\end{itemize}

\subsection{课程设计预告}

从下周开始的三周,学生将分组开发完整的自动阅卷系统,整合选择题、判断题、简答题三种题型的识别模块。

\end{document}
