\documentclass[12pt,a4paper]{article}
\usepackage[UTF8]{ctex}
\usepackage{geometry}
\usepackage{amsmath}
\usepackage{graphicx}
\usepackage{booktabs}
\usepackage{array}
\usepackage{enumitem}
\usepackage{xcolor}
\usepackage{tcolorbox}
\usepackage{listings}
\usepackage{hyperref}

\geometry{left=2.5cm,right=2.5cm,top=2.5cm,bottom=2.5cm}

\lstset{
    language=Python,
    basicstyle=\ttfamily\small,
    keywordstyle=\color{blue},
    commentstyle=\color{green!60!black},
    stringstyle=\color{orange},
    breaklines=true,
    frame=single,
    showstringspaces=false
}

\title{\textbf{\large 第10周教案:核心开发与调试}}
\author{计算机视觉课程组}
\date{}

\begin{document}

\maketitle

\section{基本信息}

\begin{tabular}{|l|p{12cm}|}
\hline
\textbf{周次} & 第10周 \\
\hline
\textbf{主题} & 核心开发与调试 \\
\hline
\textbf{学时} & 3学时(160分钟) \\
\hline
\textbf{故事问题} & 让系统真正跑起来 \\
\hline
\textbf{OBE目标} & A4-系统集成:完成完整阅卷系统开发 \\
\hline
\textbf{项目阶段} & 集中开发 + 测试验证 \\
\hline
\end{tabular}

\section{教学目标}

\begin{enumerate}
\item \textbf{知识目标}:
    \begin{itemize}
    \item 理解系统集成的方法
    \item 掌握调试技巧
    \item 了解测试策略
    \end{itemize}

\item \textbf{能力目标}:
    \begin{itemize}
    \item 能够完成模块集成
    \item 能够进行系统测试
    \item 能够调试和优化系统
    \end{itemize}

\item \textbf{素养目标}:
    \begin{itemize}
    \item 培养工程实践能力
    \item 提升问题解决能力
    \item 建立质量意识
    \end{itemize}
\end{enumerate}

\section{教学重点与难点}

\begin{tcolorbox}[colback=red!5!white,colframe=red!75!black,title=\textbf{教学重点}]
\begin{itemize}
\item 模块集成与接口对接
\item 系统测试与验证
\item 性能优化
\end{itemize}
\end{tcolorbox}

\begin{tcolorbox}[colback=yellow!5!white,colframe=yellow!75!black,title=\textbf{教学难点}]
\begin{itemize}
\item 复杂问题的调试
\item 准确率优化
\end{itemize}
\end{tcolorbox}

\section{教学过程设计}

\subsection{环节一:集成指导(30分钟)}

\subsubsection{1.1 主程序框架(15分钟)}

\begin{lstlisting}
"""
自动阅卷系统主程序
"""

import cv2
import os
import sys
sys.path.append('.')

from modules.preprocess import Preprocessor
from modules.layout import LayoutAnalyzer
from modules.choice_recognizer import ChoiceRecognizer
from modules.judge_recognizer import JudgeRecognizer
from modules.essay_recognizer import EssayRecognizer
from modules.grading import GradingModule
from utils.visualization import visualize_results
from utils.logger import setup_logger

class AutoGradingSystem:
    """自动阅卷系统"""

    def __init__(self, config_path='config.py'):
        """初始化系统"""
        # 加载配置
        self.config = self.load_config(config_path)

        # 初始化日志
        self.logger = setup_logger(self.config.get('log_file', 'app.log'))

        # 初始化各模块
        self.preprocessor = Preprocessor(self.config.get('preprocess', {}))
        self.layout_analyzer = LayoutAnalyzer(self.config.get('layout', {}))
        self.choice_recognizer = ChoiceRecognizer(self.config.get('choice', {}))
        self.judge_recognizer = JudgeRecognizer(self.config.get('judge', {}))
        self.essay_recognizer = EssayRecognizer(self.config.get('essay', {}))
        self.grading_module = GradingModule(self.config.get('grading', {}))

        self.logger.info("Auto Grading System initialized")

    def load_config(self, config_path):
        """加载配置文件"""
        # 简化实现,实际可以用configparser等
        config = {
            'preprocess': {'denoise_method': 'median'},
            'choice': {'threshold': 0.3},
            'judge': {'method': 'feature'},
            'essay': {'method': 'paddle'}
        }
        return config

    def process(self, image_path, output_dir='data/output'):
        """
        处理试卷图像

        参数:
            image_path: 试卷图像路径
            output_dir: 输出目录

        返回:
            处理结果字典
        """
        self.logger.info(f"Processing image: {image_path}")

        # 1. 读取图像
        image = cv2.imread(image_path)
        if image is None:
            raise ValueError(f"Cannot load image: {image_path}")

        # 2. 预处理
        self.logger.info("Step 1: Preprocessing")
        preprocessed = self.preprocessor.process(image)

        # 3. 版面分析
        self.logger.info("Step 2: Layout analysis")
        layout_result = self.layout_analyzer.analyze(image)
        areas = layout_result['areas']

        # 4. 选择题识别
        self.logger.info("Step 3: Choice question recognition")
        choice_results = []
        if 'choice' in areas:
            choice_results = self.choice_recognizer.recognize_all(
                preprocessed['binary'],
                areas['choice']
            )

        # 5. 判断题识别
        self.logger.info("Step 4: Judge question recognition")
        judge_results = []
        if 'judge' in areas:
            judge_results = self.judge_recognizer.recognize_all(
                preprocessed['binary'],
                areas['judge']
            )

        # 6. 简答题识别
        self.logger.info("Step 5: Essay question recognition")
        essay_results = []
        if 'essay' in areas:
            essay_results = self.essay_recognizer.recognize_all(
                preprocessed['binary'],
                areas['essay']
            )

        # 7. 评分
        self.logger.info("Step 6: Grading")
        all_answers = {
            'choice': choice_results,
            'judge': judge_results,
            'essay': essay_results
        }
        grading_result = self.grading_module.grade(all_answers)

        # 8. 整理结果
        result = {
            'image_path': image_path,
            'answers': all_answers,
            'grading': grading_result,
            'preprocessed': preprocessed,
            'layout': layout_result
        }

        # 9. 可视化结果
        self.logger.info("Step 7: Visualization")
        visualize_results(image, result, output_dir)

        self.logger.info("Processing completed")
        return result

def main():
    """主函数"""
    import argparse

    parser = argparse.ArgumentParser(description='Auto Grading System')
    parser.add_argument('image', help='Path to exam image')
    parser.add_argument('--output', default='data/output', help='Output directory')
    parser.add_argument('--config', default='config.py', help='Config file')

    args = parser.parse_args()

    # 创建系统
    system = AutoGradingSystem(args.config)

    # 处理图像
    try:
        result = system.process(args.image, args.output)
        print("\n" + "="*50)
        print("Processing completed successfully!")
        print(f"Score: {result['grading']['total_score']}")
        print(f"Results saved to: {args.output}")
        print("="*50 + "\n")

    except Exception as e:
        print(f"Error: {e}")
        import traceback
        traceback.print_exc()

if __name__ == '__main__':
    main()
\end{lstlisting}

\subsubsection{1.2 评分模块示例(15分钟)}

\begin{lstlisting}
"""
评分模块
"""

import json

class GradingModule:
    """评分模块"""

    def __init__(self, config=None):
        self.config = config or {}
        self.standard_answers = self.load_standard_answers()

    def load_standard_answers(self):
        """
        加载标准答案

        格式: {
            'choice': {1: 'A', 2: 'B', ...},
            'judge': {1: True, 2: False, ...},
            'essay': {1: '答案关键词', ...}
        }
        """
        # 从文件加载
        try:
            with open('data/templates/standard_answers.json', 'r', encoding='utf-8') as f:
                return json.load(f)
        except:
            # 默认示例
            return {
                'choice': {1: 'A', 2: 'B', 3: 'C', 4: 'D', 5: 'A'},
                'judge': {1: True, 2: False, 3: True, 4: False, 5: True},
                'essay': {1: '关键词1 关键词2'}
            }

    def grade_choice(self, student_answers):
        """
        评分选择题

        参数:
            student_answers: 学生答案列表

        返回:
            评分结果
        """
        correct = 0
        total = len(student_answers)
        details = []

        for ans in student_answers:
            question_num = ans['question']
            student_answer = ans['answer']
            standard_answer = self.standard_answers['choice'].get(question_num)

            is_correct = (student_answer == standard_answer)
            if is_correct:
                correct += 1

            details.append({
                'question': question_num,
                'student': student_answer,
                'standard': standard_answer,
                'correct': is_correct
            })

        score_per_question = self.config.get('choice_score', 2)
        total_score = correct * score_per_question

        return {
            'type': 'choice',
            'correct': correct,
            'total': total,
            'score': total_score,
            'details': details
        }

    def grade_judge(self, student_answers):
        """评分判断题"""
        correct = 0
        total = len(student_answers)
        details = []

        for ans in student_answers:
            question_num = ans['question']
            student_answer = ans['answer']
            standard_answer = self.standard_answers['judge'].get(question_num)

            is_correct = (student_answer == standard_answer)
            if is_correct:
                correct += 1

            details.append({
                'question': question_num,
                'student': student_answer,
                'standard': standard_answer,
                'correct': is_correct
            })

        score_per_question = self.config.get('judge_score', 2)
        total_score = correct * score_per_question

        return {
            'type': 'judge',
            'correct': correct,
            'total': total,
            'score': total_score,
            'details': details
        }

    def grade_essay(self, student_answers):
        """
        评分简答题

        简化实现:基于关键词匹配
        实际应用中可能需要人工复核或更复杂的NLP
        """
        results = []

        for ans in student_answers:
            question_num = ans['question']
            student_text = ans.get('text', '')
            standard_keywords = self.standard_answers['essay'].get(question_num, '')

            # 简单的关键词匹配
            keywords_found = 0
            if standard_keywords:
                keywords = standard_keywords.split()
                for keyword in keywords:
                    if keyword in student_text:
                        keywords_found += 1

            # 计算得分比例
            if len(keywords) > 0:
                score_ratio = keywords_found / len(keywords)
            else:
                score_ratio = 0

            max_score = self.config.get('essay_score', 10)
            score = int(max_score * score_ratio)

            results.append({
                'question': question_num,
                'text': student_text,
                'keywords_found': keywords_found,
                'score': score
            })

        total_score = sum(r['score'] for r in results)

        return {
            'type': 'essay',
            'total': len(results),
            'score': total_score,
            'details': results
        }

    def grade(self, all_answers):
        """
        综合评分

        参数:
            all_answers: 所有题型答案

        返回:
            评分结果
        """
        results = {}

        # 选择题评分
        if 'choice' in all_answers and all_answers['choice']:
            results['choice'] = self.grade_choice(all_answers['choice'])

        # 判断题评分
        if 'judge' in all_answers and all_answers['judge']:
            results['judge'] = self.grade_judge(all_answers['judge'])

        # 简答题评分
        if 'essay' in all_answers and all_answers['essay']:
            results['essay'] = self.grade_essay(all_answers['essay'])

        # 计算总分
        total_score = sum(
            r.get('score', 0) for r in results.values()
        )

        results['total_score'] = total_score
        results['summary'] = self.generate_summary(results)

        return results

    def generate_summary(self, results):
        """生成评分摘要"""
        summary = []

        for type_name, result in results.items():
            if type_name == 'total_score' or type_name == 'summary':
                continue

            if 'correct' in result and 'total' in result:
                summary.append(
                    f"{type_name.capitalize()}: {result['correct']}/{result['total']} correct, "
                    f"Score: {result['score']}"
                )
            elif 'score' in result:
                summary.append(
                    f"{type_name.capitalize()}: Score: {result['score']}"
                )

        return '\n'.join(summary)
\end{lstlisting}

\subsection{环节二:测试策略(30分钟)}

\subsubsection{2.1 测试集准备(10分钟)}

\textbf{测试数据准备:}

\begin{lstlisting}
"""
测试数据准备
"""

test_cases = [
    {
        'name': 'basic_choice',
        'image': 'data/test/test_choice_basic.jpg',
        'description': '基础选择题测试',
        'expected': {
            'choice': [
                {'question': 1, 'answer': 'A'},
                {'question': 2, 'answer': 'B'},
                {'question': 3, 'answer': 'C'}
            ]
        }
    },
    {
        'name': 'basic_judge',
        'image': 'data/test/test_judge_basic.jpg',
        'description': '基础判断题测试',
        'expected': {
            'judge': [
                {'question': 1, 'answer': True},
                {'question': 2, 'answer': False}
            ]
        }
    },
    {
        'name': 'full_exam',
        'image': 'data/test/test_exam_full.jpg',
        'description': '完整试卷测试',
        'expected': {
            'choice': 5,  # 至少识别出5道选择题
            'judge': 5,   # 至少识别出5道判断题
            'essay': 1    # 至少识别出1道简答题
        }
    }
]
\end{lstlisting}

\subsubsection{2.2 单元测试(10分钟)}

\begin{lstlisting}
"""
单元测试示例
"""

import unittest
import cv2
import sys
sys.path.append('.')

from modules.choice_recognizer import ChoiceRecognizer

class TestChoiceRecognizer(unittest.TestCase):
    """选择题识别器测试"""

    @classmethod
    def setUpClass(cls):
        """测试前准备"""
        cls.recognizer = ChoiceRecognizer({'threshold': 0.3})

    def test_calculate_density(self):
        """测试密度计算"""
        # 创建测试图像:半黑半白
        test_img = np.zeros((50, 50), dtype=np.uint8)
        test_img[:, :25] = 255  # 左半边白色

        density = self.recognizer.calculate_density(test_img)

        # 应该约等于0.5
        self.assertAlmostEqual(density, 0.5, places=1)

    def test_recognize_single_choice(self):
        """测试单道选择题识别"""
        # 加载测试图像
        test_img = cv2.imread('data/test/single_choice.jpg', cv2.IMREAD_GRAYSCALE)

        # 定义选项位置
        option_positions = [(10, 10, 30, 30), (50, 10, 30, 30)]

        # 识别
        answer, densities = self.recognizer.recognize_question(
            test_img,
            option_positions
        )

        # 验证返回值
        self.assertIn(answer, ['A', 'B', 'C', 'D', None])
        self.assertEqual(len(densities), 2)

    def test_recognize_all_choices(self):
        """测试多道选择题识别"""
        test_img = cv2.imread('data/test/multi_choice.jpg', cv2.IMREAD_GRAYSCALE)

        # 模拟选项组
        choice_groups = [
            [(10, 10, 30, 30), (50, 10, 30, 30), (90, 10, 30, 30), (130, 10, 30, 30)],
            [(10, 50, 30, 30), (50, 50, 30, 30), (90, 50, 30, 30), (130, 50, 30, 30)]
        ]

        results = self.recognizer.recognize_all(test_img, choice_groups)

        # 验证
        self.assertEqual(len(results), 2)
        for r in results:
            self.assertIn('question', r)
            self.assertIn('answer', r)

if __name__ == '__main__':
    unittest.main()
\end{lstlisting}

\subsubsection{2.3 集成测试(10分钟)}

\begin{lstlisting}
"""
集成测试示例
"""

import unittest
import os
from auto_grading_system import AutoGradingSystem

class TestAutoGradingSystem(unittest.TestCase):
    """系统集成测试"""

    def setUp(self):
        """每个测试前执行"""
        self.system = AutoGradingSystem()

    def test_full_pipeline(self):
        """测试完整处理流程"""
        test_image = 'data/test/test_exam_full.jpg'

        # 确保测试图像存在
        self.assertTrue(os.path.exists(test_image))

        # 处理
        result = self.system.process(test_image)

        # 验证结果结构
        self.assertIn('answers', result)
        self.assertIn('grading', result)
        self.assertIn('total_score', result['grading'])

    def test_choice_only(self):
        """测试仅选择题"""
        test_image = 'data/test/test_choice_only.jpg'
        result = self.system.process(test_image)

        # 验证选择题结果
        self.assertIn('choice', result['answers'])
        self.assertIsInstance(result['answers']['choice'], list)

    def test_error_handling(self):
        """测试错误处理"""
        # 测试不存在的图像
        with self.assertRaises(ValueError):
            self.system.process('nonexistent.jpg')

        # 测试损坏的图像
        # self.system.process('data/test/corrupted.jpg')
\end{lstlisting}

\subsection{环节三:开发实践(90分钟)}

\textbf{学生分组开发 + 教师一对一指导}

\subsubsection{3.1 开发检查点(30分钟)}

\textbf{检查点1:选择题模块完成}

\begin{itemize}
\item 能够检测填涂
\item 能够识别A/B/C/D选项
\item 输出格式正确
\end{itemize}

\textbf{检查点2:判断题模块完成}

\begin{itemize}
\item 能够检测√/×符号
\item 能够区分正误
\item 输出格式正确
\end{itemize}

\subsubsection{3.2 集成检查点(30分钟)}

\textbf{检查点3:模块集成完成}

\begin{itemize}
\item 主程序能调用各模块
\item 数据能正确流转
\item 无致命错误
\end{itemize}

\textbf{检查点4:基本功能可用}

\begin{itemize}
\item 能处理测试图像
\item 能输出评分结果
\item 能生成可视化结果
\end{itemize}

\subsubsection{3.3 优化检查点(30分钟)}

\textbf{检查点5:准确率优化}

\begin{itemize}
\item 选择题识别率 $>$80\%
\item 判断题识别率 $>$80\%
\item 简答题能提取文字
\end{itemize}

\subsection{环节四:常见问题解决(10分钟)}

\textbf{问题清单与解决方案:}

\begin{table}[htbp]
\centering
\begin{tabular}{|l|p{8cm}|}
\hline
\textbf{问题} & \textbf{解决方案} \\
\hline
填涂识别不准 & 调整阈值,检查二值化效果 \\
\hline
符号识别失败 & 检查轮廓检测,使用模板匹配作为后备 \\
\hline
OCR结果乱码 & 检查图像预处理,尝试调整分辨率 \\
\hline
模块导入错误 & 检查sys.path设置,确保\_\_init\_\_.py存在 \\
\hline
内存不足 & 处理大图像时先resize,释放不用的变量 \\
\hline
\end{tabular}
\end{table}

\section{课后任务}

\subsection{开发任务}

\begin{enumerate}
\item 完成所有模块开发
\item 完成模块集成
\item 通过基本功能测试
\item 准备演示材料
\end{enumerate}

\subsection{提交内容}

\begin{itemize}
\item 可运行的系统代码
\item 测试结果截图
\item 遇到的问题及解决方案
\item 下周演示计划
\end{itemize}

\section{教学反思}

\subsection{进度提醒}

\begin{itemize}
\item 本周是最后一周完整开发时间
\item 下周只有展示和总结
\item 务必确保系统能基本运行
\end{itemize}

\subsection{下节预告}

下周进行成果展示,每组5分钟演示 + 2分钟答辩。

\end{document}
