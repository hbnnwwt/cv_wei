\documentclass[12pt,a4paper]{article}
\usepackage[UTF8]{ctex}
\usepackage{geometry}
\usepackage{amsmath}
\usepackage{graphicx}
\usepackage{booktabs}
\usepackage{array}
\usepackage{enumitem}
\usepackage{xcolor}
\usepackage{tcolorbox}
\usepackage{listings}
\usepackage{hyperref}

\geometry{left=2.5cm,right=2.5cm,top=2.5cm,bottom=2.5cm}

\lstset{
    language=Python,
    basicstyle=\ttfamily\small,
    keywordstyle=\color{blue},
    commentstyle=\color{green!60!black},
    stringstyle=\color{orange},
    breaklines=true,
    frame=single,
    showstringspaces=false
}

\title{\textbf{\large 第11周教案:成果展示与总结}}
\author{计算机视觉课程组}
\date{}

\begin{document}

\maketitle

\section{基本信息}

\begin{tabular}{|l|p{12cm}|}
\hline
\textbf{周次} & 第11周 \\
\hline
\textbf{主题} & 成果展示与总结 \\
\hline
\textbf{学时} & 2学时(80分钟) \\
\hline
\textbf{故事问题} & 展示你的AI助教 \\
\hline
\textbf{OBE目标} & A4-系统集成:展示完整的阅卷系统 \\
\hline
\textbf{项目阶段} & 成果展示 + 课程总结 \\
\hline
\end{tabular}

\section{教学目标}

\begin{enumerate}
\item \textbf{知识目标}:
    \begin{itemize}
    \item 回顾整个课程的知识体系
    \item 总结计算机视觉技术的应用
    \end{itemize}

\item \textbf{能力目标}:
    \begin{itemize}
    \item 能够展示系统功能
    \item 能够进行技术答辩
    \item 能够撰写项目报告
    \end{itemize}

\item \textbf{素养目标}:
    \begin{itemize}
    \item 培养表达能力
    \item 提升总结反思能力
    \item 建立持续学习意识
    \end{itemize}
\end{enumerate}

\section{教学过程设计}

\subsection{环节一:成果展示(30分钟)}

\subsubsection{1.1 展示安排(5分钟)}

\textbf{展示规则:}
\begin{itemize}
\item 每组展示时间:5分钟
\item 答辩时间:2分钟
\item 总时间:7分钟/组
\end{itemize}

\textbf{展示顺序:}
\begin{enumerate}
\item 抽签决定顺序
\item 按顺序依次展示
\item 严格控制时间
\end{enumerate}

\subsubsection{1.2 展示评分表}

\begin{table}[htbp]
\centering
\begin{tabular}{|l|p{8cm}|p{2cm}|}
\hline
\textbf{评分项} & \textbf{标准} & \textbf{分值} \\
\hline
\multicolumn{3}{|c|}{\textbf{演示效果(30分)}} \\
\hline
系统演示 & 能完整演示系统功能 & 15分 \\
\hline
界面展示 & 界面清晰、操作流畅 & 10分 \\
\hline
时间控制 & 在规定时间内完成展示 & 5分 \\
\hline
\multicolumn{3}{|c|}{\textbf{技术实现(40分)}} \\
\hline
功能完整 & 实现三种题型识别 & 15分 \\
\hline
识别准确率 & 整体识别效果良好 & 15分 \\
\hline
代码质量 & 结构清晰、规范 & 10分 \\
\hline
\multicolumn{3}{|c|}{\textbf{答辩表现(20分)}} \\
\hline
问题理解 & 准确理解评委问题 & 10分 \\
\hline
回答质量 & 回答准确、思路清晰 & 10分 \\
\hline
\multicolumn{3}{|c|}{\textbf{文档报告(10分)}} \\
\hline
项目报告 & 内容完整、逻辑清晰 & 10分 \\
\hline
\textbf{合计} & & \textbf{100分} \\
\hline
\end{tabular}
\end{table}

\subsubsection{1.3 展示内容建议(20分钟)}

\textbf{推荐展示结构(5分钟):}

\begin{enumerate}
\item \textbf{开场介绍(30秒)}
    \begin{itemize}
    \item 组员介绍
    \item 项目名称
    \end{itemize}

\item \textbf{系统演示(3分钟)}
    \begin{itemize}
    \item 展示完整处理流程
    \item 演示输入输出
    \item 可视化识别结果
    \end{itemize}

\item \textbf{技术亮点(1分钟)}
    \begin{itemize}
    \item 创新点
    \item 难点及解决方案
    \end{itemize}

\item \textbf{总结致谢(30秒)}
    \begin{itemize}
    \item 不足与改进方向
    \item 感谢
    \end{itemize}
\end{enumerate}

\textbf{演示准备清单:}
\begin{itemize}
\item[$\square$] 准备演示PPT(3-5页)
\item[$\square$] 准备演示数据(测试试卷)
\item[$\square$] 测试系统稳定性
\item[$\square$] 准备备用方案(防止演示失败)
\item[$\square$] 预演一次
\end{itemize}

\subsection{环节二:互评与点评(30分钟)}

\subsubsection{2.1 学生互评(15分钟)}

\textbf{互评表(学生填写):}

\begin{table}[htbp]
\centering
\begin{tabular}{|l|p{3cm}|p{3cm}|p{3cm}|p{2cm}|}
\hline
\textbf{组别} & \textbf{功能完整} & \textbf{演示效果} & \textbf{创新亮点} & \textbf{评分} \\
\hline
第1组 & & & & \\
\hline
第2组 & & & & \\
\hline
第3组 & & & & \\
\hline
第4组 & & & & \\
\hline
第5组 & & & & \\
\hline
\end{tabular}
\end{table}

\textbf{互评说明:}
\begin{itemize}
\item 每个学生评分
\item 去掉最高分和最低分后取平均
\item 互评占最终成绩的20\%
\end{itemize}

\subsubsection{2.2 教师点评(15分钟)}

\textbf{点评维度:}

\begin{enumerate}
\item \textbf{功能完成度}
    \begin{itemize}
    \item 实现了哪些题型识别
    \item 系统的完整性如何
    \end{itemize}

\item \textbf{技术质量}
    \begin{itemize}
    \item 识别准确率
    \item 代码规范性
    \item 创新性
    \end{itemize}

\item \textbf{团队协作}
    \begin{itemize}
    \item 分工是否明确
    \item 合作是否顺畅
    \end{itemize}

\item \textbf{展示表达}
    \begin{itemize}
    \item 逻辑是否清晰
    \item 表达是否流畅
    \end{itemize}
\end{enumerate}

\subsection{环节三:课程总结(15分钟)}

\subsubsection{3.1 知识体系回顾(5分钟)}

\textbf{11周知识地图:}

\begin{center}
\begin{tabular}{|c|c|}
\hline
\textbf{周次} & \textbf{核心知识点} \\
\hline
1 & 计算机视觉导论、OpenCV基础 \\
\hline
2 & AI辅助编程工具 \\
\hline
3 & 图像预处理(去噪、二值化、矫正) \\
\hline
4 & 版面分析(边缘、轮廓、区域定位) \\
\hline
5 & 选择题识别(OMR、填涂检测) \\
\hline
6 & 判断题识别(形状匹配、符号识别) \\
\hline
7 & OCR印刷文字识别 \\
\hline
8 & 手写文字识别 \\
\hline
9-10 & 系统集成与项目开发 \\
\hline
11 & 成果展示与总结 \\
\hline
\end{tabular}
\end{center}

\textbf{核心技术栈:}

\begin{table}[htbp]
\centering
\begin{tabular}{|l|p{8cm}|}
\hline
\textbf{技术类别} & \textbf{主要内容} \\
\hline
图像处理 & 像素操作、滤波、二值化、几何变换 \\
\hline
特征提取 & 边缘检测、轮廓分析、形状特征 \\
\hline
模式识别 & 填涂检测、模板匹配、OCR \\
\hline
深度学习 & CNN、Transformer(TrOCR) \\
\hline
工程实践 & 模块化设计、测试、调试 \\
\hline
\end{tabular}
\end{table}

\subsubsection{3.2 能力提升总结(5分钟)}

\textbf{学生应该获得的能力:}

\begin{enumerate}
\item \textbf{技术能力}
    \begin{itemize}
    \item 掌握OpenCV基础操作
    \item 能够使用OCR工具
    \item 能够实现简单的CV应用
    \end{itemize}

\item \textbf{工程能力}
    \begin{itemize}
    \item 模块化设计思维
    \item 系统集成能力
    \item 测试调试能力
    \end{itemize}

\item \textbf{学习能力}
    \begin{itemize}
    \item AI辅助编程
    \item 文档查阅能力
    \item 问题解决能力
    \end{itemize}

\item \textbf{协作能力}
    \begin{itemize}
    \item 团队沟通
    \item 分工合作
    \item 项目管理
    \end{itemize}
\end{enumerate}

\subsubsection{3.3 应用展望(5分钟)}

\textbf{计算机视觉的应用领域:}

\begin{itemize}
\item \textbf{教育领域}
    \begin{itemize}
    \item 自动阅卷系统
    \item 课堂行为分析
    \item 在线考试监考
    \end{itemize}

\item \textbf{医疗领域}
    \begin{itemize}
    \item 医学影像诊断
    \item 病灶检测
    \item 健康监测
    \end{itemize}

\item \textbf{交通领域}
    \begin{itemize}
    \item 自动驾驶
    \item 车牌识别
    \item 交通流量分析
    \end{itemize}

\item \textbf{安防领域}
    \begin{itemize}
    \item 人脸识别
    \item 行为分析
    \item 异常检测
    \end{itemize}
\end{itemize}

\subsection{环节四:颁奖与寄语(5分钟)}

\subsubsection{4.1 优秀项目颁奖(3分钟)}

\textbf{奖项设置:}

\begin{itemize}
\item \textbf{最佳功能奖}:识别准确率最高、功能最完整
\item \textbf{最佳技术奖}:技术创新、代码质量高
\item \textbf{最佳展示奖}:演示效果最好、表达清晰
\item \textbf{最佳协作奖}:团队合作优秀
\end{itemize}

\subsubsection{4.2 课程寄语(2分钟)}

\begin{tcolorbox}[colback=blue!5!white,colframe=blue!75!black,title=\textbf{给同学们的话}]
\begin{enumerate}
\item \textbf{持续学习}:CV技术发展迅速,保持学习的热情

\item \textbf{实践为本}:多动手实践,在做中学

\item \textbf{善用工具}:AI是助手,不是替代

\item \textbf{关注应用}:技术服务于实际需求

\item \textbf{保持好奇}:探索未知的领域
\end{enumerate}
\end{tcolorbox}

\section{项目报告要求}

\subsection{报告结构}

\textbf{项目报告应包含以下内容:}

\begin{enumerate}
\item \textbf{摘要}(200字)
    \begin{itemize}
    \item 项目背景、目标、方法、结果
    \end{itemize}

\item \textbf{引言}
    \begin{itemize}
    \item 项目背景与意义
    \item 研究现状
    \item 项目目标
    \end{itemize}

\item \textbf{系统设计}
    \begin{itemize}
    \item 系统架构
    \item 模块设计
    \item 技术选型
    \end{itemize}

\item \textbf{实现细节}
    \begin{itemize}
    \item 各模块实现方法
    \item 关键算法说明
    \item 难点与解决方案
    \end{itemize}

\item \textbf{测试与结果}
    \begin{itemize}
    \item 测试数据
    \item 测试方法
    \item 实验结果
    \item 准确率分析
    \end{itemize}

\item \textbf{总结与展望}
    \begin{itemize}
    \item 工作总结
    \item 不足之处
    \item 改进方向
    \end{itemize}
\end{enumerate}

\subsection{提交要求}

\begin{itemize}
\item 报告格式:PDF
\item 报告篇幅:10-15页
\item 截止时间:课程结束后一周
\item 提交位置:教学平台/邮箱
\end{itemize}

\section{课程资源汇总}

\subsection{学习资源}

\textbf{官方文档:}
\begin{itemize}
\item OpenCV官方文档:\url{https://docs.opencv.org/}
\item PaddleOCR文档:\url{https://github.com/PaddlePaddle/PaddleOCR}
\item TrOCR文档:\url{https://huggingface.co/microsoft/trocr}
\end{itemize}

\textbf{在线课程:}
\begin{itemize}
\item Coursera: Deep Learning Specialization
\item Udacity: Computer Vision Nanodegree
\item B站: 计算机视觉相关课程
\end{itemize}

\subsection{开源项目}

\textbf{推荐学习项目:}
\begin{itemize}
\item Tesseract OCR
\item EasyOCR
\item MMOCR
\item PaddleOCR
\end{itemize}

\section{教学反思}

\subsection{课程总结}

\begin{itemize}
\item \textbf{成功经验}
    \begin{itemize}
    \item 项目驱动的教学方式
    \item AI辅助编程降低门槛
    \item 分组协作培养团队精神
    \end{itemize}

\item \textbf{改进方向}
    \begin{itemize}
    \item 可以增加更多深度学习内容
    \item 可以提供更多数据集支持
    \item 可以设计更多应用场景
    \end{itemize}
\end{itemize}

\subsection{后续学习建议}

\begin{enumerate}
\item \textbf{深入学习}
    \begin{itemize}
    \item 深度学习与CNN
    \item 目标检测(YOLO、Faster R-CNN)
    \item 图像分割
    \end{itemize}

\item \textbf{实践项目}
    \begin{itemize}
    \item 人脸识别系统
    \item 车牌识别系统
    \item 文档扫描与OCR
    \end{itemize}

\item \textbf{参加竞赛}
    \begin{itemize}
    \item Kaggle竞赛
    \item 天池竞赛
    \item 数学建模比赛
    \end{itemize}
\end{enumerate}

\end{document}
